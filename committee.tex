\documentclass[11pt,letterpaper]{article}

\usepackage{multicol}
\usepackage[margin=15mm]{geometry}

\usepackage[super,sort&compress]{natbib}

% define multiple reference lists
\usepackage{multibib}
\newcites{ref,own}{References,Publication list}

% define new environments
\usepackage{amsthm}
\theoremstyle{definition}
\newtheorem*{hypothesis}{Hypothesis}
\newtheorem*{objective}{Objective}

% disable chapter, section, and subsection numbering
\setcounter{secnumdepth}{-1}

\begin{document}

\begin{titlepage}

\setlength{\topmargin}{30mm}

\begin{center}
 
\textsc{\Large Committee meeting report}\bigskip

{\huge \bfseries Genomics of medulloblastoma molecular subgroups}\\[10mm]

\begin{minipage}{0.6\textwidth}
\begin{multicols}{2}
	\begin{flushleft}
		\large \emph{Student:}\\
		David J. H. \textsc{Shih}
	\end{flushleft}
	\vfill
	\columnbreak
	\begin{flushright}
		\large \emph{Supervisors:}\\
		Dr.~Michael D. \textsc{Taylor}\\
		Dr.~Gary \textsc{Bader}\\
	\end{flushright}
	\begin{flushright}
		\large \emph{Committee:}\\
		Dr.~Meredith \textsc{Irwin}\\
		Dr.~Quaid \textsc{Morris}
	\end{flushright}
\end{multicols}
\end{minipage}

\vspace{100mm}

{ \large 
April 24, 2012\\
\medskip
Donnelly Centre for Cellular and Biomolecular Research\\
160 College St, 6th floor meeting room
}

\end{center}

\end{titlepage}


\section{Overview}

Medulloblastoma is the most comon solid childhood malignancy \citeref{mainprize00}. Current therapy for medulloblastoma --- including surgical resection, radiation of the entire brain and spinal cord, and aggressive chemotherapy --- yields five-year survival rates of 60-70\% \citeref{gajjar06}. Survivors are often left with significant neurological, intellectural, and physical disabilities secondary to the effects of these non-specific, cytotoxic therapies on the developing nervous system \citeref{spiegler04,mabbott05}.

Recent evidence suggests that medulloblastoma in fact comprises a group of molecularly and clinically distinct entities whose clinical and genetic differences may require separate therapeutic strategies \citeref{thompson06,kool08,northcott11a,remke11,cho11}. Four principal subgroups\citeref{taylor12} of medulloblastoma have been identified: WNT, SHH, Group 3, and Group 4, and there is preliminary evidence for clinically significant subdivisions of the subgroups \citeref{northcott11a,cho11,remke11,taylor12,northcott11b}. Targeted therapies based on the genetics of the disease are not currently in use. However, inhibitors of the Sonic Hedgehog (Shh) pathway activator, Smoothened, have shown some early evidence of efficacy \citeref{rudin09}. With a deeper understanding of the molecular biology of medulloblastoma subgroups, we hope to herald a new era of medullblastoma treatment based on selective, specific, targeted therapy.

At present, there are at least three main obstacles that hinders the development of targeted therapy against medulloblastoma molecular subgroups:

\begin{enumerate}
	\item The lack of a clinically applicable assay for molecular subgrouping of medulloblastoma.
	\item The paucity of actionable targets for WNT, Group 3, and Group 4 medulloblastomas.
	\item Clinical trials of targeted therapy proceed without confirmed identifications of molecular targets in recurrent mebulloblastomas.
\end{enumerate}

The objectives of my thesis are to provide viable solutions to these issues and to demonstrate the clinical significance of molecular subgrouping.

\subsection{Aim 1: Molecular classification of medublloblastoma in clinical contexts}

Although the retrospective classification of medulloblastoma has been scientifically informative, molecular subgrouping has not been applied in the context of a propsective clinical trial. One major obstacle is the lack of fresh-frozen samples for the most clinical cases. Expression profiling, on which molecular classification was based, depends on the availability of high-quality RNA. In contrast, clinical samples are routinely subjected to formalin-fixation and paraffin-embedding (FFPE), which preserves tissue integrity but causes nucleic acid degradation. To faciliate the development of therapy specfically targeted against molecular subgroups, we sought to establish an molecular subgrouping assay that can be clinically applied on FFPE samples. In collaboration with Paul Northcott, I established a simple analytic pipeline to molecular subgrouping using expression data generated by Nanostring assays, and demonstrated its high classification accuracy on FFPE samples \citeown{northcott12}.

\subsection{Aim 2: Target identification through copy-number profiling of medulloblastoma}

After having established a clinically applicable molecular classification methodology, I turned to the problem of identifying molecular targets in medulloblastoma. Unlike SHH medulloblastomas, actionable targets for WNT, Group 3, and Group 4 tumours have yet been identified. However, prior attempts may have been underpowered to discriminate the genomic differences among the four moleulcar subgroups. To this end, the Medulloblastoma Advanced Genomics International Consortium (MAGIC), consisting of scientists and physicians from 43 cities across the globe, has gathered >1200 medulloblastomas. Paul Northcott and I have analyzed the genomic copy-number profiles of the tumorus by Single Nucleotide Polymorphism (SNP) arrays. We have identified genes and pathways that characterize each medulloblastoma subgroup \citeown{shih12}.

\subsection{Aim 3: Demarcating the genomic differences between primary and recurrent tumours}

To date, most clinical efforts and changes in pediatric tumour treatment has been in the optimization of chemotherapy and radiation protocols \citeref{bouffet10}. While survival rates for medulloblastoma patients have improved over the years, patients with recurrent medulloblastoma currently have no effective treatment option. These patients are often enrolled in clinical trials based on diagnostic tests of their primary disease, under the likely erroneous assumption that recurrent tumours are identical to their primary counterparts. MAGIC has collected one of the largest set of matched primary-recurrent medulloblastomas. I will analyze these tumorus by copy-number profiling and Whole-Genome Sequencing (WGS). From the data, I will identify the genomic differences between primary and recurrence and establish their biological significance.

\subsection{Aim 4: Assessing the biological disparities of medulloblastoma subgroups}

The existence of distinct medulloblastoma subgroups is supported by mounting evidence and has begun to gain widespread appreciation. An important emerging question is: does 'medulloblastoma' exist? That is, is it the diagnosis of `medulloblastoma' more clinical useful than the diagnosis of `WNT', `SHH', `Group 3', or `Group 4' pediatric brain tumour? Once effective molecular therapies are established, it may be more suitable to classify brain tumours based on their molecular biology, from which response to therapy can be more accurately determined than conventional histopathological classification. To begin to address this question, I will compare the expression profiles of various brain tumours, and determine whether medulloblastoma subgroups are interspersed among different brain tumours types. If this were the case, the disparate biologies of medulloblastoma subgroups would warrant the retirement of the label, `medulloblastoma'.

\section{Summary of progress}

Aim 1 has been completed\citeown{northcott12}, and Aim 2 is nearly completion\citeown{shih12}.
I have begun working on Aim 3. I am waiting for the generation of WGS data on the primary and recurrent samples in our collaborator's lab (Dr. Marco Marra, Michael Smith Genome Sciences Centre).
Aim 4 is pending the collection and submission of additional tumours samples for expression profiling.

\nociteown{wu12,rausch12,dubuc12}

\section{Aim 1: Molecular classification of medublloblastoma in clinical contexts}

\begin{objective}
To develop a clinically applicable assay for molecular classification of medulloblastoma.
\end{objective}

\section{Aim 2: Target identification through copy-number profiling of medulloblastoma}

\begin{hypothesis}
Each medulloblastoma molecular subgroup is characterized by specific genomic aberrations.
\end{hypothesis}

\subsection{Minor aim: Development of an improved tool for identifying recurrent somatic copy-number aberrations}

\begin{objective}
To develop a fast algorithm for population-based identification of recurrent somatic copy-number aberrations.
\end{objective}

\section{Aim 3: Demarcating the genomic differences between primary and recurrent tumours}

\begin{hypothesis}
Primary and recurrent medulloblastomas are genetically different, which may cause differential responses to therapy.
\end{hypothesis}

\section{Aim 4: Assessing the biological disparities of medulloblastoma subgroups}

\begin{hypothesis}
The diagnosis of medulloblastoma provides little information regarding its biology.
\end{hypothesis}


\bibliographystyleref{nature}
\bibliographyref{mb}

\bibliographystyleown{nature}
\bibliographyown{shih-phd}

\end{document}

