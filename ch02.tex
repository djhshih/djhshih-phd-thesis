\chapter{Molecular classification of medulloblastoma for clinicians}
\chaptermark{Molecular classification}
\label{ch:mb-class}

\begin{objective}
To develop a clinically applicable assay for molecular classification of medulloblastoma.
\end{objective}

class prediction and class discovery
compare to taxonomic identification and taxonomic classification

\section{Materials and methods}

\subsection{Patient samples}

All samples were obtained in accordance with the Research Ethics Board at the Hospital for Sick Children (Toronto, Canada).  Primary medulloblastomas comprising the training series for nanoString ($n = 101$) have been previously described.  Samples contributing to the validation series ($n = 131$) have been previously described and were obtained as total RNA extracted from fresh-frozen tissue from the DKFZ (Heidelberg, Germany; Remke series, $n = 56$) \citeref{remke11}, the Dana-Farber Cancer Institute (Boston, USA; Cho series, $n = 39$) \citeref{cho11}, and Marcel Kool (DKFZ, Heidelberg, Germany; Kool series, n=36) \citeref{kool08}.  \gls{ffpe} cases ($n = 84$) were obtained as paraffin sections from the Hospital for Sick Children (Toronto, Canada; $n = 34$), John’s Hopkins University (Baltimore, USA; $n = 25$), and the DKFZ (Heidelberg, Germany; $n = 25$).

\subsection{Tissue sample processing}

Total RNA was extracted from fresh-frozen tissue using the Trizol method (Invitrogen) according to the manufacturer’s instructions.  For FFPE samples, 3 to 5 paraffin sections per sample were first deparaffinized with xylene prior to RNA extraction using the RNeasy FFPE kit (Qiagen) as directed by the manufacturer.  RNA concentration was measured using a Nanodrop 1000 instrument (Nanodrop). Paul Northcott processed the samples.

\subsection{RNA integrity assessment}

RNA derived from FFPE material was analyzed with the Agilent Bioanalyzer to determine RNA integrity at \gls{tcag}. Smear analysis was performed using the Agilent 2100 expert software to determine the proportion of RNA C300 nucleotides (nt) within a given sample.

\subsection{nanoString CodeSet design and expression quantification}

Signature genes for each medulloblastoma subgroup were included in the CodeSet on the basis of their observed subgroup-specific expression as previously determined by Affymetrix exon array analysis.  The CodeSet was designed to consist of a total of 25 genes with 5 to 6 signature genes included for each subgroup: WNT (\gene{WIF1}, \gene{TNC}, \gene{GAD1}, \gene{DKK2}, \gene{EMX2}), SHH (\gene{PDLIM3}, \gene{EYA1}, \gene{HHIP}, \gene{ATOH1}, \gene{SFRP1}), Group3 (\gene{IMPG2}, \gene{GABRA5}, \gene{EGFL11}, \gene{NRL}, \gene{MAB21L2}, \gene{NPR3}), Group4 (\gene{KCNA1}, \gene{EOMES}, \gene{KHDRBS2}, \gene{RBM24}, \gene{UNC5D}, \gene{OAS1}).  Three housekeeping genes (\gene{ACTB}, \gene{GAPDH}, and \gene{LDHA}) were also included in the CodeSet for biological normalization purposes.  Probe sets for each gene in the CodeSet were designed and synthesized at nanoString Technologies. See Northcott \emph{et al.}\ \citeself{northcott12} for details on the subgroup-specific expression markers (note that Group C has been renamed Group3 and Group D has been renamed Group4 since the publication of this study).

Total RNA (100 ng) from fresh-frozen tissue and \gls{ffpe} material was analyzed using the nanoString nCounter Analysis System at the University Health Network Microarray Centre (Toronto, Canada), the Oncogenomics Core Facility at the University of Miami (Miami, USA), and the Frontiers in Genetics Facility at the University of Geneva (Geneva, Switzerland).  All procedures related to mRNA quantification including sample preparation, hybridization, detection, and scanning were carried out as recommended by nanoString Technologies.

\subsection{NanoString Data processing and class prediction}

Raw nanoString counts for each gene within each experiment were subjected to a technical normalization using the counts obtained for positive control probe sets prior to a biological normalization using the three housekeeping genes included in the CodeSet.  Normalized data were log2-transformed and used as input for class prediction analysis.

A series of medulloblastomas with known subgroup affiliation ($n = 101$) were used to establish a training dataset for subsequent class prediction analysis of independent cohorts utilized in the study.  Various class prediction algorithms were assessed by a 10-fold cross-validation scheme, using a set of scoring indices to establish a pipeline for prediction of medulloblastoma subgroups using nanoString data derived from the training series.  Based on superior performance in cross-validation analysis, the PAM method was selected for all downstream class prediction analyses.  A detailed summary of class prediction methods and evaluation of their performance are available in supplementary methods.

All class prediction analyses were performed in the R statistical programming environment (v2.13). Implementations of the class prediction algorithms were imported
from the following R packages: MASS v7.3 (linear discriminant analysis; LDA), class v7.3 (k-nearest neighbor; KNN), e1071 v1.5 (support vector machine; SVM), nnet v7.3 (multinomial log-linear model; MULT), and pamr v1.51 (prediction analysis for microarrays; PAM) [18]. During cross-validation, the training set of 101 samples was randomly split into 10 partitions. Each class predictor was trained on nine of the partitions, and the performance of the predictor was subsequently tested on the one remaining partition. Each of the 10 partitions was used as the testing set in turn for a round of cross-validation, for a total of 10,000 rounds of cross-validation, which was
repeated three times with reproducible results.

The scoring indices used during testing were accuracy, Jaccard similarity index, Rand index, adjusted Rand index, and Fowlkes–Mallows index. The latter four indices are different indices for determining the similarity between
two groupings, which are the known and predicted classifications of samples in the current analysis. These indices serve as more stringent measures of accuracy in multi-class prediction. Aside from the accuracy measures (validity),
the reliabilities of the predictors were also determined using Shannon entropy as a measure of uncertainty. Predictors with varying predicted classes for the same sample across the cross-validation rounds have higher entropy values, and are hence less reliable.

Since the model parameters for SVM can affect the prediction performance, these parameters were optimized by a grid search in a separate round of cross-validation. The ranges of searched parameter values were: $[2^{-5}, 2^{15}]$ for \code{C}; $[2^{-15}, 2^3]$ for \code{gamma}; $[2, 8]$ for \code{degree}; $[-1, 1]$ for \code{coef0}. Further, SVM using different kernels (linear, radial basis, polynomial, and sigmoid) were assessed, and the kernel with the best performance was selected. Similarly for KNN, the best model was selected from models with different $k$.

\subsection{Regression analysis of prediction accuracy}

Cumulative prediction accuracy was modeled as a function of FFPE sample age. The prediction accuracies were first calculated for each sample age year-group. The cumulative accuracies were determined by calculating the cumulative sum of the accuracies, weighted by the size of each year-group. The data were fitted using a 5-parameter logistic regression model, as implemented in the drc v2.1 R package. The maximum asymptote parameter ($D$) was constrained at 1 in order to reflect the high accuracy the predictor achieved with recent FFPE samples.


\section{Results}

The nanoString nCounter technology \citeref{geiss08} was used to directly measure the expression level of 22 medulloblastoma subgroup specific signature genes. The nanoString assay directly interrogates nucleic acid levels without \gls{pcr} amplification (or other enzymatic reactions) in multiplexed system, using pairs of fluorescent probes that bind to target sequences. We developed an analytic method that can accurately predict molecular subgroups of medulloblastoma, even on archival \gls{ffpe} samples \citeself{northcott12}.

A set of widely used classifiers (e.g. support-vector machine, linear discriminant analysis, multinomial logistic regression, k-nearest neighbour, pattern analsis of microarrays) were trained using a training set of 101 medulloblastomas with known subgroup affiliations. The classifiers were tuned and assessed using cross-validation. The most accurate classifier (across all tested accuracy measures) was selected for classification.

\begin{figure}[hb]
	\begin{center}
		\includegraphics[width=\textwidth]{fig/nanostr-class/nanostr-valid.pdf}
	\end{center}
	\caption[Validation of classification assay on independent medulloblastoma cohorts]
	{
	Validation of classification assay on independent medulloblastoma cohorts.
	\textbf{a-c}, Expression heatmaps of nanoString class-predicted medulloblastomas of known subgroup status as published by Remke et al.\citeref{remke11} (a), Cho et al.\citeref{cho11} (b), and Kool et al.\citeref{kool08} (c). Samples are sorted according to subgroup predictions. Known expression subgroup affiliations and erroneously classified cases are marked above the heatmap.
	\textbf{d}, \emph{Left}, Pie chart depicting the known subgroup distribution of medulloblastomas from the three independent cohorts analyzed in \textbf{a-c} ($n = 130$) and the subgroups predicted by nanoString profiling. Misclassified cases are marked within each slice according to the predicted subgroups. \emph{Right}, Pie chart of class prediction accuracy ($127/130$) from the validation set. Adapted from Northcott et al.\citeself{northcott12}
	}
	\label{fig:nanostr-valid}
\end{figure}

\clearpage

The assay was validated on an external set of 130 non-overlapping medulloblastomas, and it achieved an accuracy of 98\% (\citefig{nanostr-valid}). Further, the assay yielded reproducible predictions when repeated in three independent laboratories \citeself{northcott12}. The clinical applicability of the assay was demonstrated by its predictive accuracy on \gls{ffpe} samples of archival ages $\leq 8$ years (\citefig{nanostr-ffpe}). The accuracy decreased on older \gls{ffpe} samples, presumably due to poorer RNA integrity, though standard measurements of RNA quality were not correlated with accuracy \citeself{northcott12}.

\begin{figure}[ht]
	\begin{center}
		\includegraphics[width=\textwidth]{fig/nanostr-class/nanostr-ffpe.pdf}
	\end{center}
	\caption[Classification performance on formalin-fixed paraffin embedded archival samples]
	{
	Classification performance on formalin-fixed paraffin embedded archival samples.
	\textbf{a}, Class prediction accuracy in relation to sample age of archival medulloblastomas stored as \gls{ffpe} material ($n = 84$). Samples obtained within the past 8 years exhibit accuracies of $\geq 95\%$, as demarcated by the red vertical line.
	\textbf{b}, Heatmap of nanoString data showing class predictions for \gls{ffpe} cases of $\leq 8$ years confidently predicted by the assay ($n = 28$). Samples are sorted according to subgroup prediction. All cases satisfying prediction probability threshold were assigned to the correct subgroup ($28/28$). Adapted from Northcott et al.\citeself{northcott12}
	}
	\label{fig:nanostr-ffpe}
\end{figure}

Since the initial publication of the assay for molecular classification, we have analyzed over 1000 medulloblastoma samples and identified a few cases were replicate assays yielded conflicting results. Further examination revealed that poor sample quality and suboptimal assay conditions likely contributed to classification discrepancies. Therefore, additional quality control measures were implemented, which are especially important for developing this assay further for \gls{clia} certification.

Given that standard measurements of RNA quality were insufficient for predicting assay accuracy \citeself{northcott12}, the mean signals of the endogenous control probes included in the nanoString assay were used to assess whether sufficient quantities of intact, undegraded were present in the samples, using a outlier detection method. A Gaussian mixture model was fitted to all collected nanoString data to establish the nominal range for mean endogenous-control signals. Samples with mean signals that deviate significantly from this range at a significance level of 0.01 were identified as outliers. Such samples, due to extensive RNA degradation, cannot be assigned a molecular subgroup, and they may require classification using DNA copy-number or methylation profiling.

Samples with sufficiently high-quality RNA may yet yield uninterpretable results when suboptimal assay conditions confound the measurements. Therefore, signals from positive control and negative control probes are examined to identify assay reactions that may have failed and hence produced unreliable measurements. The current collection of nanoString data was used to establishe the nominal range of positive and negative signals, using Gaussian mixture and multiple negative binomial models, respectively. As above, measurements that deviate significantly from the nominal range at a significance level of 0.01 were considered outliers. Samples that fail this quality control criterion may simply be run again.

Furthermore, multi-sample assays are not amendable to reproducible clinical analysis of samples in real-time, owing to time constraints and potential batch effects. Single-sample nanoString assays were therefore tested for concordance with previous results. With the appropriate quality control and improved normalization procedures implemented, 100\% concordance was achieved with single-sample assays, which further enhanced the clinical utility of molecular classification assay.

\section{Discussion}

Above all, a rapid, reliable, and reproducible assay was developed for assigning molecular subgroups to clinical samples, available as frozen or recent \gls{ffpe} material, and this assay has been developed further use in a clinical laboratory. Critically, stringent quality control must accompany the nanoString classification assay, lest its potentials be shadowed by concerns of reproducibility and predictability, an ignominy that has long plagued the microarray technology \citeref{shi08,deronde10,weigelt10,ein-dor06,frantz05,michiels05,ioannidis05,marshall04,check04,tan03,tilstone03}.

Classification of medulloblastoma led to many discoveries \citeself{shih14,shih12,perreault14,kool14,ramaswamy14,ramaswamy13,remke13,dey13,zhukova13,dubuc13,dubuc12,wu12,jones12}.

\clearpage
