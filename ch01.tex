\chapter{Introduction}
\label{ch:intro}

One in 285 children are diagnosed with cancer before the age of 20, and cancer is the second leading cause of death in children \citeref{ward14, murphy13}. Owing to advances in treatment, the 5-year survival of children with cancer has steadily increased from 63\% in 1975 to 85\% in 2006 \citeref{howlader14}. The improvements in survival, however, vary considerably across cancer types. In 1975, the 5-year survival rates for childhood leukemia and \gls{cns} tumour were 50\% and 57\%, respectively. In 2006, the survival rate for the former reached 87\% while the latter lagged behind at 74\% \citeref{howlader14}. Leukemia and \gls{cns} tumour are biologically very different types of cancer: they arise from different cells within different organs, hijack different cellular signaling programs to effect uncontrolled proliferation, and reside in different locations that permit different accessibilities to treatment. It should come as no surprise then that leukemia and \gls{cns} tumour respond differently to similar modern anti-cancer treatments (consisting of chemotherapy and radiotherapy). Furthermore, leukemia and \gls{cns} tumour can each be classified into additional cancer subtypes, which also response varyingly to treatment. Within leukemias, the 5-year patient survival of acute lymphocytic leukemia is 92\% and that of acute myelogenous leukemia is 66\% \citeref{howlader14}. Within \gls{cns} tumours, the 5-year patient survival of pilocytic astrocytoma is 94\% and that of medulloblastoma is 72\% \citeref{ostrom14}. Indeed, the responses of cancers to therapy depend on not only the affected tissue (blood vs.\ \gls{cns}) but also the cellular origin (lymphoid vs.\ myeloid and astrocytic vs.\ embryonal). Using anatomical locations and cellular appearance, clinicians classify cancer into different types in order to predict the responses to treatment. Current post-surgical treatment modalities function largely through one predominant mechanism (inhibiting cellular division and inducing apoptosis), but novel anti-cancer therapies are increasing in specificity against aberrant cells and diverging in mechanisms of action. Accordingly, the classification of cancer will become increasingly important for selecting the right treatment for each patient.

While the current classification of cancer using the primary site of occurrence and the morphology of the cell have been useful for predicting patient response, the advent of molecular profiling and sequencing technologies can refine this classification further and facilitate the development of therapies targeted against specific aberrations within a cancer. Currently, the \gls{who} classify \gls{cns} tumours into 86 distinct entities based primarily on histological appearance \citeref{who07}. (In 1993, \gls{who} recognized \emph{only} 36 subtypes of \gls{cns} tumours \citeref{kleihues93}.) Despite this detailed level of classification, the 5-year survival of children with \gls{cns} tumours has remained stagnant at about 75\% since 1996 \citeref{howlader14}. The discrepancy between discovery of new \gls{cns} tumour histotypes and the lack of therapeutic improvement suggest that the histological classification of \gls{cns} tumours may be insufficient to identify biologically similar tumours and facilitate the development of novel effective therapy. In particular, it was noted as early as 1971 that medulloblastoma with desmoplastic histology exhibits longer patient survival \citeref{chatty71} in response to radiotherapy; however, it remains unclear why the desmoplastic phenotype is associated with longer survival or which novel therapeutic agent may be effective against this histological variant. In comparison, a medulloblastoma tumour with a loss-of-function mutation in \gene{PTCH1} (endogenous suppressor of \gls{shh} signaling) would depend on hyperactive \gls{shh} signaling for growth, the patient would therefore be expected to --- and indeed does --- respond to inhibition of \gls{shh} signaling \citeref{rudin09}. Characterizing the genetic mutations and understanding the biology of a tumour can therefore help guide the discovery of novel therapies and the selection of suitable treatment modality or intensity.

In order to identify cancer mutations against which therapeutic intervention may be beneficial, we would need to distinguish between mutations contribute to tumourigenesis (known as \emphterm{driver} mutations) from those that do not (known as \emphterm{passenger} mutations). Cancer cells accumulate somatic mutations by having disrupted DNA damage response or DNA repair pathways. Somatic mutations generally occur stochastically (with some exceptions including antibody diversification by activation-induced cytidine deaminase). Most mutations in a cancer cell would thus be deleterious or neutral to the cell, and mutations that confer selective advantage would increase in \emphterm{cellular frequency} (proportion of cells habouring mutation) within a tumour. Not all mutations observed at high cellular frequency, however, contribute to tumourigenesis. Since multiple mutations may occur each time a cancer cell replicate its DNA and multiple cell divisions may occur before a new driver mutation arises, a population of cancer cells arising from the same parental cell will share one driver mutation and many additional mutations that do not contribute to tumourigenesis. Additionally, not all driver mutations become clonal (i.e. reach 100\% cellular frequency) due to interaction among cells. In glioblastoma multiforme, cancer cells expressing mutant EGFR secrete cytokines to promote the growth of cells expressing wild-type EGFR \citeref{inda10}; hence, the tumour heterogeneity is sustained by this paracrine mechanism and the \gene{EGFR} mutation does not become clonal. Taking these possibilities into consideration, high cellular frequency is neither a necessary nor a sufficient condition for a mutation to be a driver. Instead, a robust method for distinguish driver from passenger mutations would be to assess its \emphterm{recurrence frequency}: the frequency that a mutation is observed across samples. If a mutation drives tumour formation, we would expect to find across multiple tumours from different patients, provided that the tumours are biologically similar and arise through similar molecular mechanisms. Above all, in order to use recurrence frequency as a criteria for identifying driver mutations and recognizing genomic patterns, we would need to first classify the tumours into homogeneous groups.

This thesis will focus on the refining the classification of \emphterm{medulloblastoma}, a malignant brain tumour occurring in the cerebellum and the posterior fossa. \gls{who} classifies medulloblastoma as a grade IV (highly aggressive) embryonal tumour \citeref{who07}. Its diagnosis is made by the anatomical location of the tumour and histological morphology of the cells. Medulloblastoma was once a universally fatal disease; today, 58\% of patients are expected to live longer than 15 years \citeref{ward14}. To predict whether a patient will respond favourably to treatment, clinicians currently categorize medulloblastoma by such features as metastatic presentation and histological variants. Medulloblastoma may be divided into classic, desmoplastic, and anaplastic (large cell) histotypes. While this histological classification of medulloblastoma provides some prognostic value, it provide scant insight into potential biological mechanisms. Instead, several studies generated RNA expression profiles of primary medulloblastoma tumours using microarrays and sought to characterize medulloblastoma by patterns of RNA expression. Since each expression profile captures a snapshot of the overall molecular state of a tumour, biologically similar tumours exhibit similar expression profiles. Therefore, tumours with similar expression profiles can be clustered (grouped) together to discover biologically homogeneous molecular classes. By these clustering analyses, four molecular classes (henceforth known as \emphterm{subgroups}) of medulloblastoma were discovered: \emphterm{WNT}, \emphterm{SHH}, \emphterm{Group3}, and \emphterm{Group4} medulloblastoma \citeref{taylor12, northcott11a, kool12, remke11, cho11}. As the names suggest, WNT medulloblastoma have hyperactive \gls{wnt} signaling and SHH medullobalstoma have increased \gls{shh} signaling compared to the other subgroups, while Group3 and Group4 are less well defined \citeref{northcott11a, kool12, cho11}. We hypothesize that classifying medulloblastoma into these (relatively) homogeneous molecular subgroups will shed light on its cancer biology and consequently improve prediction of treatment response and point to novel therapeutic targets.

The remainder of this chapter will discuss contributing genetic and non-genetic factors in medulloblastoma and their implications for the classification and treatment of medullobastoma. At the end of this chapter, the main research objectives of this thesis are outlined. In \citech{mb-class}, we will propose and validate a method whereby medulloblastoma tumours may be classified in the clinical setting. The immediate utility of this molecular classification system for patient risk stratification will be shown in \citech{clin-prog}, and the potential implication of molecular classification for therapeutic discovery will be presented in \citech{target-id}. To make this thesis accessible to a broader audience, the \emphlab{Appendix} presents clarifications for topics including gene nomenclature, mouse model notation, and computational algorithms.


\section{Epidemiology and genetic predisposition}

Medulloblastoma occurs at an annual incidence of 4.1 per million children under 20, ranking as the most common type of malignant brain tumour in childhood \citeref{ostrom14}. The majority of patients with medulloblastoma are diagnosed before the age of 20, and the median age at presentation is 8 years \citeself{shih14}. As medulloblastoma is almost 10 more likely to afflict children than adults, medulloblastoma is a disease of childhood \citeref{smoll12}, suggesting perhaps genetics may play a role in this disease. Consistent with this notion, medulloblastoma can occur simultaneously in monozygotic twins \citeref{scheurlen96}, and having an affected sibling increases a child's risk of developing medulloblastoma by 4 fold \citeref{hemminki09}. While most medulloblastoma cases are sporadic with unknown genetic contribution, several genetic disorders can predispose children to developing medulloblastoma, as well as many other malignancies (\citetab{genetic-disorders}). For instance, mutations in genes of DNA damage response and repair pathways, including \gene{TP53}, \gene{ATM}, \gene{BRAC2}, \gene{NBN}, predispose a child to develop a spectrum of tumours, including medulloblastoma, at a young age \citeref{pearson82, barel98, guran99, yamazaki00, garre09, villani11, hart87, reiman11, offit03, hirsch04, reid05, dewire09, ciara10, alexiou12}.

Medulloblastoma can also arise as a consequence of germline mutations in developmental signaling pathways. Germline loss-of-function mutations in negative regulators of \gls{shh} signaling, including \gene{PTCH1}, \gene{PTCH2}, \gene{SUFU}, causes the basal cell nevus syndrome (Gorlin syndrome) and predispose patients to medulloblastoma, basal cell carcinoma, and other cancers. In the context of these mutations, hyperactive \gls{shh} signaling during neural development presumably would cause patients to develop SHH medulloblastoma. This notion is supported by mouse models with heterozygous \gene{Ptch1} mutation \citeref{goodrich97, wetmore00, wetmore01, pazzaglia02, gibson10} and the \gene{Sufu}\high{+/-} \gene{Trp53}\high{-/-} mouse model \citeref{heby-henricson12}, which both develop medulloblastoma with active Shh signaling. Further, Gorlin syndrome is highly prevalent in human patients with SHH medulloblastoma \citeref{kool14}, though it is yet unclear whether patients with Gorlin syndrome exclusively develop SHH medullobalstoma. Additionally, germline mutations in \gene{APC} activates \gls{wnt} signaling, manifests as Turcot syndrome, and predisposes patients to medulloblastoma. The tumours arising in these patients would be expected to be of the WNT subgroup; however, the crosstalk between the \gls{shh} and the \gls{wnt} signaling pathways in the context of neural development and different genetic backgrounds could modulate the phenotypic outcome of \gene{APC} mutation.

\begin{table}[h]
	\caption[Genetic disorders predisposing to medulloblastoma]
	{
		Genetic disorders predisposing to medulloblastoma
	}
	\label{tab:genetic-disorders}
	\footnotesize
	\setlength{\extrarowheight}{0.5em}
	\centering
	\begin{tabular}{l | l | l}
		\hline
		\textbf{Genetic disorder} & \textbf{Mutated genes} & \textbf{Reference} \\
		\hline
		Li-Fraumeni syndrome & \gene{TP53} & \citeplainref{pearson82, barel98, guran99, yamazaki00, garre09, villani11} \\
		Ataxia telangiectasia (Louis-Bar syndrome) & \gene{ATM} & \citeplainref{hart87, reiman11} \\
		Fanconi anemia & \gene{BRCA2} (\gene{FANCD1}) & \citeplainref{offit03, hirsch04, reid05} \\
		Nijmegen breakage syndrome & \gene{NBN} (\gene{NBS1}) & \citeplainref{bakhshi03, distel03, ciara10} \\
		Fragile X syndrome & \gene{FMR1} & \citeplainref{garre09, alexiou12} \\
		Neurofibromatosis type 1 (Von Recklinghausen's disease) & \gene{NF1} & \citeplainref{martinez-lage02, garre09} \\
		DICER1 syndrome & \gene{DICER1} & \citeplainref{slade11} \\
		Rubinstein-Taybi syndrome & \gene{CREBBP} & \citeplainref{bourdeaut14} \\
		Basal cell nevus syndrome (Gorlin syndrome) & \gene{PTCH1}, \gene{PTCH2}, \gene{SUFU} & \citeplainref{wolter97, taylor02, crawford09, garre09, brugieres10, jones11, brugieres12} \\
		Turcot syndrome & \gene{APC} & \citeplainref{hamilton95} \\
		\hline
	\end{tabular}
\end{table}

Germline mutation in \gene{SMARCB1} (\gene{SNF5}/\gene{INI}) had been associated with medulloblastoma \citeref{sevenet99, lee02}, but the brain tumours in these patients are now classified as \gls{atrt}, a tumour type that \gls{who} first recognized in its 1993 classification \citeref{kleihues93}. \gls{atrt} and medulloblastoma are similar by histology\citeref{utsuki03} and by \gls{mri} \citeref{koral08}. Currently, the \gene{SMARCB1} mutation is widely accepted as an diagnostic indicator of \gls{atrt} and distinguishes \gls{atrt} from medulloblastoma \citeref{biegel00, kraus02}. Indeed, \gene{SMARCB1} is an example in which a genetic mutation superseded histological diagnosis and redefined cancer classification.

While mutations are pivotal factors in the formation of medulloblastoma, numerous other factors shape the context under which the mutations exert their effects. The genetic mutations listed in \citetab{genetic-disorders} all have incomplete penetrance, highlighting the contribution of genetic background (as well as environmental factors). Disparate patterns of tumour incidence arising from mutations in negative regulators of the \gls{shh} signaling pathway (\gene{SUFU} and \gene{PTCH1}) point to several possible contributing factors, including the limitation of inbred mouse models, involvement of genes in multiple biological pathways, and potential modifying genetic polymorphisms or mutations. On the C57BL/6J genetic background, \gene{SUFU}\high{+/-} mice do not develop medulloblastoma within the same time frame as \gene{PTCH1}\high{+/-} mice, even though both \gene{SUFU} and \gene{PTCH1} negatively regulate Shh signaling \citeref{svard09}. Conversely, in a Manchester cohort of 171 Gorlin syndrome patients, germline \gene{SUFU} mutation had about 20 times higher risk than germline \gene{PTCH1} mutation to lead to medulloblastoma \citeref{smith14}. In stark contrast to the C57BL/6J and 129X1/SvJ inbred mouse strains, which develops medulloblastoma at a frequency of about 30\%, only 2 (1.7\%) of 115 Manchester patients developed medulloblastoma \citeref{svard09, goodrich97, wetmore00, smith14}. The difference in the penetrances of mutations in \gene{SUFU} vs.\ \gene{PTCH1} suggests that these genes may have other functions beyond \gls{shh} pathway or they regulate the pathway in slightly different ways. However, the inversion in relative penetrances of \gene{SUFU} and \gene{PTCH1} mutations in a human population compared to the mouse models suggest that polymorphisms or mutations in possible modifier genes may influence the manifestation of mutant \gene{SUFU} and \gene{PTCH1}. Alternatively, this observation may indicate that the \gls{shh} signaling pathway regulate neural development in critically different ways in mouse and human. Above all, these possibilities prompts the need for further studies in additional human populations and mouse strains in order to tease out the intricate interplay between mutant genes and genetic background. 

The phenotypic manifestation of genetic mutations is also influenced by the state of the mutation-harbouring cells. \gene{PTEN} (endogenous inhibitor of \gls{pi3k} signaling) is frequently homozygously deleted in medulloblastoma \citeref{northcott09}, suggesting that \gene{PTEN} may be a tumour supressor. Heterozygous germline \gene{PTEN} mutation, however, leads to adult tumours (PTEN hamartoma tumour syndrome) but not medulloblastoma \citeref{tan12}. Instead, one mutant copy of \gene{PTEN} leads to Lhermitte-Duclos disease, which presents as a benign tumour of the cerebellum \citeref{zhou03}. The cells in the tumour have complete loss of wildtype \gene{PTEN} expression \citeref{zhou03}. Curiously, the granule neurons exhibit dysplasia (disorganization of tissue structure), increased size, aberrant migration, but not abnormal cellular proliferation \citeref{endersby08}, in contrast to neoplastic neurons in medulloblastoma. This example illustrates that the cell state during nervous system development influences whether the cells progresses to neoplasia (cancer). In the case of \gene{PTEN}, other mutations would need to occur with \gene{PTEN} loss-of-function in order for patients to develop medulloblastoma. Indeed, patients with germline \gene{PTEN} and \gene{Pten}\high{+/-} mice do not develop malignant brain tumours, despite predisposition to various other tumours \citeref{endersby08}. In contrast to \gene{Pten} loss alone, concomitant overexpression of SHH (human protein) causes medulloblastoma in mouse following irradiation \citeref{hambardzumyan08}. More generally, the effect of a mutant gene is modulated by the state of the cell harbouring the mutation, and this state can be shaped by genetic background, cooperating mutations, and developmental signaling. As a cell mature and differentiate down various lineages during development, it may reach a state that permits a specific mutation (germline or somatic) to transform itself into cancer with the support of signals from surrounding microenvironment.


\section{Molecular biology}

Central nervous system development involves the coordination of innumerable signaling pathways and the high controlled proliferation of maturing cells. Tumours arise when unfortunate accidents (e.g. germline mutations, genetic background, and DNA damage) coincide with conducive conditions (e.g. cell state and extracellular signals). Simply put, medulloblastoma occurs when normal neural development goes awry. Several developmental signaling pathways are important in medulloblastoma formation, including \gls{wnt}, \gls{shh}, Notch, \gls{pi3k}, and \gls{egfr} signaling. 

\subsection{\gls{wnt} signaling}

\gls{wnt} signaling is important in medulloblastoma.

\begin{figure}[H]
	\begin{center}
		%\includegraphics[width=\textwidth]{fig/pathway/wnt-pathway.pdf}
	\end{center}
	\caption[CTNNB1-dependent Wnt signaling pathway]
	{
		CTNNB1-dependent Wnt signaling pathway.
		In the absence of WNT ligand, CTNNB1 ($\beta$-catenin) is continuously marked for degradation by the CTNNB1 destruction complex, consisting of AXIN, APC, GSK, and CSNK1A1 (CK1). GSK and CSNK1A1 primes CTNNB1 by phosphorylation, which leads for ubiquinylation by the CUL1-containing E3 ligase complex and subsequent complete degradation by the proteome. Upon binding of WNT to FZD, DVL1 is activated via an unknown mechanism, and DVL1 phosphorylates LRP5/6, which then sequesters AXIN and frees CTNNB1 from the destruction complex. CTNNB1 accumulates in the cytoplasm and translocate to the nucleus to activate target genes such as \gene{MYC}, \gene{CCND1}, and \gene{AXIN2}. Two \gene{AXIN} genes exists in humans: \gene{AXIN1} and \gene{AXIN2}. GSK is composed by two subunits: GSK3A and GSK3B. CTNNB1 regulate expression in concert with the TCF/LEF family of co-transcription factors, such as TCF7, TCF7L1, TCF7L2, and LEF1. WNT signaling also activates many other pathways independently of CTNNB1; these pathways include: planar cell polarity, WNT-Ca\high{2+}, and others. Both WNT and FZD encompass a large family of proteins.
	}
	\label{fig:wnt-pathway}
\end{figure}


\subsection{\gls{shh} signaling}

\gls{shh} signaling is important in medulloblastoma.

\begin{figure}[H]
	\begin{center}
		%\includegraphics[width=\textwidth]{fig/pathway/shh-pathway.pdf}
	\end{center}
	\caption[\gls{shh} signaling pathway]
	{
		Shh signaling pathway.
		GLI transcription factors, including GLI1, GLI2, and GLI3, are downstream effectors of \gls{shh} signaling. The proteome can completely degrade GLI proteins or proteolytically process them into activators or repressors forms. depending on the post-translational modifications on the full-length GLI proteins. GLI1 and GLI2 predominantly function in the activator form and activates transcription, while GLI3 mainly functions in the repressor form and represses transcription. Shh signaling modulate the marks on full-length GLI, thereby influencing downstream transcription.
		In the absence of SHH ligand, SMO activity is repressed by PTCH. Upon SHH binding to PTCH, this repression is relieved, leading to active SMO signaling, which favours the processing of GLI into the active form; consequently, transcription of target genes including cell cycle genes (\gene{CCND1} and \gene{CCNE1}), \gene{MYC}, and negative regulators of Shh signaling (\gene{PTCH1} and \gene{HHIP}).
		The precise mechanism whereby SMO (indirectly) activates GLI is not conserved between fly and mammals and remains unknown in mammals. Four proteins bind to and regulate GLI processing: KIF7, SUFU, SPOP, and BTRC. Both KIF7 and SUFU are negative regulators of Shh signaling via unclear mechanisms. The role of SUFU remains contentious; possible roles are: nuclear export of GLI, protection of GLI from degradation, recruitment of GLI3 for processing into the repressor form. Conversely, SPOP and BTRC serve to recognize GLI and they function as subunits of E3 ligase complexes, which mark (ubiquitinates) GLI for proteolytic processing by the proteome. The CUL1 containing complex is also known as the Skp, Collin1, F-box (SCF) complex, Via BTRC-mediated recognition, this complex ubiquitinates many other proteins, including CTNNB1 from the Wnt pathway.
		Mammals have three hedgehog ligands: DHH, IHH, and SHH. PTCH encompasses PTCH1 and PTCH2. 
	}
	\label{fig:shh-pathway}
\end{figure}

\subsection{Notch signaling}

Notch signaling is important in medulloblastoma.

\begin{figure}[H]
	\begin{center}
		%\includegraphics[width=\textwidth]{fig/pathway/notch-pathway.pdf}
	\end{center}
	\caption[Notch signaling pathway]
	{
		Notch signaling pathway.
		When NOTCH1 binds DLL1 expressed on the surface of an adjacent cell, NOTCH1 is cleaved by ADAM17 (TACE). This proteolytic event frees the NOTCH intracellular domain (NICD) from the cell surface, allowing it to translocate to the nucleus in order to activate transcription of target genes such as \gene{MYC}, \gene{HES}, \gene{CDKN1A}, \gene{CCND3}.
	}
	\label{fig:notch-pathway}
\end{figure}

\subsection{\gls{pi3k} signaling}

\gls{pi3k} signaling is important in medulloblastoma.

\begin{figure}[H]
	\begin{center}
		%\includegraphics[width=\textwidth]{fig/pathway/pi3k-pathway.pdf}
	\end{center}
	\caption[PI3K signaling pathway]
	{
		PI3K signaling pathway.
		PTEN is a negative regulator of PI3K signaling. PI3K signaling is activated downstream of many cell-surface receptors.
	}
	\label{fig:pi3k-pathway}
\end{figure}


\subsection{\gls{egfr} signaling}

\gls{egfr} signaling is important in medulloblastoma.

\begin{figure}[H]
	\begin{center}
		%\includegraphics[width=\textwidth]{fig/pathway/egfr-pathway.pdf}
	\end{center}
	\caption[Egfr signaling pathway]
	{
		Egfr signaling pathway.
		ERBB2 has multiple roles: it can function as co-receptor, it can recruit ligand, and it can dimerize and activate signaling in the absence of ligand.
	}
	\label{fig:egfr-pathway}
\end{figure}


\subsection{cAMP signaling}

Both SMO and FZD are G-protein coupled receptors, and they regulate intracellular cAMP signaling.


Many of the aforementioned signaling pathways converge on the Myc family of proteins.

Although these pathways are presented as distinct linear series of signaling components, they are in fact highly interconnected by shared signaling components. How the pathways are connected are presumably dependent on the cell lineage or type.


\section{Histological classification}

Medllomyoblastoma and melanocytic medulloblastoma used to be considered histological variants of medulloblastoma in the 1993 classification; however, in the latest \gls{who} classification (2007), they are no longer considered distinct histotype of medulloblastoma \citeref{who07}.

About 70\% of medulloblastoma are of the classic histological variant. Within this histotype, patient response to treatment varies, indicating that the classic histotype encompasses a biologically heterogenous group of tumours.

kappa
In order for a classification system to be useful, the system must be straight-forward to apply and the results should be reproducible.

Molecular classification can be applied objectively, the process can be streamlined and the results are reproducible.

Medulloblastoma with extensive nodularity is an additional histological subtype not currently recognized in the WHO classification. Patients with this histotype exhibit improved survivorship.

\section{Molecular classification}

We contend that the classifying medullobalstoma into subtypes with similar molecular biology can help distinguish tumourigenic (cancer promoting) from non-tumourigenic mutations and thereby facilitate the discovery of therapeutic targets. 

Classify medulloblastoma purely by survival?

each patient is his/her own class?
classify to a sufficient granularity so long as the classes are reproducible and useful.


\section{Current and potential treatments}

Current therapy for medulloblastoma --- including surgical resection, radiation of the entire brain and spinal cord, and aggressive chemotherapy --- yields five-year survival rates of 60-70\% \citeref{gajjar06}. Survivors are often left with significant neurological, intellectual, and physical disabilities secondary to the effects of these non-specific, cytotoxic therapies on the developing nervous system \citeref{spiegler04,mabbott05}.

Rationale for targeted therapies in medulloblastoma \citeref{macdonald14}.

Targeted treatment for \gls{shh}-dependent medulloblastoma \citeref{kieran14}.
utility of smoothened inhibitors in cancer \citeref{amakye13}.

Targeting \gls{shh}-dependent medulloblastoma through inhibition of auroa and polo-like kinases \citeself{markant13}.

Survivin as a target in \gls{shh}-dependent medulloblastoma \citeref{brun14}.

Sensitivity to anticancer therapies, likely due to preservation of Bax apoptotic pathway in medulloblastoma \citeref{crowther13}.
In comparison, \gls{cnspnet} responds poorly to standard treatment \citeref{who07}.

BET bromodomain inhibition of MYC-amplified medulloblastoma \citeref{bandopadhayay14}. 


Recent evidence suggests that medulloblastoma in fact comprises a group of biologically distinct molecular entities whose clinical and genetic differences may require separate therapeutic strategies \citeref{thompson06,kool08,northcott11a,remke11,cho11}. Four principal subgroups\citeref{taylor12} of medulloblastoma have been identified: WNT, SHH, Group3, and Group4, and there is preliminary evidence for clinically significant subdivisions of the subgroups \citeref{northcott11a,cho11,remke11,taylor12}. Targeted therapies based on the genetics of the disease are not currently in use. However, inhibitors of the \gls{shh} pathway activator, smoothened, have shown some early evidence of efficacy \citeref{rudin09}. With a deeper understanding of the genomics and biology of medulloblastoma subgroups, we hope to herald a new era of medulloblastoma treatment based on selective, specific, targeted therapy.

Molecular subgroups are reproduced by collaborators.
Attempt at molecular classification is not novel. Classification of \gls{cnspnet} is not very robust \citeref{li09}.
Sometimes it helps guide diagnosis \citeself{merino15}.


\section{Risk stratification of patients}

Patients are stratified based into standard-risk and high-risk groups based on evidence of metastasis and size of the residual tumour after surgery. Additionally, infants under 3 years are not irradiated to prevent impairment of neurological function by craniospinal irradiation.


\section{Research objectives}

My study will focus on the following three obstacles that hinders the development of targeted therapy against medulloblastoma molecular subgroups:

\begin{enumerate}
	\item The lack of a clinically applicable assay for molecular subgrouping of medulloblastoma.
	\item The paucity of actionable targets for WNT, Group3, and Group4 medulloblastomas.
	\item Current clinical prognostication of medulloblastoma does not consider molecular subgroups.
\end{enumerate}

The objectives of my study are to provide viable solutions to these issues and to demonstrate the clinical significance of molecular classification.


\subsection*{Aim I: Molecular classification of medulloblastoma in clinical contexts}

Although the retrospective classification of medulloblastoma has been scientifically informative, molecular subgrouping has not been applied in the context of a prospective clinical trial. One major obstacle is the lack of fresh-frozen samples for most clinical cases. Expression profiling, on which molecular classification was based, depends on the availability of high-quality RNA. In contrast, clinical samples are routinely subjected to formalin-fixation and paraffin-embedding, which preserves tissue integrity but causes nucleic acid degradation. To facilitate the development of therapy specifically targeted against molecular subgroups, we sought to establish an molecular subgrouping assay that can be clinically applied on \gls{ffpe} samples. I have established an analytic pipeline to molecular subgrouping using expression data generated by nanoString assays, and demonstrated its high classification accuracy on \gls{ffpe} samples \citeself{northcott12}. To further make the assay clinically applicable, I have implemented several quality-control measures that identify cases which cannot be reliably assigned molecular subgroup, due to poor specimen quality or assay reaction failure.

\subsection*{Aim II: Target identification by copy-number profiling of medulloblastoma}

After having established a clinically applicable molecular classification methodology, I turned to the problem of identifying molecular targets in medulloblastoma. Unlike SHH medulloblastomas, actionable targets for WNT, Group3, and Group4 tumours have yet been identified. However, prior attempts may have been underpowered to discriminate the genomic differences among the four molecular subgroups. To this end, the \gls{magic}, consisting of scientists and physicians from 43 cities across the globe, has gathered $>1200$ medulloblastomas. Paul Northcott and I have analyzed the genomic copy-number profiles of the tumours by \gls{snp} arrays. We have identified genes and pathways that characterize each medulloblastoma subgroup \citeself{shih12}.


\subsection*{Aim III: Clinical prognostication within medulloblastoma subgroups}

Prior clinical prognostication studies in medulloblastoma have identified biomarkers without discriminating between the molecular subgroups of medulloblastoma. Given that medulloblastoma subgroups are biologically and molecular distinct disease entities, we hypothesized that incorporating molecular subgroup into prognostication can enhance the accuracy of survival prediction and improve the reliability of risk stratification. Practical and reliable identification of risk could allow for therapy intensification in high-risk children to improve survival and therapy de-escalation in low-risk children to avoid complications of therapy. By identifying clinical and molecular biomarkers within medulloblastoma subgroups, I have designed risk stratification schemes for SHH, Group3, and Group4 medulloblastoma that can achieve unpredented levels of prognostic accuracy.
