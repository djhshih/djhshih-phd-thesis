\chapter{Introduction}
\label{ch:intro}

One in 285 children are diagnosed with cancer before the age of 20, and cancer is the second leading cause of death in children \citeref{ward14, murphy13}. Owing to advances in treatment, the 5-year survival of children with cancer has steadily increased from 63\% in 1975 to 85\% in 2006 \citeref{howlader14}. The improvements in survival, however, vary considerably across cancer types. In 1975, the 5-year survival rates for childhood leukemia and \gls{cns} tumour were 50\% and 57\%, respectively. In 2006, the survival rate for the former reached 87\% while the latter lagged behind at 74\% \citeref{howlader14}. Indeed, leukemia and \gls{cns} tumour are biologically very different types of cancer: they arise from different cells within different organs, hijack different cellular signaling programs to effect uncontrolled proliferation, and reside in different locations that permit different accessibilities to treatment. It should come as no surprise then that leukemia and \gls{cns} tumour respond differently to similar modern anti-cancer treatments (consisting of chemotherapy and radiotherapy). Furthermore, leukemia and \gls{cns} tumour can each be classified into additional cancer subtypes, which also response varyingly to treatment. Within leukemias, the 5-year patient survival of acute lymphocytic leukemia is 92\% and that of acute myelogenous leukemia is 66\% \citeref{howlader14}. Within \gls{cns} tumours, the 5-year patient survival of pilocytic astrocytoma is 94\% and that of medulloblastoma is 72\% \citeref{ostrom14}. In general, the responses of cancers to therapy depend on not only the affected tissue (blood vs.\ \gls{cns}) but also the cellular origin (lymphoid vs.\ myeloid and astrocytic vs.\ embryonal). Using anatomical locations and cellular appearance, clinicians classify cancer into different types in order to predict the responses to treatment. Current post-surgical treatment modalities eradicate cancer largely by one predominant mechanism (inhibiting cellular division and inducing apoptosis), but novel anti-cancer therapies are increasing in specificity against aberrant cells and diverging in mechanisms of action. Accordingly, the classification of cancer with increasingly finer granularity will become critical for selecting the right treatment for each patient.

While the current classification of cancer using the primary site of occurrence and the morphology of the cell have been useful for predicting patient response, the advent of molecular profiling and sequencing technologies can refine this classification further and facilitate the development of therapies targeted against specific aberrations within a cancer. Currently, the \gls{who} classify \gls{cns} tumours into 86 distinct entities based primarily on histological appearance \citeref{who07}. (In 1993, \gls{who} recognized \emph{only} 36 subtypes of \gls{cns} tumours \citeref{kleihues93}.) Despite this detailed level of classification, the 5-year survival of children with \gls{cns} tumours has remained stagnant at about 75\% since 1996 \citeref{howlader14}. The discrepancy between discovery of new \gls{cns} tumour histotypes and the lack of therapeutic improvement suggest that the histological classification of \gls{cns} tumours may be insufficient to identify biologically similar tumours and facilitate the development of novel effective therapy. In particular, it was noted as early as 1971 that medulloblastoma with desmoplastic histology exhibits longer patient survival \citeref{chatty71} in response to radiotherapy; however, it remains unclear why the desmoplastic phenotype is associated with longer survival or which novel therapeutic agent may be effective against this histological variant. In comparison, a medulloblastoma tumour with a loss-of-function mutation in \gene{PTCH1} (endogenous suppressor of \gls{shh} signaling) would depend on hyperactive \gls{shh} signaling for growth, the patient would therefore be expected to --- and indeed does --- respond to inhibition of \gls{shh} signaling \citeref{rudin09}. Characterizing the genetic mutations and understanding the biology of a tumour can therefore help guide the discovery of novel therapies and the selection of suitable treatment modality or intensity.

In order to identify cancer mutations against which therapeutic intervention may be beneficial, we would need to distinguish between mutations contribute to tumourigenesis (known as \emphterm{driver} mutations) from those that do not (known as \emphterm{passenger} mutations). Cancer cells accumulate somatic mutations by having disrupted DNA damage response or DNA repair pathways. Somatic mutations generally occur stochastically (with some exceptions including antibody diversification by activation-induced cytidine deaminase). Most mutations in a cancer cell would thus be deleterious or neutral to the cell, and mutations that confer selective advantage would increase in \emphterm{cellular frequency} (proportion of cells habouring mutation) within a tumour. Not all mutations observed at high cellular frequency, however, contribute to tumourigenesis. Since multiple mutations may occur each time a cancer cell replicate its DNA and multiple cell divisions may occur before a new driver mutation arises, a population of cancer cells arising from the same parental cell will share one driver mutation and many additional mutations that do not contribute to tumourigenesis. Additionally, not all driver mutations become clonal (i.e. reach 100\% cellular frequency) due to interaction among cells. In glioblastoma multiforme, cancer cells expressing mutant EGFR secrete cytokines to promote the growth of cells expressing wild-type EGFR \citeref{inda10}; hence, the tumour heterogeneity is sustained by this paracrine mechanism and the \gene{EGFR} mutation does not become clonal. Taking these possibilities into consideration, high cellular frequency is neither a necessary nor a sufficient condition for a mutation to be a driver. Instead, a robust method for distinguish driver from passenger mutations would be to assess its \emphterm{recurrence frequency}: the frequency that a mutation is observed across samples. If a mutation drives tumour formation, we would expect to find across multiple tumours from different patients, provided that the tumours are biologically similar and arise through similar molecular mechanisms. Above all, in order to use recurrence frequency as a criteria for identifying driver mutations and recognizing genomic patterns, we would need to first classify the tumours into homogeneous groups.

This thesis will focus on the refining the classification of \emphterm{medulloblastoma}, a malignant brain tumour occurring in the cerebellum and the posterior fossa. \gls{who} classifies medulloblastoma as a grade IV (highly aggressive) embryonal tumour \citeref{who07}. Its diagnosis is made by the anatomical location of the tumour and histological morphology of the cells. Medulloblastoma was once a universally fatal disease; today, 58\% of patients are expected to live longer than 15 years \citeref{ward14}. To predict whether a patient will respond favourably to treatment, clinicians currently categorize medulloblastoma by such features as metastatic presentation and histological variants. Medulloblastoma may be divided into classic, desmoplastic, and anaplastic (large cell) histotypes. While this histological classification of medulloblastoma provides some prognostic value, it provide scant insight into potential biological mechanisms. Instead, several studies generated RNA expression profiles of primary medulloblastoma tumours using microarrays and sought to characterize medulloblastoma by patterns of RNA expression. Since each expression profile captures a snapshot of the overall molecular state of a tumour, biologically similar tumours exhibit similar expression profiles. Therefore, tumours with similar expression profiles can be clustered (grouped) together to discover biologically homogeneous molecular classes. By these clustering analyses, four molecular classes (henceforth known as \emphterm{subgroups}) of medulloblastoma were discovered: \emphterm{WNT}, \emphterm{SHH}, \emphterm{Group3}, and \emphterm{Group4} medulloblastoma \citeref{taylor12, northcott11a, kool12, remke11, cho11}. As the names suggest, WNT medulloblastoma have hyperactive \gls{wnt} signaling and SHH medullobalstoma have increased \gls{shh} signaling compared to the other subgroups, while Group3 and Group4 are less well defined \citeref{northcott11a, kool12, cho11}. We hypothesize that classifying medulloblastoma into these (relatively) homogeneous molecular subgroups will shed light on its cancer biology and consequently improve prediction of treatment response and point to novel therapeutic targets.

The remainder of this chapter will discuss contributing genetic and non-genetic factors in medulloblastoma and their implications for the classification and treatment of medullobastoma. At the end of this chapter, the main research objectives of this thesis are outlined. In \citech{mb-class}, we will propose and validate a method whereby medulloblastoma tumours may be classified in the clinical setting. The immediate utility of this molecular classification system for patient risk stratification will be shown in \citech{clin-prog}, and the potential implication of molecular classification for therapeutic discovery will be presented in \citech{target-id}. To make this thesis accessible to a broader audience, the \emphlab{Appendix} presents clarifications for topics including gene nomenclature, mouse model notation, and computational algorithms.


\section{Epidemiology and genetic predisposition}

Medulloblastoma occurs at an annual incidence of 4.1 per million children under 20, ranking as the most common type of malignant brain tumour in childhood \citeref{ostrom14}. The majority of patients with medulloblastoma are diagnosed before the age of 20, and the median age at presentation is 8 years \citeself{shih14}. As medulloblastoma is almost 10 more likely to afflict children than adults, medulloblastoma is a disease of childhood \citeref{smoll12}, suggesting perhaps genetics may play a role in this disease. Consistent with this notion, medulloblastoma can occur simultaneously in monozygotic twins \citeref{scheurlen96}, and having an affected sibling increases a child's risk of developing medulloblastoma by 4 fold \citeref{hemminki09}. While most medulloblastoma cases are sporadic with unknown genetic contribution, several genetic disorders can predispose children to developing medulloblastoma, as well as many other malignancies (\citetab{genetic-disorders}). For instance, mutations in genes of DNA damage response and repair pathways, including \gene{TP53}, \gene{ATM}, \gene{BRAC2}, \gene{NBN}, predispose a child to develop a spectrum of tumours, including medulloblastoma, at a young age \citeref{pearson82, barel98, guran99, yamazaki00, garre09, villani11, hart87, reiman11, offit03, hirsch04, reid05, dewire09, ciara10, alexiou12}.

Medulloblastoma can also arise as a consequence of germline mutations in developmental signaling pathways. Germline loss-of-function mutations in negative regulators of \gls{shh} signaling, including \gene{PTCH1}, \gene{PTCH2}, \gene{SUFU}, causes the basal cell nevus syndrome (Gorlin syndrome) and predispose patients to medulloblastoma, basal cell carcinoma, and other cancers. In the context of these mutations, hyperactive \gls{shh} signaling during neural development presumably would cause patients to develop SHH medulloblastoma. This notion is supported by mouse models with heterozygous \gene{Ptch1} mutation \citeref{goodrich97, wetmore00, wetmore01, pazzaglia02, gibson10} and the \gene{Sufu}\high{+/-} \gene{Trp53}\high{-/-} mouse model \citeref{heby-henricson12}, which both develop medulloblastoma with active Shh signaling. Further, Gorlin syndrome is highly prevalent in human patients with SHH medulloblastoma \citeref{kool14}, though it is yet unclear whether patients with Gorlin syndrome exclusively develop SHH medullobalstoma. Additionally, germline mutations in \gene{APC} activates \gls{wnt} signaling, manifests as Turcot syndrome, and predisposes patients to medulloblastoma. The tumours arising in these patients would be expected to be of the WNT subgroup; however, the crosstalk between the \gls{shh} and the \gls{wnt} signaling pathways in the context of neural development and different genetic backgrounds could modulate the phenotypic outcome of \gene{APC} mutation.

\begin{table}[h]
	\caption[Genetic disorders predisposing to medulloblastoma]
	{
		Genetic disorders predisposing to medulloblastoma
	}
	\label{tab:genetic-disorders}
	\footnotesize
	\setlength{\extrarowheight}{0.5em}
	\centering
	\begin{tabular}{l | l | l}
		\hline
		\textbf{Genetic disorder} & \textbf{Mutated genes} & \textbf{Reference} \\
		\hline
		Li-Fraumeni syndrome & \gene{TP53} & \citeplainref{pearson82, barel98, guran99, yamazaki00, garre09, villani11} \\
		Ataxia telangiectasia (Louis-Bar syndrome) & \gene{ATM} & \citeplainref{hart87, reiman11} \\
		Fanconi anemia & \gene{BRCA2} (\gene{FANCD1}) & \citeplainref{offit03, hirsch04, reid05} \\
		Nijmegen breakage syndrome & \gene{NBN} (\gene{NBS1}) & \citeplainref{bakhshi03, distel03, ciara10} \\
		Fragile X syndrome & \gene{FMR1} & \citeplainref{garre09, alexiou12} \\
		Neurofibromatosis type 1 (Von Recklinghausen's disease) & \gene{NF1} & \citeplainref{martinez-lage02, garre09} \\
		DICER1 syndrome & \gene{DICER1} & \citeplainref{slade11} \\
		Rubinstein-Taybi syndrome & \gene{CREBBP} & \citeplainref{bourdeaut14} \\
		Basal cell nevus syndrome (Gorlin syndrome) & \gene{PTCH1}, \gene{PTCH2}, \gene{SUFU} & \citeplainref{wolter97, taylor02, crawford09, garre09, brugieres10, jones11, brugieres12} \\
		Turcot syndrome & \gene{APC} & \citeplainref{hamilton95} \\
		\hline
	\end{tabular}
\end{table}

Germline mutation in \gene{SMARCB1} (\gene{SNF5}/\gene{INI}) had been associated with medulloblastoma \citeref{sevenet99, lee02}, but the brain tumours in these patients are now classified as \gls{atrt}, a tumour type that \gls{who} first recognized in its 1993 classification \citeref{kleihues93}. \gls{atrt} and medulloblastoma are similar by histology\citeref{utsuki03} and by \gls{mri} \citeref{koral08}. Currently, the \gene{SMARCB1} mutation is widely accepted as an diagnostic indicator of \gls{atrt} and distinguishes \gls{atrt} from medulloblastoma \citeref{biegel00, kraus02}. Indeed, \gene{SMARCB1} is an example in which a genetic mutation superseded histological diagnosis and redefined cancer classification.

While mutations are pivotal factors in the formation of medulloblastoma, numerous other factors shape the context under which the mutations exert their effects. The genetic mutations listed in \citetab{genetic-disorders} all have incomplete penetrance, highlighting the contribution of genetic background (as well as environmental factors). Disparate patterns of tumour incidence arising from mutations in negative regulators of the \gls{shh} signaling pathway (\gene{SUFU} and \gene{PTCH1}) point to several possible contributing factors, including the limitation of inbred mouse models, involvement of genes in multiple biological pathways, and potential modifying genetic polymorphisms or mutations. On the C57BL/6 genetic background, \gene{SUFU}\high{+/-} mice do not develop medulloblastoma within the same time frame as \gene{PTCH1}\high{+/-} mice, even though both \gene{SUFU} and \gene{PTCH1} negatively regulate Shh signaling \citeref{svard09}. Conversely, in a Manchester cohort of 171 Gorlin syndrome patients, germline \gene{SUFU} mutation had about 20 times higher risk than germline \gene{PTCH1} mutation to lead to medulloblastoma \citeref{smith14}. The observation that only 2 (1.7\%; $n = 115$) Manchester patients with \gene{PTCH1} mutation developed medulloblastoma \citeref{smith14} starkly contrasts the findings in inbred mouse strains (C57BL/6 and 129X1/Sv) with \gene{PTCH1} mutation, which spontaneously develop medulloblastoma at a frequency of about 18\% ($n = 507$; \citetab{ptch1-mmu-mb}) \citeref{goodrich97, wetmore00, svard09, pazzaglia09}. The difference in the penetrances of mutations in \gene{SUFU} vs.\ \gene{PTCH1} suggests that these genes may have other functions beyond \gls{shh} pathway or they regulate the pathway in slightly different ways. However, the inversion in relative penetrances of \gene{SUFU} and \gene{PTCH1} mutations in a human population compared to the mouse models suggest that polymorphisms or mutations in possible modifier genes may influence the manifestation of mutant \gene{SUFU} and \gene{PTCH1}. Alternatively, this observation may indicate that the \gls{shh} signaling pathway regulate neural development in critically different ways in mouse and human. Above all, these possibilities prompts the need for further studies in additional human populations and mouse strains in order to tease out the intricate interplay between mutant genes and genetic background. 

The phenotypic manifestation of genetic mutations is also influenced by the state of the mutation-harbouring cells. \gene{PTEN} (endogenous inhibitor of \gls{pi3k} signaling) is frequently homozygously deleted in medulloblastoma \citeref{northcott09}, suggesting that \gene{PTEN} may be a tumour supressor. Heterozygous germline \gene{PTEN} mutation, however, leads to adult tumours (PTEN hamartoma tumour syndrome) but not medulloblastoma \citeref{tan12}. Instead, one mutant copy of \gene{PTEN} leads to Lhermitte-Duclos disease, which presents as a benign tumour of the cerebellum \citeref{zhou03}. The cells in the tumour have complete loss of wildtype \gene{PTEN} expression \citeref{zhou03}. Curiously, the granule neurons exhibit dysplasia (disorganization of tissue structure), increased size, aberrant migration, but not abnormal cellular proliferation \citeref{endersby08}, in contrast to neoplastic neurons in medulloblastoma. This example illustrates that the cell state during nervous system development influences whether the cells progresses to neoplasia (cancer). In the case of \gene{PTEN}, other mutations would need to occur with \gene{PTEN} loss-of-function in order for patients to develop medulloblastoma. Indeed, patients with germline \gene{PTEN} and \gene{Pten}\high{+/-} mice do not develop malignant brain tumours, despite predisposition to various other tumours \citeref{endersby08}. In contrast to \gene{Pten} loss alone, concomitant overexpression of SHH (human protein) causes medulloblastoma in mouse following irradiation \citeref{hambardzumyan08}. More generally, the effect of a mutant gene is modulated by the state of the cell harbouring the mutation, and this state can be shaped by genetic background, cooperating mutations, and developmental signaling. As a cell mature and differentiate down various lineages during development, it may reach a state that permits a specific mutation (germline or somatic) to transform itself into cancer with the support of signals from surrounding microenvironment.


\section{Molecular biology}

Central nervous system development involves the coordination of innumerable signaling pathways and the highly controlled proliferation of cells in various lineages. Tumours arise when unfortunate accidents (e.g. germline mutations, genetic background, and DNA damage) coincide with conducive conditions (e.g. cell state and extracellular signals). Simply put, medulloblastoma occurs when normal neural development goes awry. Several developmental signaling pathways are important in medulloblastoma formation, including \gls{wnt}, \gls{shh}, Notch, and \gls{pi3k}. 

\subsection{\gls{wnt} signaling}

Patients with Turcot syndrome (mutation in \gene{APC}) who developed medulloblastoma provided the one of the first clues that medulloblastoma involves activation of \gls{wnt} signaling \citeref{hamilton95}. Additionally, CTNNB1 nuclear localization and \gene{CTNNB1} mutation were frequently observed medullobalstoma \citeref{thompson06, ellison05}. CTNNB1 is the transcriptional factor that serves as the downstream effector of \gls{wnt} signaling \citefig{wnt-pathway}. As part of the CTNNB1 desctruction complex containing AXIN and APC, GSK and CSNK1A1 normally phosphorylates CTNNB1 at multiple serine/threonine residues (S45, T41 S37, and S33) within a region encoded by exon 3, flagging CTNNB1 for ubiquitination and subsequent proteosomal degradation \citeref{liu02}. Mutation in APC therefore aborgates the interaction between the destruction complex and CTNNB1, preventing the degradation of CTNNB1. Similarly, CTNNB1 mutations at or near the multiple phosphorylation sites also stabilizes CTNNB1, permitting its signaling \citeref{liu02, legoix99, shih12}. Curiously, a mouse model expressing stabilized human (S37F mutant) CTNNB1 under the mouse promoter of \citeref{Prnp} did not develop medulloblastoma under a \gene{Trp53}\high{-/-} background \citeref{kratz02}. Conversely, Gibson \emph{et al.}\ generated medullobalstoma using a different mouse model expressing stabilized CTNNB1 \citeref{gibson10}. In this Cre-recombination model, mouse Ctnnb1 is converted to a stabilized form by Cre-mediated somatic deletion of exon 3 (which was flanked by loxP sequences) \citeref{harada99}. Ctnnb1 is under endogenous promoter control, but deletion of exon 3 is controlled by expressing Cre under a selected promoter; therefore, expression of stabilized Ctnnb1 can be restricted to cell types that activate the selected promoter.  \gene{Ctnnb1}\high{+/lox(ex3)} mice crossed to \gene{Atoh1-Cre} mice did not produce progeny (\gene{Atoh1-Cre};\gene{Ctnnb1}\high{+/lox(ex3)}) that develop hyperplasia or medulloblastoma, suggesting that Atoh1-expressing cells are not the developmental origin of Wnt-activated medulloblastoma \citeref{gibson10}. In contrast, \gene{Blbp-Cre};\gene{Ctnnb1}\high{+/lox(ex3)} mice showed hyperplasia in the dorsal brainstem and eventually developed medulloblastoma in the context of Trp53 deficiency, which presumably caused additional tumourigenic mutations \citeref{gibson10}. In sum, only \gene{Blbp-Cre};\gene{Ctnnb1}\high{+/lox(ex3)};\gene{Trp53}\high{flx/flx} mouse developed Wnt-activated medulloblastoma, and these tumours arise from the dorsal brainstem and not the cerebellum (in contrast to the Shh-activated medulloblastoma described later). Tumour development not only required expression of the oncogene (stablized Ctnnb1) in a suitable cell type (Blbp expressing) but may also need additional mutations (secondary to Trp53 deletion).

\begin{figure}[H]
	\begin{center}
		%\includegraphics[width=\textwidth]{fig/pathway/wnt-pathway.pdf}
	\end{center}
	\caption[CTNNB1-dependent Wnt signaling pathway]
	{
		CTNNB1-dependent Wnt signaling pathway.
		In the absence of WNT ligand, CTNNB1 ($\beta$-catenin) is continuously marked for degradation by the CTNNB1 destruction complex, consisting of AXIN, APC, GSK, and CSNK1A1 (CK1). GSK and CSNK1A1 primes CTNNB1 by phosphorylation, which leads for ubiquinylation by the CUL1-containing E3 ligase complex and subsequent complete degradation by the proteome. Upon binding of WNT to FZD, DVL1 is activated via an unknown mechanism, and DVL1 phosphorylates LRP5/6, which then sequesters AXIN and frees CTNNB1 from the destruction complex. CTNNB1 accumulates in the cytoplasm and translocate to the nucleus to activate target genes such as \gene{MYC}, \gene{CCND1}, and \gene{AXIN2}. Two \gene{AXIN} genes exists in humans: \gene{AXIN1} and \gene{AXIN2}. GSK consists of two subunits: GSK3A and GSK3B (catalytic). CTNNB1 regulate expression in concert with the TCF/LEF family of co-transcription factors, such as TCF7, TCF7L1, TCF7L2, and LEF1. WNT signaling also activates many other pathways independently of CTNNB1; these pathways include: planar cell polarity, WNT-Ca\high{2+}, and others. Both WNT and FZD encompass a large family of proteins.
	}
	\label{fig:wnt-pathway}
\end{figure}


\subsection{\gls{shh} signaling}

Medulloblastoma can also arise from activated Shh signaling \citefig{shh-pathway}. Patients with Gorlin syndrome provided the first clue that \gls{shh} signaling is activated in a subset of medulloblastoma. Patients with germline loss-of-function mutations in negative regulators of \gls{shh} signaling --- \gene{PTCH1}, \gene{PTCH2}, or \gene{SUFU} --- can develop medullobastoma during childhood, and basal cell carcinoma and other tumours later in life due to hyperactive \gls{shh} signaling \citeref{wolter97, taylor02, crawford09, garre09, brugieres10, jones11, brugieres12}. Moreover, somatic \gene{PTCH1} mutation, activating \gene{SMO} mutation, \gene{PTCH1} homozygous deletions, \gene{GLI2} amplifications are recurrently observed in medulloblastoma tumours \citeref{raffel97, reifenberger98, northcott09, shih12}, which provide further evidence that \gls{shh} signaling is activated in some medulloblastoma cases. In mouse, homozygous \gene{Ptch1} loss is embryonic lethal but heterozygous \gene{Ptch1} mutant mice exhibit hyperplasia in the external granule layer of the cerebellum, and a subset of mutant mice develop cerebellar medulloblastoma \citeref{goodrich97, wetmore00}. Frequently, the external granule layer of the developing cerebellum exhibit hyperplasia (cell mass with increased proliferation) before medulloblastoma \citeref{oliver05, yang08}. The incidence medulloblastoma in \gene{Ptch1}\high{+/-} mice ranges from 8\% to 38\% and is influenced by the genetic background \citetab{shh-mmu-mb}. (Additionally, the incidence of medulloblastoma in \gene{Ptch1}\high{+/-} models is affected by the occurrence of rhabdomyosarcoma and other tumours whose early appearance may preclude the medulloblastoma development \citeref{hahn98, goodrich97}.) This incomplete penetrance suggests that additional events beyond single-copy \gene{Ptch1} loss may be necessary for tumour progression. Indeed, \gene{Ptch1}\high{+/-} mice developed tumours with much higher frequency in the context of \gene{Trp53} deficiency \citetab{shh-mmu-mb} and when subjected to neonatal irradiation, conditional on the genetic background \citetab{ptch1-mmu-mb}. Presumably, \gene{Trp53} deficiency and irradiation generate additional mutation that cooperate with \gene{Ptch1} haploinsuffiency to promote progression of hyperplasia to medulloblastoma. Since \gene{APC} mutation did not enhance tumour incidence in \gene{Ptch1}\high{+/-} mouse, the Wnt signalling pathway may not play a major role in the development of Shh-activated medulloblastoma \citeref{wetmore01}. Similarly, homozygous loss of mouse p19\high{ARF} did not enhance medulloblastoma formation in \gene{Ptch1}\high{+/-} mice \citeref{wetmore01}, suggesting that the hyperproliferation of external granule neuron precursors caused by activated \gls{shh} signaling does not trigger p19\high{ARF}-dependent cell cycle arrest. (Mouse p19\high{ARF} is encoded by \gene{Ckdn2a} and mediates p53-dependent cell cycle arrest, similar to the human p14\high{ARF}.) Conversely, concurrent haploinsufficiency of either of two other cell cycle inhibitor, \gene{Cdkn2c} (encodes p18\high{Ink4c}) or \gene{Cdkn1b} (encodes p27\high{Kip1}), in \gene{Ptch1}{+/-} mice further increase cell proliferation and enhance medulloblastoma incidence \citeref{uziel05, ayrault09}. The tumours in these mice invariably lose the wild-type copy of \gene{Ptch1} \citeref{uziel05, ayrault09}. Indeed, complete loss of \gene{Ptch1} is highly frequently observed in the medulloblastoma tumour and may a key event for medulloblastoma progression \citeref{berman02, pazzaglia06, oliver05, uziel05, ayrault09, mille14, frappart09}. Activated \gls{shh} signaling in \gene{Ptch1}\high{-/+} granule neuron precursors induces cellular proliferation and increases accumulation of DNA breaks, which can lead to the loss of the wildtype \gene{Ptch1} allele, further induction of \gls{shh}/Gli signaling independent of the Shh ligand or the Shh coreceptor Boc, and progression of medulloblastoma \citeref{mille14}. Although homozygous loss of \gene{Ptch1} is not absolutely required for progression of hyperplasia to medulloblastoma \citeref{wetmore00, zurawel00, mille14}, conditional homozygous loss in Atoh1 expressing cells (e.g. external granule neuron precursors) or GFAP expressing cells (e.g. neural stem cells) is sufficient for developing medulloblastoma from the external granule layer with 100\% penetrance \citeref{yang08}. Similarly, activated Smo (inhibited target of Ptch1 and positive regulator of \gls{shh} signaling) induce tumour formation in a dose-dependent manner \citeref{hatton08}. When one copy of the activated mouse Smo allele habouring the W539L mutation (also known as SmoA1 and corresponds to the human SmoM2 allele) is expressed under the promoter of \gene{Neurod2} (expressed in cerebellar granule neuron precursors), the mice develop medulloblastoma with 48\% penetrance, and mice carrying two copies of the activated Smo allele (\gene{Neurod2-Smo}\high{W539L/W539L}) develop medulloblastoma with near complete penetrance \citeref{hallahan04, hatton08}. Morever, GLI (downstream effector of \gls{shh} signaling) promotes cell survival by directly upregulating the prosurvival factor BCL2 \citeref{regl04, bar07} and disabling DNA damage checkpoint \citeref{leonard08}. These results support the notion that activated \gls{shh} signaling is suffient for medulloblastoma formation and may be necessary for tumour maintainence, highlighting this pathway as a candidate for therapuetic intervention.

% medulloblastoma can occur from Atoh1-expressing \citeref{yang08}, GFAP-expressing \citeref{yang08}, or nestin-expressing \citeref{li13} cells.
% medulloblastoma arise from cerebellar granule neuron precursors \citeref{kim03}.

% Sonic hedgehog signaling pathway supports cancer cell growth during cancer radiotherapy \citeref{ma13}.

\begin{figure}[H]
	\begin{center}
		%\includegraphics[width=\textwidth]{fig/pathway/shh-pathway.pdf}
	\end{center}
	\caption[\gls{shh} signaling pathway]
	{
		Shh signaling pathway.
		GLI transcription factors, including GLI1, GLI2, and GLI3, are downstream effectors of \gls{shh} signaling. The proteome can completely degrade GLI proteins or proteolytically process them into activators or repressors forms. depending on the post-translational modifications on the full-length GLI proteins. GLI1 and GLI2 predominantly function in the activator form and activates transcription, while GLI3 mainly functions in the repressor form and represses transcription. \gls{shh} signaling modulate the marks on full-length GLI, thereby influencing downstream transcription.
		In the absence of SHH ligand, SMO activity is repressed by PTCH. Upon SHH binding to PTCH, this repression is relieved, leading to active SMO signaling, which favours the processing of GLI into the active form; consequently, transcription of target genes including cell cycle genes (\gene{CCND1} and \gene{CCNE1}), \gene{MYC}, and negative regulators of Shh signaling (\gene{PTCH1} and \gene{HHIP}).
		The precise mechanism whereby SMO (indirectly) activates GLI is not conserved between fly and mammals and remains unknown in mammals. Four proteins bind to and regulate GLI processing: KIF7, SUFU, SPOP, and BTRC. Both KIF7 and SUFU are negative regulators of Shh signaling via unclear mechanisms. The role of SUFU remains contentious; possible roles include: nuclear export of GLI, protection of GLI from degradation, recruitment of GLI3 for processing into the repressor form. Conversely, SPOP and BTRC ($beta$TRCP) serve to recognize GLI and they function as subunits of E3 ligase complexes, which mark (ubiquitinates) GLI for proteolytic processing by the proteome. The CUL1 containing complex is also known as the Skp, Collin1, F-box (SCF) complex, Via BTRC-mediated recognition, this complex ubiquitinates many other proteins, including CTNNB1 from the Wnt pathway.
		Mammals have three hedgehog ligands: DHH, IHH, and SHH. PTCH encompasses PTCH1 and PTCH2. PKA is a tetramer composed of two catalytic subunits (protein family includes PRKACA, PRKACB, and PRKACG) and two regulatory subunits (protein family includes PRKAR1A, PRKAR1B, PRKAR2A, and PRKAR2B).
	}
	\label{fig:shh-pathway}
\end{figure}

\begin{table}[h]
	\caption[Incidence of medulloblastoma in Shh-activated mouse model]
	{
		Incidence of medulloblastoma in Shh-activated mouse model
	}
	\label{tab:shh-mmu-mb}
	\footnotesize
	\setlength{\extrarowheight}{0.5em}
	\centering
	\begin{tabular}{l | l | l | l}
		\hline
		\textbf{Genotype} & \textbf{Genetic background} & \textbf{Incidence} (95\% CI) & 
		\textbf{Reference} \\
		\hline
		% \citeref{wetmore00} 10/80 (est.), 48 week follow-up
		% \citeref{goodrich97} 6/27, 25 weeks follow-up?
		% \citeref{oliver05} 4/25, 25 week follow-up
		\gene{Ptch1}\high{+/-} & 129X1/Sv & \textbf{15} (10--22) & \citeplainref{goodrich97, wetmore00, oliver05} \\
		\hline
		% \citeref{pazzaglia02}, 2/26, 38 weeks followup
		% \citeref{pazzagli09}, 4/51 (probably contains pazzaglia02)
		\gene{Ptch1}\high{+/-} & CD-1 & \textbf{8} (2--19) & \citeplainref{pazzaglia09} \\
		\hline
		% \citeref{svard09} 18/49, 1 year follow-up
		% \citeref{pazzaglia09} 9/22, 1 year follow-up
		% \citeref{mille14} 24/41 (Ptch1+/-;Boc+/+), 1 year follow-up; claims all dead mice had MB...
		% \citeself{wu12} 9/22, 1 year follow-up
		\gene{Ptch1}\high{+/-} & C57BL/6 & \textbf{45} (36--54) & \citeplainref{svard09, pazzaglia09, mille14, wu12} \\
		% \citeref{svard09} 27/56, 1 year follow-up
		\gene{Ptch1}\high{+/-};\gene{Sufu}\high{+/-} & C57BL/6 & \textbf{48} (35--62) & \citeplainref{svard09} \\
		% \citeref{mille14} 19/35, 1 year follow-up
		\gene{Ptch1}\high{+/-};\gene{Boc}\high{+/-} & C57BL/6 & \textbf{54} (37--71) & \citeplainref{mille14} \\
		% \citeref{mille14} 4/21, 1 year follow-up
		\gene{Ptch1}\high{+/-};\gene{Boc}\high{-/-} & C57BL/6 & \textbf{19} (5--42) & \citeplainref{mille14} \\
		\hline
		% \citeref{goodrich97} 9/47, 25 weeks follow-up
		\gene{Ptch1}\high{+/-} & C57BL/6 $\times$ DBA/2 (B6D2F1) & \textbf{19} (9--33) & \citeplainref{goodrich97} \\
		\hline
		% \citeref{wetmore00} 48/329 (est.), 48 weeks follow-up
		% \citeref{wetmore01} 62/440, 30 weeks follow-up (probably mostly same mouse as wetmore00)
		% \citeref{lee06} 4/62 (Ptch1+/-;Ptch2+/+), 15 months follow-up
		% \citeref{uziel05} 2/27, 9 months follow-up
		% \citeref{ayrault09} 30/70? (Ptch1+/-;Kip1+/+), 1 year follow-up; more C57BL/6 background?
		\gene{Ptch1}\high{+/-} & C57BL/6 $\times$ 129X1/Sv & \textbf{13} (10--16) & \citeplainref{wetmore01, lee06, uziel05} \\
		% \citeref{lee06} 15/97, 15 months follow-up
		\gene{Ptch1}\high{+/-};\gene{Ptch2}\high{+/-} & C57BL/6 $\times$ 129X1/Sv & \textbf{15} (9--24) & \citeplainref{lee06} \\
		% \citeref{lee06} 11/63, 15 months follow-up
		\gene{Ptch1}\high{+/-};\gene{Ptch2}\high{-/-} & C57BL/6 $\times$ 129X1/Sv & \textbf{17} (9--29) & \citeplainref{lee06} \\
		% \citeref{wetmore01} ?/68 (< 28 % of Ptch1+/-;Trp53+/-), 30 week follow-up
		%                     assume 10/68 ~= 14 % had tumour (since not different from Ptch1+/-)
		\gene{Ptch1}\high{+/-};\gene{Trp53}\high{+/-} & C57BL/6 $\times$ 129X1/Sv & \textbf{14} (7--25) & \citeplainref{wetmore01} \\
		% \citeref{wetmore01} 38/40, (Ptch1+/- Trp53-/-), 30 weeks follow-up
		% \citeref{lee07} 15/16 (Ptch1+/- Trp35-/-) (assume Ptch1+/- and Trp53+/- came from wetmore)
		\gene{Ptch1}\high{+/-};\gene{Trp53}\high{-/-} & C57BL/6 $\times$ 129X1/Sv & \textbf{95} (85--99) & \citeplainref{wetmore01, lee07} \\
		% \citeref{uziel05} 10/27, 9 months follow-up
		\gene{Ptch1}\high{+/-};\gene{Cdkn2c}\high{+/-} & C57BL/6 $\times$ 129X1/Sv & \textbf{37} (19--58) & \citeplainref{uziel05} \\
		% \citeref{uziel05} 31/68, 9 months follow-up
		\gene{Ptch1}\high{+/-};\gene{Cdkn2c}\high{-/-} & C57BL/6 $\times$ 129X1/Sv & \textbf{46} (33--58) & \citeplainref{uziel05} \\
		\hline
		% \citeref{yang08}
		% mice from \citeref{ellis03, machold05, zhuo01}
		\gene{Atoh1-Cre};\gene{Ptch1}\high{flx/flx} & C57BL/6 $\times$ FVB/N & \textbf{100} & \citeplainref{yang08} \\
		\gene{GFAP-Cre};\gene{Ptch1}\high{flx/flx} & C57BL/6 $\times$ FVB/N & \textbf{100} & \citeplainref{yang08} \\
		\hline
		% \citeref{svard09} 0/25, 1 year follow-up
		\gene{Sufu}\high{+/-} & C57BL/6 & \textbf{0} (0--14) & \citeplainref{svard09} \\
		% \citeref{lee07} 0/55, 22 weeks follow-up
		\gene{Sufu}\high{+/-};\gene{Trp53}\high{+/-} & C57BL/6 $\times$ 129P2/OlaHsd & \textbf{0} (0--6) & \citeplainref{lee07} \\
		% \citeref{lee07} 32/55, 22 weeks follow-up
		\gene{Sufu}\high{+/-};\gene{Trp53}\high{-/-} & C57BL/6 $\times$ 129P2/OlaHsd & \textbf{58} (44-71) & \citeplainref{lee07} \\
		\hline
		% \citeref{hallahan04}, 49/102, 1 year follow-up
		\gene{Neurod2-Smo}\high{+/W539L} & C57BL/6 & \textbf{48} (38--58) & \citeplainref{hallahan04} \\
		% \citeref{hatton08}, 62/66, 1 year follow-up
		\gene{Neurod2-Smo}\high{W539L/W539L} & C57BL/6 & \textbf{94} (85--98) & \citeplainref{hatton08} \\
		\hline
	\end{tabular}
\end{table}

% \gene{Smo}\high{W539L/W539L} is also known as SmoA1 (activated mouse Smo) and corresponds to human SmoM2 (activated human Smo).

% \gene{Ptch1}\high{+/-} mice develop rhabdomyosarcoma early in life, as well as basal cell carcinoma and other tumours later in life \citeref{hahn98}.

% Many medulloblastoma occur in asymptomatic mice \citeref{goodrich97}.


\begin{table}[h]
	\caption[Incidence of medulloblastoma in irradiated \gene{Ptch1}\high{+/-} mouse model]
	{
		Incidence of medulloblastoma in irradiated \gene{Ptch1}\high{+/-} mouse model
	}
	\label{tab:ptch1-mmu-mb}
	\footnotesize
	\setlength{\extrarowheight}{0.5em}
	\centering
	\begin{tabular}{l | l | l | l}
		\hline
		\textbf{Genetic background} & \textbf{Treatment} & \textbf{Incidence} (95\% CI) & \textbf{Reference} \\
		\hline
		% \citeref{pazzaglia02}, 2/26, 38 week followup
		% \citeref{pazzaglia09}, 4/51 (probably contains pazzaglia02)
		CD-1 & non-irradiated & \textbf{8} (2--19) & \citeplainref{pazzaglia09} \\ 
		% \citeref{pazzaglia02}, 26/51 (3 Gy at P4), 38 week followup
		% \citeref{pazzaglia06}, 17/21 (3 Gy at P1), 1/33 (3 Gy at P10)
		%                        80 week followup, last tumour observed before 38 week
		CD-1 & 3 Gy at P1 & \textbf{81} (58--94) & \citeplainref{pazzaglia06} \\
		CD-1 & 3 Gy at P4 & \textbf{51} (37--65) & \citeplainref{pazzaglia02} \\
		CD-1 & 3 Gy at P10 & \textbf{3} (0--16) & \citeplainref{pazzaglia06} \\
		\hline
		% \citeref{svard09} 18/49, 1 year follow-up
		% \citeref{pazzaglia09} 9/22, 1 year follow-up
		% \citeref{mille14} 24/41 (Ptch1+/-;Boc+/+), 1 year follow-up; claims all dead mice had MB...
		% \citeself{wu12} 9/22, 1 year follow-up
		C57BL/6 & non-irradiated & \textbf{45} (36--54) & \citeplainref{svard09, pazzaglia09, mille14, wu12} \\
		% \citeref{pazzaglia09} 8/15, 1 yr follow-up 
		C57BL/6 & 3 Gy at P1 & \textbf{53} (27--79) & \citeplainref{pazzaglia09} \\
		\hline
	\end{tabular}
\end{table}


\subsection{Notch signaling}

Notch signaling \citefig{notch-pathway} governs embryonic development; in the cental nervous system, Notch signaling prevents precocious neuronal differentiation and maintains the pluripotency and self-renewal of neural stem and progenitor cells \citeref{delapompa97, breunig07}. While \gene{Notch1}\high{+/-} mice are viable, homozygous \gene{Notch1} mutants die before E11.5 and possibly partly due to the depletion of neural stem cell pool \citeref{swiatek94, delapompa97, breunig07}. The embryonic lethality of \gene{Notch1}\high{-/-} also suggests that \gene{Notch1} is not completely functionally redundant with other mammalian Notch homologs (\gene{Notch2}, \gene{Notch3}, and \gene{Notch4}). Conversely, conditional activation of Notch signaling by transgenic expression of a constitutively active Notch, \gls{n1icd}, in \gene{GFAP-CreER};\gene{ACTB-N1ICD} blocked cell cycle exit of neural progenitors and inhibited neuronal differentiation \citeref{breunig07}. (In this model, \gene{N1ICD} is conditionally expressed by Cre excision of an upstream stop cassette.) Conceivably, hyperactive Notch signaling may promote tumour formation by blocking cell cycle exit and differentiation. Consistent with this proposition, Shh-activated medulloblastoma in \gene{Neurod2-Smo}\high{+/W539L} mice show increased expression of Notch pathway targets (\gene{HES1} and \gene{HES5}) as well as \gene{Notch1}, suggesting that the Notch pathway is active in medulloblastoma induced by \gls{shh} signaling \citeref{hallahan04}. When \gene{Neurod2-Smo}\high{+/W539L} were crossed to \gene{Atoh1-Cre};\gene{Notch1}\high{flx/flx} or \gene{Atoh1-Cre};\gene{Notch2}\high{flx/flx} mice, the progeny (\gene{Neurod2-Smo}\high{+/W539L};\gene{Atoh1-Cre};\gene{Notch1}\high{flx/flx} and \gene{Neurod2-Smo}\high{+/W539L};\gene{Atoh1-Cre};\gene{Notch2}\high{flx/flx}) did not develop medulloblastoma at a higher incidence \citeref{hatton10}. While the conditional knockout of one Notch homolog may be rescued by another homolog, treatment with multiple gamma-secretase inhibitors (which prevent Notch cleavage and activation) did not affect tumour incidence, tumour size, apoptosis, or cell proliferation in \gene{Neurod2-Smo}\high{+/W539L} mice, nor did it affect xenograft engraftment in immunocompromised mice \citeref{hatton10}. Similarly, constitutive knockout of an important downstream target \gene{Hes5} in \gene{Neurod2-Smo}\high{+/W539L};\gene{Hes5}\high{-/-} mutants did not reduce medulloblastoma incidence either \citeref{hatton10}. Therefore, Notch signaling is not necessary for formation, progression, or maintenance of Shh-activated medulloblastoma \citeref{hatton10, julian10}. However, expression of \gls{n1icd} in \gene{GFAP-Cre};\gene{ACTB-N1ICD};\gene{Trp53}\high{-/-} mice resulted in medulloblastoma with 69 (39--91) percent incidence, suggesting that Notch signaling is sufficient to induce medulloblastoma formation with concurrent circumvention of Trp53-dependent apoptosis \citeref{natarajan13, yang04}. Further, the Notch-activated medulloblastoma tumours exhibit transcriptomes resembling Shh-activated medulloblastoma \citeself{natarajan13}, suggest that these tumours may share activation of similar pathways or arise from common cells of origin.

% Neuralized1 causes apoptosis and downregulates Notch target genes in medulloblastoma \citeref{teider10}.

% \citeself{natarajan13}
% \gene{GFAP-Cre};\gene{ACTB-N1ICD};\gene{Trp53}\high{-/-}mice
% 129Sv \times FVB \times C57BL/6
% 9/13, 1 year follow-up
% \gene{GFAP-Cre};\gene{ACTB-N1ICD};\gene{Trp53}\high{-/-}mice
% 129Sv \times FVB \times C57BL/6
% 2/10, 1 year follow-up

% \gene{GFAP-CreER};\gene{ACTB-N1ICD}: promoter of human GFAP, promoter of chicken ACTB, human NOTCH1 intracellular domain

\begin{figure}[H]
	\begin{center}
		%\includegraphics[width=\textwidth]{fig/pathway/notch-pathway.pdf}
	\end{center}
	\caption[Notch signaling pathway]
	{
		Notch signaling pathway.
		When NOTCH binds DLL expressed on the surface of an adjacent cell, NOTCH is extracellularly cleaved by ADAM10 or ADAM17 (TACE), and it is subsequently cleaved intracellularly by the gamma-secretase/presenilin complex. This proteolytic event releases the NOTCH intracellular domain (NICD) from the cell surface, allowing it to translocate to the nucleus in order to activate transcription of target genes such as \gene{MYC}, \gene{HES}, \gene{CDKN1A}, \gene{CCND3}. Mammals have four Notch family members, NOTCH1, NOTCH2, NOTCH3, NOTCH4, and these receptors can bind the ligands, DLL1, DLL3, DLL4, JAG1, JAG2. 
	}
	\label{fig:notch-pathway}
\end{figure}

\subsection{\gls{pi3k} signaling}

Homozygous deletions of \gene{PTEN} in medulloblastoma first highlighted that \gls{pi3k} signaling \citeref{pi3k-pathway} may play a role in medulloblastoma \citeref{inda04, frappart09, northcott09}. The \gls{pi3k} family phosphorylates the plasma membrane lipid phosphatidylinositol-4,5-bisphosphate [PI(4,5)P\low{2}] to produce phosphatidylinositol-3,4,5-trisphosphate [PI(3,4,5)P\low{3} or PIP\low{3}] \citeref{cantley02}. In turn, PIP\low{3} serves an as intermediate signaling molecule for multiple pathways. AKT1 binds to PIP\low{3} via its pleckstrin-homology domain and become anchored to the plasma membrane by PIP\low{3}; consequently, PIP\low{3}-anchored PDPK1 phosphorylates and activates AKT1, which regulates numerous downstream pathways such as inhibition of apoptosis (via BAX \citeref{gardai04} and BAD \citeref{datta97}) and cell cycle arrest (via CDKN1A \citeref{rossig01, li02} and CDKN1B \citeref{liang02}). (PDKP1 is the official symbol for 3-phosphoinositide dependent protein kinase 1, and it is commonly known as PDK1, which also ambiguously refers to pyruvate dehydrogenase kinase, isozyme 1.) Conversely, PTEN (a phosphotase) inhibits the \gls{pi3k} pathway by dephosphorylating PIP\low{3} to PI(4,5)P\low{2}. Therefore, loss of \gene{PTEN} may potentiate \gls{pi3k} signaling and AKT1 activation, inducing cell survival and proliferation. Curiously, germline PTEN loss-of-function mutations underlie a collection of disorders known as PTEN hamartoma tumor syndromes and increase the risk of a diverse array of tumours that usually occur in adulthood \citeref{tan12}. In fact, patients can develop a benign overgrowth of neurons in the cerebellum but not medulloblastoma \citeref{endersby08}. Mouse models of conditional homozygous \gene{PTEN} knock-out reveal that the neuronal cells exhibit migration defect and increased cell size but not proliferation, suggesting that complete PTEN loss is insufficient for childhood malignancy in the cerebellum \citeref{endersby08}. Nonethless, \gene{PTEN} loss in Shh-activated mouse models of medulloblastoma accelerates tumour progression and contributes to resistance against inhibition of \gls{shh} signaling \citeref{metcalfe13} as well as resistance against irradiation \citeref{hambardzumyan08}.

\begin{figure}[H]
	\begin{center}
		%\includegraphics[width=\textwidth]{fig/pathway/pi3k-pathway.pdf}
	\end{center}
	\caption[PI3K signaling pathway]
	{
		PI3K signaling pathway.
		PTEN is a negative regulator of PI3K signaling. PI3K signaling is activated downstream of many cell-surface receptors.
	}
	\label{fig:pi3k-pathway}
\end{figure}


%\subsection{\gls{egfr} signaling}
%
%\gls{egfr} signaling is important in medulloblastoma.
%
%Lack of ERBB2 protein expression in tumour is associated with favourable survival under standard treatment \citeref{gajjar04}, suggesting ERBB2 inhibition may be effective alone or in combination with conventional chemotherapy.
%
%However, "Lapatinib was well-tolerated in children with recurrent or CNS malignancies, but did not inhibit target in tumor and had little single agent activity." \citeref{fouladi13}.
%
%
%\begin{figure}[H]
%	\begin{center}
%		%\includegraphics[width=\textwidth]{fig/pathway/egfr-pathway.pdf}
%	\end{center}
%	\caption[Egfr signaling pathway]
%	{
%		Egfr signaling pathway.
%		ERBB2 has multiple roles: it can function as co-receptor, it can recruit ligand, and it can dimerize and activate signaling in the absence of ligand.
%	}
%	\label{fig:egfr-pathway}
%\end{figure}


% Developmental and oncogenic effects of insulin-like growth factor-I in Ptc1+/- mouse cerebellum \citeref{tanori10}.

\subsection{Crosstalk of signaling pathways}

Although the presented pathways are described as distinct linear series of signaling moleculars, they are in fact highly interconnected by several shared components (e.g. GSK3 and BTRC). Notably, many of the aforementioned signaling pathways converge on Myc family proteins, and this convergence underscores the central role played by the Myc family in medulloblastoma. Similarly, the upstream regulators of \gls{wnt}] and \gls{shh} signaling, SMO and FZD, are both G-protein coupled receptors. Although their G-protein activity were underappreciated in the past, both receptors can regulate intracellular cAMP signaling, and these pathways may cooperate in promoting tumourigenesis. How various pathways are connected would likely depend on cell type or cell lineage, in addition to dynamic extracellular signals during development; hence, the cellular origins of medulloblastoma may dictate the molecular mechanisms of tumour initiation, progression, and maintenance.


\section{Histological classification}

The \gls{who} classification of \gls{cns} tumours standardizes the nomenclature for different cancers arising in the \gls{cns} and now rests on the premise that each tumour type arises from the transformation of a specific cell type \citeref{kleihues93, who07}. The cellular origin, in turn, may predict prognosis and guide treatment decisions. The classification evolves with new discoveries and emerging insights, though it relies primarily on morphological features and protein marker expressions analyzed by immunohistochemistry.

\gls{who} classifies now recognizes medulloblastoma as a distinct entity \citeref{who07}. Medulloblastoma and \gls{cnspnet} used to be grouped collectively as \gls{pnet}, but \gls{who} now recognize them as distint diseases \citeref{who07}. \gls{who} classification currently includes four histological variants of medulloblastoma: desmoplastic/nodular medulloblastoma, \gls{mben}, anaplastic medulloblastoma, and large cell medulloblastoma. The last two variants are often grouped together as \gls{lca}, though some evidence suggests that they may have different prognoses \citeref{vonhoff10}. Medullomyoblastoma and melanocytic medulloblastoma used to be considered histological variants of medulloblastoma in the 1993 classification \citeref{kleihues93}; however, in the latest \gls{who} classification (2007), they are no longer considered distinct histotype of medulloblastoma \citeref{who07}. About 70\% of medulloblastoma are of the classic histological variant. Within this histotype, patient response to treatment varies greatly, indicating that the classic histotype encompasses a biologically heterogenous group of tumours \citeself{shih14}. The prognostic significance of the histotypes are discussed further in \citech{clin-prog}.

% kappa
% In order for a classification system to be useful, the system must be straight-forward to apply and the results should be reproducible.


\section{Molecular classification}

More recently, principal molecular classes of medulloblastoma have been identified based on unsupervised clustering of RNA expression profiles of tumours \citeref{taylor12}. These classes consist of \emphterm{WNT}, \emphterm{SHH}, \emphterm{Group3}, and \emphterm{Group4} medulloblastoma, and they are henceforth refered to as the \emphterm{molecular subgroups} of medullobalstoma. The subgroups have been independently identified in multiple studies \citeref{thompson06, kool08, northcott11a, remke11, cho11}, and this classification system can be applied objectively using a computer algorithm (see \citech{mb-class}) so that the process is streamlined and the results are reproducible. Although each of the molecular subgroups may be further subclassified into additional molecular subtypes, the division into four classes provide groups of tumours that are sufficiently homogeneous for predicting treatment outcome and revealing biological insights \citeref{taylor12, shih12, shih14}.

Molecular classification by expression profiles have been described in numerous other cancer types, though the same classes are often not reproducibly discovered nor the molecular classes provide little scientific insight or clinical utility. For example, the classification of \gls{cnspnet} is not very robust \citeref{li09}, partly because the \gls{cnspnet} is very rare and highly heterogeneous. Sometimes, a set of tumours are transcriptomically homogeneous. Although most clustering algorithms yield clusters from the expression profiles of these tumours, these clusters may not represent biologically meaningful classes and will not be reproducible across cohorts \citeself{agnihotri14}. Above all, molecular classes discovered by unsupervised clustering should be reproduced across cohorts.

In other cancer types, many other molecular classification approaches have been used. For example, the molecular classes of breast cancer were initially discovered by expressiong profiling, but it is now predominantly classified by three protein markers --- estrogen receptor, progesterone receptor, and ERBB2 (HER2) --- as well as a marker for cellular proliferation (Ki67) into four main molecular subtypes: luminal A, luminal B, HER2 type, and basal-like \citeref{perou00, carey06, hu06, fan06, obrien10}. Prostate cancer can molecularly classified into two classes: positive or negative for a gene fusion involving two ETS transcription factors, \gene{ERG} or \gene{ETV1} \citeref{tomlins05}, which is found in approximately half of all prostate cancers. In contrast, the classification of leukemia is far more complex and involves both histological features and cytogenetic abnormalities (e.g. chromosomal rearrangements) \citeref{falini10}. Regardless of the metholody used, the ultimate objective of molecular classification is the same: separate cancers reproducibly into biologically similar classes so that they may share similar responses to treatments. Given this objective, one may consider classifying cancer types based on patient survival; however, this approach will likely not produce biologically similar classes, because patients can die for various reasons, and patient survival can be heavily influenced by the treatments rendered.

Importantly, the molecular subgroups of medulloblastoma are biologically distinct molecular entities whose clinical and genetic differences may require separate therapeutic strategies \citeref{thompson06, kool08, northcott11a, remke11, cho11}. WNT medulloblastoma response favourably to standard therapy \citeref{northcott11a, shih12} Targeted therapies based on the genetics of the disease are not currently in use. However, inhibitors of SMO have shown some early evidence of efficacy \citeref{rudin09}. With a deeper understanding of the genomics and biology of medulloblastoma subgroups, we hope to herald a new era of medulloblastoma treatment based on selective, specific therapy.

% Ependymoma can also be molecularly classified into molecular classes \citeself{mack14}.

% It does not always make sense to establish new molecular classes, though molecular profiling helps guide diagnosis \citeself{merino15}.

% Each patient is his/her own class?
% Classify to a sufficient granularity so long as the classes are reproducible and useful.


%\section{Current and potential treatments}

% Pubmed query used on 2015-04-22
%     search term: "medulloblastoma"
%     article type: "Clinical Trial" 

%Current therapy for medulloblastoma --- including surgical resection, radiation of the entire brain and spinal cord, and aggressive chemotherapy --- yields five-year survival rates of 60-70\% \citeref{gajjar06}. Survivors are often left with significant neurological, intellectual, and physical disabilities secondary to the effects of these non-specific, cytotoxic therapies on the developing nervous system \citeref{spiegler04,mabbott05}.
%
%Combination of lomustine and vincristine (with or without prednisone) were ineffective compared to radiotherapy alone \citeref{tait90, evans90}.
%Combination of vincristine and methotrexate and hydrocortisone was ineffective compared to radiotherapy alone \citeref{vaneys81}
%
%Experimented with timing of post-operative radiotherapy and timing of chemotherapy (before, during, or after radiotherapy).
%Experimented with different doses of radiotherapy
%Experimented with different combinations of chemotherapeutic agents
%
%chemotherapy
%usually delivered over the course from months to a year
%
%combination cisplatin, lomustine, and vincristine \citeref{jeng93, gajjar94, walter99}
%combination of cisplatin, cyclophosphamide, and vincristine
%cisplatin, lomustine, cyclophosphamide, etoposide, and vincristine
%
%methotrexate used in high-risk patients
%Given that methotrexate crosses the blood-brain barrier effectively, the use of this chemotherapeutic agent is usually restricted to high-risk cases.
%
%procarbazine \citeref{grill05}.
%thiotepa \citeref{razzouk95, chi04, grill05}
%ifosfamide, cisplatin, and etoposide \citeref{saito11}
%
%Most combination chemotherapy include cisplatin or carboplatin as perhaps the main chemotherapeutic agent, given its high efficacy when given alone \citeref{gaynon90, mastrangelo95}. All these are platinum deriviatives. The second-generation drug carboplatin can substitute cisplatin \citeref{fouladi09}, but the third-generation drug oxaliplatin shows increased neurotoxicity \citeref{avan15} and provided no benefit for recurrent medulloblastoma \citeref{fouladi06, hartmann11, geoerger11}.
%
%failed phase II trials for chemotherapeutic agents
%Docetaxel did not show efficacy for recurrent medulloblastoma \citeref{zwerdling06}.
%topotecan \citeref{kadota99, blaney96}.
%Idarubicin \citeref{arndt98}.
%paclitaxel \citeref{hurwitz01}.
%
%some efficacy
%irinotecan \citeref{bomgaars07, kim13, turner02}
%
%Busulfan and thiotepa too highly toxic in previously irradiated patients \citeref{valteau-couanet05}, thought it was promising in patients who were not previously irradiated \citeref{kalifa92}.
%
%Carboplatin was not effective on recurrent medulloblastoma \citeref{friedman92} but was effective upfront \citeref{mastrangelo95, fouladi09, gaynon90}.
%
%
%Cisplatin is the backbone of most combination therapy for medulloblastoma. It only crosses the blood-brain barrier under hypoxic conditions \citeref{minami96}. Another study found that cisplatin crosses blood brain barrier with repeat doses \citeref{namikawa00}.
%
%
%Temozolomide treatment for recurrent medulloblastoma showed promise \citeref{broniscer07, nicholson07, ruggiero10, cefalo14, grill13}.
%
%eight drugs in one day (vincristine, procarbazine, hydroxyurea, cisplatin, cytarabine, prednisone, cyclophosphamide, and carmustine) by itself possibly followed by radiotherapy \citeref{geyer14}. Of the treated patients, 6\% died due to chemotherapy.
%
%eight drugs in one day (vincristine, hydroxyurea, procarbazine, lomustine, cisplatin, cytarabine, methylprednisone, and either cyclophosphamide or dacarbazine), followed by methotrexate \citeref{gentet95}.
%
%eight drugs in one day last used by itself in \citeref{zeltzer99}
%"VCP plus XRT is a superior adjuvant combination compared with 8-in-1 chemotherapy plus XRT"
%
%eight drugs in one day followed by etoposide and carboplatin \citeref{oyharcabal-bourden05}.
%
%leucovorin administered to prevent harmful effect of methotrexate.
%
%no longer commonly used due to lack of efficacy or intolerable side-effects:
%carmustine, methylchlorethamine, prednisone \citeref{fewer72, cangir84, dreyer03, krischer91}. Ifosfamide was not effective as a single-agent therapy in recurrent medulloblastoma (and other pediatric brain tumours) \citeref{heidman95, chastagner93}, but remained in use with combination chemotherapy \citeref{saito11, geyer05}.
%
%phase II trial of farnesyl transferase inhibitor (tipifarnib) was not promising \citeref{fouladi07}.
%
%phase II trial of ERBB signaling (lapatinib) did not show efficacy \citeref{fouladi13}.
%
%
%autologous stem cell transplant in high-risk or recurrent cases following myeloablative chemotherapy \citeref{perez-martinez05, gajjar06, kadota08, rosenfeld10, dunkel10, guruangan98, finlay96}
%
%Sensitivity to anticancer therapies, likely due to preservation of Bax apoptotic pathway in medulloblastoma \citeref{crowther13}.
%In comparison, \gls{cnspnet} responds poorly to standard treatment \citeref{who07}.
%
%Hyperfractionated versus conventional radiotherapy followed by chemotherapy in standard-risk medulloblastoma: results from the randomized multicenter HIT-SIOP PNET 4 trial. No difference \citeref{lannergin12}.
%
%In an ongoing clinical trial, prolonged dose-intensive chemotherapy (with cisplatin, cyclophosphamide, vincristine, etoposide) for patients with medulloblastoma yielded no improvement in survival (and also resulted several treatment-associated deaths) \citeref{strother14}. The collection of active chemotherapy agents for \gls{cns} tumours has not changed much in the past two decades, aside from the relatively new inclusion of methothrexate \citeref{strother14}. Although methotrexate shows promising in high-risk medulloblastoma patients, its long-term neurotoxicity poses serious concerns \citeref{rutkowski05, mitby03, chi04}. These finding suggests that treatment of medulloblastoma with existing chemotherapeutic agents may be reaching a ceiling.
%
%Dendritic cell-based tumour vaccination was not promising \citeref{ardon10}.
%Adoptive immunotherapy using lymphokine-activated killer cells and interleukin-2 was not promising \citeref{sankhla96}.
%
%Rationale for targeted therapies in medulloblastoma \citeref{macdonald14}.


\section{Risk stratification of patients}

Current treatment protocols for medulloblastoma stratify patients based on clinical features: patient age, metastatic stage, and extent of resection. Patients are stratified based into standard-risk and high-risk groups based on evidence of metastasis and size of the residual tumour after surgery. Additionally, infants are not irradiated to prevent impairment of neurological function by craniospinal irradiation. Typically, patients whose tumour was incompletely (subtotally) resected by surgery or has metastasized are classified as high-risk \citeref{polkinghorn07}. Histological features have not been widely used to stratify patients into risk groups in prospective clinical trials.

The stratification scheme can differ across continents, however. For example, whether a patient should receive radiotherapy is usually determined by an age cutoff, but this threshold can be 2 years \citeref{pezzotta96, geyer94}, 3 years \citeref{geyer05, rutkowski05, polkinghorn07, ashley12, strother14}, 4 years \citeref{white98, kool12, vonbueren11}, or 5 years \citeref{grill05, rutkowski10}. Conversely, hospitals in Japan (and other places) do not use strict age cut-offs to determine eligibility for radiotherapy  \citeref{yasuda08}. Sometimes, infants of less than 1 year of age ocan receive reduced dose chemotherapy as well.


\section{Research objectives}

Our study will focus on the three obstacles that hinders the development of targeted therapy against molecular subgroups of medulloblastoma. First, current methods for classifying medulloblastoma into subgroups molecular classes are difficult to apply in the clinical setting. Second, current clinical prognostication of medulloblastoma poorly predicts patient survival and does not consider molecular subgroup. Third, few actionable therapeutic targets for WNT, Group3, and Group4 medulloblastomas have been discovered to date. In addressing these problems, we demonstrate the clinical significance of the molecular classification of medulloblastoma.

\subsection*{Aim 1: Molecular classification of medulloblastoma for clinicians}

Although the retrospective classification of medulloblastoma has been scientifically informative, molecular subgrouping has not been applied in the context of a prospective clinical trial. One major obstacle is the lack of fresh-frozen samples for most clinical cases. Expression profiling, on which molecular classification was based, depends on the availability of high-quality RNA. In contrast, clinical samples are routinely subjected to formalin-fixation and paraffin-embedding, which preserves tissue integrity but causes nucleic acid degradation. To facilitate the development of therapy specifically targeted against molecular subgroups, we sought to establish an molecular subgrouping assay that can be clinically applied on \gls{ffpe} samples. We have established an analytic pipeline to molecular subgrouping using expression data generated by nanoString assays, and demonstrated its high classification accuracy on \gls{ffpe} samples \citeself{northcott12}. To further make the assay clinically applicable, we have implemented several quality-control measures that identify cases which cannot be reliably assigned molecular subgroup, due to poor specimen quality or assay reaction failure.

\subsection*{Aim 2: Clinical prognostication within molecular subgroups of medulloblastoma}

Prior clinical prognostication studies in medulloblastoma have identified biomarkers without discriminating between the molecular subgroups of medulloblastoma. Given that medulloblastoma subgroups are biologically and molecular distinct disease entities, we hypothesized that incorporating molecular subgroup into prognostication can enhance the accuracy of survival prediction and improve the reliability of risk stratification. Practical and reliable identification of risk could allow for therapy intensification in high-risk children to improve survival and therapy de-escalation in low-risk children to avoid complications of therapy. By identifying clinical and molecular biomarkers within medulloblastoma subgroups, we have designed risk stratification schemes for SHH, Group3, and Group4 medulloblastoma that can achieve unprecedented levels of prognostic accuracy.

\subsection*{Aim 3: Discovering therapeutic targets by genomic profiling of medulloblastoma}

Following the adoption of the molecular classification of medulloblastoma, we then sought to identify molecular targets in medulloblastoma. Unlike SHH medulloblastomas, actionable therapeutic targets for WNT, Group3, and Group4 tumours have yet been identified. Since prior attempts have been underpowered to discriminate the genomic differences among the four molecular subgroups, the \gls{magic}, consisting of scientists and physicians from 43 cities across the globe, has gathered $>1200$ medulloblastomas. We analyzed the genomic copy-number profiles of the tumours by \gls{snp} arrays and identified genes and pathways that characterize each medulloblastoma subgroup \citeself{shih12}.

