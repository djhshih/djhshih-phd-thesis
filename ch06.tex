\chapter{Conclusions and future directions}
\label{ch:conclusion}

Molecular classification driving the next generation of clinical trials \citeref{leary12}.

Is medulloblastoma an entity?
Likely, given its distinguishing expression pattern.

What should be use for classifying cancer?
Anatomical location and histology will likely remain an integral core of classification.
What about genetics? Sure, some genes identifies a cancer type: RB (retinoblastoma), SMARCB1 (\gls{atrt}), PTCH1 (basal cell carcinoma and SHH medulloblastoma) but others are less definiing: TP53 (a spectrum of tumours; inhibiting TP53 will not restore the genomic damage already done). Most cancer mutations likely do not contribute to tumourigenesis.
What about epigenetics? reflect cell type and form contexts permissive for driver mutations. It has a similar goal as histology, but can be done high-throughput and objectively.
Brain vs. organ is useful. Drugs against brain tumour need to cross the blood-brain barrier (at least for now).

Amyotrophic lateral sclerosis has 22 types on OMIM, each is identified by mutation in a specific gene.

Many mouse models require classification \citeself{he14, chow14, northcott14, natarajan13}.

Additional issue: methylation based subgroups. Unified classifier includes expression, DNA methylation, copy-number aberrations and other features (CTNNB1 nuclear localization, histological staining?). This is to be distinguished from typical integrated approaches where all data types are needed; here, the unified approaches uses whatever data that is available.

The discovery of the four molecular subgroups of medulloblastoma has paved the road for the rational development of specific targeted therapy and promises hope for achieving personalized medicine in the treatment of medulloblastoma, where each patient will receive an individualized regimen that maximizes efficacy and minimizes side-effects. Medulloblastomas, not unlike many other malignancies, arises due to a multitude of genetic aberrations, leading to disruptions of biological processes that differ from patient to patient. As we continue to tease out the intricate biological differences among patients with medulloblastomas, we must also provide medical practitioners with the means and the motivation to realize the potential of our discoveries to date, so that both science and medicine can move forward, side by side.

Two major obstacles in the adoption of molecular subgroup classification in the clinic are the lack of applicable biospecimen for genomic analysis and the discrepancy in quality control standards between research and clinical labs. To overcome these challenges, I have tested and validated an assay suitable for subgroup classifcation of \gls{ffpe} samples, and I have implemented rigorous quality control to ensure that the assay results are reproducible, credible, and suitable for guiding clinical decision-making.

If molecular subgroup information does not influence clinical treatment, however, the assay would have little clinical utility. I have therefore identified actionable signalling pathways within medulloblastoma subgroups through copy-number profiling that may serve as rational targets for future therapeutic development. To further fuel the motivation for classifying medulloblastoma subgroups in the clinic, I have identified molecular biomarkers that, together with clinical biomarkers, can stratify patients into risk groups using schemes specific to each molecular subgroup and attain unprecedented prognostic accuracy.

Accordingly, we have addressed some of the challenges that face the treatment of patients with medulloblastoma and provided evidence that supports the classification of medulloblastoma by molecular subgroups in the clinic. By providing medical practitioners with both the means and the motivation to realize the clinical potential of our findings, we strive to usher in a new era of medulloblastoma treatment based on individualized targeted therapy that enhances the quality of care and preserves the quality of life for the patients.

