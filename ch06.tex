\chapter{Conclusions and future directions}
\label{ch:conclusions}

The discovery of the four molecular subgroups of medulloblastoma has paved the road for the rational development of specific targeted therapy and promises hope for achieving personalized medicine in the treatment of medulloblastoma, where each patient will receive an individualized regimen that maximizes efficacy and minimizes side-effects. Medulloblastomas, not unlike many other malignancies, arises due to a multitude of genetic aberrations, leading to disruptions of biological processes that differ from patient to patient. These aberrations begin to form clear patterns when viewed the context of molecular subgroups. To streamline scientific research and clinical practice, we hope to introduce molecular classification in the clinic. Currently, two major obstacles in the adoption of molecular subgroup classification in the clinic are the lack of applicable biospecimen for genomic analysis and the discrepancy in quality control standards between research and clinical labs. To overcome these challenges, we have tested and validated an assay suitable for subgroup classifcation of \gls{ffpe} samples, and we have implemented rigorous quality control to ensure that the assay results are reproducible, credible, and suitable for guiding clinical decision-making. If molecular subgroup information does not influence clinical treatment, however, the assay would have little clinical utility. We have therefore identified actionable signalling pathways within medulloblastoma subgroups through copy-number profiling that may serve as rational targets for future therapeutic development. To further fuel the motivation for classifying medulloblastoma subgroups in the clinic, we have identified molecular biomarkers that, together with clinical biomarkers, can stratify patients into risk groups using schemes specific to each molecular subgroup and attain unprecedented prognostic accuracy. Accordingly, we have addressed some of the challenges that face the treatment of patients with medulloblastoma and provided evidence that supports the classification of medulloblastoma by molecular subgroups in the clinic. We hope that the adoption of molecular classification will inform the next generation of clinical trials and facilitate the development of personalized targeted therapy.

While in pursuit of this long-term goal, we need to address some remaining questions regarding the classification of medulloblastoma. While anatomical location and histology will likely remain an integral core of \gls{cns} tumour classification, numerous other ways of categorizing cancer present the problem of choosing or appropriately integrating classification schemes. Perhaps medulloblastoma could be classified by genomic alterations as in many other cancer types. While some genes identifies a cancer type -- RB (retinoblastoma) and SMARCB1 (\gls{atrt}) -- other genes may be less useful for defining cancer types. For example, TP53 mutation or loss leads to a spectrum of tumours, and restoring TP53 function will not restore the genomic damage already incurred. Additionally, given that most observed mutations in cancer likely do not contribute to tumourigenesis, identifying and validating tumourigenic mutations may be difficult without first grouping cancers into sufficiently homogeneous subtypes. Further, disruption of multiple genes in the same pathway may lead to the same molecular phenotype, though it can be difficult to identify recurrently disrupted pathways prior to molecular classification \citeself{shih12}. The most problematic issue for defining medulloblastoma based on genomic alterations is the relative low frequency of most mutations.  Conversely, epigenetic profiles may reflect the cellular origin that shape the genomic landscape of the tumour, and it may be useful for classification of medulloblastoma. Encouragingly, DNA methylation profiles define very similar subtypes as RNA expression profiles, suggesting that both may be integrated to develop a more robust molecular classification of medulloblastoma.

Since medulloblastoma is now classified into four molecular subgroups, mouse models of medulloblastoma would also need to be classified into the same subgroups. Numerous mouse models purport to recapitulate a specific molecular subgroup of human medulloblastoma, \citeself{he14, chow14, northcott14, natarajan13} but the molecular subgroup of some mouse models are contested \citeref{poschl14}. Better comparative transcriptomics may therefore be needed to resolve controversies surrounding mouse models of medulloblastoma (and perhaps other cancers or diseases). We would need to move beyond mere description of the conservation or divergence of transcriptional programs and  attempt to draw parallels between human and mouse, in order to more precisely identify conserved molecular mechanisms and enable hypotheses regarding human diseases to be tested in mouse models.

Concurrent with ongoing scientific inquiries, the search for more effective therapy for medulloblastoma continues. With the recognition that medulloblastoma comprises four different molecular diseases, many prospective trials are now testing emerging therapies for specific medulloblastoma subgroups, consistent with the spirit of precision medicine. For Group3 medulloblastoma, a \emph{in vitro} drug screened identified  pemetrexed and gemcitabine as a potential combination therapy, and this combination showed efficacy in mouse models of Group3 medulloblastoma (but not SHH medulloblastoma) \citeref{morfouace14}. Similarly, BET bromodomain inhibition of \gene{MYC}-amplified medulloblastoma is currently under investigation \citeref{bandopadhayay14}. For SHH medulloblastoma, SMO inhibitors (e.g. vismodegib and sonidegib) showed some efficacy and present a promising avenue for further development. \citeref{kieran14, gajjar13, rodon14, amakye13}. Further, mouse models of SHH medulloblastoma appear sensitive to inhibition of Auroa and Polo-like kinases \citeself{markant13} or inhibition of BIRC5 (survivin) \citeref{brun14}. In order to expand the arsenal of anti-cancer drugs, it may be prudent to consider administering candidate drugs before standard combination chemotherapy in future clinical trials, as precedent exists for drugs to be effect on untreated tumours but not on recurrent tumours. For example, topotecan (topoisomerase inhibitor) is ineffective in recurrent medulloblastoma \citeref{kadota99, blaney96} but is effective upfront in untreated, high-risk medulloblastoma \citeref{stewart04}. Salvage treatment with chemotherapy should of course be planned so that patient survival is not compromised, and prior trials with replacing radiotherapy with chemotherapy would provide invaluable insight for planning salvage treatments. Past clinical experiences and novel scientific knowledge will together help usher in a new era of medulloblastoma treatment based on individualized targeted therapy that enhances the quality of care and preserves the quality of life for the patients.

% Since high-quality biological specimens of medulloblastoma rare and clinical diagnostic practices differ among treatment centres, we hope to able to identify some medulloblastoma samples with partial information. Unified classifier includes expression, DNA methylation, copy-number aberrations and other features (CTNNB1 nuclear localization, histological staining?). This is to be distinguished from typical integrated approaches where all data types are needed; here, the unified approaches uses whatever data that is available.

% Amyotrophic lateral sclerosis has 22 types on OMIM, each is identified by mutation in a specific gene.

% Brain vs. other organs is useful. Drugs against brain tumour need to cross the blood-brain barrier (at least for now).

% Phase I gamma secretase inhibitor trial \citeref{fouladi11}

% As we continue to tease out the intricate biological differences among patients with medulloblastomas, we hope to provide medical practitioners with the means and the motivation to realize the potential of our discoveries.

% Is medulloblastoma an entity? It is likely entity, given its distinguishing expression pattern from other brain tumours.

% Molecular classification driving the next generation of clinical trials \citeref{leary12}.
