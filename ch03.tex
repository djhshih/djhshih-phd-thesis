\chapter{Clinical prognostication within molecular subgroups of medulloblastoma}
\chaptermark{Clinical prognostication}
\label{ch:clin-prog}

\begin{objective}
We aim to stratify patients into risk groups based on clinical and molecular biomarkers within medulloblastoma subgroups for the purpose of effecting risk-adaptive treatment.
\end{objective}

Medulloblastoma was a uniformly fatal disease with a survival duration of mere months until the introduction of systematic irradiation of the entire central nervous system in the 1940s \citeref{paterson53}. Prior to the adoption of craniospinal (whole brain and spine) irradiation, medulloblastoma cases treated with surgical resection and localized radiotherapy recur with metastases in the subarachnoid space \citeref{mcfarland69}. Although the propensity of medulloblastoma to metastasize necessitated whole \gls{cns} radiotherapy, exposing the developing brain to irradiation led to long-term neuropsychological sequelae that were beginning to be documented in the 1960s \citeref{mcfarland69, gudrunardottir14}. Integration of chemotherapy in the 1970s into the standard treatment of medulloblastoma led to a concomitant rise in patient survival \citeref{gudrunardottir14}. Chemotherapeutic drugs, however, can also have immediate and long-term adverse effects on neurocognitive function \citeref{khong06, zeller13, avan15}. Today, patients with medulloblastoma are treated by surgical resection, followed by craniospinal irradiation and combination chemotherapy. While advances in imaging and surgical technologies have largely eliminated operative mortality and minimized damage to the brain during resection, craniospinal irradiation and combination chemotherapy continue to impair neural development and cause debilitating neurocognitive decline of long-term survivors \citeref{palmer13, schreiber14, knight14, gudrunardottir14}. With modern treatment, patients with medulloblastoma can be cured, but at great cost to their qualities of life.

Aside from impairing brain development, chemotherapy and radiotherapy can cause various other side-effects in long-term survivors of childhood cancer. They can cause endocrinological complications, resulting in delayed puberty, hypothyroidism, growth hormone deficiency, and stunted growth; and neurological complications, leading to symptoms including limb weakness, prolonged pain, reduced sense of touch, balance problems, permanent hearing loss, blindness, seizures, tremors, and paralysis \citeref{armstrong09, xu03, edelstein11, christopherson14, avan15, boman09, duffner98}. Ironically, these anti-cancer treatments can also predispose patients to second cancers \citeref{armstrong09, packer13, christopherson14, avan15, boman09, duffner98}.

The Childhood Cancer Survivor Study reported sobering statistics for adult survivors of childhood cancers and highlighted adverse, long-term socioeconomic consequences of chemotherapeutic and irradiation treatment \citeref{hudson03, mitby03}. The survivors, compared to unaffected siblings, are 5 times more likely to suffer from functional impairments that prohibits independent living, 2 times more likely to earn less than \$20~000 in annual household income \citeref{hudson03}. (Most participants of this study were based in the United States, in which the median household income is more than \$50~000 during the same period \citeref{denavas14}.) Specifically for childhood \gls{cns} cancers, the survivors are 18 times more likely to suffer from functional impairments \citeref{hudson03}. Moreover, 70\% of survivors diagnosed with \gls{cns} cancer before the age of 6 years require special education services to cope with learning or emotional difficulties \citeref{mitby03}. The survivors' use of special education is directly related to the treatment received: cranial irradiation treatment alone increases the odds of needing special education by 7 times, while methotrexate treatment alone increases this odds by 1.3 times, compared to unaffected siblings \citeref{mitby03}. While the long-term neurocognitive effects of chemotherapy, radiotherapy, and the brain tumour itself are intertwined, these findings suggest that cranial irradiation may be the most damaging treatment, and chemotherapy, albeit less harmful, is not entirely innocuous many years after treatment either.

Radiotherapy causes apoptosis (programmed cell death) of dividing cancer cells, but it can also cause normal dividing cells to die, leading to physical, endocrinologic, and neurologic sequelae. In a developing brain, dividing neural progenitors are sensitive to irradiation. Additionally, quiescent neural progenitors or stem cells can also incur radiation-induced DNA damage whose effect may manifest later in life. In patients with acute lymphoblastic leukemia, cranial radiation causes decline in intelligence, and this decline is progressive, showing more impairment of cognitive function with increasing time since radiation therapy \citeref{krull13}. Nowadays, cranial irradiation is reserved for the fewer than 20\% of children with acute lymphoblastic leukemia who are considered to be at high risk for \gls{cns} relapse, in order to spare the maturing brain of the neurotoxic side-effect of radiotherapy \citeref{pui08}. Conversely, brain tumours usually require irradiation for complete tumour eradication and long-term patient survival, though clinicians are increasingly aware of neurologic sequelae following radiotherapy. A recently completed prospective trial assessing 54 Gy conformal (targeted against the tumour bed) radiotherapy in low-grade glioma patients revealed a striking correlation between age at treatment and subsequent decline in \gls{iq} score: the younger the survivor was during conformal radiotherapy, the more severe was the decline in intelligence \citeref{merchant09}. Similarly, younger children with medulloblastoma treated with high-dose irradiation had worse progressive decline in intellectual outcome and academic performance compared to children of older age at diagnosis \citeref{radcliffe94, palmer13, schreiber14, knight14, mulhern98}. Even with a reduced dose of craniospinal radiotherapy, survivors continue to show progressive decline in intellectual and academic outcomes \citeref{ris13, mulhern05, ris01, mulhern98}.

Several attempts have been made over the past three decades to minimize exposure of the developing brain to irradiation. One of the first prospective trials in reducing radiotherapy for patients with medulloblastoma reported an increased rate of tumour recurrence and consequently closed early \citeref{deutsch96}, highlighting the need for a planned strategy for salvaging non-responding disease in order to maintain patient survival. While earlier attempts at reducing craniospinal irradiation led to poorer survival \citeref{thomas00, deutsch96, bailey95, bouffet92}, other attempts at reducing craniospinal irradiation achieved relative success by incorporating chemotherapy into the treatment regiment \citeref{packer06, packer94, packer99, kim10, halberg91, oyharcabal-bourden05, sung13}. Extending this approach, numerous oncology groups sought to postpone or eliminate radiotherapy in young children by using chemotherapy to control or eradicate the tumour \citeref{mulhern89, rutkowski05, jeng93, duffner93, geyer94, gentet95, duffner99, walter99, oyharcabal-bourden05, grill05, geyer05, dhall08, sands10, grundy10, vonbueren11, saha14, yasuda08, kellie02, white98, strauss91}. Unfortunately, postsurgical chemotherapy alone often cannot achieve complete response of the residual tumour, leading to eventual use of radiotherapy. For example, combination chemotherapy with vinblastine, cisplatin, and etoposide was insufficient by itself to induce complete remission of residual medulloblastoma, and patients often progress during chemotherapy. \citeref{jeng93, gajjar94, walter99}. Patients with chemoresistant tumours could be salvaged with subsequent radiotherapy; however, survivors treated with delayed radiotherapy can still suffer from neurodevelopmental deficits \citeref{gajjar94, grill05}.

In yet other trials, clinicians have successfully used high-dose combination therapy to eliminate irradiation treatment in young children with non-metastatic medulloblastoma. Geyer \emph{et al.}\ showed that two combination therapy regimens (vincristine, cisplatin, cyclophosphamide, and etoposide; or vincristine, carboplatin, ifosfamide, and etoposide) could both obviate the need for radiotherapy in patients with no metastatic or residual tumour after surgery \citeref{geyer05}.  Similarly, Grill \emph{et al.}\ used combination chemotherapy (carboplatin, procarbazine, etoposide, cisplatin, vincristine, cyclophosphamide) and salvaged progressive medulloblastoma with radiation and additional high-dose chemotherapy (busulfan, thiotepa, and melphalan) \citeref{grill05}. Patients without metastasis or residual tumour exhibited favourable outcome, and the survivors in this study also had improved intellectual outcome compared to radiotherapy-treated patients \citeref{grill05}. Rutkowski \emph{et al.}\ used combination chemotherapy alone (including vincristine, carboplatin, etoposide, cyclophosphamide, and methotrexate) and achieved favourable survival outcomes for children without metastasis or residual tumour \citeref{rutkowski05}. Decline in \gls{iq} was still evident among the non-irradiated survivors, albeit less severe than those who had received radiotherapy \citeref{rutkowski05}. Chi \emph{et al.}\ using combination chemotherapy (vincristine, cisplatin, etoposide, cyclophosphamide, methotrexate, and thiotepa) and autologous stem cell transplant following the consolidation chemotherapy, yielded complete response in macroscopically metastatic medulloblastomas without the need for radiotherapy \citeref{chi04}. Hence, efforts in using chemotherapy to ward off radiotherapy led to the adoption of the practice of withholding or postponing radiotherapy for treating young children with medulloblastoma in most of North America and Europe.

Based on the results of more recent trials, however, clinicians remain divided on the use of radiotherapy in young children. The HIT~2000 trial (2001--2005) confirmed that combination chemotherapy (cyclophosphamide, vincristine, carboplatin, etoposide, and methotrexate) can maintain favourable survivorship without radiotherapy in nonmetastatic medulloblastoma \citeref{vonbueren11}. Unless complete remission was achieved following induction, the authors recommended contingent treatment with local radiotherapy, secondary surgery, and consolidation chemotherapy (cisplatin, lomustine, and vincristine) \citeref{vonbueren11}. Conversely, the COG-P9934 trial (2000--2006) brought back unconditional, planned irradiation and showed that conformal radiotherapy (localized to posterior fossa and tumour bed) in addition to chemotherapy achieved superior progression-free survival than chemotherapy alone by comparing against the POG-9233 trial \citeref{ashley12}. Despite known risk for long-term neurotoxicity in children, the authors contended that radiotherapy is still required for optimal survival of patients with nonmetastatic medulloblastoma \citeref{ashley12}.

Given the limited capability for chemotherapy to replace radiotherapy in all patients, it is important to select the patients who are at low risk of progressive or recurrent medulloblastoma and evaluate their candidacy for therapy de-escalation. Currently, the main risk factors for medulloblastoma relapse are residual tumour after subtotal surgical resection and presentation with metastasis either macroscopically or in the cerebral spinal fluid. As earlier trials have shown, patients with residual disease or metastasis are poor candidates for reduced or withheld radiotherapy \citeref{grill05, rutkowski05}, though one trial had some success with metastatic medulloblastoma \citeref{chi04}. Additionally, the 5-year overall survival rate for nonmetastatic, completely resected medulloblastoma in young children (age $< 3$) of the HITSKK92 trial (1992--1997) was an impressive 93\% ($n = 17$), while the POG-9233 trial (1992--ongoing) reported a 5-year overall survival rate of 43\% ($n = 37$) for nonmetastatic, completely resected medulloblastoma. Indeed, the results from the two trials are difficult to compare, given differences in the chemotherapy regimen, surgical resection, imaging technologies, and supportive care. Nonetheless, the striking difference in survival for what should be similar cases of medulloblastoma suggest that the two cohorts are, in fact, biologically dissimilar. Accordingly, the prognostic factors currently used in patient risk stratification, metastasis at diagnosis and extent of resection, fail to accurately identify favourable responders to chemotherapeutic treatment.

What has been missing in past clinical trials is the classification of medulloblastoma into molecular subgroups. As medulloblastoma subgroups exhibit different survivorships (\citefig{surv_mb-subgroups}), we believe that subgroups may be useful for risk stratification of patients. Further, given the distinct origins of subgroups \citeref{gibson10}, we hypothesize that prognostic markers would be influenced by subgroups. That is, some markers may be prognostic only in specific subgroups while others may be surrogate markers of subgroup status and have no prognostic value themselves. Therefore, by incorporating molecular subgroups into risk stratification, we believe that we would be able to more accurately predict favourable responders to treatment and obviate the need for indiscriminate administration of intensive treatment to children, who will suffer long-term treatment-induced toxicities. By improving risk stratification and tuning treatment intensity, we hope to minimize collateral damage to the patient's developing brain and preserve the survivor's quality of life.

\begin{figure}[h]
	\begin{center}
		\includegraphics[width=0.7\textwidth]{fig/magic-clin/surv_mb-subgroups.pdf}
	\end{center}
	\caption[Overall survival curves for molecular subgroups of medulloblastoma]
	{
		Overall survival curves for molecular subgroups of medulloblastoma.
		Numbers below x-axis represent patients at risk of event; statistical significances are evaluated by log-rank tests; \gls{hr} estimates are derived from Cox proportional-hazards analyses.
	}
	\label{fig:surv_mb-subgroups}
\end{figure}

\clearpage

The current paucity of markers used in risk stratification is not due to a lack of biomarker studies. Indeed, the medulloblastoma literature is rife with reports of prognostic markers. Most of the purported markers, however, do not reproducibly predict survival in different cohorts due to small sample sizes and distributional differences in underlying covariates \citeself{shih14}. We purport that disagreements in prior biomarker identification attempts may be explained by differences in the composition of medulloblastoma subgroups in the different cohorts. For example, patients with desmoplastic medulloblastoma often exhibit better survival \citeref{chatty71, rutkowski05, kool12, pietsch14, dhall08, vonbueren11, ashley12, grundy10}, and the discordant survival outcomes between the POG-9233 and the COG-P9934 trials could be due the latter having a higher proportion of desmoplastic medulloblastoma, leading to better patient survival \citeref{ashley12}. The aforementioned difference in survival outcomes between the POG-9233 and the HITSKK92 trial could also be due to a similar reason. Indeed, the small sample sizes in these trials (POG-9233, $n = 112$; HITSKK92, $n = 62 $; COG-P9934, $n = 82$) makes uneven distribution of covariates likely, and these covariates may be responsible for the outcome differences. Further, while desmoplastic histology has been proposed to be one such covariate, it is likely that unobserved covariates such as molecular subgroups and genetic mutations may better explain the differences in response. In addition to its status as a favourable prognostic factor, desmoplasia has also been reported to be a prognostic factor for poor survival in medulloblastoma by some \citeref{park83, gajjar06, rutkowski10} and an insignificant factor by others \citeref{pietsch14, lannering12}. High interobserver and intraobserver variability in histological examination may be responsible for these discrepancies. Therefore, difficulty and variability in assessment can limit the utility of a biomarker, and competing covariates should be assessed carefully.

In order to mitigate the problems of small sample size and competing unobserved covariates, we assembled an large, international cohort of 673 medulloblastoma samples with clinical annotation. The cytogenetic and focal copy-number events were determined using copy-number profiling on this discovery cohort. We identified subgroup-specific cytogenetic events and integrated them with clinical variables to develop subgroup-specific, multivariate risk-stratification models based on the discovery cohort. In order to validate the models and ensure that the technique was generalizable to routine pathology laboratories, we then studied a panel of six cytogenetic biomarkers (\gene{GLI2}, \gene{MYC}, 11, 14, 17p, and 17q) using interphase \gls{fish} on an \gls{ffpe} medulloblastoma tissue microarray that includes a set of 453 medulloblastomas that were treated at a single center and does not overlap with the discovery cohort.

The size of our discovery and validation cohorts provides unprecedented power for clinical prognostication and enables comprehensive, multivariate modeling of patient survival to identify robust prognostic markers \citefig{meta_cyto-markers}. In this retrospective study, we wish to comprehensively assess cytogenetic markers in the context of the molecular subgroups of medulloblastoma and determine whether subgroup affiliation could complement clinical variables for more accurate risk stratification of patients and predict favourable responders for de-escalation of radiotherapy, in order to improve the quality of life for survivors. Being retrospective, our cohort is subject to recall bias (for cases with frozen samples), and it encompasses heterogeneously treated patients from multiple centres and continents. Our histology records were not centrally reviewed; nevertheless, our study reflects the typical clinical experience more closely and implicitly reveals the weakness of histological diagnosis in decentralized clinical practice. Further, the lack of surgical details and treatment protocols precludes analyses on how specific treatments and extent of surgical resection affect survival outcome. Notwithstanding the limitations of our discovery cohort, the findings are highly reproducible in an independent cohort of patients.

\begin{SCfigure}[5][t]
	\includegraphics[width=0.7\textwidth]{fig/magic-clin/meta_cyto-markers.pdf}
	\caption[Sample sizes of recent prognostic marker studies]
	{
	Sample sizes of recent prognostic marker studies.
	This meta-analysis was performed by Marc Remke.
	}
	\label{fig:meta_cyto-markers}
\end{SCfigure}

\clearpage

\section{Materials and methods}

\subsection{Patient information}

All tissues and clinicopathological information were collected in accordance with institutional review boards from various contributing centers. In the discovery set, although precise treatment dates were often unavailable, at least 95\% of the patients were treated within the past 15 years using modern treatment protocols, including surgical resection, craniospinal (whole brain and spine) irradiation, or chemotherapy. Discovery set samples were collected between 2005 and 2013, with a focus on samples with available fresh-frozen material. Among the samples with treatment details, the earliest diagnosis is July 1997 and the latest is August 2012. Samples in the validation set were all obtained from the Burdenko institute with no selection criterion applied. All patients in the validation set were treated between 1995 and 2010 according to standardized therapy protocols of the German HIT study group.

\subsection{Tumor material and patient characteristics}

A discovery set of 673 medulloblastoma samples with clinical follow-up was acquired retrospectively from 43 cities around the globe. These samples were copy-number profiled on the Affymetrix SNP6 array platform in order to identify potential molecular biomarkers \citeself{shih12}. An independent validation set of 453 samples with clinical follow-up on a medulloblastoma tissue microarray was analyzed by FISH as previously described \citeself{northcott11a}. The validation set consisted only of patients treated in Burdenko, Moscow. Tumors were classified based on signature marker expression into molecular subgroups as previously described \citeself{northcott12}; additional tumours were classified based on cytogenetic aberrations using standard conditional probability models. Subgroup affiliation was not available for 162 discovery samples. The validation set includes an additional set of 50 WNT tumours that were not on the tissue microarray. Nucleic acid isolation, tissue microarray construction, and $\beta$-catenin mutation analysis were performed as previously described \citeself{shih12}. Tissue microarray analysis was performed by collaborators Andrey Korshunov and Stefan Pfister.

\subsection{Prognostic biomarker identification}

Cytogenetic events and copy-number aberrations were identified as previously described in the discovery set \citeself{shih12}. All chromosomal events (or chromosome arm events) were compared against reference samples with balanced copy-number for the chromosome (or chromosome arm); samples with copy-number changes in the opposite direction were specifically excluded from each comparison. Subsequent to biomarker discovery, cross-validation with correction for multiple hypothesis testing was performed to estimate the reproducibility and generalizability of the potential biomarkers in an independent cohort. During cross-validation, the discovery set was split randomly into two subsets. First, the biomarkers are tested by the log-rank test on the first subset. Then, statistically significant biomarkers ($p < 0.05$) are tested again by the log-rank test on the second subset, with correction for multiple hypotheses testing. This process was repeated 10~000 times to estimate the expected validation rate of each biomarker. The expected validation rate of each biomarker is $n_v / n_d$, where $n_d$ is the number of times a biomarker is significant in the first subset and $n_v$ is the number of times a discovered biomarker is also significant in the second subset. The final set of biomarkers was further validated in the external validation set. Additionally, sample size estimates for prospective trials of each biomarker were calculated under univariate Cox models based on the observed hazard ratios. (See page \pageref{sec:prognostication} in the \emphlab{Appendix} for more details on prognostic biomarker discovery.)

\subsection{Multiple hypothesis testing correction}

Within each biomarker identification analysis, correction for multiple hypothesis testing was performed by the Benjamini-Hochberg method\citeref{benjamini95} during cross-validation. Independent analyses were corrected for multiple hypotheses testing independently: clinical biomarker identification across medulloblastoma, within WNT medulloblastoma, within SHH medulloblastoma, within Group3, and within Group4; molecular biomarker identification across medulloblastoma, within WNT medulloblastoma, within SHH medulloblastoma, within Group3 medulloblastoma, and within Group4 medulloblastoma.

\subsection{Time-dependent ROC analysis}

Time-dependent \gls{roc} analyses were performed using the \code{CoxWeights} function provided in the risksetROC (v1.0.4) R package. This function calculates areas under time-dependent ROC curves as described by Heagerty and Zheng \citeref{heagerty05}. Estimates of \gls{auc} for the fitted multivariate Cox model being assessed were calculated every month, from 1 month to 60 months, in order to determine the collective predictive performance of the biomarkers in the Cox models. Differences in \gls{auc} estimates among Cox models across time points were tested by Friedman rank sum tests.

\subsection{Risk stratification model selection}

Candidate prognositic markers, including all cytogenetic events, focal copy-number events, and all clinical features, were first tested by univariate survival analyses (log-rank tests) individually. Significant unviariate markers were tested under multivariate Cox proportional-hazards models including age and gender as covariates. All significant multivariate markers (including age) were included in the model selection step. The optimal survival models for each medulloblastoma subgroup were determined by unbiased model selection procedures: stepwise regression using forward selection, backward elimination, and bidirectional elimination. To make the final model practical for use with \gls{fish} in diagnostic laboratories, a maximum of three cytogenetic or copy-number markers were included in each candidate model. The selected models were compared by analyses of deviance tests to determine the model that best fits the discovery cohort. Finally, the survival data was re-analyzed by the optimal model in order to assess the survival of patients with each combination of variable levels; the risk-stratification trees were manually designed in order to group patients into distinct risk groups.

\subsection{Statistical analysis}

Patient survival characteristics were right-censored at 5 years (or 10 years) and analyzed by the Kaplan-Meier method. Univariate comparison of two or more survival curves were performed using log-rank tests and the Cox proportional-hazards regression models. The predictive values of biomarkers were assessed by analyses of deviance tests under multivariate Cox models and by time-dependent \gls{roc} analyses. Associations between covariates and risk groups were tested by the Fisher's exact test. All statistical analyses were performed in the R software environment (v2.15), using R packages survival (v2.36), risksetROC (v1.0.4), powerSurvEpi (v0.0.6), and ggplot2 (v0.9.3).


\section{Results}

\subsection{Prognostic significance of clinical variables within medulloblastoma subgroups}

Many prior medulloblastoma biomarker publications were limited by sample size, a problem that will only be exacerbated once cohorts are divided into their molecular subgroups. The current study includes 1126 medulloblastoma patients (673 discovery plus 453 validation patients), which is more than double the sample size of any prior medulloblastoma biomarker publication, and one of only a very few that includes a validation cohort (\citefig{meta_cyto-markers}). Although the discovery cohort accumulated by \gls{magic} consists of medulloblastomas gathered from 43 different treating centers from around the world, the subgroup-specific outcome mirrors what has been previously published with very good outcomes for WNT patients, poor outcomes for Group3 patients, and intermediate outcomes for SHH and Group4 patients (\citefig{surv_mb-subgroups}) suggesting that the discovery cohort is a representative sample \citeself{shih14}.

\begin{figure}[h]
	\begin{center}
		\includegraphics[width=\textwidth]{fig/magic-clin/surv_ageg_mstat_wnt.pdf}
	\end{center}
	\caption[Ten-year overall survival curves for WNT medulloblastoma]
	{
	Ten-year overall survival curves for WNT medulloblastoma, split by age group or metastatic status.
	Numbers below x-axis represent patients at risk of event; statistical significances are evaluated by log-rank tests; \gls{hr} estimates are derived from Cox proportional-hazards analyses.
	}
	\label{fig:surv_ageg_mstat_wnt}
\end{figure}

In order to assess long-term survivors, WNT patients were followed for up to 10 years, and only two deaths were observed, both late in the follow-up period and due to recurrence of medulloblastoma (\citefig{surv_ageg_mstat_wnt}). Among the SHH tumours, there is a significantly better outcome in the adult patients as compared to children or infants (\citefig{surv_ageg_shh_group3_group4}). There is a trend towards a worse outcome for infants with Group3 tumours that is not statistically significant (\citefig{surv_ageg_shh_group3_group4}). Infants with Group4 tumours have a significantly worse outcome than children or adults (\citefig{surv_ageg_shh_group3_group4}), suggesting that radiation therapy is critical in the treatment of Group4 medulloblastoma. There is no reproducible association between gender and prognosis in any of the four subgroups (\citeself{shih14}). Desmoplastic histology portends a more favorable prognosis than classic histology, which is more favorable than anaplastic histology among SHH tumours \citeself{shih14}. Large cell/anaplastic histology has prognostic significance for Group3 medulloblastomas in the discovery cohort, but does not validate as significant in the validation cohort.

\begin{figure}[ht]
	\begin{center}
		\includegraphics[width=\textwidth]{fig/magic-clin/surv_ageg_shh_group3_group4.pdf}
	\end{center}
	\caption[Overall survival curves for age groups within SHH, Group3, and Group4 subgroups]
	{
		Overall survival curves for age groups within SHH, Group3, and Group4 subgroups.
		Numbers below x-axis represent patients at risk of event; statistical significances are evaluated by log-rank tests; \gls{hr} estimates are derived from Cox proportional-hazards analyses.
	}
	\label{fig:surv_ageg_shh_group3_group4}
\end{figure}

While metastatic status is not prognostic for patients with WNT medulloblastoma, macroscopic metastasis (M2/M3) is consistently associated with poor survival in all non-WNT subgroups, though the clinical effect is very slight among patients with Group4 disease (\citefig{surv_mstat_shh_group3_group4}). While the prognostic significance of M0 disease as compared to M2/3 disease is very clear across SHH, Group3, and Group4, the prognostic significance of isolated M1 disease is less clear (\citefig{surv_mstat_shh_group3_group4}). Isolated M1 disease is associated with increased risk in Group3 in the discovery cohort, but not the validation cohort, with the opposite pattern seen in the SHH patients. However, for both discovery and validation cohorts, there are no survival differences survival between M0 and M1 patients with Group4 disease. There are no CNAs in any of the subgroups that are associated with an increased risk of leptomeningeal dissemination (not shown). Overall, many clinical biomarkers continue to exhibit prognostic significance when medulloblastoma is analyzed in a subgroup-specific fashion.

\begin{figure}[ht]
	\begin{center}
		\includegraphics[width=\textwidth]{fig/magic-clin/surv_mstat_shh_group3_group4.pdf}
	\end{center}
	\caption[Overall survival curves for metastatic status within SHH, Group3, and Group4 subgroups]
	{
	Overall survival curves for metastatic status within SHH, Group3, and Group4 subgroups.
	Numbers below x-axis represent patients at risk of event; statistical significances are evaluated by log-rank tests; \gls{hr} estimates are derived from Cox proportional-hazards analyses.
	}
	\label{fig:surv_mstat_shh_group3_group4}
\end{figure}

\subsection{Subgroup and metastatic status are the most predictive markers}

Multivariate survival analyses were conducted in order to dissect the relative predictive value of clinical variables (age, gender, metastatic status, and histotype) and molecular subgroup affiliation. Stepwise Cox proportional-hazards regressions revealed that molecular subgroup significantly contributes to multivariate survival prediction, on top of a regression model already parameterized by clinical variables gender, age, metastatic status, and histology (\citefig{subgroup-specific_cox}\emphlab{a}). Further, Cox models parameterized with both clinical biomarkers and molecular subgroup achieve higher prediction accuracy in time-dependent \gls{roc} analyses (\citefig{subgroup-specific_cox}\emphlab{b}). In isolation, each biomarker has modest prediction accuracy (\citefig{subgroup-specific_cox}\emphlab{c}) compared to the complete multivariate model (\citefig{subgroup-specific_cox}\emphlab{b}). In the complete model, the removals of metastatic status and subgroup lead to the greatest decreases in predictive accuracy (\citefig{subgroup-specific_cox}\emphlab{d}). Taken together, these results suggest that subgroup affiliation and metastatic status are the most important predictive biomarkers, and that they make non-redundant contributions to the prediction of survival. We conclude that combining both clinical biomarkers (metastatic status) and molecular biomarkers (subgroup affiliation) will make the optimal tool for predicting survival of medulloblastoma patients.

\begin{figure}[h]
	\begin{center}
		\includegraphics[width=\textwidth]{fig/magic-clin/subgroup-specific_cox.pdf}
	\end{center}
	\caption[Molecular subgroup and metastatic status are the most important prognostic biomarkers]
	{
	Molecular subgroup and metastatic status are the most important prognostic biomarkers.
	\textbf{a}, Multivariate Cox proportional-hazards survival analysis of predictor variables. Starting with the null model, each variable is added stepwise (from top to bottom) to the survival model. Model likelihood values assess the degree to which each Cox model fits the survival data. Increments in model likelihoods are tested by analysis of deviance. 
	\textbf{b}, Average areas under time-dependent receiver operating characteristic curves (AUC) for multivariate Cox models parameterized by only clinical variables, or both clinical and subgroup variables.
	\textbf{c}, Average time-dependent AUCs for univariate Cox models parameterized by each variable.
	\textbf{d}, Predictive importance of each variable in the fully-parameterized multivariate Cox models, as determined by the average decrease in time-dependent AUC when the variable is omitted from the model.
	Differences in time-dependent AUC and predictive importance are evaluated by the Friedman rank sum test.
	}
	\label{fig:subgroup-specific_cox}
\end{figure}

\clearpage

\subsection{Subgroup specificity of published molecular biomarkers}

Several cytogenetic biomarkers have been previously reported to be associated with patient survival across medulloblastoma, but their prognostic values have seldom been assessed in the context of medulloblastoma subgroups \citeself{shih14}. Loss of chromosome 6 is significantly associated with improved survival across all medulloblastoma (\citefig{subgroup-specific_eg}\emphlab{a}). However, the prognostic value of chr6 loss can be completely attributed to its enrichment in WNT medulloblastomas, as chr6 loss has no prognostic value among WNT patients, or among non-WNT patients, when compared to their respective controls with balanced chr6 (\citefig{subgroup-specific_eg}\emphlab{b}). We would suggest that chr6 loss is a subgroup-driven biomarker in that its prognostic significance is driven by its enrichment in a particular subgroup, and it thus has no further significance in subgroup-specific analysis. Further, based on these results, we would caution against using chr6 loss as the lone diagnostic criteria for WNT medulloblastoma, since it is also observed in non-WNT medulloblastoma (7/49 chr6 loss medulloblastomas were not WNT (14\%)), and chr6 loss is only present in 42/53 WNT tumours (79\%). The prognostic role of isochromosome 17q (iso17q) has been very controversial; in our cohort, iso17q is a statistically significant predictor of poor outcome overall (\citefig{subgroup-specific_eg}\emphlab{c}). However, subgroup-specific analysis demonstrates that iso17q is highly prognostic for Group3 medulloblastoma, but not for Group4 medulloblastoma (\citefig{subgroup-specific_eg}\emphlab{d}), indicating that it is a subgroup-specific molecular biomarker. Similarly, while chr10q loss is a modestly significant predictor of poor outcome across medulloblastoma subgroups (\citefig{subgroup-specific_eg}\emphlab{e}), its prognostic power is limited to the SHH subgroup of tumours in a subgroup-specific analysis (\citefig{subgroup-specific_eg}\emphlab{f}). We conclude that determination of molecular subgroup is crucial in the evaluation and implementation of molecular biomarkers for patients with medulloblastoma, as some putative biomarkers are merely enriching for a specific subgroup (\emphterm{subgroup driven}) while most others are only significant within a specific subgroup (\emphterm{subgroup specific}).

\clearpage

\begin{figure}[h]
	\begin{center}
		\includegraphics[width=\textwidth]{fig/magic-clin/subgroup-specific_eg.pdf}
	\end{center}
	\caption[Subgroup-driven and subgroup-specific molecular biomarkers]
	{
	Subgroup-driven and subgroup-specific molecular biomarkers.
	\textbf{a}, Overall survival curves and frequency distribution of chr6 status across the entire cohort.
	\textbf{b}, Overall survival curves for chr6 status in WNT and non-WNT medulloblastomas.		
	\textbf{c}, Overall survival curves and frequency distribution of isolated chr17q gain across the entire cohort.
	\textbf{d}, Overall survival curves for chr17q status in Group3 and Group4 subgroups. 
	\textbf{e}, Overall survival curves for chr10q status across the entire cohort.
	\textbf{f}, Overall survival curves for chr10q status in SHH and non-SHH medulloblastomas.
	Numbers below x-axis represent patients at risk of event; statistical significances are evaluated by log-rank tests; \gls{hr} estimates are derived from Cox proportional-hazards analyses.
	}
	\label{fig:subgroup-specific_eg}
\end{figure}

\clearpage

\subsection{SHH patients can be stratified into three distinct risk groups}

We identified 11 CNAs that are prognostically significant in our SHH medulloblastoma discovery set in univariate survival analyses (\citefig{shh-markers}, \citefig{shh-all-markers}). Given the considerable number of candidates, the reproducibility of the identified biomarkers was assessed by cross-validation. Furthremore, the expected sample sizes required for validation in future prospective trials were estimated using power analyses under Cox proportional-hazards models, in order to guide candidate prioritization in future prospective trials. Specific amplifications but not broad gains encompassing \gene{GLI2} or \gene{MYCN} are associated with bleak prognosis (\citefig{shh-markers}\emphlab{a--b}). Loss of chr14q confers significantly inferior survival (\citefig{shh-markers}\emphlab{c}). There is no minimal region of deletion on chr14 in SHH patients, and recent medulloblastoma re-sequencing efforts have not identified any recurrent SNVs on chr14 in SHH medulloblastoma \citeself{shih14}. The presence of chromothripsis (chromosome shattering) is associated with worse survival in SHH patients (\citefig{shh-markers}\emphlab{d}).

To integrate the individual biomarkers into a risk stratification model, multivariate Cox proportional-hazards analyses were performed on all significant prognostic markers. Through multiple stepwise regression procedures, a consensus set of biomarkers was selected for inclusion in the model in an unbiased manner. The proposed risk stratification scheme represents the model that was most consistent with available data in the discovery cohort, from among many possible alternatives (\citefig{shh-risk-strat}\emphlab{a}) \citeself{shih14}. \gene{GLI2} amplification, chr14q loss, and leptomeningeal dissemination (M+ disease) identify high and standard risk patients. Specifically, \gene{GLI2} amplification alone can identify patients with bleak prognosis (\citefig{shh-risk-strat}\emphlab{a}) \citeself{shih14}. Absence of these markers demarcates a low-risk group of patients who exhibit survivorship reminiscent of WNT patients. Importantly, none of the covariates, particularly age and anaplastic histology, can explain the survival differences observed among the risk groups (\citefig{shh-risk-strat}\emphlab{a}) \citeself{shih14}. Direct application of the proposed risk stratification scheme on the independent validation cohort yields distinct survivorships for the three risk groups, thereby validating the model (\citefig{shh-risk-strat}\emphlab{c}).

\begin{figure}[h]
	\begin{center}
		\includegraphics[width=\textwidth]{fig/magic-clin/shh-markers.pdf}
	\end{center}
	\caption[Overall survival curves for molecular biomarkers in SHH medulloblastoma]
	{
	Overall survival curves for molecular biomarkers in SHH medulloblastoma:
	\textbf{a}, \gene{GLI2} copy number status;
	\textbf{b}, \gene{MYCN} copy number status;
	\textbf{c}, chr14q status; and
	\textbf{d}, chromothripsis status.
	Numbers below x-axis represent patients at risk of event; statistical significances are evaluated by log-rank tests; \gls{hr} estimates are derived from Cox proportional-hazards analyses.
	}
	\label{fig:shh-markers}
\end{figure}

Two additional stratification schemes were constructed using only clinical biomarkers or only cytogenetic markers; however, the proposed model, which combines both types of biomarkers, yields the highest prediction accuracy (\citefig{shh-risk-strat}\emphlab{b}) \citeself{shih14}. Furthermore, the accuracy of the combined risk model is reduced when applied across non-SHH patients, further underscoring the importance of taking subgroup into consideration during risk stratification. By using two molecular biomarkers (\gene{GLI2} and 14q \gls{fish}) and metastatic status, we can reliably predict prognosis for patients with SHH medulloblastoma.

\clearpage

\begin{figure}[h]
	\begin{center}
		\includegraphics[width=\textwidth]{fig/magic-clin/shh-all-markers.pdf}
	\end{center}
	\caption[Overall survival curves for significant cytogenetic biomarkers in SHH medulloblastoma]
	{
		Overall survival curves for significant cytogenetic biomarkers in SHH medulloblastoma.
		Numbers below x-axis represent patients at risk of event; statistical significances are evaluated by log-rank tests; \gls{hr} estimates are derived from Cox proportional-hazards analyses.
	}
	\label{fig:shh-all-markers}
\end{figure}

\clearpage

\begin{figure}[h]
	\begin{center}
		\includegraphics[width=\textwidth]{fig/magic-clin/shh-risk-strat.pdf}
	\end{center}
	\caption[Combined clinical and molecular biomarkers improve risk-stratification of SHH patients]
	{
	Combined clinical and molecular biomarkers improve risk-stratification of SHH patients.
	\textbf{a}, Risk stratification of SHH medulloblastomas by molecular and clinical prognostic markers. \emph{Top-left}, decision tree; \emph{bottom-left}, events plot depicting status of molecular and clinical markers across the risk groups; \emph{right}, overall survival curves for SHH risk groups.
	\textbf{b}, Average time-dependent AUCs for risk groups stratified using only clinical or molecular markers, or both. Risk stratification regimens are applied to SHH and non-SHH medulloblastomas. ***, $p < 0.001$, Friedman rank sum tests.
	\textbf{c}, Survival curves for SHH risk groups in the validation cohort.
	Numbers below x-axis represent patients at risk of event; statistical significances are evaluated by log-rank tests; \gls{hr} estimates are derived from Cox proportional-hazards analyses.
	}
	\label{fig:shh-risk-strat}
\end{figure}

\clearpage

\subsection{Three biomarkers demarcate high-risk Group3 patients}

In Group3 patients, iso17q and \emph{MYC} amplification remain the only cytogenetic markers associated with poor survival, whereas chr8q loss and chr1q gain are the only good prognosis markers (\citefig{group3-markers}) \citeself{shih14}. In multivariate survival analyses, patients with metastasis, iso17q, or MYC amplification represent the high-risk group (\citefig{group3-risk-strat}\emphlab{a}). Critically, absence of these markers can identify a population of Group3 patients who have a survivorship much longer than Group3 taken as a whole. The risk groups are not associated with any clinical covariates, including age (\citefig{group3-risk-strat}\emphlab{a}) \citeself{shih14}. Consistent with the findings in SHH patients, optimal risk stratification in Group3 patients requires the use of both clinical and molecular prognostic markers, which have reduced or no prognostic value outside of Group3 (\citefig{group3-risk-strat}\emphlab{b}) \citeself{shih14}. Our proposed risk stratification scheme was validated on the non-overlapping validation cohort using three molecular biomarkers (\emph{MYC}, 17p, and 17q \gls{fish}) and metastatic status (\citefig{group3-risk-strat}\emphlab{c}).

\bigskip

\begin{figure}[h]
	\begin{center}
		\includegraphics[width=\textwidth]{fig/magic-clin/group3-markers.pdf}
	\end{center}
	\caption[Overall survival curves for molecular biomarkers in Group3 medulloblastoma]
	{
	Overall survival curves for molecular biomarkers in Group3 medulloblastoma:
	\textbf{a}, chr17 copy number aberrations;
	\textbf{b}, \emph{MYC} copy number status; and 
	\textbf{c}, chr8q status.
	\textbf{d}, Risk stratification of Group3 medulloblastomas by molecular and clinical prognostic markers.
	Numbers below x-axis represent patients at risk of event; statistical significances are evaluated by log-rank tests; \gls{hr} estimates are derived from Cox proportional-hazards analyses.
	}
	\label{fig:group3-markers}
\end{figure}

\clearpage

\begin{figure}[h]
	\begin{center}
		\includegraphics[width=\textwidth]{fig/magic-clin/group3-risk-strat.pdf}
	\end{center}
	\caption[Combined clinical and molecular biomarkers improve risk-stratification of Group3 patients.]
	{
	Combined clinical and molecular biomarkers improve risk-stratification of Group3 patients.
	\textbf{a}, Risk stratification of Group3 medulloblastomas by molecular and clinical prognostic markers.	\emph{Top-left}, decision tree; \emph{bottom-left}, events plot depicting status of molecular and clinical markers across the risk groups; \emph{right}, overall survival curves for Group3 risk groups.
	\textbf{b}, Average time-dependent AUCs for risk groups stratified using only clinical or molecular markers, or both. Risk stratification regimens are applied to Group3 and non-Group3 medulloblastomas. ***, $p < 0.001$, Friedman rank sum tests.
	\textbf{c}, Survival curves for Group3 risk groups in the validation cohort.
	Numbers below x-axis represent patients at risk of event; statistical significances are evaluated by log-rank tests; \gls{hr} estimates are derived from Cox proportional-hazards analyses.
	}
	\label{fig:group3-risk-strat}
\end{figure}

\clearpage


\subsection{Identification of a low-risk group of metastatic Group4 patients}

Group4 patients with whole chromosome loss of chr11 or gain of chr17 exhibit better survival under univariate Cox models (\citefig{group4-markers}\emphlab{a}), in addition to chr10p loss \citeself{shih14}. There is no cytogenetic marker associated with poor prognosis \citeself{shih14}. Specifically, neither \gene{MYCN} gain nor amplification is associated with poorer survival in Group4, in stark contrast to SHH patients, reinforcing the distinction in their underlying biology (\citefig{group4-markers}\emphlab{b}) \citeself{shih14}. Similarly, none of the cytogenetic biomarkers identified for Group3 patients (e.g. iso17q) have any prognostic value in Group4 \citeself{shih14}. Following unbiased model selection, the consensus set of biomarkers results in a risk stratification scheme in which leptomeningeal dissemination identifies high-risk Group4 patients, except in the context of chr11 loss or chr17 gain (\citefig{group4-risk-strat}\emphlab{a}). The biology underlying chr11 loss is not apparent as there is no obvious minimal common region of deletion, nor are there any recurrent SNVs on chr11 reported in the recent medulloblastoma re-sequencing publications \citeself{shih14}. Group4 patients with either chr17 gain or chr11 loss, irrespective of their metastatic statuses, exhibit survivorship that is characteristic of WNT patients in both the discovery and validation cohorts (\citefig{group4-risk-strat}\emphlab{a},\emphlab{c}), and the survival differences are not explainable by covariates \citeself{shih14}. Significantly, the low-risk Group4 cohort also included some patients with anaplastic histology, suggesting that anaplasia may not be universally prognostic for poor outcome. Consistent with other subgroups, the risk stratification model using both clinical and molecular biomarkers achieve the highest accuracy (\citefig{group4-risk-strat}\emphlab{b}). Critically, the cytogenetic biomarkers identify low-risk Group4 patients whom would be otherwise designated as high-risk by evidence of metastasis and/or anaplastic histology, and this finding cannot be extrapolated to SHH and Group3 patients (\citefig{group4-risk-strat}). We conclude that through the use of three molecular biomarkers (chr11, 17p, and 17q \gls{fish}) and metastatic status, we can accurately and reliably predict the survival of patients with Group4 medulloblastoma.

\bigskip

\begin{figure}[h]
	\begin{center}
		\includegraphics[width=\textwidth]{fig/magic-clin/group4-markers.pdf}
	\end{center}
	\caption[Overall survival curves for molecular biomarkers in Group4 medulloblastoma]
	{
	Overall survival curves for molecular biomarkers in Group4 medulloblastoma:
	\textbf{a}, whole chr11 status and whole chr17 status; and
	\textbf{b}, \gene{MYCN} copy number status.
	Numbers below x-axis represent patients at risk of event; statistical significances are evaluated by log-rank tests; \gls{hr} estimates are derived from Cox proportional-hazards analyses.
	}
	\label{fig:group4-markers}
\end{figure}

\clearpage

\begin{figure}[h]
	\begin{center}
		\includegraphics[width=\textwidth]{fig/magic-clin/group4-risk-strat.pdf}
	\end{center}
	\caption[Combined clinical and molecular biomarkers improve risk-stratification of Group4 patients]
	{Combined clinical and molecular biomarkers improve risk-stratification of Group4 patients.
	\textbf{a}, Risk stratification of Group4 medulloblastomas by molecular and clinical prognostic markers. \emph{Top-left}, decision tree; \emph{bottom-left}, events plot depicting status of molecular and clinical markers across the risk groups; \emph{right}, overall survival curves for Group4 risk groups.
	\textbf{b}, Average time-dependent AUCs for risk groups stratified using only clinical or molecular markers, or both. Risk stratification regimens are applied to Group4 and non-Group4 medulloblastomas. ***, $p < 0.001$, Friedman rank sum tests.
	\textbf{c}, Survival curves for Group4 risk groups in the validation cohort. 
	Numbers below x-axis represent patients at risk of event; statistical significances are evaluated by log-rank tests; \gls{hr} estimates are derived from Cox proportional-hazards analyses.
	}
	\label{fig:group4-risk-strat}
\end{figure}

\clearpage


\section{Discussion}

The analysis of $> 1000$ medulloblastoma patients clearly demonstrates that subgroup affiliation enhances prognostication with clinical biomarkers and that the majority of published molecular biomarkers are only relevant in the setting of a single subgroup. The combination of clinical variables, molecular subgroup, and six cytogenetic markers analyzed on \gls{ffpe} tissues can achieve an unprecedented level of prognostic prediction for medulloblastoma patients that is practical, reliable, and reproducible. The proposed risk stratification models represent those that best fit the available data in the discovery cohort. Despite the large size of our discovery cohort, missing data and the complexity of multivariate analyses may necessitate the use of even larger cohorts to assess the inclusion of additional prognostic markers. Moreover, while we strive to include the most important markers in multivariate models, we cannot exclude the possibility that alternative markers may perform equally well. Our results nonetheless elucidate the prognostic potential of known and novel markers and highlight clinically useful risk-stratification schemes.

The prognostic significance of M1 status (presence of cells in the cerebrospinal fluid) has long been controversial. Most reports agree that presence of metastasis portends poor prognosis and warrants intensified treatment \citeref{grill05, rutkowski05, salama06, rutkowski10, kool12, pietsch14, vonhoff09, bouffet94, zeltzer99}; however, it is unclear whether M1 disease has the same prognosis as M2/M3 (macroscopic metastasis). Kortmann \emph{et al.}\ contended in a prospective trial that M2/M3 status were indicators of poor outcome in medulloblastoma, but residual disease or M1 status were not \citeref{kortmann00}. In another prospective trial, Zeltzer \emph{et al.}\ maintained that both M1 and M2/M3 statuses were prognostically unfavourable \citeref{zeltzer99}. In subsequent studies, some investigators group M0 and M1 together in one category \citeref{rutkowski05, pietsch14}, while others group M1, M2 and M3 together as M+ \citeref{rutkowski10, kool12, pietsch14, strother14}. In a retrospective review, Sanders \emph{et al.}\ reported that M1 patients do not have better survival than M2/M3 patients under the same treatment \citeref{sanders08}. In our cohorts, the prognostic significance of M1 disease may be subgroup specific, though the small sample size of M1 patients hinders a definitive conclusion \citeself{shih14}. Based on our data, it is unlikely that M1 status is a universal indicator of poor outcome. Nevertheless, irrespective of whether M1 was categorized with M0 or M2/M3, our risk-stratification models can reproducibly and robustly predict patient survival.

Controversy also surrounds the prognostic value of anaplastic histology. Large cell and anaplastic histologies are often grouped together because these histological features often co-occur and can be difficult to distinguish. Several studies reveal that large cell/anaplastic histology is prognostically unfavourable \citeref{gajjar04, jakacki12, kool12, pietsch14}. von Hoff \emph{et al.}\ distinguished between large cell and anaplastic histologies and reported that large cell histology was a negative prognostic factor while anaplasia was not \citeref{vonhoff10}. The authors further suggested that \emph{MYC} amplification, which co-occurred with large cell histology, may be the underlying cause of poor prognosis; however, the precise definition of \emph{MYC} amplification remains contentious \citeref{vonhoff10, raabe10}. We clarify this issue by demonstrating that only high-level \emph{MYC} amplification but not single copy gains of \emph{MYC} (focal or broad) is prognostically significant. Additionally, large cell/anaplastic histology has no prognostic value in a multivariate model accounting for \emph{MYC} amplification. Possibly, \emph{MYC} amplification may be a marker for apoptotic resistance, leading to resistance against radiotherapy and chemotherapy. Although MYC promotes cell proliferation, it also normally induces apoptosis; therefore, \emph{MYC} amplification is incompatible with tumour formation except in the context of apoptotic pathway disruption \citeref{pei12}.

Another marker notably absent from our risk-stratification schemes is \gene{TP53} mutation, which is a well-known indicator of poor prognosis \citeref{zhukova13}. Loss of TP53 function abrogates the apoptotic pathway and contributes to resistance against chemotherapy and radiotherapy. While we had some \gene{TP53} mutation data in our cohorts, a substantial proportion of samples were not interrogated for \gene{TP53} status. Additionally, \gene{TP53} mutation appears to be predominately prognostic for long-term survivors \citeref{zhukova13}, and the follow-up lengths in our cohorts were insufficient to evaluate the long-term prognostic impact of \gene{TP53} mutation. Therefore, the utility of \gene{TP53} mutation in multivariate patient risk-stratification should be further assessed in a cohort with more complete data and longer follow-up.

The absence of age in our risk stratification schemes bodes well for the modern practice of restricting the role of radiotherapy in the treatment of young children. Indeed, age groups are not associated with the risk groups defined by our models, and the survival differences among the risk groups cannot be explained by differences in age distributions. Furthermore, age (discretized or otherwise) has no prognostic value in multivariate survival models once such prognostic factors such as metastatic status is included \citeself{shih14}. The univariate significance of age for poor outcome may be explained by the notation that early age at diagnosis is simply a proxy for tumour aggressiveness. While some aggressive tumours arise early, not all early arising tumours are aggressive. Accordingly, age has little independent prognostic value in a multivariate survival model, despite that infants often receive less intensive treatment in contemporary protocols. Our results suggest that elimination of radiotherapy and dose reduction in chemotherapy has not contributed to poorer survival of infants; rather, aggressive tumours that respond poorly to treatment sometimes present in infants.

Above all, our risk stratification models identify patient groups who are promising candidates for de-escalation or elimination of irradiation during treatment. In particular, WNT patients exhibit excellent long term survival; with careful monitoring, these patients may respond well to reduced radiotherapy with postsurgical chemotherapy. For SHH medulloblastoma, the finding that infants in the low-risk group under our model respond favourably to multimodal treatment (with presumably reduced or eliminated radiotherapy) points to the tantalizing possibility that the remaining patients in this low risk group (defined by absence of all unfavourable markers) may similarly respond well to chemotherapy alone. Given that SHH medulloblastoma tends not to recur with metastasis \citeref{ramaswamy13}, localized radiotherapy may be sufficient to prevent recurrence. Among patients with Group4 medulloblastoma, some patients with metastatic disease show excellent survival. Since patients presenting with metastasis are traditionally considered high risk, their apparent favourable outcome in our cohorts begs the question: Did these patients need the intensified radiotherapy for tumour eradication? If their favourable survival is not attributable to intensified treatment, these patients may benefit from radiotherapy de-escalation and survive with improved qualities of life. Encouragingly, recent findings suggest that the dose of craniospinal irradiation might be reduced in high-risk medulloblastoma (metastatic or residual disease) without compromising survival by supplementing the treatment with tandem high dose chemotherapy (and autologous stem cell transplantation) \citeref{sung13}.

To conclude, we demonstrate that medulloblastoma molecular subgroups are highly informative for predicting patient outcome, and we can dramatically improve the accuracy of survival prediction by incorporating molecular subgroup with conventional clinical parameters for patient risk stratification. Moreover, we proposed, tested, and validated novel subgroup-specific risk stratification models that consider both clinical and molecular variables. These models perform robustly and reproducibly both in the discovery cohort consisting of a heterogeneously treated group of patients and in a large, non-overlapping validation cohort of patients treated at a single institution according to a single treatment protocol. Given that we do not have detailed treatment information for patients in these cohorts, it is highly possible that treatment effects (type, duration, or intensity) could impact our results. We would suggest that this can only be accounted through examination of our stratification model in a sufficiently large prospectively followed cohort of patients with medulloblastoma. While the current study uses either \gls{snp} arrays, or interphase \gls{fish} on \gls{ffpe} sections, it is possible that other approaches such as \gls{acgh} could also be used to determine the copy-number status of the six cytogenetic markers. Our results demonstrate the utility of incorporating tumour biology into clinical decision-making and offer a novel perspective on risk stratification using \gls{fish} (applicable on paraffin sections), and thus should be validated in prospective multi-centre trials and be translated into routine clinical practice.


% Material not included:

% The model can be fine-tuned by additional of more covariates. Due to missing data, some covariates may have underappreciated significance due to reduced power, and they did not make the final model. Nonetheless, we know that the model works and is reproducible in an independent cohort.

% In China, WNT Medulloblastoma continues to have superior survival compared to other medulloblastoma subgroups \citeref{zhang14}.

% Large cell histology (as opposed to anaplastic) is unfavourable \citeref{pietsch14}.

% Subtotal resection is prognostically unfavourable \citeref{park83, grill05, rutkowski05, roldan08, rutkowski10, vonbueren11, lannering12, strother14, zeltzer99}. However, gross total resection in cases with brain stem involvement does not yield survival benefit \citeref{gajjar96}.

% Desmoplastic histology overruled extraneural metastasis (M4) \citeref{young15}.

\clearpage
