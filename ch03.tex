\chapter{Clinical prognostication within molecular subgroups of medulloblastoma}
\chaptermark{Clinical prognostication}
\label{ch:clin-prog}

\begin{objective}
To stratify patients into risk groups based on clinical and molecular biomarkers within medulloblastoma subgroups.
\end{objective}


Current medulloblastoma protocols stratify patients based on clinical features: patient age, metastatic stage, extent of resection, and histological variant. The stark prognostic and genetic differences between the subgroups observed in \textbf{Aim II} suggest that subgroup-specific molecular biomarkers could improve patient prognostication.

\begin{SCfigure}[5][t]
	\includegraphics[width=0.25\textwidth]{fig/magic-clin/meta_cyto-markers.pdf}
	\caption[Sample sizes of recent prognostic marker studies]
	{
	Sample sizes of recent prognostic marker studies.
	This meta-analysis was performed by Marc Remke.
	}
	\label{fig:meta_cyto-markers}
\end{SCfigure}


To determine whether subgroup affiliation could support or supplant clinical variables for prognostication in medulloblastoma patients and to determine the effects of subgroup affiliation on cytogenetic biomarkers, we assembled an international discovery cohort of 673 medulloblastomas through MAGIC, for which we had both clinical follow-up and copy number data (Affymetrix SNP 6.0). To begin, I identified subgroup-specific copy-number aberrations (CNAs) and integrated them with clinical variables to develop subgroup-specific risk models based on the discovery cohort. In order to validate the models and ensure that the technique was generalizable to routine pathology laboratories, our colloborators (Andrey Korshunov and Stefan Pfister) then studied a panel of six cytogenetic biomarkers (\gene{GLI2}, \gene{MYC}, 11, 14, 17p, and 17q) using interphase \gls{fish} on an \gls{ffpe} medulloblastoma tissue microarray that includes a set of 453 medulloblastomas that were treated at a single center and does not overlap with the discovery cohort.


The analysis of $> 1000$ medulloblastoma patients clearly demonstrates that subgroup affiliation enhances prognostication with clinical biomarkers, and that the majority of published molecular biomarkers are only relevant in the setting of a single subgroup. The combination of clinical variables, subgroup affiliation, and six cytogenetic markers analyzed on gls{ffpe} tissues can achieve an unprecedented level of prognostic prediction for medulloblastoma patients that is practical, reliable, and reproducible.

\begin{SCfigure}[5][b]
	\includegraphics[width=0.28\textwidth]{fig/magic-clin/surv_mb-subgroups.pdf}
	\caption[Overall survival curves for molecular subgroups of medulloblastoma]
	{
	Overall survival curves for molecular subgroups of medulloblastoma.
	Numbers below x-axis represent patients at risk of event; statistical significances are evaluated by log-rank tests; \gls{hr} estimates are derived from Cox proportional-hazards analyses.
	}
	\label{fig:surv_mb-subgroups}
\end{SCfigure}

In China, WNT Medulloblastoma continues to have superior survival compared to other medulloblastoma subgroups \citeref{zhang14}.

\clearpage

\section{Material and methods}

\section{Results}

\subsection{Prognostic significance of clinical variables within medulloblastoma subgroups}

Many prior medulloblastoma biomarker publications were limited by sample size, a problem that will only be exacerbated once cohorts are divided into their molecular subgroups.  The current study includes 1126 medulloblastoma patients (673 discovery plus 453 validation patients), which is more than double the sample size of any prior medulloblastoma biomarker publication, and one of only a very few that includes a validation cohort (\citefig{meta_cyto-markers}). Although the discovery cohort accumulated by MAGIC consists of medulloblastomas gathered from 43 different treating centers from around the world, the subgroup-specific outcome mirrors what has been previously published with very good outcomes for WNT patients, poor outcomes for Group~3 patients, and intermediate outcomes for SHH and Group~4 patients (\citefig{surv_mb-subgroups}) suggesting that the discovery cohort is a representative sample (appendix table not shown).

\begin{figure}[h]
	\begin{center}
		\includegraphics{fig/magic-clin/surv_ageg_mstat_wnt.pdf}
	\end{center}
	\caption[Ten-year overall survival curves for WNT medulloblastoma]
	{
	Ten-year overall survival curves for WNT medulloblastoma, split by age group or metastatic status.
	Numbers below x-axis represent patients at risk of event; statistical significances are evaluated by log-rank tests; \gls{hr} estimates are derived from Cox proportional-hazards analyses.
	}
	\label{fig:surv_ageg_mstat_wnt}
\end{figure}

In order to assess long-term survivors, WNT patients were followed for up to 10 years, and only two deaths were observed, both late in the follow-up period and due to recurrence of medulloblastoma (\citefig{surv_ageg_mstat_wnt}, appendix table not shown).  Among the SHH tumors, there is a significantly better outcome in the adult patients as compared to children or infants (\citefig{surv_ageg_shh_group3_group4}).  There is a trend towards a worse outcome for infants with Group~3 tumors that is not statistically significant (\citefig{surv_ageg_shh_group3_group4}).  Infants with Group~4 tumors have a significantly worse outcome than children or adults (\citefig{surv_ageg_shh_group3_group4}), suggesting that radiation therapy is critical in the treatment of Group~4 medulloblastoma. There is no reproducible association between gender and prognosis in any of the four subgroups (appendix figure not shown). Desmoplastic histology portends a more favorable prognosis than classic histology, which is more favorable than anaplastic histology among SHH tumors (appendix figure not shown). Large cell/anaplastic histology has prognostic significance for Group~3 medulloblastomas in the discovery cohort, but does not validate as significant in the validation cohort.

While metastatic status is not prognostic for patients with WNT medulloblastoma, macroscopic metastasis (M2/M3) is consistently associated with poor survival in all non-WNT subgroups, though the clinical effect is very slight among patients with Group~4 disease (\citefig{surv_mstat_shh_group3_group4}).  While the prognostic significance of M0 disease as compared to M2/3 disease is very clear across SHH, Group~3, and Group~4, the prognostic significance of isolated M1 disease is less clear (\citefig{surv_mstat_shh_group3_group4}, appendix figure not shown). Isolated M1 disease is associated with increased risk in Group~3 in the discovery cohort, but not the validation cohort, with the opposite pattern seen in the SHH patients. However, for both discovery and validation cohorts, there are no survival differences survival between M0 and M1 patients with Group~4 disease. There are no CNAs in any of the subgroups that are associated with an increased risk of leptomeningeal dissemination (appendix table not shown). Overall, many clinical biomarkers continue to exhibit prognostic significance when medulloblastoma is analyzed in a subgroup-specific fashion (appendix table not shown).

\bigskip

\begin{figure}[ht]
	\begin{center}
		\includegraphics{fig/magic-clin/surv_ageg_shh_group3_group4.pdf}
	\end{center}
	\caption[Overall survival curves for age groups within SHH, Group~3, and Group~4 subgroups]
	{
	Overall survival curves for age groups within SHH, Group~3, and Group~4 subgroups.
	Numbers below x-axis represent patients at risk of event; statistical significances are evaluated by log-rank tests; \gls{hr} estimates are derived from Cox proportional-hazards analyses.
	}
	\label{fig:surv_ageg_shh_group3_group4}
\end{figure}

\begin{figure}[ht]
	\begin{center}
		\includegraphics{fig/magic-clin/surv_mstat_shh_group3_group4.pdf}
	\end{center}
	\caption[Overall survival curves for metastatic status within SHH, Group~3, and Group~4 subgroups]
	{
	Overall survival curves for metastatic status within SHH, Group~3, and Group~4 subgroups.
	Numbers below x-axis represent patients at risk of event; statistical significances are evaluated by log-rank tests; \gls{hr} estimates are derived from Cox proportional-hazards analyses.
	}
	\label{fig:surv_mstat_shh_group3_group4}
\end{figure}

\clearpage

\subsection{Subgroup and metastatic status are the most powerful predictive prognostic biomarkers}

Multivariate survival analyses were conducted in order to dissect the relative predictive value of clinical variables (age, gender, metastatic status, and histotype) and molecular subgroup affiliation. Stepwise Cox proportional-hazards (PH) regressions revealed that molecular subgroup significantly contributes to multivariate survival prediction, on top of a regression model already parameterized by clinical variables: gender, age, metastatic status, and histology (\citefig{subgroup-specific_cox}\emphlab{a}). Further, Cox PH models parameterized with both clinical biomarkers and molecular subgroup achieve higher prediction accuracy in time-dependent \gls{roc} analyses (\citefig{subgroup-specific_cox}\emphlab{b}, appendix figure not shown). In isolation, each biomarker has modest prediction accuracy (\citefig{subgroup-specific_cox}\emphlab{c}), compared to the complete multivariate model (\citefig{subgroup-specific_cox}\emphlab{b}). In the complete model, the removals of metastatic status and subgroup lead to the greatest decreases in predictive accuracy (\citefig{subgroup-specific_cox}\emphlab{d}). Taken together, these results suggest that subgroup affiliation and metastatic status are the most important predictive biomarkers, and that they make non-redundant contributions to the prediction of survival. We conclude that combining both clinical biomarkers (metastatic status) and molecular biomarkers (subgroup affiliation) will make the optimal tool for predicting survival of medulloblastoma patients.

\begin{figure}[h]
	\begin{center}
		\includegraphics{fig/magic-clin/subgroup-specific_cox.pdf}
	\end{center}
	\caption[Molecular subgroup and metastatic status are the most important prognostic biomarkers]
	{
	Molecular subgroup and metastatic status are the most important prognostic biomarkers.
	\textbf{a}, Multivariate Cox proportional-hazards survival analysis of predictor variables. Starting with the null model, each variable is added stepwise (from top to bottom) to the survival model. Model likelihood values assess the degree to which each Cox model fits the survival data. Increments in model likelihoods are tested by analysis of deviance. 
	\textbf{b}, Average areas under time-dependent receiver operating characteristic curves (AUC) for multivariate Cox models parameterized by only clinical variables, or both clinical and subgroup variables.
	\textbf{c}, Average time-dependent AUCs for univariate Cox models parameterized by each variable.
	\textbf{d}, Predictive importance of each variable in the fully-parameterized multivariate Cox models, as determined by the average decrease in time-dependent AUC when the variable is omitted from the model.
	Differences in time-dependent AUC and predictive importance are evaluated by the Friedman rank sum test.
	}
	\label{fig:subgroup-specific_cox}
\end{figure}

\clearpage

\subsection{Subgroup specificity of published molecular biomarkers}

Several cytogenetic biomarkers have been previously reported to be associated with patient survival across medulloblastoma, but their prognostic values have seldom been assessed in the context of medulloblastoma subgroups (appendix table not shown). Monosomy for chromosome 6 is significantly associated with improved survival across medulloblastoma in toto (\citefig{subgroup-specific_eg}\emphlab{a}, appendix table not shown). However, the prognostic value of chr6 loss can be completely attributed to its enrichment in WNT medulloblastomas (\citefig{subgroup-specific_eg}\emphlab{b}, appendix data not shown), as loss of chr6 has no prognostic value among WNT patients, or among non-WNT patients, when compared to their respective controls with balanced chr6.  We would suggest that monosomy 6 is subgroup-driven biomarker in that its prognostic significance is driven by its enrichment in a particular subgroup, and it thus holds no further significance in subgroup-specific analysis.  Further, these results would caution against using monosomy 6 as the lone diagnostic criteria for a WNT medulloblastoma, since it is also observed in non-WNT medulloblastoma (7/49 monosomy 6 medulloblastomas were not WNT (14\%)), and monosomy 6 is only present in 42/53 WNT tumors (79\%).  The prognostic role of isochromosome 17q (iso17q) has been very controversial; in our cohort in toto, iso17q is a statistically significant predictor of poor outcome (\citefig{subgroup-specific_eg}\emphlab{c}).  However, subgroup-specific analysis demonstrates that iso17q is highly prognostic for Group~3 medulloblastoma, but not for Group~4 medulloblastoma (\citefig{subgroup-specific_eg}\emphlab{d}), indicating that it is a subgroup-specific molecular biomarker.  Similarly, while 10q loss is a modestly significant predictor of poor outcome across medulloblastoma subgroups (\citefig{subgroup-specific_eg}\emphlab{e}), its prognostic power is limited to the SHH subgroup of tumors in a subgroup-specific analysis (\citefig{subgroup-specific_eg}\emphlab{f}).  We conclude that determination of molecular subgroup affiliation is crucial in the evaluation and implementation of molecular biomarkers for patients with medulloblastoma (appendix data not shown), as some putative biomarkers are merely enriching for a specific subgroup (subgroup driven) while most others are only significant within a specific subgroup (subgroup specific).

\clearpage

\begin{figure}[h]
	\begin{center}
		\includegraphics[width=\textwidth]{fig/magic-clin/subgroup-specific_eg.pdf}
	\end{center}
	\caption[Subgroup-driven and subgroup-specific molecular biomarkers]
	{
	Subgroup-driven and subgroup-specific molecular biomarkers.
	\textbf{a}, Overall survival curves and frequency distribution of chr6 status across the entire cohort.
	\textbf{b}, Overall survival curves for chr6 status in WNT and non-WNT medulloblastomas.		
	\textbf{c}, Overall survival curves and frequency distribution of isolated chr17q gain across the entire cohort.
	\textbf{d}, Overall survival curves for chr17q status in Group~3 and Group~4 subgroups. 
	\textbf{e}, Overall survival curves for chr10q status across the entire cohort.
	\textbf{f}, Overall survival curves for chr10q status in SHH and non-SHH medulloblastomas.
	Numbers below x-axis represent patients at risk of event; statistical significances are evaluated by log-rank tests; \gls{hr} estimates are derived from Cox proportional-hazards analyses.
	}
	\label{fig:subgroup-specific_eg}
\end{figure}

\clearpage

\subsection{SHH patients can be stratified into three distinct risk groups}

We identified 11 CNAs that are prognostically significant in our SHH medulloblastoma discovery set (\citefig{shh-markers}, appendix figure not shown) in univariate survival analyses. Given the considerable number of candidates, the reproducibility of the identified biomarkers was assessed by cross-validation, and the expected sample sizes required for validation in future prospective trials were estimated to facilitate candidate prioritization (appendix table not shown). Specific amplifications but not broad gains encompassing \gene{GLI2} or \gene{MYCN} are associated with bleak prognosis (\citefig{shh-markers}\emphlab{a--b}, appendix figure not shown). Loss of chr14q confers significantly inferior survival (\citefig{shh-markers}\emphlab{c}). There is no minimal region of deletion on chr14 in SHH patients (appendix figure not shown), and recent medulloblastoma re-sequencing efforts have not identified any recurrent SNVs on chr14 in SHH medulloblastoma . The presence of chromothripsis (chromosome shattering) is associated with worse survival in SHH patients (\citefig{shh-markers}\emphlab{d}).

To integrate the individual biomarkers into a risk stratification model, multivariate Cox PH analyses were performed on all significant prognostic markers. Through multiple stepwise regression procedures, a consensus set of biomarkers was selected for inclusion in the model in an unbiased manner. The proposed risk stratification scheme represents the model that was most consistent with available data in the discovery cohort, from among many possible alternatives (\citefig{shh-risk-strat}\emphlab{a}, appendix data not shown). \gene{GLI2} amplification, 14q loss, and leptomeningeal dissemination (M+ disease) identify high and standard risk patients. Specifically, \gene{GLI2} amplification alone can identify patients with bleak prognosis (\citefig{shh-risk-strat}\emphlab{a}, appendix figure not shown). Absence of these markers demarcates a low-risk group of patients who exhibit survivorship reminiscent of WNT patients. Importantly, none of the covariates, particularly age and anaplastic histology, can explain the survival differences observed among risk groups (\citefig{shh-risk-strat}\emphlab{a}, appendix figures not shown). Direct application of the proposed risk stratification scheme on the independent validation cohort yields distinct survivorships for the three risk groups, thereby validating the model (\citefig{shh-risk-strat}\emphlab{c}).

\begin{figure}[h]
	\begin{center}
		\includegraphics{fig/magic-clin/shh-markers.pdf}
	\end{center}
	\caption[Overall survival curves for molecular biomarkers in SHH medulloblastoma]
	{
	Overall survival curves for molecular biomarkers in SHH medulloblastoma:
	\textbf{a}, \gene{GLI2} copy number status;
	\textbf{b}, \gene{MYCN} copy number status;
	\textbf{c}, chr14q status; and
	\textbf{d}, chromothripsis status.
	Numbers below x-axis represent patients at risk of event; statistical significances are evaluated by log-rank tests; \gls{hr} estimates are derived from Cox proportional-hazards analyses.
	}
	\label{fig:shh-markers}
\end{figure}

Two additional stratification schemes were constructed using only clinical biomarkers or only cytogenetic markers; however, the proposed model, which combines both types of biomarkers, yields the highest prediction accuracy (\citefig{shh-risk-strat}\emphlab{b}, appendix figure not shown). Furthermore, the accuracy of the combined risk model is drastically reduced when applied across non-SHH patients, further underscoring the importance of taking subgroup into consideration during risk stratification. We conclude that by using two molecular biomarkers (\gene{GLI2} and 14q \gls{fish}) and metastatic status, we can practically and reliably predict prognosis for patients with SHH medulloblastoma.

\clearpage

\begin{figure}[h]
	\begin{center}
		\includegraphics[width=\textwidth]{fig/magic-clin/shh-risk-strat.pdf}
	\end{center}
	\caption[Combined clinical and molecular biomarkers improve risk-stratification of SHH patients]
	{
	Combined clinical and molecular biomarkers improve risk-stratification of SHH patients.
	\textbf{a}, Risk stratification of SHH medulloblastomas by molecular and clinical prognostic markers. \emph{Top-left}, decision tree; \emph{bottom-left}, events plot depicting status of molecular and clinical markers across the risk groups; \emph{right}, overall survival curves for SHH risk groups.
	\textbf{b}, Average time-dependent AUCs for risk groups stratified using only clinical or molecular markers, or both. Risk stratification regimens are applied to SHH and non-SHH medulloblastomas. ***, $p < 0.001$, Friedman rank sum tests.
	\textbf{c}, Survival curves for SHH risk groups in the validation cohort.
	Numbers below x-axis represent patients at risk of event; statistical significances are evaluated by log-rank tests; \gls{hr} estimates are derived from Cox proportional-hazards analyses.
	}
	\label{fig:shh-risk-strat}
\end{figure}

\clearpage

\subsection{Metastatic status, iso17q, and \emph{MYC} amplification demarcate high-risk Group~3 patients}

In Group~3 patients, iso17q and \emph{MYC} amplification remain the only cytogenetic markers associated with poor survival, whereas chr8q loss and chr1q gain are the only good prognosis markers (\citefig{group3-markers}, appendix data not shown). In multivariate survival analyses, patients with metastasis, iso17q, or MYC amplification represent the high-risk group (\citefig{group3-risk-strat}\emphlab{a}). Critically, absence of these markers can identify a population of Group~3 patients who have a survivorship much longer than Group~3 taken as a whole. The risk groups are not associated with any clinical covariates, including age (\citefig{group3-risk-strat}\emphlab{a}, appendix figures not shown). Consistent with the findings in SHH patients, optimal risk stratification in Group~3 patients requires the use of both clinical and molecular prognostic markers, which have reduced or no prognostic value outside of Group~3 (\citefig{group3-risk-strat}\emphlab{b}, appendix figure not shown). Our proposed risk stratification scheme was validated on the non-overlapping validation cohort using three molecular biomarkers (\emph{MYC}, 17p, and 17q \gls{fish}) and metastatic status (\citefig{group3-risk-strat}\emphlab{c}).

\bigskip

\begin{figure}[h]
	\begin{center}
		\includegraphics[width=\textwidth]{fig/magic-clin/group3-markers.pdf}
	\end{center}
	\caption[Overall survival curves for molecular biomarkers in Group~3 medulloblastoma]
	{
	Overall survival curves for molecular biomarkers in Group~3 medulloblastoma:
	\textbf{a}, chr17 copy number aberrations;
	\textbf{b}, \emph{MYC} copy number status; and 
	\textbf{c}, chr8q status.
	\textbf{d}, Risk stratification of Group~3 medulloblastomas by molecular and clinical prognostic markers.
	Numbers below x-axis represent patients at risk of event; statistical significances are evaluated by log-rank tests; \gls{hr} estimates are derived from Cox proportional-hazards analyses.
	}
	\label{fig:group3-markers}
\end{figure}

\clearpage

\begin{figure}[h]
	\begin{center}
		\includegraphics{fig/magic-clin/group3-risk-strat.pdf}
	\end{center}
	\caption[Combined clinical and molecular biomarkers improve risk-stratification of Group~3 patients.]
	{
	Combined clinical and molecular biomarkers improve risk-stratification of Group~3 patients.
	\textbf{a}, Risk stratification of Group~3 medulloblastomas by molecular and clinical prognostic markers.	\emph{Top-left}, decision tree; \emph{bottom-left}, events plot depicting status of molecular and clinical markers across the risk groups; \emph{right}, overall survival curves for Group~3 risk groups.
	\textbf{b}, Average time-dependent AUCs for risk groups stratified using only clinical or molecular markers, or both. Risk stratification regimens are applied to Group~3 and non-Group~3 medulloblastomas. ***, $p < 0.001$, Friedman rank sum tests.
	\textbf{c}, Survival curves for Group~3 risk groups in the validation cohort.
	Numbers below x-axis represent patients at risk of event; statistical significances are evaluated by log-rank tests; \gls{hr} estimates are derived from Cox proportional-hazards analyses.
	}
	\label{fig:group3-risk-strat}
\end{figure}

\clearpage


\subsection{Identification of a low-risk group of metastatic Group~4 patients}

Group~4 patients with whole chromosome loss of chr11 or gain of chr17 exhibit better survival under univariate Cox PH models (\citefig{group4-markers}\emphlab{a}), in addition to chr10p loss (\citefig{group4-markers}\emphlab{b}). There is no cytogenetic marker associated with poor prognosis (appendix data not shown). Specifically, neither \gene{MYCN} gain nor amplification is associated with poorer survival in Group~4, in stark contrast to SHH patients, reinforcing the distinction in their underlying biology (\citefig{group4-markers}\emphlab{b}, appendix figure not shown). Similarly, none of the cytogenetic biomarkers identified for Group~3 patients (e.g. iso17q) have any prognostic value in Group~4 (appendix table not shown). Following unbiased model selection, the consensus set of biomarkers results in a risk stratification scheme in which leptomeningeal dissemination identifies high-risk Group~4 patients, except in the context of chr11 loss or chr17 gain (\citefig{group4-risk-strat}\emphlab{a}). The biology underlying chr11 loss is not apparent as there is no obvious minimal common region of deletion (appendix figure not shown), nor are there any recurrent SNVs on chr11 reported in the recent medulloblastoma re-sequencing publications. Group~4 patients with either chr17 gain or chr11 loss, irrespective of their metastatic statuses exhibit survivorship that is characteristic of WNT patients in both the discovery and validation cohorts (\citefig{group4-risk-strat}\emphlab{a},\emphlab{c}), and the survival differences are not explainable by covariates (appendix figure not shown). Significantly, the low-risk Group~4 cohort also included some patients with anaplastic histology. Consistent with other subgroups, the risk stratification model using both clinical and molecular biomarkers achieve the highest accuracy (\citefig{group4-risk-strat}\emphlab{b}). Critically, the cytogenetic biomarkers identify low-risk Group~4 patients whom would be otherwise designated as high-risk by evidence of metastasis and/or anaplastic histology; this finding cannot be extrapolated to SHH and Group~3 patients (\citefig{group4-risk-strat}, appendix figure not shown).  We conclude that through the use of three molecular biomarkers (chr11, 17p, and 17q \gls{fish}) and metastatic status, we can accurately and reliably predict the prognosis of patients with Group~4 medulloblastoma.

\bigskip

\begin{figure}[h]
	\begin{center}
		\includegraphics{fig/magic-clin/group4-markers.pdf}
	\end{center}
	\caption[Overall survival curves for molecular biomarkers in Group~4 medulloblastoma]
	{
	Overall survival curves for molecular biomarkers in Group~4 medulloblastoma:
	\textbf{a}, whole chr11 status and whole chr17 status; and
	\textbf{b}, \gene{MYCN} copy number status.
	Numbers below x-axis represent patients at risk of event; statistical significances are evaluated by log-rank tests; \gls{hr} estimates are derived from Cox proportional-hazards analyses.
	}
	\label{fig:group4-markers}
\end{figure}

\clearpage

\begin{figure}[h]
	\begin{center}
		\includegraphics[width=\textwidth]{fig/magic-clin/group4-risk-strat.pdf}
	\end{center}
	\caption[Combined clinical and molecular biomarkers improve risk-stratification of Group~4 patients]
	{Combined clinical and molecular biomarkers improve risk-stratification of Group~4 patients.
	\textbf{a}, Risk stratification of Group~4 medulloblastomas by molecular and clinical prognostic markers. \emph{Top-left}, decision tree; \emph{bottom-left}, events plot depicting status of molecular and clinical markers across the risk groups; \emph{right}, overall survival curves for Group~4 risk groups.
	\textbf{b}, Average time-dependent AUCs for risk groups stratified using only clinical or molecular markers, or both. Risk stratification regimens are applied to Group~4 and non-Group~4 medulloblastomas. ***, $p < 0.001$, Friedman rank sum tests.
	\textbf{c}, Survival curves for Group~4 risk groups in the validation cohort. 
	Numbers below x-axis represent patients at risk of event; statistical significances are evaluated by log-rank tests; \gls{hr} estimates are derived from Cox proportional-hazards analyses.
	}
	\label{fig:group4-risk-strat}
\end{figure}

\clearpage


To conclude, we demonstrate that medulloblastoma subgroup affiliation is significantly more informative for predicting patient outcome than existing clinical variables, and that by incorporating subgroup status with conventional clinical parameters for patient risk stratification, the accuracy of survival prediction can be dramatically improved.  Moreover, we propose, test, and validate novel subgroup-specific risk stratification models that incorporate both clinical and molecular variables.  These models perform robustly and reproducibly both in the discovery cohort consisting of a heterogeneously treated group of patients and in a large non-overlapping validation cohort of patients treated at a single institution according to a single treatment protocol.  We do not have detailed treatment information for patients in these cohorts.  It is highly possible that treatment effects (type, duration, or intensity) could impact our results.  We would suggest that this can only be accounted through examination of our stratification model in a sufficiently large prospectively followed cohort of medulloblastoma patients.  While the current study uses either SNP arrays, or interphase \gls{fish} on gls{ffpe} sections, it is possible that other approaches such as array CGH could also be used to determine the copy number status of the six markers.  Our findings demonstrate the utility of incorporating tumor biology into clinical decision-making and offer a novel perspective on risk stratification using \gls{fish} applicable on paraffin sections, and thus could be translated immediately into routine clinical practice.

\clearpage



\clearpage


\section{Discussion}

Age did not appear to be prognostically important, contrary to earlier studies \citeself{northcott11}.
Why wasn't p53 included in the model? Compare against early study \citeself{zhukova13}.
