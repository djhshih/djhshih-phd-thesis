\chapter{Clinical prognostication within molecular subgroups of medulloblastoma}
\chaptermark{Clinical prognostication}
\label{ch:clin-prog}

\begin{objective}
To stratify patients into risk groups based on clinical and molecular biomarkers within medulloblastoma subgroups.
\end{objective}

Medulloblastoma was a uniformly fatal disease with a survival duration of mere months until the introduction of systematic irradiation of the entire central nervous system in the 1940s \citeref{paterson53}. Prior to the adoption of craniospinal (whole brain and spine) irradiation, medulloblastoma cases treated surgical resection and localized radiotherapy recur with metastases in the subarachnoid space \citeref{mcfarland69}. Although the propensity of medulloblastoma to metastasize necessitated whole \gls{cns} radiotherapy, exposing the entire brain and spine to irradiation led to long-term neuropsychological sequelae that were beginning to be documented in the 1960s \citeref{mcfarland69, gudrunardottir14}. The integration of chemotherapy in the 1970s into the standard treatment of medulloblastoma led to a concomittant rise in patient survival \citeref{gudrunardottir14}. Chemotherapeutic drugs, however, can also have immediate or long-term adverse effect on neurocognitive function \citeref{khong06, zeller13, avan15}. Today, patients with medulloblastoma are treated by surgical resection, followed by craniospinal irradiation and combination chemotherapy. While advances in imaging and surgical technologies have largely eliminated operative mortality and minimized damage to the brain during resection, craniospinal irradiation and combination chemotherapy continue to cause debilitating neurocognitive decline of long-term survivors \citeref{palmer13, schreiber14, knight14, gudrunardottir14}. With modern treatment, patients with medulloblastoma can be cured, but often at great cost to their quality of life.

Radiotherapy causes apoptosis (programmed cell death) of divinging tumour cells, but it can also cause normal dividing cells to die leading to physical, endocrinologic, and neurologic sequalae. In a developing brain, dividing neural progenitors are sensitive to irradation. Additionally, quiescent neural progenitors or stem cells can also incur radiation-induced DNA damage whose effect may manifest later in life. Therefore, radiotherapy is avoided in young children to spare the maturing brain of the neurotoxic side-effect of treatment.

Attempts to delay or elminate radiotherapy by using chemotherapy to control the tumour. Often chemotherapy cannot achieve complete response of the residual tumour, leading to eventual use of radiotherapy. Others have been successful in using high dose combination therapy to elminate the need for irradiation.

Chemotherapy itself, however, also have some degree of neurotxocity. In patients with acute lymphoplastic leukemia, chemotherapy alone can cause reduction in volumes of several neuroanatomic structures of the brain and consequent decline in neurocognitive function \citeref{zeller13}.

Due to the awareness of neurologic sequalae following radiotherapy, earlier trials were aimed at postponing or eliminating radiotherapy in young children, who are particularly sensitive to the neurotoxic side-effect of irradiation \citeref{mulhern89, rutkowski05, jeng93, duffner93, geyer94, gentet95, duffner99, walter99, oyharcabal-bourden05, grill05, geyer05, dhall08, sands10, grundy10, vonbueren11, saha14, yasuda08, kellie02, white98, strauss91}. Indeed, a recently completed prospective trial assessing 54 Gy conformal (targeted against the tumour bed) radiotherapy in low-grade glioma patients revealed a striking correlation between age at treatment and subsequent decline in \gls{iq} score: the younger the survivor was during conformal radiotherapy, the more severe was the decline in intelligence \citeref{merchant09}. Similarly, younger children with medulloblastoma treated with high-dose irradiation had worse progressive decline in intellectual outcome and academic performance compared to children older in age at diagnosis \citeref{radcliffe94, palmer13, schreiber14, knight14, mulhern98}. Even with a reduced dose of craniospinal radiotherapy, survivors continue to show progressive decline in intellectual and academic outcomes \citeref{ris13, mulhern05, ris01, mulhern98}.

In acute lymphoblastic leukemia patients, cranial radiation causes decline in intelligence, and this decline is progressive, showing more impairment of cognitive function with increasing time since radiation therapy \citeref{krull13}.

Cranial irradiation is now reserved for the fewer than 20\% of children who are considered at high risk for \gls{cns} relapse \citeref{pui08}.

chemotherapy alone (with vinblastine, cisplatin, and etoposide) is usually insufficient for complete remission of residual medulloblastoma, and patients often progress during chemotherapy. \citeref{jeng93, gajjar94, walter99}. Radiotherapy on chemotherapy-resistance young patients can salvage the patient and lead to long-term survival; however, survivors frequently suffer from neurodevelopmental deficits \citeref{gajjar94, grill05}.

treatment with high-dose combination chemotherapy alone (including vincristine, carboplatin, etoposide, cyclophosphamide, and methotrexate) can be sufficient for treating non-metastatic medulloblastoma, though decline in \gls{iq} is still evident, though less severe than those who had received radiotherapy \citeref{rutkowski05}.

Similar attempt at using combination chemotherapy (carboplatin, procarbazine, etoposide, cisplatin, vincristine, cyclophosphamide) to obviate need for radiotherapy \citeref{grill05}. Salvage with radiation and high-dose chemotherapy (busulfan, thiotepa, and melphalan). Survivors similarly have improved intellectual outcome compared to radiotherapy-treated patients \citeref{grill05}.

Combination therapy (vincristine, cisplatin, cyclophosphamide, and etoposide; or vincristine, carboplatin, ifosfamide, etoposide) obviated need for radiotherapy in some patients \citeref{geyer05}.

Yet another combination chemotherapy (cyclophosphamide, vincristine, methotrexate, carboplatin, and etoposide) trial showed promising survivorship without radiotherapy in non-metastatic medulloblastoma \citeref{vonbueren11}. Salvage with radiation, second surgery, and consolidation chemotherapy (with cisplatin, lomustine, and vincristine).


treatment with vincristine, cisplatin, etoposide, cyclophosphamide, methotrexate, and thiotepa yielded complete response in some patients without need for radiotherapy, but this protocol required autologous stem cell transplant following the consolidation chemotherapy due to the chemotherapeutic intensity \citeref{chi04}.

chemotherapy increases the risk of infection due to suppression of immune system

dose intensive chemotherapy can also lead to high toxic mortality (death due to chemotherapy) \citeref{dhall08, strother14}.

Although earlier attempt at reducing craniospinal irradiation led to poorer survival \citeref{thomas00, deutsch96, bailey95, bouffet92}, other attempts at reducing craniospinal irradiation did not compromise progression-free survival by inclusion of chemotherapy \citeref{packer06, packer94, packer99, kim10, halberg91}.

hearing loss due to cisplatin or carboplatin treatment \citeref{avan15}

lower dose of cisplatin did not compromise survival among average-risk medulloblastoma patients \citeref{nageswararao14}.

Despite known risk for long-term neurotoxicity in children with cranial irradiation, such treatment as increase progression-free survival \citeref{ashley12}.

Addition of chemotherapy to craniospinal irradiation caused long-term survivors of medulloblastoma to exhibit decrease in health status \citeref{bull07}. Chemotherapy is not without long term consequences.

reason for risk stratification, deiver intensified treatment only when needed, reduce effects, and maintain quality of life

Aside from imparing brain development, chemotherapy and radiation can cause endocrinological complications resulting in growth hormone deficiency and stunted growth, neurological complications including limb weakness, reduced sense of touch, prolonged pained, problem with balance, permanent hearing loss, seizures, tremors, and paralysis, and predispose the patient to second malignancies \citeref{armstrong09, xu03, edelstein11, packer13, christopherson14, avan15, boman09, duffner98}.

Adult survivors of childhood cancers, compared to unaffected siblings, are 5 times more likely to suffer from functional impairment, 2 times more likely to earn less than \$20~000 in annual household income \citeref{hudson03}. (Most participants of this 2003 study were based in the United States, in which the median household income is more than \$50~000 during the same period \citeref{denavas14}.) Strikingly, survivors of chilhood \cls{cns} cancers are 18 times more likely to suffer from functional impairment, which encompasses any condition that results in requiring help for personal care or daily routines or that precludes holding long-term employment or attending school \citeref{hudson03}.

Moreover, 70\% of survivors diagnosed with \cls{cns} cancer before age 6 years require special education services to cope with learning or emotional difficulties \citeref{mitby03}. Survivors' use of special education is directly related to the treatment received: cranial irradiation treatment alone increases the likelihood of requiring special education by 7 times, while methotrexate treatment alone increases this likelihood by 1.3 times, compared to unaffected siblings \citeref{mitby03}. Given that methotrexate crosses the blood-brain barrier effectively, the use of this chemotherapeutic agent is usually restricted to high-risk cases.

Differences between subjects

current risk stratification of patients

Current medulloblastoma protocols stratify patients based on clinical features: patient age, metastatic stage, extent of resection, and histological variant. The stark prognostic and genetic differences between the subgroups observed in \textbf{Aim II} suggest that subgroup-specific molecular biomarkers could improve patient prognostication.

age < 3, gross total resection, anaplastic histology, presentation with metastasis \citeref{polkinghorn07}
subtotal resction and metastasis classified as high-risk \citeref{gajjar06}.

scheme different aross countries

some groups use different cutoffs: infant as age < 4 \citeref{white98, kool12}

Japan does not use age cutoff for irradiation \citeref{yasuda08}.
No irradiation for age < 2 \citeref{pezzotta96}.


previous studies, may disagreements
small sample size, different cohorts
different results may be attributable to different composition of subgroups

previously known: group3 patients have survival
WNT patients have excellent survival

regression to the mean?
No Group3 patient survival past 5 years
But current cohort have Group3 patietns with longer survival
follow WNT patients longer, how do they recur?

Metastasis is prognostically unfavourable \citeref{grill05, rutkowski05, salama06, rutkowski10, kool12, pietsch14, vonhoff09, bouffet94, zeltzer99}.

"Young age and M2/3 stage were negative prognostic factors in medulloblastoma, but residual or M1 disease was not" \citeref{kortmann00}. prospective trial
in another prospective trail, both M1 and M2/3 were prognostically unfavourable \citeref{zeltzer99}.

Demosplastic is prognostically favourable in many studies \citeref{chatty71, rutkowski05, kool12, pietsch14, dhall08, vonbueren11, ashley12, grundy10}, unfavourable in some \citeref{park83, gajjar06, rutkowski10}, and not prognostically significant in others \citeref{pietsch14, lannering12}.

Does c-myc amplification overrule anaplastic histology \citeref{vonhoff10, raabe10}?

Large cell/anaplastic histologis is prognostically unfavourable \citeref{gajjar04, jakacki12, kool12}.
Anaplastic histology is not prognostic, but \emph{MYC} amplification and large cell histology are \citeref{vonhoff10}. Small sample size ($n = 28$). Large cell histology is unfavourable \citeref{pietsch14}.

Subtotal resection is prognostically unfavourable \citeref{park83, grill05, rutkowski05, roldan08, rutkowski10, vonbueren11, lannering12, strother14, zeltzer99}. However, gross total resection in cases with brain stem involvement does not yield survival benefit \citeref{gajjar96}.

"Restriction of the role of radiotherapy in the management of brain tumours, especially in young children, has been the main aim of contemporary protocols in this age-group" \citeref{grill05}.

"Therefore, children with metastatic disease cannot be cured with the strategy described here, and should probably be treated with more aggressive regimens" \citeref{grill05}.

\gene{TP53} mutation is prognostically unfavourable \citeref{zhukova13}.

Results in a prospective trial in reducing radiotherapy for standard-risk medulloblastoma (no metastasis and near total resection) was discouraging: trial was closed early due to increased in medullobalstoma \citeref{deutsch96}.
Therefore, need better way of identifying low-risk patients, and need plan for salvaging relapsing patients to maintain high survival.


\begin{SCfigure}[5][t]
	\includegraphics[width=0.25\textwidth]{fig/magic-clin/meta_cyto-markers.pdf}
	\caption[Sample sizes of recent prognostic marker studies]
	{
	Sample sizes of recent prognostic marker studies.
	This meta-analysis was performed by Marc Remke.
	}
	\label{fig:meta_cyto-markers}
\end{SCfigure}

overall approach
discovery cohort: heterogeneously ttreated, no pathology review, retrospective
short comings and limitations of dataset
nevertheless, study reflects typical clinical experience more closely
large and diverse sample, validation in homogeneous cohort

To determine whether subgroup affiliation could support or supplant clinical variables for prognostication in medulloblastoma patients and to determine the effects of subgroup affiliation on cytogenetic biomarkers, we assembled an international discovery cohort of 673 medulloblastomas through MAGIC, for which we had both clinical follow-up and copy number data (Affymetrix SNP 6.0). To begin, I identified subgroup-specific copy-number aberrations (CNAs) and integrated them with clinical variables to develop subgroup-specific risk models based on the discovery cohort. In order to validate the models and ensure that the technique was generalizable to routine pathology laboratories, our colloborators (Andrey Korshunov and Stefan Pfister) then studied a panel of six cytogenetic biomarkers (\gene{GLI2}, \gene{MYC}, 11, 14, 17p, and 17q) using interphase \gls{fish} on an \gls{ffpe} medulloblastoma tissue microarray that includes a set of 453 medulloblastomas that were treated at a single center and does not overlap with the discovery cohort.

The analysis of $> 1000$ medulloblastoma patients clearly demonstrates that subgroup affiliation enhances prognostication with clinical biomarkers, and that the majority of published molecular biomarkers are only relevant in the setting of a single subgroup. The combination of clinical variables, subgroup affiliation, and six cytogenetic markers analyzed on gls{ffpe} tissues can achieve an unprecedented level of prognostic prediction for medulloblastoma patients that is practical, reliable, and reproducible.

\begin{SCfigure}[5][b]
	\includegraphics[width=0.28\textwidth]{fig/magic-clin/surv_mb-subgroups.pdf}
	\caption[Overall survival curves for molecular subgroups of medulloblastoma]
	{
	Overall survival curves for molecular subgroups of medulloblastoma.
	Numbers below x-axis represent patients at risk of event; statistical significances are evaluated by log-rank tests; \gls{hr} estimates are derived from Cox proportional-hazards analyses.
	}
	\label{fig:surv_mb-subgroups}
\end{SCfigure}

In China, WNT Medulloblastoma continues to have superior survival compared to other medulloblastoma subgroups \citeref{zhang14}.

\clearpage

\section{Material and methods}

\subsection{Patient information}

All tissues and clinicopathological information were serially collected in accordance with institutional review boards from various contributing centers. In the discovery set, although precise treatment dates were often unavailable, at least 95\% of the patients were treated within the past 15 years using modern treatment protocols, including surgical resection, craniospinal (whole brain and spine) irradiation, and/or chemotherapy. Discovery set samples were collected between 2005 and 2013, with a focus on samples with available fresh-frozen material. Among the samples with treatment details, the earliest diagnosis is July 1997 and the latest is August 2012. Samples in the validation set were all obtained from the Burdenko institute with no selection criterion applied. All patients in the validation set were treated between 1995 and 2010 according to standardized therapy protocols of the German HIT study group.

\subsection{Tumor material and patient characteristics}

A discovery set of 673 medulloblastoma samples with clinical follow-up was acquired retrospectively from 43 cities around the globe. These samples were copy-number profiled on the Affymetrix SNP6 array platform, in order to identify potential molecular biomarkers \citeself{shih12}. An independent validation set of 453 samples with clinical follow-up on a medulloblastoma TMA was analyzed by FISH as previously described \citeself{northcott11a}. The validation set consisted only of patients treated in Burdenko, Moscow. Tumors were classified based on signature marker expression into molecular subgroups as previously described \citeself{northcott12}; additional tumors were classified based on cytogenetic aberrations using standard conditional probability models. Subgroup affiliation was not available for 162 discovery samples. The validation set includes an additional set of 50 WNT tumours that were not on the TMA. Details on clinical data are listed in Appendix Table A1-2. The availability of clinical and cytogenetic data is shown in Appendix Fig 20. Nucleic acid isolation, TMA construction, and $\beta$-catenin mutation analysis were performed as previously described \citeself{shih12}. 

\subsection{Prognostic biomarker identification}

Cytogenetic events and copy-number aberrations were identified as previously described in the discovery set \citeself{shih12}. Subsequent to biomarker discovery, cross-validation was performed to estimate the reproducibility of the potential biomarkers in an independent cohort, with multiple hypothesis correction. Additionally, sample size estimates for prospective trials of each biomarker were calculated under univariate Cox models based on the observed hazard ratios.

During the identification of cytogenetic events and copy-number aberrations in the discovery set, all chromosomal events (or chromosome arm events) were compared against reference samples with balanced copy-number for the chromosome (or chromosome arm); samples with copy-number changes in the opposite direction were specifically excluded from each comparison. Subsequent to biomarker discovery, cross-validation was performed to estimate the reproducibility and generalizability of the potential biomarkers in an independent cohort. During cross-validation, the discovery set was split randomly into two subsets. First, the biomarkers are tested by the log-rank test on the first subset. Then, statistically significant biomarkers ($p < 0.05$) are tested again by the log-rank test on the second subset, with correction for multiple hypotheses testing. This process was repeated 10~000 times to estimate the expected validation rate of each biomarker. The expected validation rate of each biomarker is $n_v / n_d$, where $n_d$ is the number of times a biomarker is significant in the first subset and $n_v$ is the number of times a discovered biomarker is also significant in the second subset. The final set of biomarkers was further validated in the external validation set.

\subsection{Multiple hypothesis testing correction}

Within each biomarker identification analysis, correction for multiple hypothesis testing was performed by the Benjamini-Hochberg method during the cross-validation procedure. Independent analyses were corrected for multiple hypothesese testing independently: clinical biomarker identification across medulloblastoma, within WNT medulloblastoma, within SHH medulloblastoma, within Group3, and within Group4; molecular biomarker identification across medulloblastoma, within WNT medulloblastoma, within SHH medulloblastoma, within Group3 medulloblastoma, and within Group4 medulloblastoma.

\subsection{Statistical analysis}

The patient survival characteristics were right-censored at 5 years (or 10 years) and analyzed by the Kaplan-Meier method. Univariate comparison of two or more survival curves were performed using log-rank tests and the Cox proportional-hazards regression models. The predictive values of biomarkers were assessed by analyses of deviance tests under multivariate Cox models and by time-dependent \gls{roc} analyses. Associations between covariates and risk groups were tested by the Fisher’s exact test. All statistical analyses were performed in the R software environment (v2.15), using R packages survival (v2.36), risksetROC (v1.0.4), powerSurvEpi (v0.0.6), and ggplot2 (v0.9.3).

\subsection{Time-dependent ROC analysis}

Time-dependent \gls{roc} analyses were performed using the CoxWeights function provided in the risksetROC (v1.0.4) R package. This function calculates areas under time-dependent ROC curves as described by Heagerty and Zheng \citeref{heagerty05}. \gls{auc} estimates of the fitted multivariate Cox models being assessed were calculated every month, from 1 month to 60 months, in order to determine the collective predictive performance of the biomarkers in the Cox models. Differences in \gls{auc} estimates among Cox models across time points were tested by Friedman rank sum tests.

\subsection{Risk stratification model selection}

Biomarkers identified in univariate survival analyses were tested by multivariate Cox proportional-hazards models. All discovered biomarkers were tested for inclusion in the risk stratification model by multiple unbiased procedures: stepwise regression using forward selection, backward elimination and bidirectional elimination with the \gls{aic}, as well as analyses of deviance tests.


\section{Results}

\subsection{Prognostic significance of clinical variables within medulloblastoma subgroups}

Many prior medulloblastoma biomarker publications were limited by sample size, a problem that will only be exacerbated once cohorts are divided into their molecular subgroups.  The current study includes 1126 medulloblastoma patients (673 discovery plus 453 validation patients), which is more than double the sample size of any prior medulloblastoma biomarker publication, and one of only a very few that includes a validation cohort (\citefig{meta_cyto-markers}). Although the discovery cohort accumulated by MAGIC consists of medulloblastomas gathered from 43 different treating centers from around the world, the subgroup-specific outcome mirrors what has been previously published with very good outcomes for WNT patients, poor outcomes for Group3 patients, and intermediate outcomes for SHH and Group4 patients (\citefig{surv_mb-subgroups}) suggesting that the discovery cohort is a representative sample (appendix table not shown).

\begin{figure}[h]
	\begin{center}
		\includegraphics[width=\textwidth]{fig/magic-clin/surv_ageg_mstat_wnt.pdf}
	\end{center}
	\caption[Ten-year overall survival curves for WNT medulloblastoma]
	{
	Ten-year overall survival curves for WNT medulloblastoma, split by age group or metastatic status.
	Numbers below x-axis represent patients at risk of event; statistical significances are evaluated by log-rank tests; \gls{hr} estimates are derived from Cox proportional-hazards analyses.
	}
	\label{fig:surv_ageg_mstat_wnt}
\end{figure}

In order to assess long-term survivors, WNT patients were followed for up to 10 years, and only two deaths were observed, both late in the follow-up period and due to recurrence of medulloblastoma (\citefig{surv_ageg_mstat_wnt}, appendix table not shown).  Among the SHH tumors, there is a significantly better outcome in the adult patients as compared to children or infants (\citefig{surv_ageg_shh_group3_group4}).  There is a trend towards a worse outcome for infants with Group3 tumors that is not statistically significant (\citefig{surv_ageg_shh_group3_group4}).  Infants with Group4 tumors have a significantly worse outcome than children or adults (\citefig{surv_ageg_shh_group3_group4}), suggesting that radiation therapy is critical in the treatment of Group4 medulloblastoma. There is no reproducible association between gender and prognosis in any of the four subgroups (appendix figure not shown). Desmoplastic histology portends a more favorable prognosis than classic histology, which is more favorable than anaplastic histology among SHH tumors (appendix figure not shown). Large cell/anaplastic histology has prognostic significance for Group3 medulloblastomas in the discovery cohort, but does not validate as significant in the validation cohort.

While metastatic status is not prognostic for patients with WNT medulloblastoma, macroscopic metastasis (M2/M3) is consistently associated with poor survival in all non-WNT subgroups, though the clinical effect is very slight among patients with Group4 disease (\citefig{surv_mstat_shh_group3_group4}).  While the prognostic significance of M0 disease as compared to M2/3 disease is very clear across SHH, Group3, and Group4, the prognostic significance of isolated M1 disease is less clear (\citefig{surv_mstat_shh_group3_group4}, appendix figure not shown). Isolated M1 disease is associated with increased risk in Group3 in the discovery cohort, but not the validation cohort, with the opposite pattern seen in the SHH patients. However, for both discovery and validation cohorts, there are no survival differences survival between M0 and M1 patients with Group4 disease. There are no CNAs in any of the subgroups that are associated with an increased risk of leptomeningeal dissemination (appendix table not shown). Overall, many clinical biomarkers continue to exhibit prognostic significance when medulloblastoma is analyzed in a subgroup-specific fashion (appendix table not shown).

\bigskip

\begin{figure}[ht]
	\begin{center}
		\includegraphics[width=\textwidth]{fig/magic-clin/surv_ageg_shh_group3_group4.pdf}
	\end{center}
	\caption[Overall survival curves for age groups within SHH, Group3, and Group4 subgroups]
	{
	Overall survival curves for age groups within SHH, Group3, and Group4 subgroups.
	Numbers below x-axis represent patients at risk of event; statistical significances are evaluated by log-rank tests; \gls{hr} estimates are derived from Cox proportional-hazards analyses.
	}
	\label{fig:surv_ageg_shh_group3_group4}
\end{figure}

\begin{figure}[ht]
	\begin{center}
		\includegraphics[width=\textwidth]{fig/magic-clin/surv_mstat_shh_group3_group4.pdf}
	\end{center}
	\caption[Overall survival curves for metastatic status within SHH, Group3, and Group4 subgroups]
	{
	Overall survival curves for metastatic status within SHH, Group3, and Group4 subgroups.
	Numbers below x-axis represent patients at risk of event; statistical significances are evaluated by log-rank tests; \gls{hr} estimates are derived from Cox proportional-hazards analyses.
	}
	\label{fig:surv_mstat_shh_group3_group4}
\end{figure}

\clearpage

\subsection{Subgroup and metastatic status are the most predictive biomarkers}

Multivariate survival analyses were conducted in order to dissect the relative predictive value of clinical variables (age, gender, metastatic status, and histotype) and molecular subgroup affiliation. Stepwise Cox proportional-hazards (PH) regressions revealed that molecular subgroup significantly contributes to multivariate survival prediction, on top of a regression model already parameterized by clinical variables: gender, age, metastatic status, and histology (\citefig{subgroup-specific_cox}\emphlab{a}). Further, Cox PH models parameterized with both clinical biomarkers and molecular subgroup achieve higher prediction accuracy in time-dependent \gls{roc} analyses (\citefig{subgroup-specific_cox}\emphlab{b}, appendix figure not shown). In isolation, each biomarker has modest prediction accuracy (\citefig{subgroup-specific_cox}\emphlab{c}), compared to the complete multivariate model (\citefig{subgroup-specific_cox}\emphlab{b}). In the complete model, the removals of metastatic status and subgroup lead to the greatest decreases in predictive accuracy (\citefig{subgroup-specific_cox}\emphlab{d}). Taken together, these results suggest that subgroup affiliation and metastatic status are the most important predictive biomarkers, and that they make non-redundant contributions to the prediction of survival. We conclude that combining both clinical biomarkers (metastatic status) and molecular biomarkers (subgroup affiliation) will make the optimal tool for predicting survival of medulloblastoma patients.

\begin{figure}[h]
	\begin{center}
		\includegraphics[width=\textwidth]{fig/magic-clin/subgroup-specific_cox.pdf}
	\end{center}
	\caption[Molecular subgroup and metastatic status are the most important prognostic biomarkers]
	{
	Molecular subgroup and metastatic status are the most important prognostic biomarkers.
	\textbf{a}, Multivariate Cox proportional-hazards survival analysis of predictor variables. Starting with the null model, each variable is added stepwise (from top to bottom) to the survival model. Model likelihood values assess the degree to which each Cox model fits the survival data. Increments in model likelihoods are tested by analysis of deviance. 
	\textbf{b}, Average areas under time-dependent receiver operating characteristic curves (AUC) for multivariate Cox models parameterized by only clinical variables, or both clinical and subgroup variables.
	\textbf{c}, Average time-dependent AUCs for univariate Cox models parameterized by each variable.
	\textbf{d}, Predictive importance of each variable in the fully-parameterized multivariate Cox models, as determined by the average decrease in time-dependent AUC when the variable is omitted from the model.
	Differences in time-dependent AUC and predictive importance are evaluated by the Friedman rank sum test.
	}
	\label{fig:subgroup-specific_cox}
\end{figure}

\clearpage

\subsection{Subgroup specificity of published molecular biomarkers}

Several cytogenetic biomarkers have been previously reported to be associated with patient survival across medulloblastoma, but their prognostic values have seldom been assessed in the context of medulloblastoma subgroups (appendix table not shown). Monosomy for chromosome 6 is significantly associated with improved survival across medulloblastoma in toto (\citefig{subgroup-specific_eg}\emphlab{a}, appendix table not shown). However, the prognostic value of chr6 loss can be completely attributed to its enrichment in WNT medulloblastomas (\citefig{subgroup-specific_eg}\emphlab{b}, appendix data not shown), as loss of chr6 has no prognostic value among WNT patients, or among non-WNT patients, when compared to their respective controls with balanced chr6.  We would suggest that monosomy 6 is subgroup-driven biomarker in that its prognostic significance is driven by its enrichment in a particular subgroup, and it thus holds no further significance in subgroup-specific analysis.  Further, these results would caution against using monosomy 6 as the lone diagnostic criteria for a WNT medulloblastoma, since it is also observed in non-WNT medulloblastoma (7/49 monosomy 6 medulloblastomas were not WNT (14\%)), and monosomy 6 is only present in 42/53 WNT tumors (79\%).  The prognostic role of isochromosome 17q (iso17q) has been very controversial; in our cohort in toto, iso17q is a statistically significant predictor of poor outcome (\citefig{subgroup-specific_eg}\emphlab{c}).  However, subgroup-specific analysis demonstrates that iso17q is highly prognostic for Group3 medulloblastoma, but not for Group4 medulloblastoma (\citefig{subgroup-specific_eg}\emphlab{d}), indicating that it is a subgroup-specific molecular biomarker.  Similarly, while 10q loss is a modestly significant predictor of poor outcome across medulloblastoma subgroups (\citefig{subgroup-specific_eg}\emphlab{e}), its prognostic power is limited to the SHH subgroup of tumors in a subgroup-specific analysis (\citefig{subgroup-specific_eg}\emphlab{f}).  We conclude that determination of molecular subgroup affiliation is crucial in the evaluation and implementation of molecular biomarkers for patients with medulloblastoma (appendix data not shown), as some putative biomarkers are merely enriching for a specific subgroup (subgroup driven) while most others are only significant within a specific subgroup (subgroup specific).

\clearpage

\begin{figure}[h]
	\begin{center}
		\includegraphics[width=\textwidth]{fig/magic-clin/subgroup-specific_eg.pdf}
	\end{center}
	\caption[Subgroup-driven and subgroup-specific molecular biomarkers]
	{
	Subgroup-driven and subgroup-specific molecular biomarkers.
	\textbf{a}, Overall survival curves and frequency distribution of chr6 status across the entire cohort.
	\textbf{b}, Overall survival curves for chr6 status in WNT and non-WNT medulloblastomas.		
	\textbf{c}, Overall survival curves and frequency distribution of isolated chr17q gain across the entire cohort.
	\textbf{d}, Overall survival curves for chr17q status in Group3 and Group4 subgroups. 
	\textbf{e}, Overall survival curves for chr10q status across the entire cohort.
	\textbf{f}, Overall survival curves for chr10q status in SHH and non-SHH medulloblastomas.
	Numbers below x-axis represent patients at risk of event; statistical significances are evaluated by log-rank tests; \gls{hr} estimates are derived from Cox proportional-hazards analyses.
	}
	\label{fig:subgroup-specific_eg}
\end{figure}

\clearpage

\subsection{SHH patients can be stratified into three distinct risk groups}

We identified 11 CNAs that are prognostically significant in our SHH medulloblastoma discovery set (\citefig{shh-markers}, appendix figure not shown) in univariate survival analyses. Given the considerable number of candidates, the reproducibility of the identified biomarkers was assessed by cross-validation, and the expected sample sizes required for validation in future prospective trials were estimated to facilitate candidate prioritization (appendix table not shown). Specific amplifications but not broad gains encompassing \gene{GLI2} or \gene{MYCN} are associated with bleak prognosis (\citefig{shh-markers}\emphlab{a--b}, appendix figure not shown). Loss of chr14q confers significantly inferior survival (\citefig{shh-markers}\emphlab{c}). There is no minimal region of deletion on chr14 in SHH patients (appendix figure not shown), and recent medulloblastoma re-sequencing efforts have not identified any recurrent SNVs on chr14 in SHH medulloblastoma . The presence of chromothripsis (chromosome shattering) is associated with worse survival in SHH patients (\citefig{shh-markers}\emphlab{d}).

To integrate the individual biomarkers into a risk stratification model, multivariate Cox PH analyses were performed on all significant prognostic markers. Through multiple stepwise regression procedures, a consensus set of biomarkers was selected for inclusion in the model in an unbiased manner. The proposed risk stratification scheme represents the model that was most consistent with available data in the discovery cohort, from among many possible alternatives (\citefig{shh-risk-strat}\emphlab{a}, appendix data not shown). \gene{GLI2} amplification, 14q loss, and leptomeningeal dissemination (M+ disease) identify high and standard risk patients. Specifically, \gene{GLI2} amplification alone can identify patients with bleak prognosis (\citefig{shh-risk-strat}\emphlab{a}, appendix figure not shown). Absence of these markers demarcates a low-risk group of patients who exhibit survivorship reminiscent of WNT patients. Importantly, none of the covariates, particularly age and anaplastic histology, can explain the survival differences observed among risk groups (\citefig{shh-risk-strat}\emphlab{a}, appendix figures not shown). Direct application of the proposed risk stratification scheme on the independent validation cohort yields distinct survivorships for the three risk groups, thereby validating the model (\citefig{shh-risk-strat}\emphlab{c}).

\begin{figure}[h]
	\begin{center}
		\includegraphics[width=\textwidth]{fig/magic-clin/shh-markers.pdf}
	\end{center}
	\caption[Overall survival curves for molecular biomarkers in SHH medulloblastoma]
	{
	Overall survival curves for molecular biomarkers in SHH medulloblastoma:
	\textbf{a}, \gene{GLI2} copy number status;
	\textbf{b}, \gene{MYCN} copy number status;
	\textbf{c}, chr14q status; and
	\textbf{d}, chromothripsis status.
	Numbers below x-axis represent patients at risk of event; statistical significances are evaluated by log-rank tests; \gls{hr} estimates are derived from Cox proportional-hazards analyses.
	}
	\label{fig:shh-markers}
\end{figure}

Two additional stratification schemes were constructed using only clinical biomarkers or only cytogenetic markers; however, the proposed model, which combines both types of biomarkers, yields the highest prediction accuracy (\citefig{shh-risk-strat}\emphlab{b}, appendix figure not shown). Furthermore, the accuracy of the combined risk model is drastically reduced when applied across non-SHH patients, further underscoring the importance of taking subgroup into consideration during risk stratification. We conclude that by using two molecular biomarkers (\gene{GLI2} and 14q \gls{fish}) and metastatic status, we can practically and reliably predict prognosis for patients with SHH medulloblastoma.

\clearpage

\begin{figure}[h]
	\begin{center}
		\includegraphics[width=\textwidth]{fig/magic-clin/shh-risk-strat.pdf}
	\end{center}
	\caption[Combined clinical and molecular biomarkers improve risk-stratification of SHH patients]
	{
	Combined clinical and molecular biomarkers improve risk-stratification of SHH patients.
	\textbf{a}, Risk stratification of SHH medulloblastomas by molecular and clinical prognostic markers. \emph{Top-left}, decision tree; \emph{bottom-left}, events plot depicting status of molecular and clinical markers across the risk groups; \emph{right}, overall survival curves for SHH risk groups.
	\textbf{b}, Average time-dependent AUCs for risk groups stratified using only clinical or molecular markers, or both. Risk stratification regimens are applied to SHH and non-SHH medulloblastomas. ***, $p < 0.001$, Friedman rank sum tests.
	\textbf{c}, Survival curves for SHH risk groups in the validation cohort.
	Numbers below x-axis represent patients at risk of event; statistical significances are evaluated by log-rank tests; \gls{hr} estimates are derived from Cox proportional-hazards analyses.
	}
	\label{fig:shh-risk-strat}
\end{figure}

\clearpage

\subsection{Three biomarkers demarcate high-risk Group3 patients}

In Group3 patients, iso17q and \emph{MYC} amplification remain the only cytogenetic markers associated with poor survival, whereas chr8q loss and chr1q gain are the only good prognosis markers (\citefig{group3-markers}, appendix data not shown). In multivariate survival analyses, patients with metastasis, iso17q, or MYC amplification represent the high-risk group (\citefig{group3-risk-strat}\emphlab{a}). Critically, absence of these markers can identify a population of Group3 patients who have a survivorship much longer than Group3 taken as a whole. The risk groups are not associated with any clinical covariates, including age (\citefig{group3-risk-strat}\emphlab{a}, appendix figures not shown). Consistent with the findings in SHH patients, optimal risk stratification in Group3 patients requires the use of both clinical and molecular prognostic markers, which have reduced or no prognostic value outside of Group3 (\citefig{group3-risk-strat}\emphlab{b}, appendix figure not shown). Our proposed risk stratification scheme was validated on the non-overlapping validation cohort using three molecular biomarkers (\emph{MYC}, 17p, and 17q \gls{fish}) and metastatic status (\citefig{group3-risk-strat}\emphlab{c}).

\bigskip

\begin{figure}[h]
	\begin{center}
		\includegraphics[width=\textwidth]{fig/magic-clin/group3-markers.pdf}
	\end{center}
	\caption[Overall survival curves for molecular biomarkers in Group3 medulloblastoma]
	{
	Overall survival curves for molecular biomarkers in Group3 medulloblastoma:
	\textbf{a}, chr17 copy number aberrations;
	\textbf{b}, \emph{MYC} copy number status; and 
	\textbf{c}, chr8q status.
	\textbf{d}, Risk stratification of Group3 medulloblastomas by molecular and clinical prognostic markers.
	Numbers below x-axis represent patients at risk of event; statistical significances are evaluated by log-rank tests; \gls{hr} estimates are derived from Cox proportional-hazards analyses.
	}
	\label{fig:group3-markers}
\end{figure}

\clearpage

\begin{figure}[h]
	\begin{center}
		\includegraphics[width=\textwidth]{fig/magic-clin/group3-risk-strat.pdf}
	\end{center}
	\caption[Combined clinical and molecular biomarkers improve risk-stratification of Group3 patients.]
	{
	Combined clinical and molecular biomarkers improve risk-stratification of Group3 patients.
	\textbf{a}, Risk stratification of Group3 medulloblastomas by molecular and clinical prognostic markers.	\emph{Top-left}, decision tree; \emph{bottom-left}, events plot depicting status of molecular and clinical markers across the risk groups; \emph{right}, overall survival curves for Group3 risk groups.
	\textbf{b}, Average time-dependent AUCs for risk groups stratified using only clinical or molecular markers, or both. Risk stratification regimens are applied to Group3 and non-Group3 medulloblastomas. ***, $p < 0.001$, Friedman rank sum tests.
	\textbf{c}, Survival curves for Group3 risk groups in the validation cohort.
	Numbers below x-axis represent patients at risk of event; statistical significances are evaluated by log-rank tests; \gls{hr} estimates are derived from Cox proportional-hazards analyses.
	}
	\label{fig:group3-risk-strat}
\end{figure}

\clearpage


\subsection{Identification of a low-risk group of metastatic Group4 patients}

Group4 patients with whole chromosome loss of chr11 or gain of chr17 exhibit better survival under univariate Cox PH models (\citefig{group4-markers}\emphlab{a}), in addition to chr10p loss (\citefig{group4-markers}\emphlab{b}). There is no cytogenetic marker associated with poor prognosis (appendix data not shown). Specifically, neither \gene{MYCN} gain nor amplification is associated with poorer survival in Group4, in stark contrast to SHH patients, reinforcing the distinction in their underlying biology (\citefig{group4-markers}\emphlab{b}, appendix figure not shown). Similarly, none of the cytogenetic biomarkers identified for Group3 patients (e.g. iso17q) have any prognostic value in Group4 (appendix table not shown). Following unbiased model selection, the consensus set of biomarkers results in a risk stratification scheme in which leptomeningeal dissemination identifies high-risk Group4 patients, except in the context of chr11 loss or chr17 gain (\citefig{group4-risk-strat}\emphlab{a}). The biology underlying chr11 loss is not apparent as there is no obvious minimal common region of deletion (appendix figure not shown), nor are there any recurrent SNVs on chr11 reported in the recent medulloblastoma re-sequencing publications. Group4 patients with either chr17 gain or chr11 loss, irrespective of their metastatic statuses exhibit survivorship that is characteristic of WNT patients in both the discovery and validation cohorts (\citefig{group4-risk-strat}\emphlab{a},\emphlab{c}), and the survival differences are not explainable by covariates (appendix figure not shown). Significantly, the low-risk Group4 cohort also included some patients with anaplastic histology. Consistent with other subgroups, the risk stratification model using both clinical and molecular biomarkers achieve the highest accuracy (\citefig{group4-risk-strat}\emphlab{b}). Critically, the cytogenetic biomarkers identify low-risk Group4 patients whom would be otherwise designated as high-risk by evidence of metastasis and/or anaplastic histology; this finding cannot be extrapolated to SHH and Group3 patients (\citefig{group4-risk-strat}, appendix figure not shown).  We conclude that through the use of three molecular biomarkers (chr11, 17p, and 17q \gls{fish}) and metastatic status, we can accurately and reliably predict the prognosis of patients with Group4 medulloblastoma.

\bigskip

\begin{figure}[h]
	\begin{center}
		\includegraphics[width=\textwidth]{fig/magic-clin/group4-markers.pdf}
	\end{center}
	\caption[Overall survival curves for molecular biomarkers in Group4 medulloblastoma]
	{
	Overall survival curves for molecular biomarkers in Group4 medulloblastoma:
	\textbf{a}, whole chr11 status and whole chr17 status; and
	\textbf{b}, \gene{MYCN} copy number status.
	Numbers below x-axis represent patients at risk of event; statistical significances are evaluated by log-rank tests; \gls{hr} estimates are derived from Cox proportional-hazards analyses.
	}
	\label{fig:group4-markers}
\end{figure}

\clearpage

\begin{figure}[h]
	\begin{center}
		\includegraphics[width=\textwidth]{fig/magic-clin/group4-risk-strat.pdf}
	\end{center}
	\caption[Combined clinical and molecular biomarkers improve risk-stratification of Group4 patients]
	{Combined clinical and molecular biomarkers improve risk-stratification of Group4 patients.
	\textbf{a}, Risk stratification of Group4 medulloblastomas by molecular and clinical prognostic markers. \emph{Top-left}, decision tree; \emph{bottom-left}, events plot depicting status of molecular and clinical markers across the risk groups; \emph{right}, overall survival curves for Group4 risk groups.
	\textbf{b}, Average time-dependent AUCs for risk groups stratified using only clinical or molecular markers, or both. Risk stratification regimens are applied to Group4 and non-Group4 medulloblastomas. ***, $p < 0.001$, Friedman rank sum tests.
	\textbf{c}, Survival curves for Group4 risk groups in the validation cohort. 
	Numbers below x-axis represent patients at risk of event; statistical significances are evaluated by log-rank tests; \gls{hr} estimates are derived from Cox proportional-hazards analyses.
	}
	\label{fig:group4-risk-strat}
\end{figure}

\clearpage


\section{Discussion}

Why was age not included?
Not one of the better biomarkers. It may be a surrogate for something else
age at onset may be an effect of tumour aggressiveness rather than cause of aggressiveness (e.g. reduced treatment; no irradiation)
Age did not appear to be prognostically important, contrary to earlier studies \citeself{northcott11}.

Why wasn't p53 included in the model? Compare against early study \citeself{zhukova13}.

local recurrence in therapy descalation, watch for recurrence of tumour
WNT patient do well under current treatment
does not imply they will do well under reduced treatment
is high dose of radidation required for WNT survival? particular those with met

some patients with metastasis survive under current (intensified) treatment
e.g. WNT and Group4 low risk patients. Do they need the intensified treatment?

M0 vs. M1 - presence of cells in cerebrospinal fluid meaningful?

Why is \gene{MYCN} amplification prognostically significant in SHH and \gene{MYC} amplification in Group3, but MYCN is not prognostically significant in Group4? Different functions of \gene{MYC} and \gene{MYCN}?

Perhaps low-risk patients can be treated in the same way that young children with medulloblastoma are treated with chemotherapy with possible salvage radiation therapy.

Encouragingly, recent finding suggest that the dose of craniospinal irradiation might be reduced without compromising survival by supplementing with tandem high dose chemotherapy and autologous stem cell transplantation \citeref{sung13}.

Contrasting finding: demoplastic histology overruled extraneural metastasis (M4) \citeref{young15}.


To conclude, we demonstrate that medulloblastoma subgroup affiliation is significantly more informative for predicting patient outcome than existing clinical variables, and that by incorporating subgroup status with conventional clinical parameters for patient risk stratification, the accuracy of survival prediction can be dramatically improved.  Moreover, we propose, test, and validate novel subgroup-specific risk stratification models that incorporate both clinical and molecular variables.  These models perform robustly and reproducibly both in the discovery cohort consisting of a heterogeneously treated group of patients and in a large non-overlapping validation cohort of patients treated at a single institution according to a single treatment protocol.  We do not have detailed treatment information for patients in these cohorts.  It is highly possible that treatment effects (type, duration, or intensity) could impact our results.  We would suggest that this can only be accounted through examination of our stratification model in a sufficiently large prospectively followed cohort of medulloblastoma patients.  While the current study uses either SNP arrays, or interphase \gls{fish} on gls{ffpe} sections, it is possible that other approaches such as array CGH could also be used to determine the copy number status of the six markers.  Our findings demonstrate the utility of incorporating tumor biology into clinical decision-making and offer a novel perspective on risk stratification using \gls{fish} applicable on paraffin sections, and thus could be translated immediately into routine clinical practice.
