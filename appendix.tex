\chapter{Appendix}
\label{ch:appendix}

\section{Nomenclature}
\label{sec:nomenclature}

\subsection{Gene nomenclature}

Although a discussion of nomenclature may be pedantic, unambiguous nomenclature is important to avoid confusion and misinterpretation of scientific results, as exemplified by a retracted Nature publication \citeref{kawasaki03}. This thesis uses standard notations commonly seen in the literature to distinguish between genes and proteins from different species, but not all publications (including those cited herein) use the notations presented here. For the most part, gene symbols serve merely as identifiers for genes in this thesis; accordingly, full gene names are usually not provided. (In particular, some genes such as \gene{AKT1} have no full names due to the systematic nature of their discoveries.) Genes are often known by several names owing to multiple discoveries in different research laboratories. Although one unique official symbol is used to identify a gene, genes often have pleiotropic functions, and some researchers prefer to refer to a gene by a particular alias that reflects one facet of the gene's function of interest. On other occasions, genes were initially assigned generic names such as p21, p27, p53 (referring to proteins with masses 21 kDa, 27 kDa, and 53 kDa, respectively), and these names have become standard in the literature. This thesis instead uses official gene symbols to ensure that genes can be uniquely identified and that no unfortunate misinterpretation will arise, as in the study of Kawasaki \emph{et al} \citeref{kawasaki03}. To learn more about a particular gene, readers may wish to query the Entrez Gene database ({http://www.ncbi.nlm.nih.gov/gene}).

As much as possible, official gene symbols as approved by the Human Genome Organisation Gene Nomenclature Committee or the Mouse Genome Informatics are used, and where appropriate, followed by common gene name aliases in parentheses. Human gene symbols appear in uppercase, and mouse gene symbols, in title case. Italicized names refer to genes, whereas non-italicized names refer to gene products. For example, the human gene on chromosome 9 (at location 98205264--98279247 of the GRCh37 human genome assembly) encoding the first family member of the patched 12-pass transmembrance receptor is denoted \gene{PTCH1}, and its protein counterpart, PTCH1. The mouse gene encoding the patched homologue gene is denoted \gene{Ptch1} and its protein, Ptch1. Specifically in this thesis, SHH and WNT usually refer to the molecular subgroups of medulloblastoma and not the protein or protein family.

\subsection{Animal model nomenclature}

The nomenclature for genetically engineered mouse models is extensive \citeref{mgi13, mgi14}. This thesis will use the following simplified notation: \gene{Ptch1}\high{-/-} is homozygous mutant, \gene{Ptch1}\high{+/-} is heterozygous mutant, and \gene{Ptch}\high{+/+} is homozygous wildtype at the \gene{Ptch} locus. The (+) symbol denotes a wildtype or non-mutated allele, and the (-) symbol denotes a null or loss-of-function allele. Multiple genetic modifications are separated by a semicolon: \gene{Ptch1}\high{+/-};\gene{Trp53}\high{+/-} is heterozygous mutant for both \gene{Ptch} and \gene{Trp53}. While these notations omit the mouse strain, the genetic background of the mouse model can indeed modulate the mutant phenotype \citeref{pazzaglia02b, pazzaglia04, pazzaglia09}.

Gene function may be disrupted constitutively (\emphterm{knocked out}). Homozygous knock-outs, which abrogate gene function completely, may cause embryonic lethality and preclude further study (e.g. \gene{Ptch1}\high{-/-} mice die before birth) \citeref{goodrich97}. In addition to studying heterozygous knock-outs (e.g. \gene{Ptch1}\high{+/-}), researchers may also conditionally disrupt of gene function using Cre recombination. A sequence flanked by loxP sites is knocked into the native gene locus. Cre recombinase recognizes loxP sites and recombines the DNA, causing deletion of the region \emphterm{f}lanked by \emphterm{lox}P sites (the region is \emphterm{floxed} and subsequently deleted). Whenever and wherever Cre is expressed, the floxed region of the gene would be deleted. Such a model requires two genetic contructs: the floxed gene (e.g. \gene{Ptch1}\high{flx/flx} where flx denotes `floxed') and Cre with a promoter (e.g. \gene{Atoh1-Cre}). The floxed region may encompass the entire gene or (usually) critical exons of the gene so that the no functional products would result from the recombined gene. The promoter is often named after the gene from which it was derived. That is, \gene{Atoh1-Cre} indicates that Cre is placed downstream of the promoter for \gene{Atoh1} (and not the coding sequence of the \gene{Atoh1} gene itself). Therefore, in the \gene{Atoh1-Cre};\gene{Ptch1}\high{flx/flx} mouse \citeref{yang08}, Cre would be expressed in Atoh1-expressing cells and delete both of the floxed \gene{Ptch1} alleles; consequently, the Atoh1-expressing cells would incur a homozygous deletion of \gene{Ptch1} and lose Ptch1 function completely.
		
A transgenic construct may also be integrated (randomly) into the genome to express a gene. For example, the \gene{Neurod2-Smo}\high{+/W539L} mouse has the \gene{Neurod2-Smo}\high{W539L} construct integrated into the genome once (later mapped to chr14) \citeref{hatton08}, the activated Smo (with a substitution mutation from tryptophan to leucine at residue 539) is expressed under the promoter of \gene{Neurod2}. The Cre recombination system may also be used to conditonally express a gene using a more complex design, such as the \gene{CAGGS-CreER};\gene{Rosa26-SMO}\high{W535L} mouse \citeref{mao06}. In this mouse, the human \gene{SMO} gene with an activating W535L mutation is knocked into the ubiquitously expressed \gene{Rosa26} locus, whose official symbol is Gt(ROSA)26Sor. (This human \gene{SMO} gene is known as SmoM2 \citeref{xie98} and corresponds to mouse \gene{Smo}\high{W539L}, which is known as SmoA1 \citeref{hallahan04}.) However, since a floxed polyadenylation stop sequence cassette is placed upstream of the \gene{SMO}\high{W535L} gene, the latter is only expressed subsequent to Cre-mediated deletion of the cassette. Cre is expressed here under the synthetic CAGGS promoter (which drives high-level, generalized expression), and its activity is (somewhat) tamoxifen-dependent because it is fused to the estrogen receptor. Therefore, the expression of \gene{SMO}\high{W535L} is ultimately controlled by the administration of tamoxifen (at least in design).


\section{Signaling Pathways}
\label{sec:signal-pathways}

\subsection{\gls{wnt} signaling (CTNNB1-dependent)}

%\begin{figure}[H]
%	\begin{center}
%		%\includegraphics[width=\textwidth]{fig/pathway/wnt-pathway.pdf}
%	\end{center}
%	\caption[CTNNB1-dependent Wnt signaling pathway]
%	{
%		CTNNB1-dependent Wnt signaling pathway.
		In the absence of WNT ligand, CTNNB1 ($\beta$-catenin) is continuously marked for degradation by the CTNNB1 destruction complex, consisting of AXIN, APC, GSK, and CSNK1A1 (CK1). GSK and CSNK1A1 primes CTNNB1 by phosphorylation, which leads to ubiquinylation by the CUL1-containing E3 ligase complex and subsequent complete degradation by the proteome. Upon binding of WNT to FZD, DVL1 is activated via an unknown mechanism, and DVL1 phosphorylates LRP5/6, which then sequesters AXIN and frees CTNNB1 from the destruction complex. CTNNB1 accumulates in the cytoplasm and translocates to the nucleus to activate target genes such as \gene{MYC}, \gene{CCND1}, and \gene{AXIN2}. Two \gene{AXIN} genes exist in humans: \gene{AXIN1} and \gene{AXIN2}. GSK consists of two subunits: GSK3A and GSK3B (catalytic). CTNNB1 regulates expression in concert with the TCF/LEF family of co-transcription factors, such as TCF7, TCF7L1, TCF7L2, and LEF1. WNT signaling also activates many other pathways independently of CTNNB1; these pathways include: planar cell polarity, WNT-Ca\high{2+}, and others. Both WNT and FZD encompass a large family of proteins.
%	}
%	\label{fig:wnt-pathway}
%\end{figure}


\subsection{\gls{shh} signaling}

%\begin{figure}[H]
%	\begin{center}
%		%\includegraphics[width=\textwidth]{fig/pathway/shh-pathway.pdf}
%	\end{center}
%	\caption[\gls{shh} signaling pathway]
%	{
%		Shh signaling pathway.
		GLI transcription factors, including GLI1, GLI2, and GLI3, are downstream effectors of \gls{shh} signaling. The proteome can completely degrade GLI proteins or proteolytically process them into activators or repressors forms, depending on post-translational modifications on the full-length GLI proteins. GLI1 and GLI2 predominantly function in the activator form and activates transcription, while GLI3 mainly functions in the repressor form and represses transcription. \gls{shh} signaling modulates the marks on full-length GLI, thereby influencing downstream transcription.
		In the absence of SHH ligand, SMO activity is repressed by PTCH. Upon SHH binding to PTCH, this repression is relieved, leading to active SMO signaling, which favours the processing of GLI into the active form; consequently, transcription of target genes including cell cycle genes (\gene{CCND1} and \gene{CCNE1}), \gene{MYC}, and negative regulators of Shh signaling (\gene{PTCH1} and \gene{HHIP}) is induced.
		The precise mechanism whereby SMO (indirectly) activates GLI is not conserved between fly and mammals and remains unknown in mammals. Four proteins bind to and regulate GLI processing: KIF7, SUFU, SPOP, and BTRC. Both KIF7 and SUFU are negative regulators of Shh signaling via unclear mechanisms. The role of SUFU remains contentious; possible roles include: nuclear export of GLI, protection of GLI from degradation, recruitment of GLI3 for processing into the repressor form. Conversely, SPOP and BTRC ($\beta$TRCP) serve to recognize GLI and they function as subunits of E3 ligase complexes, which mark (by ubiquitination) GLI for proteolytic processing by the proteome. The CUL1 containing complex is also known as the Skp, Cullin1, F-box (SCF) complex, Via BTRC-mediated recognition, this complex ubiquitinates many other proteins, including CTNNB1 from the Wnt pathway.
		Mammals have three hedgehog ligands: DHH, IHH, and SHH. PTCH encompasses PTCH1 and PTCH2. PKA is a tetramer composed of two catalytic subunits (protein family includes PRKACA, PRKACB, and PRKACG) and two regulatory subunits (protein family includes PRKAR1A, PRKAR1B, PRKAR2A, and PRKAR2B).
%	}
%	\label{fig:shh-pathway}
%\end{figure}


\subsection{Notch signaling}

%\begin{figure}[H]
%	\begin{center}
%		%\includegraphics[width=\textwidth]{fig/pathway/notch-pathway.pdf}
%	\end{center}
%	\caption[Notch signaling pathway]
%	{
%		Notch signaling pathway.
		When NOTCH binds DLL that is expressed on the surface of an adjacent cell, NOTCH is extracellularly cleaved by ADAM10 or ADAM17 (TACE), and it is subsequently cleaved intracellularly by the gamma-secretase/presenilin complex. This proteolytic event releases the NOTCH intracellular domain (NICD) from the cell surface, allowing it to translocate to the nucleus in order to activate transcription of target genes such as \gene{MYC}, \gene{HES}, \gene{CDKN1A}, \gene{CCND3}. Mammals have four Notch family members (NOTCH1, NOTCH2, NOTCH3, and NOTCH4), and these receptors can bind ligands DLL1, DLL3, DLL4, JAG1, and JAG2. 
%	}
%	\label{fig:notch-pathway}
%\end{figure}


\subsection{\gls{pi3k} signaling}

%\begin{figure}[H]
%	\begin{center}
%		%\includegraphics[width=\textwidth]{fig/pathway/pi3k-pathway.pdf}
%	\end{center}
%	\caption[PI3K signaling pathway]
%	{
%		PI3K signaling pathway.
		Kinases from the PI3K family phosphorylate PIP\low{2} to produce PIP\low{3}, which serves as the intermediate signaling molecule for many pathways. One well-studied binding target for PIP\low{3} is AKT1, which in turn inhibits apoptosis and promotes cell cycle progression. PI3K signaling is activated downstream of many cell-surface receptors, including receptor tyrosine kinases, cytokine receptors, integrins, and G-protein coupled receptors.
		PTEN is a negative regulator of PI3K signaling and functions by dephosphorylating PIP\low{3} to form PIP\low{2}. 
%	}
%	\label{fig:pi3k-pathway}
%\end{figure}


%\subsection{\gls{egfr} signaling}
%
%\gls{egfr} signaling is important in medulloblastoma.
%
%Lack of ERBB2 protein expression in tumour is associated with favourable survival under standard treatment \citeref{gajjar04}, suggesting ERBB2 inhibition may be effective alone or in combination with conventional chemotherapy.
%
%However, "Lapatinib was well-tolerated in children with recurrent or CNS malignancies, but did not inhibit target in tumor and had little single agent activity." \citeref{fouladi13}.
%
%
%\begin{figure}[H]
%	\begin{center}
%		%\includegraphics[width=\textwidth]{fig/pathway/egfr-pathway.pdf}
%	\end{center}
%	\caption[Egfr signaling pathway]
%	{
%		Egfr signaling pathway.
%		ERBB2 has multiple roles: it can function as co-receptor, it can recruit ligand, and it can dimerize and activate signaling in the absence of ligand.
%	}
%	\label{fig:egfr-pathway}
%\end{figure}


% Developmental and oncogenic effects of insulin-like growth factor-I in Ptc1+/- mouse cerebellum \citeref{tanori10}.


\section{Classification}
\label{sec:classification}

Classification is the central preoccupation of one of the oldest disciplines, \emphterm{taxonomy}, which seeks to place organisms into groups based on shared characteristics. Just as classifying organisms facilitates their study, classifying diseases also helps clinicians understand common mechanisms underlying disease and develop rational treatment against each type of disease. Classification encompasses two parts: first, the classes are established by exploring features and grouping the samples based on similarity of features; second, a method is created to classify new sample using features that discriminate between classes. In bioinformatics, the first part is referred to as \emphterm{class discovery}, and the second, \emphterm{class prediction}. In taxonomy, the term \emphterm{classification} only refers to the first part while the term \emphterm{identification} refers to the second part. In machine learning, the first part is equivalent to \emphterm{unsupervised learning}, and the second, \emphterm{supervised learning}. In statistics, the term \emphterm{classification} only refers to the second part. Notwithstanding the different terminologies in various fields and the conflation of different meanings of the term `classification', the two components have distinct objectives. While some algorithms can be used for both class discovery and class prediction, algorithms specifically designed for prediction usually outperform other algorithms in the class prediction problem. Furthermore, class prediction algorithms are well developed for the case in which the classes are known for a set of samples. Conversely, the class discovery process may be contentious: when using different features results in discovering different sets of classes, which set of classes is correct? Indeed, a classification system is practically useful insofar as it is widely accepted and stable. Once such a classification system is discovered and becomes established, one can focus on predicting the class of a new sample (as least until a new classification system arises).

\subsection{Class discovery}

The objective of class discovery is to group (or split) samples into disjoint classes. One systematic and unbiased way of achieving this goal is by \emphterm{clustering} (or \emphterm{partitioning}) a set of samples. Given measurements for various features (characteristics), clustering groups samples with similar features together, whereas partitioning split the samples into dissimilar groups. The distinction between clustering and partitioning is moot because the same overall objectives are achieved; however, different algorithms often produce different results.  A variety of clustering (or partitioning) algorithms are available, including hierarchical clustering, $k$-means, self-organizing map, affinity propagation, spectral clustering, graph clustering, mixture models, and consensus clustering. The two most popular clustering algorithms are hierarchical clustering and $k$-means (these are also among the oldest). These clustering algorithms have many additional variants. Hierarchical clustering may be divisive or agglomerative; it may calculate distance between groups using various \emphterm{linkage methods} (single, complete, average or others). $k$-means have variants such as $k$-medians or $k$-medoids, where different methods are used to represent the center of a group. $k$-means uses the mean of all the group members (i.e. centroid), $k$-medians uses the median, and $k$-medoids uses the group member whose average distance to all other group members is minimal. A variant of $k$-medoids is \emphterm{partitioning around medoids}, which uses a particular method for initializing the groups and updating the clusters. Furthermore, the aforementioned clustering methods may use different measures of distance between individual samples. Self-organizing map, affinity propagation, and spectral clustering are less popular in computational biology (at least for now). Graph clustering encompasses a myriad of algorithms that cluster a \emphterm{graph} (a collection of nodes with interconnecting edges, known more informally as a `network'). These algorithms rely on known independence (lack of edges) between samples (nodes), whereas most of the other algorithms consider connections between all pairs of samples as presented in a distance matrix or similarity matrix. Mixture models attempt to model the data as a mixture of statistical distributions, such as multivariate Gaussian distributions, and the class assignments are probabilistic. Consensus clustering runs multiple clustering algorithms or variants (with different parameters) to generate multiple clustering results; then, it define classes based on the census of the clustering results. A type of consensus clustering algorithm that has gained popularity recently is \emphterm{\gls{nmf} consensus clustering}. In this method, the data matrix is randomly factorized into two matrices using \gls{nmf} and samples are assigned preliminary groups based on one of the factor matrices (using an arbitrary rule); subsequent to multiple \gls{nmf} runs, the clustering results are combined using consensus clustering, by apply hierarchical clustering on the census matrix, which counts the number of times that each pair of samples were assigned the same group.

The foremost decision in class discovery is deciding which type of features to use. For example, one could run clustering analysis with expression data, DNA methylation data, gene mutation data, patient, and other data. One could also combine different types of features and perform clustering in an integrative manner (while taking care to weight the different types appropriately). Due to the exploratory nature of class discovery, there are no formal guidelines for the choice of features (though interval-scale features are more mathematically amendable to clustering analysis than nominal-scale features). Often, the type of features can be chosen to discover classes for specific objectives, such as using expression classes to identify tumour types with activation of different biological pathways. Indeed, a rational researcher guided by the same objective would be unlikely to group patients into classes based on their birth days, months, and years.

Another challenge of clustering analysis is to determine the number of classes represented in the data, $k$. Some algorithms begin by putting all samples into one class and proceed to partition the samples into smaller classes recursively (e.g. divisive hierarchical clustering), some algorithms begin by putting each sample into its own class and proceed to cluster the samples into bigger classes recursively (e.g. agglomerative hierarchical clustering). Either way, a decision must be made as to when to stop partitioning or clustering. In the case of hierarchical clustering, this decision is usually made informally. Other algorithms require the number of classes $k$ to be specified \emph{a priori} (e.g. $k$-means, finite mixture models). One recent algorithm that circumvents this requirement is \emphterm{Dirichlet Process infinite mixture model}, which allows for all possible values of $k$ and discovers $k$ from the data. However, Dirichlet Process can be very computationally intensive. An alternative approach is evaluate the clustering results for different values of $k$. (For algorithms in which $k$ is not specified \emph{a prior}, one runs the algorithm until $k$ clusters are created.)

There is no concensus on the best method for evaluating the quality of the clusters. Often, practitioners evaluate clustering results by considering external information. This external evaluation of the data should be done with care in order to avoid overfitting and optimistic bias. Additionally, since the objective of class discovery is to discover new classes, comparison with known classes would not likely help guide the evaluation of clustering results. As an example, suppose we evaluate clustering results by assessing the association of the discovered classes with survival (i.e. how different are the survival times among the classes). Using this criterion, we compare the association with survival among different clustering algorithms or parameter settings, and we choose the algorithm with the parameter setting that achieves the most significant association with survival. Unless we have additional samples not used during clustering to validate the association of the classes with survival, we may be \emph{overfitting} the algorithm (or parameters) to the existing data, and the optimistically biased association may disappear in a new dataset. In other words, we may have discovered classes that are spuriously associated with survival in our available data, and we may not observe this association again in a new dataset. We may mitigate this problem by using a subset of the data for optimizing the clustering algorithm and evaluating the results on the remaining data. Alternatively, we may also evaluate the clustering results without using external data. This internal evaluation of the data involves measuring the distance between sample pairs within each cluster as well as the distance between sample pairs from different clusters. The best cluster assignments should minimize the within-cluster distance and maximize the between-cluster distance. Different measures for evaluating and combining these two objectives have been proposed, including the silhouette width and the Dunn index. However, these internal measures depend on the definition of distance between samples: the results would change if a different distance measure were used. Indeed, evaluating the results of a clustering algorithm can be difficult and far from straightforward. Due to this difficulty in evaluating clustering results, there is also no concensus on the best clustering method. There is simply no substitute for applying class discovery on multiple datasets and confirming that the same classes are discovered in each dataset.

\subsection{Class prediction}

Given a set of samples with known classes (\emphterm{labeled data}), the objective of class prediction is to learn the association between the labels and the features of the samples and subsequently predict the class of a new sample. The class labels could be defined by expert opinion, discovered by systematic clustering, or determined by other means. Class prediction involves three steps: \emphterm{training}, \emphterm{testing}, and \emphterm{application}. That is, a classifier must be trained on a labeled dataset and tested on another labeled dataset before it can be applied on samples with unknown classes (\emphterm{unlabeled data}). During training, the classifier learns the association between labels and features in a first labeled dataset (\emphterm{training data}). During testing, we evaluate how well the classifier generalizes its learned association to a separate labeled dataset (\emphterm{testing data}). Finally, only when we are certain that the classifier can predict classes accurately, we apply it to unknown samples and make downstream decisions.

One simple measure of classifier performance is \emphterm{accuracy}: the percentage of samples classified correctly. A high accuracy on the training data indicates the classifier has sufficient capacity to learn from the data, and a high accuracy on the testing data implies that the classifier can generalize well. When a classifier has high training accuracy but low testing accuracy, the classifier is likely overfitted to the training data. In such an event, possible remedies include simplifying the complexity of the classification algorithm (by permitting fewer free parameters) or collect more training data.

For a classification problem with two classes (positive and negative), common evaluation methods include receiver-operating characteristics curve or precision-recall curve. These methods simultaneously evaluate how often a classifier falsely assigns a positive label to a true negative sample and how often a classifier falsely assign a negative label to a true positive sample. For multiclass problems, common evaluation methods include Rand index or Jaccard index. Alternatively, a multiclass problem can be split into multiple two-class sub-problems, in which each sub-problem addresses whether a sample belongs to a given class.


\section{Cancer treatment}
\label{sec:cancer-treatment}

\subsection{Chemotherapy}

Chemotherapy is the use of drugs that target general cellular processes such as cell division in order to eradicate cancer cells. The drugs may damage DNA directly (by alkylation or forming DNA adducts), interfere with DNA synthesis (by inhibiting folic acid, substituting for a nucleotide, intercalating into DNA, or inhibiting topoisomerase), prevent mitosis (by inhibiting microtubules), or various other mechanisms. Often, chemotherapy relies on triggering the (hopefully intact) apoptotic program of the cell in response to recognition of DNA damage. For example, loss-of-function mutation in \gene{TP53} (responsible for triggering apoptosis) can lead to resistance against chemotherapy. Overexpression of efflux pumps is another major mechanism of chemoresistance.

Chemotherapeutic treatment typically lasts from weeks to months, and it is given at different stages. The purpose of induction chemotherapy is to induce a remission, while consolidation chemotherapy is given, typically at high dose, at the end of induction to complete remission. Conversely, maintenance chemotherapy is given at low dose over a long period to prevent cancer recurrence.

Sometimes, chemotherapy are described as adjuvant or neoadjuvant. Adjuvant therapy is given to assist the main treatment in eradicating the cancer and is given during or after the latter; neoadjuvant therapy is given before the main treatment. In solid tumours, some literature consider surgical resection as the main treatment while others consider radiotherapy to be of prime importance.


\section{Prognostic biomarker discovery}
\label{sec:prognostication}

\subsection{Log-rank tests vs. Cox proportional-hazards test}

The survival analyses presented were based on log-rank tests and Cox proportional-hazards tests, which may yield considerably different p-values. As log-rank tests do not assume proportional hazards, their results were presented instead of those of Cox proportional-hazards tests. Univariate Cox proportional-hazards analyses were performed to estimate hazard ratios and sample sizes required for prospective studies.

\subsection{Construction and validation of risk stratification models}

In order to identify novel and robust prognostic biomarkers, the present study examined a discovery set and a validation set of medulloblastoma cases. The discovery set consisted of cases with patient survival follow-up, whole-genome copy-number profiles, and varying degree of clinical details, including age, gender, metastatic status, and histological subtype. This set of cases was acquired from several hospitals and tumour banks around the globe. Therefore, patients in the discovery set represent a heterogeneously treated group with diverse ethnic backgrounds. In contrast, the validation set consisted of medulloblastoma patients who were uniformly treated at a single institution in Moscow (Burdenko Hospital).

All available clinical variables and molecular markers were tested for prognostic association in the discovery set. Several clinical variables, such as metastatic status and age group, were categorized in multiple different ways, due to disagreements in the literature and clinical practice across continents. Due to the large number of candidate markers tested, a rigorous selection procedure was applied in order to select a small number of candidates to be validated in the external validation set using fluorescence in situ hybridization (FISH), which is routinely performed in modern pathology laboratories within hospitals.

Accordingly, the clinical and molecular candidate biomarkers were assessed by three approaches. First, the candidates were assessed by a cross-validation method, in order to estimate the expected validation rate of the biomarker. That is, whether the biomarker will likely validate in an independent cohort. Second, the sample size required for further validation in a prospective study was estimated for each candidate. Prognostic markers with small effect size (i.e. hazard ratio) or with low frequency may need impractically large sizes and are thus clinically irrelevant. Third, the candidates were combined in multivariate Cox proportional-hazards models in order to assess whether the biomarkers have prognostic values independent of one another. Biomarkers were prioritized by high validation rates, reasonably small sample sizes, and/or prognostically significance in multivariate models. The selected biomarkers were then used to construct the risk stratification models for each medulloblastoma subgroup.

The proposed risk stratification models represent promising candidates for future prospective trials. The constituent biomarkers were selected based on analyses within a heterogeneous discovery set, and are likely generalizable to different patient populations. For a specific treatment protocol within a specific patient population, there may be prognostic markers that have better prognostic value, particularly those that were not assessed in the present study due to scope. Notwithstanding these limitations, the proposed risk stratification models have been validated in an independent cohort, and can serve as the basis for the informed design of a future prospective trial.

\subsection{Rare cytogenetic events}

Some molecular biomarker candidates (e.g. \gene{MYC} amplification, chr17 gain) have only been observed in a relatively small number of patients ($\approx 10$). Notwithstanding their infrequency in specific subgroups of medulloblastoma (a rare disease), their prognostic significances are supported by log-rank tests, likely due to their large `effect size' (i.e. hazard ratio). Such biomarker candidates, however, have low expected validation rates from cross-validation and large estimated sample sizes from power analysis. On accounts of their potential therapeutic impact, these candidates were nonetheless included in the risk stratification models based on their independent prognostic significance under multivariate Cox models. Indeed, the candidates were ultimately validated to be \emph{bona fide} prognostic biomarkers in the external validation set.

\subsection{Isolated vs. non-isolated events}

Isolated arm events occur in the absence of whole-chromosome event; non-isolated events may occur in the context of a whole-chromosome event.

%In the appendix tables, chr17q\|G denotes the gain of chr17q or gain of chr17, whereas chr17Q\|G denotes the gain of chr17q without concurrent gain of the whole chr17.

%\subsection{Isochromosome events}

%The analyses of isochromosomes (e.g. iso17q) differ between the figures and the tables. In the tables, samples with iso17q were compared against samples without iso17q, irrespective of other cytogenetic aberrations on chr17. However, iso17q in Group4 medulloblastoma, albeit statistically significantly associated with poor survival in Group4 patients, have a low validation rate (Appendix Table 10), suggesting that its statistical association may be indirect. This statistical significance may be due to the inclusion of patients with tumors harboring chr17 gain in the reference group, since chr17 gain is associated with good prognosis (Figure 5A). Therefore, samples with iso17q were compared against samples with balanced chr17 and samples with broad gain or loss of chr17 were excluded from Figure 2G-H.
