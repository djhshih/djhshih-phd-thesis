%% SUMMARY OF FEATURES:
%%
%% All environments, commands, and options provided by the `ut-thesis'
%% class will be described below, at the point where they should appear
%% in the document.  See the file `ut-thesis.cls' for more details.
%%
%% To explicitly set the pagestyle of any blank page inserted with
%% \cleardoublepage, use one of \clearemptydoublepage,
%% \clearplaindoublepage, \clearthesisdoublepage, or
%% \clearstandarddoublepage (to use the style currently in effect).
%%
%% For single-spaced quotes or quotations, use the `longquote' and
%% `longquotation' environments.


%%%%%%%%%%%%         PREAMBLE         %%%%%%%%%%%%

%%  - Default settings format a final copy (single-sided, normal
%%    margins, one-and-a-half-spaced with single-spaced notes).
%%  - For a rough copy (double-sided, normal margins, double-spaced,
%%    with the word "DRAFT" printed at each corner of every page), use
%%    the `draft' option.
%%  - The default global line spacing can be changed with one of the
%%    options `singlespaced', `onehalfspaced', or `doublespaced'.
%%  - Footnotes and marginal notes are all single-spaced by default, but
%%    can be made to have the same spacing as the rest of the document
%%    by using the option `standardspacednotes'.
%%  - The size of the margins can be changed with one of the options:
%%     . `narrowmargins' (1 1/4" left, 3/4" others),
%%     . `normalmargins' (1 1/4" left, 1" others),
%%     . `widemargins' (1 1/4" all),
%%     . `extrawidemargins' (1 1/2" all).
%%  - The pagestyle of "cleared" pages (empty pages inserted in
%%    two-sided documents to put the next page on the right-hand side)
%%    can be set with one of the options `cleardoublepagestyleempty',
%%    `cleardoublepagestyleplain', or `cleardoublepagestylestandard'.
%%  - Any other standard option for the `report' document class can be
%%    used to override the default or draft settings (such as `10pt',
%%    `11pt', `12pt'), and standard LaTeX packages can be used to
%%    further customize the layout and/or formatting of the document.


\documentclass[12pt]{ut-thesis}

\degree{Doctor of Philosophy}
\department{Laboratory Medicine and Pathobiology}
\gradyear{2015}
\author{J. H. David Shih}
\title{\textbf{Clinical Genomics of Medulloblastoma}}

%% External formatting definitions
\usepackage{graphicx}

\usepackage{multicol}

\usepackage{setspace}

% use captions
\usepackage[font={small,sf}]{caption}
\captionsetup[figure]{labelfont=bf}
\captionsetup[table]{labelfont=bf}

% use side captions
\usepackage{sidecap}
\sidecaptionvpos{figure}{c}

% define new environments
\usepackage{amsthm}
\theoremstyle{definition}
\newtheorem*{hypothesis}{Hypothesis}
\newtheorem*{objective}{Objective}

% per-chapter figure numbering
\usepackage{chngcntr}
\counterwithin{figure}{chapter}
% use arabic section numbering for figures
\renewcommand{\thefigure}{\arabic{chapter}.\arabic{figure}}

% use roman number formatting for sections
\renewcommand\thesection{\Roman{section}}

% disable subsection numbering
\setcounter{secnumdepth}{1}

\usepackage[super,sort&compress]{natbib}

% define multiple reference lists
\usepackage{multibib}
\newcites{self,ref}{Publications Arising from Thesis,References}


% emphasize author name
\newcommand{\emphname}[1]{\underline{\textbf{#1}}}

% emphasize label
\newcommand{\emphlab}[1]{\textbf{\textsf{#1}}}

% italicize gene name
\newcommand{\gene}[1]{\emph{#1}}

% cite figure and table
\newcommand{\citefig}[1]{\emphlab{Figure~\ref{fig:#1}}}
\newcommand{\citetable}[1]{\emphlab{Table~\ref{tbl:#1}}}
\newcommand{\citech}[1]{\emphlab{Chapter~\ref{ch:#1}}}

% paragraph formatting
\setlength{\parindent}{3em}
\setlength{\parskip}{0.5em}

% margins
\geometry{headsep=0.4in, top=1.25in}

% change section header font size
\usepackage{sectsty}
\chapternumberfont{\large}
\chaptertitlefont{\LARGE}
\sectionfont{\Large}
\subsectionfont{\large}

% add glossary
\usepackage[toc,acronym,nomain]{glossaries}
\makeglossaries

% add list of figures and list of tables to table of contents
\usepackage[nottoc]{tocbibind}



%% List only down to subsections in the table of contents;
%% 0=chapter, 1=section, 2=subsection, 3=subsubsection, etc.
\setcounter{tocdepth}{2}

%% Make each page fill up the entire page.
\flushbottom


%%%%%%%%%%%%      MAIN  DOCUMENT      %%%%%%%%%%%%

\begin{document}

%% This sets the page style and numbering for preliminary sections.
\begin{preliminary}

%% This generates the title page from the information given above.
\maketitle

\cleardoublepage

\begin{abstract}
Medulloblastoma is the most common malignant brain tumour in children. Standard treatments cure 60-70\% of patients; however, survivors suffer long-term developmental deficits due to treatment-associated toxicity. Recent studies identified four subgroups of medulloblastoma with distinct expression patterns: WNT, SHH, Group3, and Group4. We believe these subgroups represent different molecular entities that arise through and rely on different oncogenic processes. Accordingly, we aim to improve prediction of patient survival by identifying prognostic markers for each subgroup, and we hope to abrogate non-specific cytotoxic treatments by discovering candidates for targeted intervention against each subgroup.

Molecular subgroups of medulloblastoma tumours were determined with a trained classifier using select subgroup-specific marker expressions. DNA copy-number profiles ($n=1087$) were analyzed to identify somatic copy-number aberrations (SCNAs) and recurrently disrupted genes and pathways. Prognostic SCNAs were identified by Kaplan-Meier survival analyses on a discovery set ($n=673$), and the candidates were validated by FISH on a tissue microarray of validation samples ($n=453$).

Tumours of each subgroup harbour recurrent SCNAs disrupting different pathways. WNT medulloblastoma is characterized by Beta-catenin mutation, SHH medulloblastoma by activated Gli signaling, Group3 medulloblastoma by Myc activation, and Group4 medulloblastoma by SNCAIP duplication. Further, patients of different subgroups exhibit differential response to standard treatments. Incorporating subgroup data into survival models significantly improved predictive performance. Using six FISH biomarkers on FFPE tissues, we reproducibly stratified patients into groups with distinct survivorships.

The stark differences in genetic alterations among medulloblastoma subgroups suggest that each subgroup arises through different biological mechanisms. The molecular classification of medulloblastoma not only improved survival prediction but also revealed pathways for intervention. We have identified a panel of prognostic markers that can be used to select patients for therapy de-escalation in future trials, and we have also discovered candidates for the development of targeted therapy.    
\end{abstract}

%% Anything placed between the abstract and table of contents will
%% appear on a separate page since the abstract ends with \newpage and
%% the table of contents starts with \clearpage.  Use \cleardoublepage
%% for anything that you want to appear on a right-hand page.

\begin{dedication}
\begin{center}
\vspace{1in}
\emph{To my mentor, Professor Hon Kwan.}
\end{center}
\end{dedication}

\newpage

\begin{acknowledgements}
I would like to thank my advisors, Michael and Gary, for all their support and guidance throughout my studies. I would like to acknowledge the contributions of my thesis committee -- Dr.\ Meredith Irwin, Dr.\ Quaid Morris -- for their insightful inputs. I would also like to acknowledge all my labmates for their help and feedback. Most of all, I wish to thank Paul for his contribution to our collaborative pursuits.

I must thank all members of the Medulloblastoma Advanced Genomics Consortium for their contributions, without which the studies described herein would not be possible. I am especially thankful for work done by Dr.\ Andrey Korshunov, Dr.\ Stefan Pfister, and their teams at the German Cancer Research Center.

I would also like to thank all my teachers who imparted invaluable knowledge and expertise. I especially thank Dr.\ Derek Corneil, Dr.\ Geoffrey Hinton, Dr.\ Boris Steipe, and Dr.\ Reinhold Vieth at the University of Toronto. I would also like to highlight Massive Open Online Courses offered by Stanford University, Coursera, and edX, as well as course notes graciously made openly available by instructors at other institutions. I owe a debt of gratitude to inspirational teachers such as Dr.\ Andrew Ng (Stanford University), Dr.\ Daphne Koller (Stanford University), Dr.\ Robert Pich\'{e} (Tampere University of Technology), Dr.\ Brain Caffo (Johns Hopkins Bloomberg School of Public Health), Dr.\ Rafael Irizarry (Harvard School of Public Health), and Dr.\ Tom Mitchell (Carnegie Mellon University).

I am grateful to open source software communities for openly sharing their work, including authors, contributors, and maintainers on CRAN, Bioconductor, PyPI, GitHub, and numerous other repositories. Their spirit of openness is infectious.

This work is supported by the Frederick Banting and Charles Best Canada Graduate Scholarship, the Michael Smith Foreign Study Supplement, the Ontario Graduate Scholarship, and the University of Toronto Fellowship.
\end{acknowledgements}


\tableofcontents

\listoftables

\listoffigures

% General biological and medical terms
\newacronym{acgh}{aCGH}{array comparative genomic hybridization}
\newacronym{cna}{CNA}{copy-number aberration}
\newacronym{cns}{CNS}{central nervous system}
\newacronym{dna}{DNA}{dexoyribonucleic acid}
\newacronym{ffpe}{FFPE}{formalin-fixed, paraffin-embedded}
\newacronym{fish}{FISH}{fluorescence \emph{in situ} hybridization}
\newacronym{loh}{LOH}{loss of heterozygosity}
\newacronym{mri}{MRI}{magnetic resonance imaging}
\newacronym{nos}{NOS}{not otherwise specified}
\newacronym{pcr}{PCR}{polymerase chain reaction}
\newacronym{rna}{RNA}{ribonucleic acid}
\newacronym{scna}{SCNA}{somatic copy-number aberration}
\newacronym{scnas}{SCNAs}{somatic copy-number aberrations}
\newacronym{snp}{SNP}{single nucleotide polymorphism}
\newacronym{iq}{IQ}{intelligence quotient}

% Diseases
\newacronym{atrt}{ATRT}{atypical teratoid/rhabdoid tumour}
\newacronym{cnspnet}{CNS-PNET}{central nervous system primitive neuroectodermal tumour}
\newacronym{mb}{MB}{medulloblastoma}
\newacronym{pnet}{PNET}{primitive neuroectodermal tumour}
\newacronym{etmr}{ETMR}{embryonal tumours with multilayered rosettes}
\newacronym{mben}{MBEN}{medulloblastoma with extensive nodularity}
\newacronym{lca}{LCA}{large cell/anaplastic medullobalstoma}

% Molecular pathways
\newacronym{shh}{Shh}{Sonic hedgehog}
\newacronym{wnt}{Wnt}{Wingless}
\newacronym{pi3k}{PI3K}{phosphoinositide 3-kinase}
\newacronym{akt}{Akt}{Protein kinase B}
\newacronym{tgfb}{Tgf-$\beta$}{Transforming growth factor $\beta$}
\newacronym{egfr}{Egfr}{Epidermal growth factor receptor}
\newacronym{n1icd}{NICD}{\gene{NOTCH1} intracellular domain}

% Statistical methods and terms
\newacronym{auc}{AUC}{area under the curve}
\newacronym{hr}{HR}{hazard ratio}
\newacronym{roc}{ROC}{receiver-operating characteristics}
\newacronym{aic}{AIC}{Akaike information criterion}

% Computer algorithms
\newacronym{cbs}{CBS}{circular binary segmentation}
\newacronym{emd}{EMD}{empirical mode decomposition}
\newacronym{gistic}{GISTIC}{Genomic Identification of Significant Targets in Cancer}
\newacronym{glad}{GLAD}{Gain and Loss Analysis of DNA}
\newacronym{knn}{KNN}{k-nearest neighbour}
\newacronym{lda}{LDA}{linear discriminant analysis}
\newacronym{pam}{PAM}{prediction analysis of microarrays}
\newacronym{pca}{PCA}{principal component analysis}
\newacronym{nmf}{NMF}{nonnegative matrix factorization}
\newacronym{svm}{SVM}{support-vector machine}

% Organizations
\newacronym{clia}{CLIA}{Clinical Laboratory Improvement Amendments}
\newacronym{tcag}{TCAG}{The Centre for Applied Genomics}
\newacronym{who}{WHO}{World Health Organization}
\newacronym{magic}{MAGIC}{Medulloblastoma Advanced Genomics International Consortium}


%% End of the preliminary sections: reset page style and numbering.
\end{preliminary}


%% Chapter files

\nociteself{shih14,shih12,northcott12,diaz12}

\chapter{Introduction}
\label{ch:intro}

One in 285 children are diagnosed with cancer before the age of 20, and cancer is the second leading cause of death in children \citeref{ward14, murphy13}. Owing to advances in treatment, the 5-year survival of children with cancer has steadily increased from 63\% in 1975 to 85\% in 2006 \citeref{howlader14}. The improvements in survival, however, vary considerably across cancer types. In 1975, the 5-year survival rates for childhood leukemia and \gls{cns} tumour were 50\% and 57\%, respectively. In 2006, the survival rate for the former reached 87\% while the latter lagged behind at 74\% \citeref{howlader14}. Leukemia and \gls{cns} tumour are biologically very different types of cancer: they arise from different cells within different organs, hijack different cellular signaling programs to effect uncontrolled proliferation, and reside in different locations that permit different accessibilities to treatment. It should come as no surprise then that leukemia and \gls{cns} tumour respond differently to similar modern anti-cancer treatments (consisting of chemotherapy and radiotherapy). Furthermore, leukemia and \gls{cns} tumour can each be classified into additional cancer subtypes, which also response varyingly to treatment. Within leukemias, the 5-year patient survival of acute lymphocytic leukemia is 92\% and that of acute myelogenous leukemia is 66\% \citeref{howlader14}. Within \gls{cns} tumours, the 5-year patient survival of pilocytic astrocytoma is 94\% and that of medulloblastoma is 72\% \citeref{ostrom14}. Indeed, the responses of cancers to therapy depend on not only the affected tissue (blood vs.\ \gls{cns}) but also the cellular origin (lymphoid vs.\ myeloid and astrocytic vs.\ embryonal). Using anatomical locations and cellular appearance, clinicians classify cancer into different types in order to predict the responses to treatment. Current post-surgical treatment modalities function largely through one predominant mechanism (inhibiting cellular division and inducing apoptosis), but novel anti-cancer therapies are increasing in specificity against aberrant cells and diverging in mechanisms of action. Accordingly, the classification of cancer will become increasingly important for selecting the right treatment for each patient.

While the current classification of cancer using the primary site of occurrence and the morphology of the cell have been useful for predicting patient response, the advent of molecular profiling and sequencing technologies can refine this classification further and facilitate the development of therapies targeted against specific aberrations within a cancer. Currently, the \gls{who} classify \gls{cns} tumours into 86 distinct entities based primarily on histological appearance \citeref{who07}. (In 1993, \gls{who} recognized \emph{only} 36 subtypes of \gls{cns} tumours \citeref{kleihues93}.) Despite this detailed level of classification, the 5-year survival of children with \gls{cns} tumours has remained stagnant at about 75\% since 1996 \citeref{howlader14}. The discrepancy between discovery of new \gls{cns} tumour histotypes and the lack of therapeutic improvement suggest that the histological classification of \gls{cns} tumours may be insufficient to identify biologically similar tumours and facilitate the development of novel effective therapy. In particular, it was noted as early as 1971 that medulloblastoma with desmoplastic histology exhibits longer patient survival \citeref{chatty71} in response to radiotherapy; however, it remains unclear why the desmoplastic phenotype is associated with longer survival or which novel therapeutic agent may be effective against this histological variant. In comparison, a medulloblastoma tumour with a loss-of-function mutation in \gene{PTCH1} (endogenous suppressor of \gls{shh} signaling) would depend on hyperactive \gls{shh} signaling for growth, the patient would therefore be expected to --- and indeed does --- respond to inhibition of \gls{shh} signaling \citeref{rudin09}. Characterizing the genetic mutations and understanding the biology of a tumour can therefore help guide the discovery of novel therapies and the selection of suitable treatment modality or intensity.

In order to identify cancer mutations against which therapeutic intervention may be beneficial, we would need to distinguish between mutations contribute to tumourigenesis (known as \emphterm{driver} mutations) from those that do not (known as \emphterm{passenger} mutations). Cancer cells accumulate somatic mutations by having disrupted DNA damage response or DNA repair pathways. Somatic mutations generally occur stochastically (with some exceptions including antibody diversification by activation-induced cytidine deaminase). Most mutations in a cancer cell would thus be deleterious or neutral to the cell, and mutations that confer selective advantage would increase in \emphterm{cellular frequency} (proportion of cells habouring mutation) within a tumour. Not all mutations observed at high cellular frequency, however, contribute to tumourigenesis. Since multiple mutations may occur each time a cancer cell replicate its DNA and multiple cell divisions may occur before a new driver mutation arises, a population of cancer cells arising from the same parental cell will share one driver mutation and many additional mutations that do not contribute to tumourigenesis. Additionally, not all driver mutations become clonal (i.e. reach 100\% cellular frequency) due to interaction among cells. In glioblastoma multiforme, cancer cells expressing mutant EGFR secrete cytokines to promote the growth of cells expressing wild-type EGFR \citeref{inda10}; hence, the tumour heterogeneity is sustained by this paracrine mechanism and the \gene{EGFR} mutation does not become clonal. Taking these possibilities into consideration, high cellular frequency is neither a necessary nor a sufficient condition for a mutation to be a driver. Instead, a robust method for distinguish driver from passenger mutations would be to assess its \emphterm{recurrence frequency}: the frequency that a mutation is observed across samples. If a mutation drives tumour formation, we would expect to find across multiple tumours from different patients, provided that the tumours are biologically similar and arise through similar molecular mechanisms. Above all, in order to use recurrence frequency as a criteria for identifying driver mutations and recognizing genomic patterns, we would need to first classify the tumours into homogeneous groups.

This thesis will focus on the refining the classification of \emphterm{medulloblastoma}, a malignant brain tumour occurring in the cerebellum and the posterior fossa. \gls{who} classifies medulloblastoma as a grade IV (highly aggressive) embryonal tumour \citeref{who07}. Its diagnosis is made by the anatomical location of the tumour and histological morphology of the cells. Medulloblastoma was once a universally fatal disease; today, 58\% of patients are expected to live longer than 15 years \citeref{ward14}. To predict whether a patient will respond favourably to treatment, clinicians currently categorize medulloblastoma by such features as metastatic presentation and histological variants. Medulloblastoma may be divided into classic, desmoplastic, and anaplastic (large cell) histotypes. While this histological classification of medulloblastoma provides some prognostic value, it provide scant insight into potential biological mechanisms. Instead, several studies generated RNA expression profiles of primary medulloblastoma tumours using microarrays and sought to characterize medulloblastoma by patterns of RNA expression. Since each expression profile captures a snapshot of the overall molecular state of a tumour, biologically similar tumours exhibit similar expression profiles. Therefore, tumours with similar expression profiles can be clustered (grouped) together to discover biologically homogeneous molecular classes. By these clustering analyses, four molecular classes (henceforth known as \emphterm{subgroups}) of medulloblastoma were discovered: \emphterm{WNT}, \emphterm{SHH}, \emphterm{Group3}, and \emphterm{Group4} medulloblastoma \citeref{taylor12, northcott11a, kool12, remke11, cho11}. As the names suggest, WNT medulloblastoma have hyperactive \gls{wnt} signaling and SHH medullobalstoma have increased \gls{shh} signaling compared to the other subgroups, while Group3 and Group4 are less well defined \citeref{northcott11a, kool12, cho11}. We hypothesize that classifying medulloblastoma into these (relatively) homogeneous molecular subgroups will shed light on its cancer biology and consequently improve prediction of treatment response and point to novel therapeutic targets.

The remainder of this chapter will discuss contributing genetic and non-genetic factors in medulloblastoma and their implications for the classification and treatment of medullobastoma. At the end of this chapter, the main research objectives of this thesis are outlined. In \citech{mb-class}, we will propose and validate a method whereby medulloblastoma tumours may be classified in the clinical setting. The immediate utility of this molecular classification system for patient risk stratification will be shown in \citech{clin-prog}, and the potential implication of molecular classification for therapeutic discovery will be presented in \citech{target-id}. To make this thesis accessible to a broader audience, the \emphlab{Appendix} presents clarifications for topics including gene nomenclature, mouse model notation, and computational algorithms.


\section{Epidemiology and genetic predisposition}

Medulloblastoma occurs at an annual incidence of 4.1 per million children under 20, ranking as the most common type of malignant brain tumour in childhood \citeref{ostrom14}. The majority of patients with medulloblastoma are diagnosed before the age of 20, and the median age at presentation is 8 years \citeself{shih14}. As medulloblastoma is almost 10 more likely to afflict children than adults, medulloblastoma is a disease of childhood \citeref{smoll12}, suggesting perhaps genetics may play a role in this disease. Consistent with this notion, medulloblastoma can occur simultaneously in monozygotic twins \citeref{scheurlen96}, and having an affected sibling increases a child's risk of developing medulloblastoma by 4 fold \citeref{hemminki09}. While most medulloblastoma cases are sporadic with unknown genetic contribution, several genetic disorders can predispose children to developing medulloblastoma, as well as many other malignancies (\citetab{genetic-disorders}). For instance, mutations in genes of DNA damage response and repair pathways, including \gene{TP53}, \gene{ATM}, \gene{BRAC2}, \gene{NBN}, predispose a child to develop a spectrum of tumours, including medulloblastoma, at a young age \citeref{pearson82, barel98, guran99, yamazaki00, garre09, villani11, hart87, reiman11, offit03, hirsch04, reid05, dewire09, ciara10, alexiou12}.

Medulloblastoma can also arise as a consequence of germline mutations in developmental signaling pathways. Germline loss-of-function mutations in negative regulators of \gls{shh} signaling, including \gene{PTCH1}, \gene{PTCH2}, \gene{SUFU}, causes the basal cell nevus syndrome (Gorlin syndrome) and predispose patients to medulloblastoma, basal cell carcinoma, and other cancers. In the context of these mutations, hyperactive \gls{shh} signaling during neural development presumably would cause patients to develop SHH medulloblastoma. This notion is supported by mouse models with heterozygous \gene{Ptch1} mutation \citeref{goodrich97, wetmore00, wetmore01, pazzaglia02, gibson10} and the \gene{Sufu}\high{+/-} \gene{Trp53}\high{-/-} mouse model \citeref{heby-henricson12}, which both develop medulloblastoma with active Shh signaling. Further, Gorlin syndrome is highly prevalent in human patients with SHH medulloblastoma \citeref{kool14}, though it is yet unclear whether patients with Gorlin syndrome exclusively develop SHH medullobalstoma. Additionally, germline mutations in \gene{APC} activates \gls{wnt} signaling, manifests as Turcot syndrome, and predisposes patients to medulloblastoma. The tumours arising in these patients would be expected to be of the WNT subgroup; however, the crosstalk between the \gls{shh} and the \gls{wnt} signaling pathways in the context of neural development and different genetic backgrounds could modulate the phenotypic outcome of \gene{APC} mutation.

\begin{table}[h]
	\caption[Genetic disorders predisposing to medulloblastoma]
	{
		Genetic disorders predisposing to medulloblastoma
	}
	\label{tab:genetic-disorders}
	\footnotesize
	\setlength{\extrarowheight}{0.5em}
	\centering
	\begin{tabular}{l | l | l}
		\hline
		\textbf{Genetic disorder} & \textbf{Mutated genes} & \textbf{Reference} \\
		\hline
		Li-Fraumeni syndrome & \gene{TP53} & \citeplainref{pearson82, barel98, guran99, yamazaki00, garre09, villani11} \\
		Ataxia telangiectasia (Louis-Bar syndrome) & \gene{ATM} & \citeplainref{hart87, reiman11} \\
		Fanconi anemia & \gene{BRCA2} (\gene{FANCD1}) & \citeplainref{offit03, hirsch04, reid05} \\
		Nijmegen breakage syndrome & \gene{NBN} (\gene{NBS1}) & \citeplainref{bakhshi03, distel03, ciara10} \\
		Fragile X syndrome & \gene{FMR1} & \citeplainref{garre09, alexiou12} \\
		Neurofibromatosis type 1 (Von Recklinghausen's disease) & \gene{NF1} & \citeplainref{martinez-lage02, garre09} \\
		DICER1 syndrome & \gene{DICER1} & \citeplainref{slade11} \\
		Rubinstein-Taybi syndrome & \gene{CREBBP} & \citeplainref{bourdeaut14} \\
		Basal cell nevus syndrome (Gorlin syndrome) & \gene{PTCH1}, \gene{PTCH2}, \gene{SUFU} & \citeplainref{wolter97, taylor02, crawford09, garre09, brugieres10, jones11, brugieres12} \\
		Turcot syndrome & \gene{APC} & \citeplainref{hamilton95} \\
		\hline
	\end{tabular}
\end{table}

Germline mutation in \gene{SMARCB1} (\gene{SNF5}/\gene{INI}) had been associated with medulloblastoma \citeref{sevenet99, lee02}, but the brain tumours in these patients are now classified as \gls{atrt}, a tumour type that \gls{who} first recognized in its 1993 classification \citeref{kleihues93}. \gls{atrt} and medulloblastoma are similar by histology\citeref{utsuki03} and by \gls{mri} \citeref{koral08}. Currently, the \gene{SMARCB1} mutation is widely accepted as an diagnostic indicator of \gls{atrt} and distinguishes \gls{atrt} from medulloblastoma \citeref{biegel00, kraus02}. Indeed, \gene{SMARCB1} is an example in which a genetic mutation superseded histological diagnosis and redefined cancer classification.

While mutations are pivotal factors in the formation of medulloblastoma, numerous other factors shape the context under which the mutations exert their effects. The genetic mutations listed in \citetab{genetic-disorders} all have incomplete penetrance, highlighting the contribution of genetic background (as well as environmental factors). Disparate patterns of tumour incidence arising from mutations in negative regulators of the \gls{shh} signaling pathway (\gene{SUFU} and \gene{PTCH1}) point to several possible contributing factors, including the limitation of inbred mouse models, involvement of genes in multiple biological pathways, and potential modifying genetic polymorphisms or mutations. On the C57BL/6 genetic background, \gene{SUFU}\high{+/-} mice do not develop medulloblastoma within the same time frame as \gene{PTCH1}\high{+/-} mice, even though both \gene{SUFU} and \gene{PTCH1} negatively regulate Shh signaling \citeref{svard09}. Conversely, in a Manchester cohort of 171 Gorlin syndrome patients, germline \gene{SUFU} mutation had about 20 times higher risk than germline \gene{PTCH1} mutation to lead to medulloblastoma \citeref{smith14}. The observation that only 2 (1.7\%; $n = 115$) Manchester patients with \gene{PTCH1} mutation developed medulloblastoma \citeref{smith14} starkly contrasts the findings in inbred mouse strains (C57BL/6 and 129X1/Sv) with \gene{PTCH1} mutation, which spontaneously develop medulloblastoma at a frequency of about 18\% ($n = 507$; \citetab{ptch1-mmu-mb}) \citeref{goodrich97, wetmore00, svard09, pazzaglia09}. The difference in the penetrances of mutations in \gene{SUFU} vs.\ \gene{PTCH1} suggests that these genes may have other functions beyond \gls{shh} pathway or they regulate the pathway in slightly different ways. However, the inversion in relative penetrances of \gene{SUFU} and \gene{PTCH1} mutations in a human population compared to the mouse models suggest that polymorphisms or mutations in possible modifier genes may influence the manifestation of mutant \gene{SUFU} and \gene{PTCH1}. Alternatively, this observation may indicate that the \gls{shh} signaling pathway regulate neural development in critically different ways in mouse and human. Above all, these possibilities prompts the need for further studies in additional human populations and mouse strains in order to tease out the intricate interplay between mutant genes and genetic background. 

The phenotypic manifestation of genetic mutations is also influenced by the state of the mutation-harbouring cells. \gene{PTEN} (endogenous inhibitor of \gls{pi3k} signaling) is frequently homozygously deleted in medulloblastoma \citeref{northcott09}, suggesting that \gene{PTEN} may be a tumour supressor. Heterozygous germline \gene{PTEN} mutation, however, leads to adult tumours (PTEN hamartoma tumour syndrome) but not medulloblastoma \citeref{tan12}. Instead, one mutant copy of \gene{PTEN} leads to Lhermitte-Duclos disease, which presents as a benign tumour of the cerebellum \citeref{zhou03}. The cells in the tumour have complete loss of wildtype \gene{PTEN} expression \citeref{zhou03}. Curiously, the granule neurons exhibit dysplasia (disorganization of tissue structure), increased size, aberrant migration, but not abnormal cellular proliferation \citeref{endersby08}, in contrast to neoplastic neurons in medulloblastoma. This example illustrates that the cell state during nervous system development influences whether the cells progresses to neoplasia (cancer). In the case of \gene{PTEN}, other mutations would need to occur with \gene{PTEN} loss-of-function in order for patients to develop medulloblastoma. Indeed, patients with germline \gene{PTEN} and \gene{Pten}\high{+/-} mice do not develop malignant brain tumours, despite predisposition to various other tumours \citeref{endersby08}. In contrast to \gene{Pten} loss alone, concomitant overexpression of SHH (human protein) causes medulloblastoma in mouse following irradiation \citeref{hambardzumyan08}. More generally, the effect of a mutant gene is modulated by the state of the cell harbouring the mutation, and this state can be shaped by genetic background, cooperating mutations, and developmental signaling. As a cell mature and differentiate down various lineages during development, it may reach a state that permits a specific mutation (germline or somatic) to transform itself into cancer with the support of signals from surrounding microenvironment.


\section{Molecular biology}

Central nervous system development involves the coordination of innumerable signaling pathways and the highly controlled proliferation of cells in various lineages. Tumours arise when unfortunate accidents (e.g. germline mutations, genetic background, and DNA damage) coincide with conducive conditions (e.g. cell state and extracellular signals). Simply put, medulloblastoma occurs when normal neural development goes awry. Several developmental signaling pathways are important in medulloblastoma formation, including \gls{wnt}, \gls{shh}, Notch, \gls{pi3k}, and \gls{egfr} signaling. 

\subsection{\gls{wnt} signaling}

Patients with Turcot's syndrome (mutation in \gene{APC}) who developed medulloblastoma provided the one of the first clues that medulloblastoma involves activation of \gls{wnt} signaling \citeref{hamilton95}. Additionally, CTNNB1 nuclear localization and \gene{CTNNB1} mutation were frequently observed medullobalstoma \citeref{thompson06, ellison05}. CTNNB1 is the transcriptional factor that serves as the downstream effector of \gls{wnt} signaling \citefig{wnt-pathway}. As part of the CTNNB1 desctruction complex containing AXIN and APC, GSK and CSNK1A1 normally phosphorylates CTNNB1 at multiple serine/threonine residues (S45, T41 S37, and S33) within a region encoded by exon 3, flagging CTNNB1 for ubiquitination and subsequent proteosomal degradation \citeref{liu02}. Mutation in APC therefore aborgates the interaction between the destruction complex and CTNNB1, preventing the degradation of CTNNB1. Similarly, CTNNB1 mutations at or near the multiple phosphorylation sites also stabilizes CTNNB1, permitting its signaling \citeref{liu02, legoix99, shih12}. Curiously, a mouse model expressing stabilized human (S37F mutant) CTNNB1 under the mouse Prnp promoter did not develop medulloblastoma under a Trp53\high{-/-} background \citeref{kratz02}. Conversely, Gibson \emph{et al.}\ generated medullobalstoma using a different mouse model expressing stabilized CTNNB1 \citeref{gibson10}. In this Cre-recombination model, mouse Ctnnb1 is converted to a stabilized form by Cre-mediated somatic deletion of exon 3 (which was flanked by loxP sequences) \citeref{harada99}. Ctnnb1 is under endogenous promoter control, but deletion of exon 3 is controlled by expressing Cre under a selected promoter; therefore, expression of stabilized Ctnnb1 can be restricted to cell types that activate the selected promoter.  Ctnnb1\high{+/lox(ex3)} mice crossed to Atoh1-Cre mice did not produce progeny (Atoh1-Cre;Ctnnb1\high{+/lox(ex3)}) that develop hyperplasia or medulloblastoma, suggesting that Atoh1 expressing cells are not the developmental origin of Wnt-activated medulloblastoma \citeref{gibson10}. In contrast, Blbp-Cre;Ctnnb1\high{+/lox(ex3)} mice showed hyperplasia in the dorsal brainstem and eventually developed medulloblastoma in the context of Trp53 deficiency, which presumably caused additional tumourigenic mutations \citeref{gibson10}. In sum, only Blbp-Cre;Ctnnb1\high{+/lox(ex3)};Trp53\high{flx/flx} mouse developed Wnt-activated medulloblastoma, and these tumours arise from the dorsal brainstem and not the cerebellum (in contrast to the Shh-activated medulloblastoma described later). Tumour development not only required expression of the oncogene (stablized Ctnnb1) in a suitable cell type (Blbp expressing) but may also need additional mutations (secondary to Trp53 deletion).

\begin{figure}[H]
	\begin{center}
		%\includegraphics[width=\textwidth]{fig/pathway/wnt-pathway.pdf}
	\end{center}
	\caption[CTNNB1-dependent Wnt signaling pathway]
	{
		CTNNB1-dependent Wnt signaling pathway.
		In the absence of WNT ligand, CTNNB1 ($\beta$-catenin) is continuously marked for degradation by the CTNNB1 destruction complex, consisting of AXIN, APC, GSK, and CSNK1A1 (CK1). GSK and CSNK1A1 primes CTNNB1 by phosphorylation, which leads for ubiquinylation by the CUL1-containing E3 ligase complex and subsequent complete degradation by the proteome. Upon binding of WNT to FZD, DVL1 is activated via an unknown mechanism, and DVL1 phosphorylates LRP5/6, which then sequesters AXIN and frees CTNNB1 from the destruction complex. CTNNB1 accumulates in the cytoplasm and translocate to the nucleus to activate target genes such as \gene{MYC}, \gene{CCND1}, and \gene{AXIN2}. Two \gene{AXIN} genes exists in humans: \gene{AXIN1} and \gene{AXIN2}. GSK consists of two subunits: GSK3A and GSK3B (catalytic). CTNNB1 regulate expression in concert with the TCF/LEF family of co-transcription factors, such as TCF7, TCF7L1, TCF7L2, and LEF1. WNT signaling also activates many other pathways independently of CTNNB1; these pathways include: planar cell polarity, WNT-Ca\high{2+}, and others. Both WNT and FZD encompass a large family of proteins.
	}
	\label{fig:wnt-pathway}
\end{figure}


\subsection{\gls{shh} signaling}

Medulloblastoma can also arise from activated Shh signaling. Patients with Gorlin syndrome provided the first clue that \gls{shh} signaling is activated in a subset of medulloblastoma. Patients with germline mutations in \gene{PTCH1}, \gene{PTCH2}, or \gene{SUFU} can develop medullobastoma during childhood, and basal cell carcinoma and other tumours later in life \citeref{wolter97, taylor02, crawford09, garre09, brugieres10, jones11, brugieres12}. Moreover, somatic \gene{PTCH1} mutation, \gene{PTCH1} homozygous deletions and \gene{GLI2} amplifications are recurrently observed in medulloblastoma tumours \citeref{raffel97, northcott09, shih12}, which provide further evidence that \gls{shh} signaling is activated in some medulloblastoma cases. In mouse, homozygous Ptch1 loss-of-function mutation is embryonic lethal but a subset of heterozygous Ptch1 mutant mice develop cerebellar medulloblastoma \citeref{goodrich97, wetmore00}. svard09  
Loss of p53 but not ARF accelerates medulloblastoma in mice heterozygous for patched \citeref{wetmore01, berman02}.
Exhibit hyperplasia (cell mass with increased proliferation) in the external granule layer before medulloblastoma \citeref{oliver05, yang08}.

\begin{figure}[H]
	\begin{center}
		%\includegraphics[width=\textwidth]{fig/pathway/shh-pathway.pdf}
	\end{center}
	\caption[\gls{shh} signaling pathway]
	{
		Shh signaling pathway.
		GLI transcription factors, including GLI1, GLI2, and GLI3, are downstream effectors of \gls{shh} signaling. The proteome can completely degrade GLI proteins or proteolytically process them into activators or repressors forms. depending on the post-translational modifications on the full-length GLI proteins. GLI1 and GLI2 predominantly function in the activator form and activates transcription, while GLI3 mainly functions in the repressor form and represses transcription. Shh signaling modulate the marks on full-length GLI, thereby influencing downstream transcription.
		In the absence of SHH ligand, SMO activity is repressed by PTCH. Upon SHH binding to PTCH, this repression is relieved, leading to active SMO signaling, which favours the processing of GLI into the active form; consequently, transcription of target genes including cell cycle genes (\gene{CCND1} and \gene{CCNE1}), \gene{MYC}, and negative regulators of Shh signaling (\gene{PTCH1} and \gene{HHIP}).
		The precise mechanism whereby SMO (indirectly) activates GLI is not conserved between fly and mammals and remains unknown in mammals. Four proteins bind to and regulate GLI processing: KIF7, SUFU, SPOP, and BTRC. Both KIF7 and SUFU are negative regulators of Shh signaling via unclear mechanisms. The role of SUFU remains contentious; possible roles include: nuclear export of GLI, protection of GLI from degradation, recruitment of GLI3 for processing into the repressor form. Conversely, SPOP and BTRC ($beta$TRCP) serve to recognize GLI and they function as subunits of E3 ligase complexes, which mark (ubiquitinates) GLI for proteolytic processing by the proteome. The CUL1 containing complex is also known as the Skp, Collin1, F-box (SCF) complex, Via BTRC-mediated recognition, this complex ubiquitinates many other proteins, including CTNNB1 from the Wnt pathway.
		Mammals have three hedgehog ligands: DHH, IHH, and SHH. PTCH encompasses PTCH1 and PTCH2. PKA is a tetramer composed of two catalytic subunits (protein family includes PRKACA, PRKACB, and PRKACG) and two regulatory subunits (protein family includes PRKAR1A, PRKAR1B, PRKAR2A, and PRKAR2B).
	}
	\label{fig:shh-pathway}
\end{figure}

\begin{table}[h]
	\caption[Incidence of medulloblastoma in Shh-activated mouse model]
	{
		Incidence of medulloblastoma in Shh-activated mouse model
	}
	\label{tab:shh-mmu-mb}
	\footnotesize
	\setlength{\extrarowheight}{0.5em}
	\centering
	\begin{tabular}{l | l | l | l}
		\hline
		\textbf{Genotype} & \textbf{Genetic background} & \textbf{Percent incidence} (95\% CI) \\
		\hline
		% \citeref{wetmore00} 10/80 (est.), 48 week follow-up
		% \citeref{goodrich97} 6/27, 25 week follow-up?
		% \citeref{oliver05} 4/25, 25 week follow-up
		\gene{Ptch1}\high{+/-} & 129X1/Sv & 15 (10--22) \\
		% \citeref{pazzaglia02}, 2/26, 38 week followup
		% \citeref{pazzagli09}, 4/51 (probably contains pazzaglia02)
		\gene{Ptch1}\high{+/-} & CD-1 & 8 (2--19) \\
		% \citeref{svard09} 18/49, 1 yr follow-up
		% \citeref{pazzaglia09} 9/22, 1 yr follow-up
		\gene{Ptch1}\high{+/-} & C57BL/6 & 38 (27--50) \\
		% \citeref{svard09} 27/56, 1 yr follow-up
		\gene{Ptch1}\high{+/-};\gene{Sufu}\high{+/-} & C57BL/6 & 48 (35--62) \\
		% \citeref{goodrich97} 9/47, 25 week follow-up
		\gene{Ptch1}\high{+/-} & C57BL/6 $\times$ DBA/2 (B6D2F1) & 19 (9--33) \\
		% \citeref{wetmore00} 48/329 (est.), 48 week follow-up
		% \citeref{wetmore01} 62/440, 30 week follow-up (probably mostly same mouse as wetmore00)
		% \citeref{lee06} 4/62 (Ptch1+/- Ptch2+/+), 15 month follow-up
		\gene{Ptch1}\high{+/-} & C57BL/6 $\times$ 129X1/Sv & 13 (10--16) \\
		% \citeref{lee06} 15/97, 15 month follow-up
		\gene{Ptch1}\high{+/-};\gene{Ptch2}\high{+/-} & C57BL/6 $\times$ 129X1/Sv & 15 (9--24) \\
		% \citeref{lee06} 11/63, 15 month follow-up
		\gene{Ptch1}\high{+/-};\gene{Ptch2}\high{-/-} & C57BL/6 $\times$ 129X1/Sv & 17 (9--29) \\
		% \citeref{wetmore01} ?/68 (< 28 % of Ptch1+/-;Trp53+/-), 30 week follow-up
		%                     assume 10/68 ~= 14 % had tumour (since not different from Ptch1+/-)
		\gene{Ptch1}\high{+/-};\gene{Trp53}\high{+/-} & C57BL/6 $\times$ 129X1/Sv & 14 (7--25) \\
		% \citeref{wetmore01} 38/40, (Ptch1+/- Trp53-/-), 30 week follow-up
		% \citeref{lee07} 15/16 (Ptch1+/- Trp35-/-) (assume Ptch1+/- and Trp53+/- came from wetmore)
		\gene{Ptch1}\high{+/-};\gene{Trp53}\high{-/-} & C57BL/6 $\times$ 129X1/Sv & 95 (85-99) \\
		% \citeref{yang08}
		% mice from \citeref{ellis03, machold05, zhuo01}
		\gene{Atoh1-Cre};\gene{Ptch1}\high{flx/flx} & C57BL/6 $\times$ FVB/N & 100 \\
		\gene{GFAP-Cre};\gene{Ptch1}\high{flx/flx} & C57BL/6 $\times$ FVB/N & 100 \\
		% \citeref{svard09} 0/25, 1 yr follow-up
		\gene{Sufu}\high{+/-} & C57BL/6 & 0 (0--14) \\
		% \citeref{lee07} 0/55, 22 weeks follow-up
		\gene{Sufu}\high{+/-};\gene{Trp53}\high{+/-} & C57BL/6 $\times$ 129P2/OlaHsd & 0 (0--6) \\
		% \citeref{lee07} 32/55, 22 weeks follow-up
		\gene{Sufu}\high{+/-};\gene{Trp53}\high{-/-} & C57BL/6 $\times$ 129P2/OlaHsd & 58 (44-71) \\
		% \citeref{hallahan04}, 49/102, 1 year follow-up
		\gene{Neurod2-Smo}\high{+/W539L} & C57BL/6 & 48 (38-58) \\
		% \citeref{hatton08}, 62/66, 1 year follow-up
		\gene{Neurod2-Smo}\high{W539L/W539L} & C57BL/6 & 94 (85--98) \\
		\hline
	\end{tabular}
\end{table}

Incidence within a year.
\gene{Smo}\high{W539L/W539L} is also known as SmoA1 (activated mouse Smo) and corresponds to human SmoM2 (activated human Smo).


\begin{table}[h]
	\caption[Incidence of medulloblastoma in irradiated \gene{Ptch1}\high{+/-} mouse model]
	{
		Incidence of medulloblastoma in irradiated \gene{Ptch1}\high{+/-} mouse model
	}
	\label{tab:ptch1-mmu-mb}
	\footnotesize
	\setlength{\extrarowheight}{0.5em}
	\centering
	\begin{tabular}{l | l | l | l}
		\hline
		\textbf{Genetic background} & \textbf{Treatment} & \textbf{Percent incidence} (95\% CI) \\
		\hline
		% \citeref{pazzaglia02}, 2/26, 38 week followup
		% \citeref{pazzagli09}, 4/51 (probably contains pazzaglia02)
		CD-1 & non-irradiated & 8 (2--19) \\ 
		% \citeref{pazzaglia02}, 26/51 (3 Gy at P4), 38 week followup
		% \citeref{pazzaglia06}, 17/21 (3 Gy at P1), 1/33 (3 Gy at P10)
		%                        80 week followup, last tumour observed before 38 week
		CD-1 & 3 Gy at P1 & 81 (58--94) \\
		CD-1 & 3 Gy at P4 & 51 (37--65) \\
		CD-1 & 3 Gy at P10 & 3 (0--16) \\
		% \citeref{svard09} 18/49, 1 yr follow-up
		% \citeref{pazzaglia09} 9/22, 1 yr follow-up
		C57BL/6 & non-irradiated & 38 (27--50) \\
		% \citeref{pazzaglia09} 8/15, 1 yr follow-up 
		C57BL/6 & 3 Gy at P1 & 53 (27--79) \\
		\hline
	\end{tabular}
\end{table}


\subsection{Notch signaling}

Notch signaling is important in medulloblastoma.

Hallahan AR, Pritchard JI, Hansen S, et al. The SmoA1 mouse model reveals that notch signaling is critical for the growth and survival of sonic hedgehog鄄induced medulloblastomas [J]. Cancer Res, 2004,64(21):7794-7800.

\begin{figure}[H]
	\begin{center}
		%\includegraphics[width=\textwidth]{fig/pathway/notch-pathway.pdf}
	\end{center}
	\caption[Notch signaling pathway]
	{
		Notch signaling pathway.
		When NOTCH binds DLL expressed on the surface of an adjacent cell, NOTCH is extracellularly cleaved by ADAM10 or ADAM17 (TACE), and it is subsequently cleaved intracellularly by the gamma-secretase/presenilin complex. This proteolytic event releases the NOTCH intracellular domain (NICD) from the cell surface, allowing it to translocate to the nucleus in order to activate transcription of target genes such as \gene{MYC}, \gene{HES}, \gene{CDKN1A}, \gene{CCND3}. Mammals have four Notch family members, NOTCH1, NOTCH2, NOTCH3, NOTCH4, and these receptors can bind the ligands, DLL1, DLL3, DLL4, JAG1, JAG2. 
	}
	\label{fig:notch-pathway}
\end{figure}

\subsection{\gls{pi3k} signaling}

\gls{pi3k} signaling is important in medulloblastoma.

\begin{figure}[H]
	\begin{center}
		%\includegraphics[width=\textwidth]{fig/pathway/pi3k-pathway.pdf}
	\end{center}
	\caption[PI3K signaling pathway]
	{
		PI3K signaling pathway.
		PTEN is a negative regulator of PI3K signaling. PI3K signaling is activated downstream of many cell-surface receptors.
	}
	\label{fig:pi3k-pathway}
\end{figure}


\subsection{\gls{egfr} signaling}

\gls{egfr} signaling is important in medulloblastoma.

\begin{figure}[H]
	\begin{center}
		%\includegraphics[width=\textwidth]{fig/pathway/egfr-pathway.pdf}
	\end{center}
	\caption[Egfr signaling pathway]
	{
		Egfr signaling pathway.
		ERBB2 has multiple roles: it can function as co-receptor, it can recruit ligand, and it can dimerize and activate signaling in the absence of ligand.
	}
	\label{fig:egfr-pathway}
\end{figure}

Lack of ERBB2 protein expression in tumour is associated with favourable survival under standard treatment \citeref{gajjar04}, suggesting ERBB2 inhibition may be effective alone or in combination with conventional chemotherapy.

However, "Lapatinib was well-tolerated in children with recurrent or CNS malignancies, but did not inhibit target in tumor and had little single agent activity." \citeref{fouladi13}.


\subsection{cAMP signaling}

Both SMO and FZD are G-protein coupled receptors, and they regulate intracellular cAMP signaling.


Many of the aforementioned signaling pathways converge on the Myc family of proteins.

Although these pathways are presented as distinct linear series of signaling components, they are in fact highly interconnected by shared signaling components. How the pathways are connected are presumably dependent on the cell lineage or type.


\section{Histological classification}

Medulloblastoma and \gls{cnspnet} used to be grouped collectively as \gls{pnet}, but \gls{who} now recognizes these as distint diseases.

Medllomyoblastoma and melanocytic medulloblastoma used to be considered histological variants of medulloblastoma in the 1993 classification; however, in the latest \gls{who} classification (2007), they are no longer considered distinct histotype of medulloblastoma \citeref{who07}.

About 70\% of medulloblastoma are of the classic histological variant. Within this histotype, patient response to treatment varies, indicating that the classic histotype encompasses a biologically heterogenous group of tumours.

kappa
In order for a classification system to be useful, the system must be straight-forward to apply and the results should be reproducible.

Molecular classification can be applied objectively, the process can be streamlined and the results are reproducible.

Medulloblastoma with extensive nodularity is an additional histological subtype not currently recognized in the WHO classification. Patients with this histotype exhibit improved survivorship.

\section{Molecular classification}

We contend that the classifying medullobalstoma into subtypes with similar molecular biology can help distinguish tumourigenic (cancer promoting) from non-tumourigenic mutations and thereby facilitate the discovery of therapeutic targets. 

want molecular groups with an acceptable level of homogeneity

Classify medulloblastoma purely by survival?

each patient is his/her own class?
classify to a sufficient granularity so long as the classes are reproducible and useful.

Recent evidence suggests that medulloblastoma in fact comprises a group of biologically distinct molecular entities whose clinical and genetic differences may require separate therapeutic strategies \citeref{thompson06,kool08,northcott11a,remke11,cho11}. Four principal subgroups\citeref{taylor12} of medulloblastoma have been identified: WNT, SHH, Group3, and Group4, and there is preliminary evidence for clinically significant subdivisions of the subgroups \citeref{northcott11a,cho11,remke11,taylor12}. Targeted therapies based on the genetics of the disease are not currently in use. However, inhibitors of the \gls{shh} pathway activator, smoothened, have shown some early evidence of efficacy \citeref{rudin09}. With a deeper understanding of the genomics and biology of medulloblastoma subgroups, we hope to herald a new era of medulloblastoma treatment based on selective, specific, targeted therapy.

Molecular subgroups are reproduced by collaborators.
Attempt at molecular classification is not novel. Classification of \gls{cnspnet} is not very robust \citeref{li09}.

It does not always make sense to establish new molecular classes, though molecular profiling helps guide diagnosis \citeself{merino15}.

Sometimes molecular profiling does not produce different classes \citeself{agnihotri14}.

Ependymoma can also be molecularly classified into molecular classes \citeself{mack14}.


\section{Current and potential treatments}

% Pubmed query used on 2015-04-22
%     search term: "medulloblastoma"
%     article type: "Clinical Trial" 

Current therapy for medulloblastoma --- including surgical resection, radiation of the entire brain and spinal cord, and aggressive chemotherapy --- yields five-year survival rates of 60-70\% \citeref{gajjar06}. Survivors are often left with significant neurological, intellectual, and physical disabilities secondary to the effects of these non-specific, cytotoxic therapies on the developing nervous system \citeref{spiegler04,mabbott05}.

Combination of lomustine and vincristine (with or without prednisone) were ineffective compared to radiotherapy alone \citeref{tait90, evans90}.
Combination of vincristine and methotrexate and hydrocortisone was ineffective compared to radiotherapy alone \citeref{vaneys81}


Experimented with timing of post-operative radiotherapy and timing of chemotherapy (before, during, or after radiotherapy).
Experimented with different doses of radiotherapy
Experimented with different combinations of chemotherapeutic agents

chemotherapy
usually delivered over the course from months to a year

combination cisplatin, lomustine, and vincristine \citeref{jeng93, gajjar94, walter99}
combination of cisplatin, cyclophosphamide, and vincristine
cisplatin, lomustine, cyclophosphamide, etoposide, and vincristine

methotrexate used in high-risk patients
Given that methotrexate crosses the blood-brain barrier effectively, the use of this chemotherapeutic agent is usually restricted to high-risk cases.

procarbazine \citeref{grill05}.
thiotepa \citeref{razzouk95, chi04, grill05}
ifosfamide, cisplatin, and etoposide \citeref{saito11}

Most combination chemotherapy include cisplatin or carboplatin as perhaps the main chemotherapeutic agent, given its high efficacy when given alone \citeref{gaynon90, mastrangelo95}. All these are platinum deriviatives. The second-generation drug carboplatin can substitute cisplatin \citeref{fouladi09}, but the third-generation drug oxaliplatin shows increased neurotoxicity \citeref{avan15} and provided no benefit for recurrent medulloblastoma \citeref{fouladi06, hartmann11, geoerger11}.

failed phase II trials for chemotherapeutic agents
Docetaxel did not show efficacy for recurrent medulloblastoma \citeref{zwerdling06}.
topotecan \citeref{kadota99, blaney96}.
Idarubicin \citeref{arndt98}.
paclitaxel \citeref{hurwitz01}.

some efficacy
irinotecan \citeref{bomgaars07, kim13, turner02}

Busulfan and thiotepa too highly toxic in previously irradiated patients \citeref{valteau-couanet05}, thought it was promising in patients who were not previously irradiated \citeref{kalifa92}.

Carboplatin was not effective on recurrent medulloblastoma \citeref{friedman92} but was effective upfront \citeref{mastrangelo95, fouladi09, gaynon90}.


Cisplatin is the backbone of most combination therapy for medulloblastoma. It only crosses the blood-brain barrier under hypoxic conditions \citeref{minami96}. Another study found that cisplatin crosses blood brain barrier with repeat doses \citeref{namikawa00}.


Temozolomide treatment for recurrent medulloblastoma showed promise \citeref{broniscer07, nicholson07, ruggiero10, cefalo14, grill13}.

eight drugs in one day (vincristine, procarbazine, hydroxyurea, cisplatin, cytarabine, prednisone, cyclophosphamide, and carmustine) by itself possibly followed by radiotherapy \citeref{geyer14}. Of the treated patients, 6\% died due to chemotherapy.

eight drugs in one day (vincristine, hydroxyurea, procarbazine, lomustine, cisplatin, cytarabine, methylprednisone, and either cyclophosphamide or dacarbazine), followed by methotrexate \citeref{gentet95}.

eight drugs in one day last used by itself in \citeref{zeltzer99}
"VCP plus XRT is a superior adjuvant combination compared with 8-in-1 chemotherapy plus XRT"

eight drugs in one day followed by etoposide and carboplatin \citeref{oyharcabal-bourden05}.

leucovorin administered to prevent harmful effect of methotrexate.

no longer commonly used due to lack of efficacy or intolerable side-effects:
carmustine, methylchlorethamine, prednisone \citeref{fewer72, cangir84, dreyer03, krischer91}. Ifosfamide was not effective as a single-agent therapy in recurrent medulloblastoma (and other pediatric brain tumours) \citeref{heidman95, chastagner93}, but remained in use with combination chemotherapy \citeref{saito11, geyer05}.

phase II trial of farnesyl transferase inhibitor (tipifarnib) was not promising \citeref{fouladi07}.

phase II trial of ERBB signaling (lapatinib) did not show efficacy \citeref{fouladi13}.


autologous stem cell transplant in high-risk or recurrent cases following myeloablative chemotherapy \citeref{perez-martinez05, gajjar06, kadota08, rosenfeld10, dunkel10, guruangan98, finlay96}

Sensitivity to anticancer therapies, likely due to preservation of Bax apoptotic pathway in medulloblastoma \citeref{crowther13}.
In comparison, \gls{cnspnet} responds poorly to standard treatment \citeref{who07}.

Hyperfractionated versus conventional radiotherapy followed by chemotherapy in standard-risk medulloblastoma: results from the randomized multicenter HIT-SIOP PNET 4 trial. No difference \citeref{lannergin12}.

In an ongoing clinical trial, prolonged dose-intensive chemotherapy (with cisplatin, cyclophosphamide, vincristine, etoposide) for patients with medulloblastoma yielded no improvement in survival (and also resulted several treatment-associated deaths) \citeref{strother14}. The collection of active chemotherapy agents for \gls{cns} tumours has not changed much in the past two decades, aside from the relatively new inclusion of methothrexate \citeref{strother14}. Although methotrexate shows promising in high-risk medulloblastoma patients, its long-term neurotoxicity poses serious concerns \citeref{rutkowski05, mitby03, chi04}. These finding suggests that treatment of medulloblastoma with existing chemotherapeutic agents may be reaching a ceiling.

Dendritic cell-based tumour vaccination was not promising \citeref{ardon10}.
Adoptive immunotherapy using lymphokine-activated killer cells and interleukin-2 was not promising \citeref{sankhla96}.

Rationale for targeted therapies in medulloblastoma \citeref{macdonald14}.


\section{Risk stratification of patients}

Patients are stratified based into standard-risk and high-risk groups based on evidence of metastasis and size of the residual tumour after surgery. Additionally, infants under 3 years are not irradiated to prevent impairment of neurological function by craniospinal irradiation.


Current medulloblastoma protocols stratify patients based on clinical features: patient age, metastatic stage, and extent of resection.

scheme different aross countries

age < 3, gross total resection, anaplastic histology, presentation with metastasis \citeref{polkinghorn07}
subtotal resection and metastasis classified as high-risk \citeref{gajjar06}.

age < 3 \citeref{geyer05, rutkowski05, polkinghorn07, ashley12, strother14}
some groups use different cutoffs: infant as age < 4 \citeref{white98, kool12, vonbueren11}
age < 5 \citeref{grill05, rutkowski10}

Japan does not use age cutoff for irradiation \citeref{yasuda08}.
No irradiation for age < 2 \citeref{pezzotta96, geyer94}.
infants of less than 1 year of age often receive reduced dose chemotherapy as well


\section{Research objectives}

Our study will focus on the three obstacles that hinders the development of targeted therapy against molecular subgroups of medulloblastoma. First, current methods for classifying medulloblastoma into subgroups molecular classes are difficult to apply in the clinical setting. Second, there are few actionable drug targets for WNT, Group3, and Group4 medulloblastomas. Third, current clinical prognostication of medulloblastoma does not consider molecular subgroups. The objectives are to provide viable solutions to these issues and to demonstrate the clinical significance of molecular classification.

\subsection*{Aim I: Molecular classification of medulloblastoma for clinicians}

Although the retrospective classification of medulloblastoma has been scientifically informative, molecular subgrouping has not been applied in the context of a prospective clinical trial. One major obstacle is the lack of fresh-frozen samples for most clinical cases. Expression profiling, on which molecular classification was based, depends on the availability of high-quality RNA. In contrast, clinical samples are routinely subjected to formalin-fixation and paraffin-embedding, which preserves tissue integrity but causes nucleic acid degradation. To facilitate the development of therapy specifically targeted against molecular subgroups, we sought to establish an molecular subgrouping assay that can be clinically applied on \gls{ffpe} samples. We have established an analytic pipeline to molecular subgrouping using expression data generated by nanoString assays, and demonstrated its high classification accuracy on \gls{ffpe} samples \citeself{northcott12}. To further make the assay clinically applicable, we have implemented several quality-control measures that identify cases which cannot be reliably assigned molecular subgroup, due to poor specimen quality or assay reaction failure.

\subsection*{Aim II: Clinical prognostication within molecular subgroups of medulloblastoma}

Prior clinical prognostication studies in medulloblastoma have identified biomarkers without discriminating between the molecular subgroups of medulloblastoma. Given that medulloblastoma subgroups are biologically and molecular distinct disease entities, we hypothesized that incorporating molecular subgroup into prognostication can enhance the accuracy of survival prediction and improve the reliability of risk stratification. Practical and reliable identification of risk could allow for therapy intensification in high-risk children to improve survival and therapy de-escalation in low-risk children to avoid complications of therapy. By identifying clinical and molecular biomarkers within medulloblastoma subgroups, we have designed risk stratification schemes for SHH, Group3, and Group4 medulloblastoma that can achieve unpredented levels of prognostic accuracy.

\subsection*{Aim III: Discovering therapeutic targets by genomic profiling of medulloblastoma}

Following the adoption of the molecular classification of medulloblastoma, we then sought to identify molecular targets in medulloblastoma. Unlike SHH medulloblastomas, actionable therapeutic targets for WNT, Group3, and Group4 tumours have yet been identified. Since prior attempts have been underpowered to discriminate the genomic differences among the four molecular subgroups, the \gls{magic}, consisting of scientists and physicians from 43 cities across the globe, has gathered $>1200$ medulloblastomas. We analyzed the genomic copy-number profiles of the tumours by \gls{snp} arrays and identified genes and pathways that characterize each medulloblastoma subgroup \citeself{shih12}.


\chapter{Molecular classification of medulloblastoma for clinicians}
\chaptermark{Molecular classification}
\label{ch:mb-class}

\begin{objective}
To develop a clinically applicable assay for molecular classification of medulloblastoma.
\end{objective}

class prediction and class discovery
compare to taxonomic identification and taxonomic classification

\section{Materials and methods}


\section{Results}

The nanoString nCounter technology \citeref{geiss08} was used to directly measure the expression level of 22 medulloblastoma subgroup specific signature genes. The nanoString assay directly interrogates nucleic acid levels without \gls{pcr} amplification (or other enzymatic reactions) in multiplexed system, using pairs of fluorescent probes that bind to target sequences. We developed an analytic method that can accurately predict molecular subgroups of medulloblastoma, even on archival \gls{ffpe} samples \citeself{northcott12}.

A set of widely used classifiers (e.g. support-vector machine, linear discriminant analysis, multinomial logistic regression, k-nearest neighbour, pattern analsis of microarrays) were trained using a training set of 101 medulloblastomas with known subgroup affiliations. The classifiers were tuned and assessed using cross-validation. The most accurate classifier (across all tested accuracy measures) was selected for classification.

\begin{figure}[hb]
	\begin{center}
		\includegraphics[width=\textwidth]{fig/nanostr-class/nanostr-valid.pdf}
	\end{center}
	\caption[Validation of classification assay on independent medulloblastoma cohorts]
	{
	Validation of classification assay on independent medulloblastoma cohorts.
	\textbf{a-c}, Expression heatmaps of nanoString class-predicted medulloblastomas of known subgroup status as published by Remke et al.\citeref{remke11} (a), Cho et al.\citeref{cho11} (b), and Kool et al.\citeref{kool08} (c). Samples are sorted according to subgroup predictions. Known expression subgroup affiliations and erroneously classified cases are marked above the heatmap.
	\textbf{d}, \emph{Left}, Pie chart depicting the known subgroup distribution of medulloblastomas from the three independent cohorts analyzed in \textbf{a-c} ($n = 130$) and the subgroups predicted by nanoString profiling. Misclassified cases are marked within each slice according to the predicted subgroups. \emph{Right}, Pie chart of class prediction accuracy ($127/130$) from the validation set. Adapted from Northcott et al.\citeself{northcott12}
	}
	\label{fig:nanostr-valid}
\end{figure}

\clearpage

The assay was validated on an external set of 130 non-overlapping medulloblastomas, and it achieved an accuracy of 98\% (\citefig{nanostr-valid}). Further, the assay yielded reproducible predictions when repeated in three independent laboratories \citeself{northcott12}. The clinical applicability of the assay was demonstrated by its predictive accuracy on \gls{ffpe} samples of archival ages $\leq 8$ years (\citefig{nanostr-ffpe}). The accuracy decreased on older \gls{ffpe} samples, presumably due to poorer RNA integrity, though standard measurements of RNA quality were not correlated with accuracy \citeself{northcott12}.

\begin{figure}[ht]
	\begin{center}
		\includegraphics[width=\textwidth]{fig/nanostr-class/nanostr-ffpe.pdf}
	\end{center}
	\caption[Classification performance on formalin-fixed paraffin embedded archival samples]
	{
	Classification performance on formalin-fixed paraffin embedded archival samples.
	\textbf{a}, Class prediction accuracy in relation to sample age of archival medulloblastomas stored as \gls{ffpe} material ($n = 84$). Samples obtained within the past 8 years exhibit accuracies of $\geq 95\%$, as demarcated by the red vertical line.
	\textbf{b}, Heatmap of nanoString data showing class predictions for \gls{ffpe} cases of $\leq 8$ years confidently predicted by the assay ($n = 28$). Samples are sorted according to subgroup prediction. All cases satisfying prediction probability threshold were assigned to the correct subgroup ($28/28$). Adapted from Northcott et al.\citeself{northcott12}
	}
	\label{fig:nanostr-ffpe}
\end{figure}

Since the initial publication of the assay for molecular classification, we have analyzed over 1000 medulloblastoma samples and identified a few cases were replicate assays yielded conflicting results. Further examination revealed that poor sample quality and suboptimal assay conditions likely contributed to classification discrepancies. Therefore, additional quality control measures were implemented, which are especially important for developing this assay further for \gls{clia} certification.

Given that standard measurements of RNA quality were insufficient for predicting assay accuracy \citeself{northcott12}, the mean signals of the endogenous control probes included in the nanoString assay were used to assess whether sufficient quantities of intact, undegraded were present in the samples, using a outlier detection method. A Gaussian mixture model was fitted to all collected nanoString data to establish the nominal range for mean endogenous-control signals. Samples with mean signals that deviate significantly from this range at a significance level of 0.01 were identified as outliers. Such samples, due to extensive RNA degradation, cannot be assigned a molecular subgroup, and they may require classification using DNA copy-number or methylation profiling.

Samples with sufficiently high-quality RNA may yet yield uninterpretable results when suboptimal assay conditions confound the measurements. Therefore, signals from positive control and negative control probes are examined to identify assay reactions that may have failed and hence produced unreliable measurements. The current collection of nanoString data was used to establishe the nominal range of positive and negative signals, using Gaussian mixture and multiple negative binomial models, respectively. As above, measurements that deviate significantly from the nominal range at a significance level of 0.01 were considered outliers. Samples that fail this quality control criterion may simply be run again.

Furthermore, multi-sample assays are not amendable to reproducible clinical analysis of samples in real-time, owing to time constraints and potential batch effects. Single-sample nanoString assays were therefore tested for concordance with previous results. With the appropriate quality control and improved normalization procedures implemented, 100\% concordance was achieved with single-sample assays, which further enhanced the clinical utility of molecular classification assay.

\section{Discussion}

Above all, a rapid, reliable, and reproducible assay was developed for assigning molecular subgroups to clinical samples, available as frozen or recent \gls{ffpe} material, and this assay has been developed further use in a clinical laboratory. Critically, stringent quality control must accompany the nanoString classification assay, lest its potentials be shadowed by concerns of reproducibility and predictability, an ignominy that has long plagued the microarray technology \citeref{shi08,deronde10,weigelt10,ein-dor06,frantz05,michiels05,ioannidis05,marshall04,check04,tan03,tilstone03}.

Classification of medulloblastoma led to many discoveries \citeself{shih14,shih12,perreault14,kool14,ramaswamy14,ramaswamy13,remke13,dey13,zhukova13,dubuc13,dubuc12,wu12,jones12}.

\clearpage

\chapter{Clinical prognostication within molecular subgroups of medulloblastoma}
\chaptermark{Clinical prognostication}
\label{ch:clin-prog}

\begin{objective}
We aim to stratify patients into risk groups based on clinical and molecular biomarkers within medulloblastoma subgroups for the purpose of effecting risk-adaptive treatment.
\end{objective}

Medulloblastoma was a uniformly fatal disease with a survival duration of mere months until the introduction of systematic irradiation of the entire central nervous system in the 1940s \citeref{paterson53}. Prior to the adoption of craniospinal (whole brain and spine) irradiation, medulloblastoma cases treated surgical resection and localized radiotherapy recur with metastases in the subarachnoid space \citeref{mcfarland69}. Although the propensity of medulloblastoma to metastasize necessitated whole \gls{cns} radiotherapy, exposing the developing brain to irradiation led to long-term neuropsychological sequelae that were beginning to be documented in the 1960s \citeref{mcfarland69, gudrunardottir14}. The integration of chemotherapy in the 1970s into the standard treatment of medulloblastoma led to a concomitant rise in patient survival \citeref{gudrunardottir14}. Chemotherapeutic drugs, however, can also have immediate and long-term adverse effects on neurocognitive function \citeref{khong06, zeller13, avan15}. Today, patients with medulloblastoma are treated by surgical resection, followed by craniospinal irradiation and combination chemotherapy. While advances in imaging and surgical technologies have largely eliminated operative mortality and minimized damage to the brain during resection, craniospinal irradiation and combination chemotherapy continue to impair neural development and cause debilitating neurocognitive decline of long-term survivors \citeref{palmer13, schreiber14, knight14, gudrunardottir14}. With modern treatment, patients with medulloblastoma can be cured, but at great cost to their qualities of life.

Aside from impairing brain development, chemotherapy and radiotherapy can cause various other side-effects in long-term survivors of childhood cancer. They can cause endocrinological complications, resulting in delayed puberty, hypothyroidism, growth hormone deficiency, and stunted growth; and neurological complications, leading to symptoms including limb weakness, prolonged pain, reduced sense of touch, balance problems, permanent hearing loss, blindness, seizures, tremors, and paralysis \citeref{armstrong09, xu03, edelstein11, christopherson14, avan15, boman09, duffner98}. Ironically, these anti-cancer treatments can also predispose patients to second cancers \citeref{armstrong09, packer13, christopherson14, avan15, boman09, duffner98}.

The Childhood Cancer Survivor Study reported sobering statistics for adult survivors of childhood cancers and highlighted adverse, long-term socioeconomic consequences of chemotherapeutic and irradiation treatment \citeref{hudson03, mitby03}. The survivors, compared to unaffected siblings, are 5 times more likely to suffer from functional impairments that prohibits independent living, 2 times more likely to earn less than \$20~000 in annual household income \citeref{hudson03}. (Most participants of this study were based in the United States, in which the median household income is more than \$50~000 during the same period \citeref{denavas14}.) Specifically for childhood \gls{cns} cancers, the survivors are 18 times more likely to suffer from functional impairments \citeref{hudson03}. Moreover, 70\% of survivors diagnosed with \gls{cns} cancer before age 6 years require special education services to cope with learning or emotional difficulties \citeref{mitby03}. The survivors' use of special education is directly related to the treatment received: cranial irradiation treatment alone increases the odds of needing special education by 7 times, while methotrexate treatment alone increases this odds by 1.3 times, compared to unaffected siblings \citeref{mitby03}. While the long-term neurocognitive effects of chemotherapy, radiotherapy, and brain tumour itself are intertwined, these findings suggest that cranial irradiation may be the most damaging treatment, and chemotherapy, albeit less harmful, is not entirely innocuous many years after treatment either.

Radiotherapy causes apoptosis (programmed cell death) of dividing cancer cells, but it can also cause normal dividing cells to die, leading to physical, endocrinologic, and neurologic sequelae. In a developing brain, dividing neural progenitors are sensitive to irradiation. Additionally, quiescent neural progenitors or stem cells can also incur radiation-induced DNA damage whose effect may manifest later in life. In patients with acute lymphoblastic leukemia, cranial radiation causes decline in intelligence, and this decline is progressive, showing more impairment of cognitive function with increasing time since radiation therapy \citeref{krull13}. Nowadays, cranial irradiation is reserved for the fewer than 20\% of children with acute lymphoblastic leukemia who are considered to be at high risk for \gls{cns} relapse, in order to spare the maturing brain of the neurotoxic side-effect of radiotherapy \citeref{pui08}. Conversely, brain tumours usually require irradiation for complete tumour eradication and long-term patient survival, though clinicians are increasingly aware of neurologic sequelae following radiotherapy. A recently completed prospective trial assessing 54 Gy conformal (targeted against the tumour bed) radiotherapy in low-grade glioma patients revealed a striking correlation between age at treatment and subsequent decline in \gls{iq} score: the younger the survivor was during conformal radiotherapy, the more severe was the decline in intelligence \citeref{merchant09}. Similarly, younger children with medulloblastoma treated with high-dose irradiation had worse progressive decline in intellectual outcome and academic performance compared to children of older age at diagnosis \citeref{radcliffe94, palmer13, schreiber14, knight14, mulhern98}. Even with a reduced dose of craniospinal radiotherapy, survivors continue to show progressive decline in intellectual and academic outcomes \citeref{ris13, mulhern05, ris01, mulhern98}.

Several attempts have been made over the past three decades to minimize exposure of the developing brain to irradiation. One of the first prospective trial in reducing radiotherapy for patients with medulloblastoma reported an increased rate of tumour recurrence and consequently closed early \citeref{deutsch96}, highlighting the need for a planned strategy for salvaging non-responding disease in order to maintain patient survival. While earlier attempts at reducing craniospinal irradiation led to poorer survival \citeref{thomas00, deutsch96, bailey95, bouffet92}, other attempts at reducing craniospinal irradiation did not compromise patient survival by incorporating chemotherapy into the treatment regiment \citeref{packer06, packer94, packer99, kim10, halberg91, oyharcabal-bourden05, sung13}. Extending this approach, numerous oncology groups sought to postpone or eliminate radiotherapy in young children by using chemotherapy to control or eradicate the tumour \citeref{mulhern89, rutkowski05, jeng93, duffner93, geyer94, gentet95, duffner99, walter99, oyharcabal-bourden05, grill05, geyer05, dhall08, sands10, grundy10, vonbueren11, saha14, yasuda08, kellie02, white98, strauss91}. Unfortunately, postsurgical chemotherapy alone often cannot achieve complete response of the residual tumour, leading to eventual use of radiotherapy. For example, combination chemotherapy with vinblastine, cisplatin, and etoposide was insufficient by itself to induce complete remission of residual medulloblastoma, and patients often progress during chemotherapy. \citeref{jeng93, gajjar94, walter99}. Patients with chemoresistance tumours could be salvaged with subsequent radiotherapy; however, survivors treated with delayed radiotherapy can still suffer from neurodevelopmental deficits \citeref{gajjar94, grill05}.

In yet other trials, clinicians have successfully used high-dose combination therapy to eliminate irradiation treatment in young children with non-metastatic medulloblastoma. Geyer \emph{et al.}\ showed that two combination therapy regimens (vincristine, cisplatin, cyclophosphamide, and etoposide; or vincristine, carboplatin, ifosfamide, and etoposide) could both obviated the need for radiotherapy in patients with no metastatic or residual tumour after surgery \citeref{geyer05}.  Similarly, Grill \emph{et al.}\ used combination chemotherapy (carboplatin, procarbazine, etoposide, cisplatin, vincristine, cyclophosphamide) and salvaged progressive medulloblastoma with radiation and additional high-dose chemotherapy (busulfan, thiotepa, and melphalan) \citeref{grill05}. Patients without metastasis or residual tumour exhibited favourable outcome, and the survivors in this study also had improved intellectual outcome compared to radiotherapy-treated patients \citeref{grill05}. Rutkowski \emph{et al.}\ used combination chemotherapy alone (including vincristine, carboplatin, etoposide, cyclophosphamide, and methotrexate) and achieved great survival outcome for children without metastasis or residual tumour \citeref{rutkowski05}. Decline in \gls{iq} was still evident among the non-irradiated survivors, albeit less severe than those who had received radiotherapy \citeref{rutkowski05}. Chi \emph{et al.}\ using combination chemotherapy (vincristine, cisplatin, etoposide, cyclophosphamide, methotrexate, and thiotepa) and autologous stem cell transplant following the consolidation chemotherapy, yielded complete response in macroscopically metastatic medulloblastomas without the need for radiotherapy \citeref{chi04}. The efforts in using chemotherapy to ward off radiotherapy resulted in the adoption of withholding or postponing radiotherapy for treating young children with medulloblastoma in most of North America and Europe.

Based on the results of more recent trials, however, clinicians remain divided on the use of radiotherapy in young children. The HIT~2000 trial (2001--2005) confirmed that combination chemotherapy (cyclophosphamide, vincristine, carboplatin, etoposide, and methotrexate) can maintain favourable survivorship without radiotherapy in nonmetastatic medulloblastoma \citeref{vonbueren11}. Unless complete remission was achieved following induction, the authors recommended contingent treatment with local radiotherapy, secondary surgery, and consolidation chemotherapy (cisplatin, lomustine, and vincristine) \citeref{vonbueren11}. Conversely, the COG-P9934 trial (2000--2006) brought back unconditional, planned irradiation and showed that conformal radiotherapy (localized to posterior fossa and tumour bed) in addition to chemotherapy achieved superior progression-free survival than chemotherapy alone by comparing against the POG-9233 trial \citeref{ashley12}. Despite known risk for long-term neurotoxicity in children, the authors contended that radiotherapy is still required for optimal survival of patients with nonmetastatic medulloblastoma \citeref{ashley12}.

Given the limited capability for chemotherapy to replace radiotherapy in all patients, it is important to select the patients who are at low risk of progressive or recurrent medulloblastoma and evaluate their candidacy for therapy de-escalation. Currently, the main risk factors for medulloblastoma relapse are residual tumour after subtotal surgical resection and presentation with metastasis either macroscopically or in the cerebral spinal fluid. As earlier trials have shown, patients with residual disease or metastasis are poor candidates for reduced or withheld radiotherapy \citeref{grill05, rutkowski05}, though one trial have some success with metastatic medulloblastoma \citeref{chi04}. Additionally, the 5-year overall survival rate for nonmetastatic, completely resected medulloblastoma in young children (age $< 3$) of the HITSKK92 trial (1992--1997) was an impressive 93\% ($n = 17$), while the POG-9233 trial (1992--ongoing) reported a 5-year overall survival rate of 43\% ($n = 37$) for nonmetastatic, completely resected medulloblastoma. Indeed, the results from the two trials are difficult to compare, given differences in the chemotherapy regimen, surgical resection, imaging technologies, and supportive care. Nonetheless, the striking difference in survival for what should be similar cases of medulloblastoma suggest that the two cohorts are, in fact, biologically dissimilar. Accordingly, the prognostic factors currently used in patient risk stratification, metastasis at diagnosis and extent of resection, fail to accurately identify favourable responders to chemotherapeutic treatment.

What has been missing in past clinical trials is the classification of medulloblastoma into molecular subgroups. As medulloblastoma subgroups exhibit different survivorships (\citefig{surv_mb-subgroups}), we believe that subgroups may be useful for risk stratification of patients. Further, given the distinct origins of subgroups \citeref{gibson10}, we hypothesize that prognostic markers would be influenced by subgroups. That is, some markers may be prognostic only in specific subgroups while others may be surrogate markers of subgroup status and have no prognostic value themselves. Therefore, by incorporating molecular subgroups into risk stratification, we believe that we would be able to more accurately predict favourable responders to treatment and obviate the need for indiscriminate administration of intensive treatment to children, who will suffer long-term treatment-induced toxicities. By improving risk stratification and tuning treatment intensity, we hope to minimize collateral damage to the patient's developing brain and preserve the survivor's quality of life.

\begin{figure}[h]
	\begin{center}
		\includegraphics[width=0.7\textwidth]{fig/magic-clin/surv_mb-subgroups.pdf}
	\end{center}
	\caption[Overall survival curves for molecular subgroups of medulloblastoma]
	{
		Overall survival curves for molecular subgroups of medulloblastoma.
		Numbers below x-axis represent patients at risk of event; statistical significances are evaluated by log-rank tests; \gls{hr} estimates are derived from Cox proportional-hazards analyses.
	}
	\label{fig:surv_mb-subgroups}
\end{figure}

\clearpage

The current paucity of markers used in risk stratification is not due to a lack of biomarker studies. Indeed, the medulloblastoma literature is rife with reports of prognostic markers. Most of the purported markers, however, do not reproducibly predict survival in different cohorts due to small sample sizes and distributional differences in underlying covariates \citeself{shih14}. We purport that disagreements in prior biomarker identification attempts may be explained by differences in the composition of medulloblastoma subgroups in the different cohorts. For example, patients with desmoplastic medulloblastoma often exhibit better survival \citeref{chatty71, rutkowski05, kool12, pietsch14, dhall08, vonbueren11, ashley12, grundy10}, and the discordant survival outcomes between the POG-9233 and the COG-P9934 trials could be due the latter having a higher proportion of desmoplastic medulloblastoma \citeref{ashley12}. The aforementioned difference in survival outcomes between the POG-9233 and the HITSKK92 trial could also be due to a similar reason. Indeed, the small sample sizes in these trials (POG-9233, $n = 112$; HITSKK92, $n = 62 $; COG-P9934, $n = 82$) makes uneven distribution of covariates likely, and these covariates may be responsible for the outcome differences. Further, while desmoplastic histology has been proposed to be one such covariate, it is likely that unobserved covariates such as molecular subgroups and genetic mutations may better explain the differences in response. In addition to its status as a favourable prognostic factor, desmoplasia has also been reported to be a prognostic factor for poor survival in medulloblastoma by some \citeref{park83, gajjar06, rutkowski10} and an insignificant factor by others \citeref{pietsch14, lannering12}. High interobserver and intraobserver variability in histological examination may be responsible for these discrepancies. Therefore, difficulty and variability in assessment can limit the utility of a biomarker, and competing covariates should be assessed carefully.

In order to mitigate the problems of small sample size and competing unobserved covariates, we assembled an large, international cohort of 673 medulloblastoma samples with clinical annotation. The cytogenetic and focal copy-number events were determined using copy-number profiling on this discovery cohort. We identified subgroup-specific cytogenetic events and integrated them with clinical variables to develop subgroup-specific, multivariate risk-stratification models based on the discovery cohort. In order to validate the models and ensure that the technique was generalizable to routine pathology laboratories, we then studied a panel of six cytogenetic biomarkers (\gene{GLI2}, \gene{MYC}, 11, 14, 17p, and 17q) using interphase \gls{fish} on an \gls{ffpe} medulloblastoma tissue microarray that includes a set of 453 medulloblastomas that were treated at a single center and does not overlap with the discovery cohort.

The size of our discovery and validation cohorts provides unprecedented power for clinical prognostication and enables comprehensive, multivariate modeling of patient survival to identify robust prognostic markers \citefig{meta_cyto-markers}. In this retrospective study, we wish to comprehensively assess cytogenetic markers in the context of the molecular subgroups of medulloblastoma and determine whether subgroup affiliation could complement clinical variables for more accurate risk stratification of patients and predict favourable responders for de-escalation of radiotherapy in order to improve the quality of life for survivors. Being retrospective, our cohort is subject to recall bias (for cases with frozen samples), and it encompasses heterogeneously treated patients from multiple centres and continents. Our histology records were not centrally reviewed; nevertheless, our study reflects the typical clinical experience more closely and implicitly reveals the weakness of histological diagnosis in decentralized clinical practice. Further, the lack of surgical details and treatment protocols precludes analyses on how specific treatments and extent of surgical resection affect survival outcome. Notwithstanding the limitations of our discovery cohort, the findings are highly reproducible in an independent cohort of patients.

\begin{SCfigure}[5][t]
	\includegraphics[width=0.7\textwidth]{fig/magic-clin/meta_cyto-markers.pdf}
	\caption[Sample sizes of recent prognostic marker studies]
	{
	Sample sizes of recent prognostic marker studies.
	This meta-analysis was performed by Marc Remke.
	}
	\label{fig:meta_cyto-markers}
\end{SCfigure}

\clearpage

\section{Material and methods}

\subsection{Patient information}

All tissues and clinicopathological information were collected in accordance with institutional review boards from various contributing centers. In the discovery set, although precise treatment dates were often unavailable, at least 95\% of the patients were treated within the past 15 years using modern treatment protocols, including surgical resection, craniospinal (whole brain and spine) irradiation, or chemotherapy. Discovery set samples were collected between 2005 and 2013, with a focus on samples with available fresh-frozen material. Among the samples with treatment details, the earliest diagnosis is July 1997 and the latest is August 2012. Samples in the validation set were all obtained from the Burdenko institute with no selection criterion applied. All patients in the validation set were treated between 1995 and 2010 according to standardized therapy protocols of the German HIT study group.

\subsection{Tumor material and patient characteristics}

A discovery set of 673 medulloblastoma samples with clinical follow-up was acquired retrospectively from 43 cities around the globe. These samples were copy-number profiled on the Affymetrix SNP6 array platform in order to identify potential molecular biomarkers \citeself{shih12}. An independent validation set of 453 samples with clinical follow-up on a medulloblastoma tissue microarray was analyzed by FISH as previously described \citeself{northcott11a}. The validation set consisted only of patients treated in Burdenko, Moscow. Tumors were classified based on signature marker expression into molecular subgroups as previously described \citeself{northcott12}; additional tumours were classified based on cytogenetic aberrations using standard conditional probability models. Subgroup affiliation was not available for 162 discovery samples. The validation set includes an additional set of 50 WNT tumours that were not on the tissue microarray. Nucleic acid isolation, tissue microarray construction, and $\beta$-catenin mutation analysis were performed as previously described \citeself{shih12}. Tissue microarray analysis was performed by collaborators Andrey Korshunov and Stefan Pfister.

\subsection{Prognostic biomarker identification}

Cytogenetic events and copy-number aberrations were identified as previously described in the discovery set \citeself{shih12}. All chromosomal events (or chromosome arm events) were compared against reference samples with balanced copy-number for the chromosome (or chromosome arm); samples with copy-number changes in the opposite direction were specifically excluded from each comparison. Subsequent to biomarker discovery, cross-validation with correction for multiple hypothesis testing was performed to estimate the reproducibility and generalizability of the potential biomarkers in an independent cohort. During cross-validation, the discovery set was split randomly into two subsets. First, the biomarkers are tested by the log-rank test on the first subset. Then, statistically significant biomarkers ($p < 0.05$) are tested again by the log-rank test on the second subset, with correction for multiple hypotheses testing. This process was repeated 10~000 times to estimate the expected validation rate of each biomarker. The expected validation rate of each biomarker is $n_v / n_d$, where $n_d$ is the number of times a biomarker is significant in the first subset and $n_v$ is the number of times a discovered biomarker is also significant in the second subset. The final set of biomarkers was further validated in the external validation set. Additionally, sample size estimates for prospective trials of each biomarker were calculated under univariate Cox models based on the observed hazard ratios. (See page \pageref{sec:prognostication} in the \emphlab{Appendix} for more details on prognostic biomarker discovery.)

\subsection{Multiple hypothesis testing correction}

Within each biomarker identification analysis, correction for multiple hypothesis testing was performed by the Benjamin-Hochberg method during the cross-validation procedure. Independent analyses were corrected for multiple hypotheses testing independently: clinical biomarker identification across medulloblastoma, within WNT medulloblastoma, within SHH medulloblastoma, within Group3, and within Group4; molecular biomarker identification across medulloblastoma, within WNT medulloblastoma, within SHH medulloblastoma, within Group3 medulloblastoma, and within Group4 medulloblastoma.

\subsection{Statistical analysis}

Patient survival characteristics were right-censored at 5 years (or 10 years) and analyzed by the Kaplan-Meier method. Univariate comparison of two or more survival curves were performed using log-rank tests and the Cox proportional-hazards regression models. The predictive values of biomarkers were assessed by analyses of deviance tests under multivariate Cox models and by time-dependent \gls{roc} analyses. Associations between covariates and risk groups were tested by the Fisher's exact test. All statistical analyses were performed in the R software environment (v2.15), using R packages survival (v2.36), risksetROC (v1.0.4), powerSurvEpi (v0.0.6), and ggplot2 (v0.9.3).

\subsection{Time-dependent ROC analysis}

Time-dependent \gls{roc} analyses were performed using the \code{CoxWeights} function provided in the risksetROC (v1.0.4) R package. This function calculates areas under time-dependent ROC curves as described by Heagerty and Zheng \citeref{heagerty05}. Estimates of \gls{auc} for the fitted multivariate Cox model being assessed were calculated every month, from 1 month to 60 months, in order to determine the collective predictive performance of the biomarkers in the Cox models. Differences in \gls{auc} estimates among Cox models across time points were tested by Friedman rank sum tests.

\subsection{Risk stratification model selection}

Biomarkers identified in univariate survival analyses were tested by multivariate Cox proportional-hazards models. All discovered biomarkers were tested for inclusion in the risk stratification model by multiple unbiased procedures: stepwise regression using forward selection, backward elimination and bidirectional elimination with the \gls{aic}, as well as analyses of deviance tests.


\section{Results}

\subsection{Prognostic significance of clinical variables within medulloblastoma subgroups}

Many prior medulloblastoma biomarker publications were limited by sample size, a problem that will only be exacerbated once cohorts are divided into their molecular subgroups. The current study includes 1126 medulloblastoma patients (673 discovery plus 453 validation patients), which is more than double the sample size of any prior medulloblastoma biomarker publication, and one of only a very few that includes a validation cohort (\citefig{meta_cyto-markers}). Although the discovery cohort accumulated by \gls{magic} consists of medulloblastomas gathered from 43 different treating centers from around the world, the subgroup-specific outcome mirrors what has been previously published with very good outcomes for WNT patients, poor outcomes for Group3 patients, and intermediate outcomes for SHH and Group4 patients (\citefig{surv_mb-subgroups}) suggesting that the discovery cohort is a representative sample \citeself{shih14}.

\begin{figure}[h]
	\begin{center}
		\includegraphics[width=\textwidth]{fig/magic-clin/surv_ageg_mstat_wnt.pdf}
	\end{center}
	\caption[Ten-year overall survival curves for WNT medulloblastoma]
	{
	Ten-year overall survival curves for WNT medulloblastoma, split by age group or metastatic status.
	Numbers below x-axis represent patients at risk of event; statistical significances are evaluated by log-rank tests; \gls{hr} estimates are derived from Cox proportional-hazards analyses.
	}
	\label{fig:surv_ageg_mstat_wnt}
\end{figure}

In order to assess long-term survivors, WNT patients were followed for up to 10 years, and only two deaths were observed, both late in the follow-up period and due to recurrence of medulloblastoma (\citefig{surv_ageg_mstat_wnt}).  Among the SHH tumours, there is a significantly better outcome in the adult patients as compared to children or infants (\citefig{surv_ageg_shh_group3_group4}).  There is a trend towards a worse outcome for infants with Group3 tumours that is not statistically significant (\citefig{surv_ageg_shh_group3_group4}).  Infants with Group4 tumours have a significantly worse outcome than children or adults (\citefig{surv_ageg_shh_group3_group4}), suggesting that radiation therapy is critical in the treatment of Group4 medulloblastoma. There is no reproducible association between gender and prognosis in any of the four subgroups (\citeself{shih14}). Desmoplastic histology portends a more favorable prognosis than classic histology, which is more favorable than anaplastic histology among SHH tumours \citeself{shih14}. Large cell/anaplastic histology has prognostic significance for Group3 medulloblastomas in the discovery cohort, but does not validate as significant in the validation cohort.

\begin{figure}[ht]
	\begin{center}
		\includegraphics[width=\textwidth]{fig/magic-clin/surv_ageg_shh_group3_group4.pdf}
	\end{center}
	\caption[Overall survival curves for age groups within SHH, Group3, and Group4 subgroups]
	{
		Overall survival curves for age groups within SHH, Group3, and Group4 subgroups.
		Numbers below x-axis represent patients at risk of event; statistical significances are evaluated by log-rank tests; \gls{hr} estimates are derived from Cox proportional-hazards analyses.
	}
	\label{fig:surv_ageg_shh_group3_group4}
\end{figure}

While metastatic status is not prognostic for patients with WNT medulloblastoma, macroscopic metastasis (M2/M3) is consistently associated with poor survival in all non-WNT subgroups, though the clinical effect is very slight among patients with Group4 disease (\citefig{surv_mstat_shh_group3_group4}).  While the prognostic significance of M0 disease as compared to M2/3 disease is very clear across SHH, Group3, and Group4, the prognostic significance of isolated M1 disease is less clear (\citefig{surv_mstat_shh_group3_group4}). Isolated M1 disease is associated with increased risk in Group3 in the discovery cohort, but not the validation cohort, with the opposite pattern seen in the SHH patients. However, for both discovery and validation cohorts, there are no survival differences survival between M0 and M1 patients with Group4 disease. There are no CNAs in any of the subgroups that are associated with an increased risk of leptomeningeal dissemination (not shown). Overall, many clinical biomarkers continue to exhibit prognostic significance when medulloblastoma is analyzed in a subgroup-specific fashion.

\begin{figure}[ht]
	\begin{center}
		\includegraphics[width=\textwidth]{fig/magic-clin/surv_mstat_shh_group3_group4.pdf}
	\end{center}
	\caption[Overall survival curves for metastatic status within SHH, Group3, and Group4 subgroups]
	{
	Overall survival curves for metastatic status within SHH, Group3, and Group4 subgroups.
	Numbers below x-axis represent patients at risk of event; statistical significances are evaluated by log-rank tests; \gls{hr} estimates are derived from Cox proportional-hazards analyses.
	}
	\label{fig:surv_mstat_shh_group3_group4}
\end{figure}

\subsection{Subgroup and metastatic status are the most predictive markers}

Multivariate survival analyses were conducted in order to dissect the relative predictive value of clinical variables (age, gender, metastatic status, and histotype) and molecular subgroup affiliation. Stepwise Cox proportional-hazards regressions revealed that molecular subgroup significantly contributes to multivariate survival prediction, on top of a regression model already parameterized by clinical variables gender, age, metastatic status, and histology (\citefig{subgroup-specific_cox}\emphlab{a}). Further, Cox models parameterized with both clinical biomarkers and molecular subgroup achieve higher prediction accuracy in time-dependent \gls{roc} analyses (\citefig{subgroup-specific_cox}\emphlab{b}). In isolation, each biomarker has modest prediction accuracy (\citefig{subgroup-specific_cox}\emphlab{c}) compared to the complete multivariate model (\citefig{subgroup-specific_cox}\emphlab{b}). In the complete model, the removals of metastatic status and subgroup lead to the greatest decreases in predictive accuracy (\citefig{subgroup-specific_cox}\emphlab{d}). Taken together, these results suggest that subgroup affiliation and metastatic status are the most important predictive biomarkers, and that they make non-redundant contributions to the prediction of survival. We conclude that combining both clinical biomarkers (metastatic status) and molecular biomarkers (subgroup affiliation) will make the optimal tool for predicting survival of medulloblastoma patients.

\begin{figure}[h]
	\begin{center}
		\includegraphics[width=\textwidth]{fig/magic-clin/subgroup-specific_cox.pdf}
	\end{center}
	\caption[Molecular subgroup and metastatic status are the most important prognostic biomarkers]
	{
	Molecular subgroup and metastatic status are the most important prognostic biomarkers.
	\textbf{a}, Multivariate Cox proportional-hazards survival analysis of predictor variables. Starting with the null model, each variable is added stepwise (from top to bottom) to the survival model. Model likelihood values assess the degree to which each Cox model fits the survival data. Increments in model likelihoods are tested by analysis of deviance. 
	\textbf{b}, Average areas under time-dependent receiver operating characteristic curves (AUC) for multivariate Cox models parameterized by only clinical variables, or both clinical and subgroup variables.
	\textbf{c}, Average time-dependent AUCs for univariate Cox models parameterized by each variable.
	\textbf{d}, Predictive importance of each variable in the fully-parameterized multivariate Cox models, as determined by the average decrease in time-dependent AUC when the variable is omitted from the model.
	Differences in time-dependent AUC and predictive importance are evaluated by the Friedman rank sum test.
	}
	\label{fig:subgroup-specific_cox}
\end{figure}

\clearpage

\subsection{Subgroup specificity of published molecular biomarkers}

Several cytogenetic biomarkers have been previously reported to be associated with patient survival across medulloblastoma, but their prognostic values have seldom been assessed in the context of medulloblastoma subgroups \citeself{shih14}. Loss of chromosome 6 is significantly associated with improved survival across all medulloblastoma (\citefig{subgroup-specific_eg}\emphlab{a}). However, the prognostic value of chr6 loss can be completely attributed to its enrichment in WNT medulloblastomas, as chr6 loss has no prognostic value among WNT patients, or among non-WNT patients, when compared to their respective controls with balanced chr6 (\citefig{subgroup-specific_eg}\emphlab{b}). We would suggest that chr6 loss is a subgroup-driven biomarker in that its prognostic significance is driven by its enrichment in a particular subgroup, and it thus has no further significance in subgroup-specific analysis. Further, based on these results, we would caution against using chr6 loss as the lone diagnostic criteria for WNT medulloblastoma, since it is also observed in non-WNT medulloblastoma (7/49 chr6 loss medulloblastomas were not WNT (14\%)), and chr6 loss is only present in 42/53 WNT tumours (79\%). The prognostic role of isochromosome 17q (iso17q) has been very controversial; in our cohort, iso17q is a statistically significant predictor of poor outcome overall (\citefig{subgroup-specific_eg}\emphlab{c}).  However, subgroup-specific analysis demonstrates that iso17q is highly prognostic for Group3 medulloblastoma, but not for Group4 medulloblastoma (\citefig{subgroup-specific_eg}\emphlab{d}), indicating that it is a subgroup-specific molecular biomarker.  Similarly, while chr10q loss is a modestly significant predictor of poor outcome across medulloblastoma subgroups (\citefig{subgroup-specific_eg}\emphlab{e}), its prognostic power is limited to the SHH subgroup of tumours in a subgroup-specific analysis (\citefig{subgroup-specific_eg}\emphlab{f}).  We conclude that determination of molecular subgroup is crucial in the evaluation and implementation of molecular biomarkers for patients with medulloblastoma, as some putative biomarkers are merely enriching for a specific subgroup (\emphterm{subgroup driven}) while most others are only significant within a specific subgroup (\emphterm{subgroup specific}).

\clearpage

\begin{figure}[h]
	\begin{center}
		\includegraphics[width=\textwidth]{fig/magic-clin/subgroup-specific_eg.pdf}
	\end{center}
	\caption[Subgroup-driven and subgroup-specific molecular biomarkers]
	{
	Subgroup-driven and subgroup-specific molecular biomarkers.
	\textbf{a}, Overall survival curves and frequency distribution of chr6 status across the entire cohort.
	\textbf{b}, Overall survival curves for chr6 status in WNT and non-WNT medulloblastomas.		
	\textbf{c}, Overall survival curves and frequency distribution of isolated chr17q gain across the entire cohort.
	\textbf{d}, Overall survival curves for chr17q status in Group3 and Group4 subgroups. 
	\textbf{e}, Overall survival curves for chr10q status across the entire cohort.
	\textbf{f}, Overall survival curves for chr10q status in SHH and non-SHH medulloblastomas.
	Numbers below x-axis represent patients at risk of event; statistical significances are evaluated by log-rank tests; \gls{hr} estimates are derived from Cox proportional-hazards analyses.
	}
	\label{fig:subgroup-specific_eg}
\end{figure}

\clearpage

\subsection{SHH patients can be stratified into three distinct risk groups}

We identified 11 CNAs that are prognostically significant in our SHH medulloblastoma discovery set in univariate survival analyses (\citefig{shh-markers}) \citeself{shih14}. Given the considerable number of candidates, the reproducibility of the identified biomarkers was assessed by cross-validation, and the expected sample sizes required for validation in future prospective trials were estimated to facilitate candidate prioritization \citeself{shih14}. Specific amplifications but not broad gains encompassing \gene{GLI2} or \gene{MYCN} are associated with bleak prognosis (\citefig{shh-markers}\emphlab{a--b}). Loss of chr14q confers significantly inferior survival (\citefig{shh-markers}\emphlab{c}). There is no minimal region of deletion on chr14 in SHH patients, and recent medulloblastoma re-sequencing efforts have not identified any recurrent SNVs on chr14 in SHH medulloblastoma \citeself{shih14}. The presence of chromothripsis (chromosome shattering) is associated with worse survival in SHH patients (\citefig{shh-markers}\emphlab{d}).

To integrate the individual biomarkers into a risk stratification model, multivariate Cox proportional-hazards analyses were performed on all significant prognostic markers. Through multiple stepwise regression procedures, a consensus set of biomarkers was selected for inclusion in the model in an unbiased manner. The proposed risk stratification scheme represents the model that was most consistent with available data in the discovery cohort, from among many possible alternatives (\citefig{shh-risk-strat}\emphlab{a}) \citeself{shih14}. \gene{GLI2} amplification, chr14q loss, and leptomeningeal dissemination (M+ disease) identify high and standard risk patients. Specifically, \gene{GLI2} amplification alone can identify patients with bleak prognosis (\citefig{shh-risk-strat}\emphlab{a}) \citeself{shih14}. Absence of these markers demarcates a low-risk group of patients who exhibit survivorship reminiscent of WNT patients. Importantly, none of the covariates, particularly age and anaplastic histology, can explain the survival differences observed among the risk groups (\citefig{shh-risk-strat}\emphlab{a}) \citeself{shih14}. Direct application of the proposed risk stratification scheme on the independent validation cohort yields distinct survivorships for the three risk groups, thereby validating the model (\citefig{shh-risk-strat}\emphlab{c}).

\begin{figure}[h]
	\begin{center}
		\includegraphics[width=\textwidth]{fig/magic-clin/shh-markers.pdf}
	\end{center}
	\caption[Overall survival curves for molecular biomarkers in SHH medulloblastoma]
	{
	Overall survival curves for molecular biomarkers in SHH medulloblastoma:
	\textbf{a}, \gene{GLI2} copy number status;
	\textbf{b}, \gene{MYCN} copy number status;
	\textbf{c}, chr14q status; and
	\textbf{d}, chromothripsis status.
	Numbers below x-axis represent patients at risk of event; statistical significances are evaluated by log-rank tests; \gls{hr} estimates are derived from Cox proportional-hazards analyses.
	}
	\label{fig:shh-markers}
\end{figure}

Two additional stratification schemes were constructed using only clinical biomarkers or only cytogenetic markers; however, the proposed model, which combines both types of biomarkers, yields the highest prediction accuracy (\citefig{shh-risk-strat}\emphlab{b}) \citeself{shih14}. Furthermore, the accuracy of the combined risk model is drastically reduced when applied across non-SHH patients, further underscoring the importance of taking subgroup into consideration during risk stratification. We conclude that by using two molecular biomarkers (\gene{GLI2} and 14q \gls{fish}) and metastatic status, we can practically and reliably predict prognosis for patients with SHH medulloblastoma.

\clearpage

\begin{figure}[h]
	\begin{center}
		\includegraphics[width=\textwidth]{fig/magic-clin/shh-risk-strat.pdf}
	\end{center}
	\caption[Combined clinical and molecular biomarkers improve risk-stratification of SHH patients]
	{
	Combined clinical and molecular biomarkers improve risk-stratification of SHH patients.
	\textbf{a}, Risk stratification of SHH medulloblastomas by molecular and clinical prognostic markers. \emph{Top-left}, decision tree; \emph{bottom-left}, events plot depicting status of molecular and clinical markers across the risk groups; \emph{right}, overall survival curves for SHH risk groups.
	\textbf{b}, Average time-dependent AUCs for risk groups stratified using only clinical or molecular markers, or both. Risk stratification regimens are applied to SHH and non-SHH medulloblastomas. ***, $p < 0.001$, Friedman rank sum tests.
	\textbf{c}, Survival curves for SHH risk groups in the validation cohort.
	Numbers below x-axis represent patients at risk of event; statistical significances are evaluated by log-rank tests; \gls{hr} estimates are derived from Cox proportional-hazards analyses.
	}
	\label{fig:shh-risk-strat}
\end{figure}

\clearpage

\subsection{Three biomarkers demarcate high-risk Group3 patients}

In Group3 patients, iso17q and \emph{MYC} amplification remain the only cytogenetic markers associated with poor survival, whereas chr8q loss and chr1q gain are the only good prognosis markers (\citefig{group3-markers}) \citeself{shih14}. In multivariate survival analyses, patients with metastasis, iso17q, or MYC amplification represent the high-risk group (\citefig{group3-risk-strat}\emphlab{a}). Critically, absence of these markers can identify a population of Group3 patients who have a survivorship much longer than Group3 taken as a whole. The risk groups are not associated with any clinical covariates, including age (\citefig{group3-risk-strat}\emphlab{a}) \citeself{shih14}. Consistent with the findings in SHH patients, optimal risk stratification in Group3 patients requires the use of both clinical and molecular prognostic markers, which have reduced or no prognostic value outside of Group3 (\citefig{group3-risk-strat}\emphlab{b}) \citeself{shih14}. Our proposed risk stratification scheme was validated on the non-overlapping validation cohort using three molecular biomarkers (\emph{MYC}, 17p, and 17q \gls{fish}) and metastatic status (\citefig{group3-risk-strat}\emphlab{c}).

\bigskip

\begin{figure}[h]
	\begin{center}
		\includegraphics[width=\textwidth]{fig/magic-clin/group3-markers.pdf}
	\end{center}
	\caption[Overall survival curves for molecular biomarkers in Group3 medulloblastoma]
	{
	Overall survival curves for molecular biomarkers in Group3 medulloblastoma:
	\textbf{a}, chr17 copy number aberrations;
	\textbf{b}, \emph{MYC} copy number status; and 
	\textbf{c}, chr8q status.
	\textbf{d}, Risk stratification of Group3 medulloblastomas by molecular and clinical prognostic markers.
	Numbers below x-axis represent patients at risk of event; statistical significances are evaluated by log-rank tests; \gls{hr} estimates are derived from Cox proportional-hazards analyses.
	}
	\label{fig:group3-markers}
\end{figure}

\clearpage

\begin{figure}[h]
	\begin{center}
		\includegraphics[width=\textwidth]{fig/magic-clin/group3-risk-strat.pdf}
	\end{center}
	\caption[Combined clinical and molecular biomarkers improve risk-stratification of Group3 patients.]
	{
	Combined clinical and molecular biomarkers improve risk-stratification of Group3 patients.
	\textbf{a}, Risk stratification of Group3 medulloblastomas by molecular and clinical prognostic markers.	\emph{Top-left}, decision tree; \emph{bottom-left}, events plot depicting status of molecular and clinical markers across the risk groups; \emph{right}, overall survival curves for Group3 risk groups.
	\textbf{b}, Average time-dependent AUCs for risk groups stratified using only clinical or molecular markers, or both. Risk stratification regimens are applied to Group3 and non-Group3 medulloblastomas. ***, $p < 0.001$, Friedman rank sum tests.
	\textbf{c}, Survival curves for Group3 risk groups in the validation cohort.
	Numbers below x-axis represent patients at risk of event; statistical significances are evaluated by log-rank tests; \gls{hr} estimates are derived from Cox proportional-hazards analyses.
	}
	\label{fig:group3-risk-strat}
\end{figure}

\clearpage


\subsection{Identification of a low-risk group of metastatic Group4 patients}

Group4 patients with whole chromosome loss of chr11 or gain of chr17 exhibit better survival under univariate Cox models (\citefig{group4-markers}\emphlab{a}), in addition to chr10p loss \citeself{shih14}. There is no cytogenetic marker associated with poor prognosis \citeself{shih14}. Specifically, neither \gene{MYCN} gain nor amplification is associated with poorer survival in Group4, in stark contrast to SHH patients, reinforcing the distinction in their underlying biology (\citefig{group4-markers}\emphlab{b}) \citeself{shih14}. Similarly, none of the cytogenetic biomarkers identified for Group3 patients (e.g. iso17q) have any prognostic value in Group4 \citeself{shih14}. Following unbiased model selection, the consensus set of biomarkers results in a risk stratification scheme in which leptomeningeal dissemination identifies high-risk Group4 patients, except in the context of chr11 loss or chr17 gain (\citefig{group4-risk-strat}\emphlab{a}). The biology underlying chr11 loss is not apparent as there is no obvious minimal common region of deletion, nor are there any recurrent SNVs on chr11 reported in the recent medulloblastoma re-sequencing publications \citeself{shih14}. Group4 patients with either chr17 gain or chr11 loss, irrespective of their metastatic statuses, exhibit survivorship that is characteristic of WNT patients in both the discovery and validation cohorts (\citefig{group4-risk-strat}\emphlab{a},\emphlab{c}), and the survival differences are not explainable by covariates \citeself{shih14}. Significantly, the low-risk Group4 cohort also included some patients with anaplastic histology, suggesting that anaplasia may not be universally prognostic for poor outcome. Consistent with other subgroups, the risk stratification model using both clinical and molecular biomarkers achieve the highest accuracy (\citefig{group4-risk-strat}\emphlab{b}). Critically, the cytogenetic biomarkers identify low-risk Group4 patients whom would be otherwise designated as high-risk by evidence of metastasis and/or anaplastic histology, and this finding cannot be extrapolated to SHH and Group3 patients (\citefig{group4-risk-strat}).  We conclude that through the use of three molecular biomarkers (chr11, 17p, and 17q \gls{fish}) and metastatic status, we can accurately and reliably predict the survival of patients with Group4 medulloblastoma.

\bigskip

\begin{figure}[h]
	\begin{center}
		\includegraphics[width=\textwidth]{fig/magic-clin/group4-markers.pdf}
	\end{center}
	\caption[Overall survival curves for molecular biomarkers in Group4 medulloblastoma]
	{
	Overall survival curves for molecular biomarkers in Group4 medulloblastoma:
	\textbf{a}, whole chr11 status and whole chr17 status; and
	\textbf{b}, \gene{MYCN} copy number status.
	Numbers below x-axis represent patients at risk of event; statistical significances are evaluated by log-rank tests; \gls{hr} estimates are derived from Cox proportional-hazards analyses.
	}
	\label{fig:group4-markers}
\end{figure}

\clearpage

\begin{figure}[h]
	\begin{center}
		\includegraphics[width=\textwidth]{fig/magic-clin/group4-risk-strat.pdf}
	\end{center}
	\caption[Combined clinical and molecular biomarkers improve risk-stratification of Group4 patients]
	{Combined clinical and molecular biomarkers improve risk-stratification of Group4 patients.
	\textbf{a}, Risk stratification of Group4 medulloblastomas by molecular and clinical prognostic markers. \emph{Top-left}, decision tree; \emph{bottom-left}, events plot depicting status of molecular and clinical markers across the risk groups; \emph{right}, overall survival curves for Group4 risk groups.
	\textbf{b}, Average time-dependent AUCs for risk groups stratified using only clinical or molecular markers, or both. Risk stratification regimens are applied to Group4 and non-Group4 medulloblastomas. ***, $p < 0.001$, Friedman rank sum tests.
	\textbf{c}, Survival curves for Group4 risk groups in the validation cohort. 
	Numbers below x-axis represent patients at risk of event; statistical significances are evaluated by log-rank tests; \gls{hr} estimates are derived from Cox proportional-hazards analyses.
	}
	\label{fig:group4-risk-strat}
\end{figure}

\clearpage


\section{Discussion}

The analysis of $> 1000$ medulloblastoma patients clearly demonstrates that subgroup affiliation enhances prognostication with clinical biomarkers and that the majority of published molecular biomarkers are only relevant in the setting of a single subgroup. The combination of clinical variables, molecular subgroup, and six cytogenetic markers analyzed on \gls{ffpe} tissues can achieve an unprecedented level of prognostic prediction for medulloblastoma patients that is practical, reliable, and reproducible. The proposed risk stratification models represent those that best fit the available data in the discovery cohort. The performances of our models are robust against the choice of prognostic markers. Despite the large size of our discovery cohort, missing data and the complexity of multivariate analyses may necessitate the use of even larger cohorts to assess the inclusion of additional prognostic markers. Moreover, while we strive to include the most important markers in multivariate models, we cannot exclude the possibility of that alternative markers may perform equally well. Our results nonetheless elucidate the prognostic potential of known and novel markers and highlight clinically useful risk-stratification schemes.

The prognostic significance of M1 status (presence of cells in the cerebrospinal fluid) has long been controversial. Most reports agree that presence of metastasis portends poor prognosis and warrants intensified treatment \citeref{grill05, rutkowski05, salama06, rutkowski10, kool12, pietsch14, vonhoff09, bouffet94, zeltzer99}; however, it is unclear whether M1 disease has the same prognosis as M2/M3 (macroscopic metastasis). Kortmann \emph{et al.}\ contended in a prospective trial that M2/M3 status were indicators of poor outcome in medulloblastoma, but residual disease or M1 status were not \citeref{kortmann00}. In another prospective trial, Zeltzer \emph{et al.}\ maintained that both M1 and M2/M3 statuses were prognostically unfavourable \citeref{zeltzer99}. In subsequent studies, some investigators group M0 and M1 together in one category \citeref{rutkowski05, pietsch14}, while others group M1, M2 and M3 together as M+ \citeref{rutkowski10, kool12, pietsch14, strother14}. In a retrospective review, Sanders \emph{et al.}\ reported that M1 patients do not have better survival than M2/M3 patients under the same treatment \citeref{sanders08}. In our cohorts, the prognostic significance of M1 disease may be subgroup specific, though the small sample size of M1 patients hinders a definitive conclusion \citeself{shih14}. Based on our data, it is unlikely that M1 status is a universal indicator of poor outcome. Nevertheless, irrespective of whether M1 was categorized with M0 or M2/M3, our risk-stratification models can reproducibly and robustly predict patient survival.

Controversy also surrounds the prognostic value of anaplastic histology. Large cell and anaplastic histologies are often grouped together because these histological features often co-occur and can be difficult to distinguish. Several studies reveal that large cell/anaplastic histology is prognostically unfavourable \citeref{gajjar04, jakacki12, kool12, pietsch14}. von Hoff \emph{et al.}\ distinguished between large cell and anaplastic histologies and reported that large cell histology was a negative prognostic factor while anaplasia was not \citeref{vonhoff10}. The authors further suggested that \emph{MYC} amplification, which co-occurred with large cell histology, may be the underlying cause of poor prognosis; however, the precise definition of \emph{MYC} amplification remains contentious \citeref{vonhoff10, raabe10}. We clarify this issue by demonstrating that only high-level \emph{MYC} amplification but not single copy gains of \emph{MYC} (focal or broad) is prognostically significant. Additionally, large cell/anaplastic histology has no prognostic value in a multivariate model accounting for \emph{MYC} amplification. Possibly, \emph{MYC} amplification may be a marker for apoptotic resistance, leading to resistance against radiotherapy and chemotherapy. Although MYC promotes cell proliferation, it also normally induces apoptosis; therefore, \emph{MYC} amplification is incompatible with tumour formation except in the context of apoptotic pathway disruption \citeref{pei12}.

Another marker notably absent from our risk-stratification schemes is \gene{TP53} mutation, which is a well-known indicator of poor prognosis \citeref{zhukova13}. Loss of TP53 function abrogates the apoptotic pathway and contributes to resistance against chemotherapy and radiotherapy. While we had some \gene{TP53} mutation data in our cohorts, a substantial proportion of samples were not interrogated for \gene{TP53} status. Additionally, \gene{TP53} mutation appears to be predominately prognostic for long-term survivors \citeref{zhukova13}, and the follow-up lengths in our cohorts were insufficient to evaluate the prognostic impact of \gene{TP53} mutation. Therefore, the utility of \gene{TP53} mutation in multivariate patient risk-stratification should be further assessed in a cohort with more complete data and longer follow-up.

The absence of age in our risk stratification schemes bodes well for the modern practice of restricting the role of radiotherapy in the treatment of young children. Indeed, age groups are not associated with the risk groups defined by our models, and the survival differences among the risk groups cannot be explained by differences in age distributions. Furthermore, age (discretized or otherwise) has no prognostic value in multivariate survival models once such prognostic factors such as metastatic status is included \citeself{shih14}. The univariate significance of age for poor outcome may be explained by the notation that early age at diagnosis is simply a proxy for tumour aggressiveness. While some aggressive tumours arise early, not all early arising tumours are aggressive. Accordingly, age has little independent prognostic value in a multivariate survival model, despite that infants often receive less intensive treatment in contemporary protocols. Our results suggest that elimination of radiotherapy and dose reduction in chemotherapy has not contributed to poorer survival of infants; rather, aggressive tumours that respond poorly to treatment sometimes present in infants.

Above all, our risk stratification models identify patient groups who are promising candidates for de-escalation or elimination of irradiation during treatment. In particular, WNT patients exhibit excellent long term survival; with careful monitoring, these patients may respond well to reduced radiotherapy with postsurgical chemotherapy. For SHH medulloblastoma, the finding that infants in the low-risk group under our model respond favourably to multimodal treatment (with presumably reduced or eliminated radiotherapy) points to the tantalizing possibility that the remaining patients in this low risk group (defined by absence of all unfavourable markers) may similarly respond well to chemotherapy alone. Given that SHH medulloblastoma tend not to recur with metastasis \citeref{ramaswamy13}, localized radiotherapy may be sufficient to prevent recurrence. Among patients with Group4 medulloblastoma, some patients with metastatic disease show excellent survival. Since patients presenting with metastasis are traditionally considered high risk, their apparent favourable outcome in our cohorts begs the question: Did they need the intensified radiotherapy for tumour eradication? If their favourable survival is not attributable to intensified treatment, these patients may benefit from radiotherapy de-escalation and survive with improved qualities of life. Encouragingly, recent findings suggest that the dose of craniospinal irradiation might be reduced in high-risk medulloblastoma (metastatic or residual disease) without compromising survival by supplementing the treatment with tandem high dose chemotherapy (and autologous stem cell transplantation) \citeref{sung13}.

To conclude, we demonstrate that medulloblastoma molecular subgroup is highly informative for predicting patient outcome, and we can dramatically improve the accuracy of survival prediction by incorporating molecular subgroup with conventional clinical parameters for patient risk stratification. Moreover, we proposed, tested, and validated novel subgroup-specific risk stratification models that consider both clinical and molecular variables.  These models perform robustly and reproducibly both in the discovery cohort consisting of a heterogeneously treated group of patients and in a large non-overlapping validation cohort of patients treated at a single institution according to a single treatment protocol. Given that we do not have detailed treatment information for patients in these cohorts, it is highly possible that treatment effects (type, duration, or intensity) could impact our results. We would suggest that this can only be accounted through examination of our stratification model in a sufficiently large prospectively followed cohort of patients with medulloblastoma. While the current study uses either \gls{snp} arrays, or interphase \gls{fish} on \gls{ffpe} sections, it is possible that other approaches such as \gls{acgh} could also be used to determine the copy-number status of the six cytogenetic markers. Our results demonstrate the utility of incorporating tumour biology into clinical decision-making and offer a novel perspective on risk stratification using \gls{fish} (applicable on paraffin sections), and thus should be validated in prospective multi-centre trials and be translated into routine clinical practice.


% Material not included:

% The model can be fine-tuned by additional of more covariates. Due to missing data, some covariates may have underappreciated significance due to reduced power, and they did not make the final model. Nonetheless, we know that the model works and is reproducible in an independent cohort.

% In China, WNT Medulloblastoma continues to have superior survival compared to other medulloblastoma subgroups \citeref{zhang14}.

% Large cell histology (as opposed to anaplastic) is unfavourable \citeref{pietsch14}.

% Subtotal resection is prognostically unfavourable \citeref{park83, grill05, rutkowski05, roldan08, rutkowski10, vonbueren11, lannering12, strother14, zeltzer99}. However, gross total resection in cases with brain stem involvement does not yield survival benefit \citeref{gajjar96}.

% Desmoplastic histology overruled extraneural metastasis (M4) \citeref{young15}.

\clearpage

\chapter{Discovering therapeutic targets by genomic profiling of medulloblastoma}
\chaptermark{Discovering therapeutic targets}
\label{ch:target-id}

\begin{objective}
We hypothesize that each medulloblastoma molecular subgroup is characterized by specific genomic alterations, and we aim to identify potential therapeutic targets specific to each molecular subgroup.
\end{objective}

Increased awareness of long-term neurotoxcity of craniospinal irradiation has motivated the de-escalation of radiotherapy by incorporating dose-intensive, combination chemotherpay into the treatment regime. In order to eradicate tumour cells, chemotherapy is often intensified to the point where autologous stem cell transplants are required to circumvent patient mortality. Albeit less harmful, chemotherapy is not without neurocognitive and functional consequences for survivors. In one trial, addition of chemotherapy to craniospinal irradiation caused long-term survivors of medulloblastoma suffer poorer health status \citeref{bull07}. In another trial, patients with medulloblastoma who were treated with chemotherapy alone (vincristine, carboplatin, etoposide, cyclophosphamide, and methotrexate) suffered from declines in neurocognitive function \citeref{rutkowski05}. Perhaps part of the decline could be due to the disease itself; however, even for patients with acute lymphoplastic leukemia, chemotherapy alone (vincristine, methotrexate, prednisone, and antracyclines) can cause reduction in volumes of several neuroanatomic structures of the brain and consequent decline in processing speed, executive function, learning and memory \citeref{zeller13}. Furthermore, chemotherapy causes a myarid of other side-effects; for example, it increases the risk of infection for patients due to suppression of immune system. Chemotherapeutic drugs such as cisplatin can cause permanent hearing loss and peripheral neurotoxicity \citeref{avan15}.

Aside from mounting evidence for long-term adverse effects, combination chemotherapy encountered yet another setback when a prospective trial reported that prolonged dose-intensive chemotherapy (with cisplatin, cyclophosphamide, vincristine, etoposide) for patients with medulloblastoma yielded no improvement in survival (and also resulted several treatment-associated deaths) \citeref{strother14}. In this trial, the authors treated the patients with 72 weeks of dose-intensity, combination therapy, and patients with medulloblastoma showed no improvement in survival (though patients with epedymoma did) \citeref{strother14}. This negative finding highlight the limitation of chemotherapy in the treatment of medulloblastoma, whose arsenal has essentially remained unchanged for decades.

To faciliate discovery of therapeutic targets, we sought to identify recurrently dysregulated genes and pathways by analyzing the DNA copy-number profiles of $> 1000$ primary medulloblastoma samples (before radiotherapy and chemotherapy). Genes amplified recurrently across tumours are candidate proto-oncogenes, against which therapeutic agents may be developed. Conversely, genes deleted recurrently are candidate tumour suppressors, and therapeutic interventions may be designed against downstream pathways. Earlier studies aimed at identifying recurrent \gls{cns} or mutations did not identify any highly recurrent focal genetic lesions across medulloblastoma tumours (not classified into additional subtypes); most focal genetic lesions occur at frequency of less than 10\%, with \gene{MYC} amplification and \gene{PTCH1} mutation being the most commonly observed events in medulloblastoma \citeref{northcott09, parsons11}. These observations suggest that medulloblastom is genetic heterogeneous, and classifying medulloblastoma into molecular subgroups may aid the identification of genes and pathways disrupted above the background mutation rate in medulloblastoma.

The four molecular subgroups of medulloblastoma exhibit activation of different transcriptional progrms, which may be shaped by different genetic abnormalities. We therefore stratify our medulloblastoma samples by subgroup status into relatively more homogeneous groups, and we sought to identify recurrent \gls{cns} enriched in or specific to each molecular subgroup. We hypothesize that each medulloblastoma subgroup incur different genetic events. Accordingly, we hope to reveal dysregulated genes and pathways underlying the tumourigenesis of each subgroup in order to facilitate the development of precise, targeted therapeutic interventions.

% dose intensive chemotherapy can also lead to high toxic mortality (death due to chemotherapy) \citeref{dhall08, strother14}.

% lower dose of cisplatin did not compromise survival among patients with nonmetastatic, near totally resected medulloblastoma \citeref{nageswararao14}.


\section{Materials and methods}

\subsection{Patient samples and nucleic acid extraction}

All patient samples were procured in accordance with the Research Ethics Board at The Hospital for Sick Children (Toronto, Canada).  Samples were obtained as frozen tissue biopsies at the time of diagnosis and stored at -80 $^{\circ}$C until processed for purification of nucleic acids.  Frozen tissue was available for ~75-80\% of cases included in the study; the remaining cases were shipped as pre-isolated DNA and/or RNA.
Whenever possible, tumour isolates were partitioned for both standard DNA and RNA extraction.  Tissues were either manually homogenized using a mortar and pestle in the presence of liquid nitrogen or in an automated manner using a Precellys 24 tissue homogenizer (Bertin Technologies, France), according to the manufacturer’s instructions.  High molecular weight DNA was extracted by SDS/Proteinase K digestion followed by 2-3 phenol extractions and ethanol precipitation.  Total RNA was isolated using the Trizol method (Invitrogen, USA) using standard protocols.  DNA and RNA were quantified using a NanoDrop 1000 instrument (Thermo Scientific, USA) and integrity assessed either by agarose gel electrophoresis (DNA) or Agilent 2100 Bioanalyzer (RNA; Agilent, USA) at The Centre for Applied Genomics (TCAG, Toronto, Canada).  RNA with an RNA Integrity Number (RIN) $\ge 7.0$ was required for analysis by either Affymetrix Gene array or RNASeq. Paul Northcott performed the nucleic acid extractions.

\subsection{DNA copy number analysis}

\subsubsection{SNP array processing and quality control}

Genotyping and copy-number profiling of DNA samples was performed on the Affymetrix Genome-Wide Human SNP Array 6.0 platform, which includes more than 906~600 probes for the genotyping of \gls{snp} loci and more than 946~000 probes for the detection of copy number variations. With a total of 1.8 million probe markers, the median distance between markers is less than 7000 bases. DNA was prepared, labeled, and hybridized to the Affymetrix SNP 6.0 arrays as previously described \citeref{shih12}. Sample quality control was assessed using Affymetrix Genotyping Console (GTC) as previously described \citeself{shih12}. \gls{tcag} performed the SNP array processing and quality control.

\subsubsection{Generation and normalization of copy number profiles}

Affymetrix SNP6 CEL files were processed in dChip \citeref{lin04} to obtain raw copy number estimates.  Arrays were normalized by quantile normalization and signal intensities computed using the MBEI method (PM-only) inherent to the dChip program.  To generate a diploid reference baseline for copy number analysis of medulloblastoma samples, we used Affymetrix SNP6 data from 132 individuals from the Ontario Population Genomics Platform (OPGP) epidemiological project and the HapMap PROJECT.  Germline DNA from these samples was genotyped in the same microarray facility as the tumour samples, using identical experimental protocols as described above.

The normalized copy number estimates from dChip were subsequently imported into the R environment, and the copy number profiles were segmented using the \gls{cbs} algorithm from the DNAcopy (v1.24) \citeref{venkatraman07} package, with the undo split option enabled (\code{method = sdundo, undo.SD = 1}). The resulting segmentation profiles were further processed to reduce artificial segments. Segments with fewer than 10 markers were removed. Adjacent segments whose copy number states differed by less than 0.25 were merged together using their size-weighted mean. This merging step is repeated iteratively until no more segments can be merged. The segmentation profile of each sample was then median-centered. Further, normal CNVs reported in the Database of Genomic Variants (DGV, v10) \citeref{iafrate04} were filtered from the segmentation profiles. Segments that reciprocally overlapped (Dice coefficient > 0.5) with normal CNVs were removed. CNVs reported on BAC End Sequencing, BAC Array CGH, ROMA, and FISH were excluded from this filter. Upon removing a segment, the upstream segment was merged to the downstream segment using a size-weighted mean. The aforementioned merging and CNV-filtering steps helped reduce the occurrence of broad segments being broken into non-contiguous pieces by artificial segments or normal CNVs (which influences the downstream GISTIC2 analysis and segment classification). The copy number segments were classified as balanced or one of 6 copy number aberrations based on the criteria in \citetab{cna-criteria}.

\begin{table}[H]
	\caption[Criteria for DNA copy number aberrations]
	{
		Criteria for DNA copy number aberrations
	}
	\label{tab:cna-criteria}
	\footnotesize
	\setlength{\extrarowheight}{0.5em}
	\centering
	\begin{tabular}{l | l | l}
		\hline
		\textbf{Class} & \textbf{Log R ratio ($r$)} & \textbf{Size in Mbp ($s$)} \\
		\hline
		Balanced & $| r | \le 0.2$ & --- \\
		Gain & $r > 0.2$ & --- \\
		Loss & $r < -0.2$ & --- \\
		Focal gain & $r > 0.2$ & $s < 12$ \\
		Focal loss & $r < -0.2$ & $s < 12$ \\
		High level amplification & $r > log_2(5/2)$ & $s < 12$ \\
		Homozygous deletion & $r < -log_2(0.7/2)$ & $s < 12$ \\
		\hline
	\end{tabular}
\end{table}

\subsubsection{Identification of recurrent copy number aberrations}

The post-processed segmentation files profiles were analyzed using two algorithms, GISTIC2 and modified CMDS, to identify recurrent copy number events. GISTIC2 requires prior single sample segmentation, and hence may be affected by the presence of segmentation artifacts. In contrast, CMDS requires no single sample segmentation and works with the raw copy number profiles directly. Many post-processing steps were lacking from the distributed CMDS program (e.g. multiple hypothesis correction, peak calling, CNV filtering); therefore, a number of post-processing steps were added.

GISTIC2 (v2.0.12) \citeref{mermel11} was run with default parameters unless specified otherwise (\code{brlen = 0.5, conf = 0.9}), on the segmentation profiles of the entire cohort and of each subgroup separately. The significant peaks were filtered based on the gene content ($ge 1$ gene spanned), size of the wide peak ($< 12$ Mbp), and additionally based on containment within a normal CNV region (one-way overlap $> 90$\%) reported in the filtered DGV database. This second CNV-filtering step ensures that amalgamated regions reported in DGV are considered, as these regions would often have poor reciprocal overlap with individual query segments and would be missed by the previous CNV-filtering procedure. It is not possible to apply this secondary CNV-filtering procedure directly on the segmentation file, as the CNV regions are often very large and this filtering can cause the loss of many informative CNA segments.
CMDS (v 1.0) \citeref{zhang10} was run using default parameters unless specified otherwise (\code{w = 40, s = 1}), on the unstratified and subgroup-stratified segmentation profiles, for each chromosome separately. Since the raw outputs had a high false positive rate, the outputs were processed further. The z-scores were recalculated (separately for each chromosome) from the Fisher’s z-transformed of the Pearson correlations, using the means of the dominant components from Gaussian mixture models ($k = 3$). This recalculation ensured that the z-scores were not skewed due to the presence of multiple modes in the raw output. The z-scores were further detrended using \gls{emd} \citeref{wu07}, which corrects for trends due to recurring broad events. The p-values derived from the z-scores were then corrected for multiple hypothesis correction using the qvalue R package(v1.1) \citeref{storey03}. Finally, the peak regions were identified using a simple q-value threshold ($FDR = 0.05$).

\subsubsection{Estimation of the copy number states of genes}

The copy number states of all RefSeq genes in each sample were inferred from the respective post-processed segmentation profiles. The state for a given gene was determined by the copy number segment (classified as described earlier) that spans the greatest proportion of the gene (if multiple segments span the gene). Further, a gene is considered lost/deleted only if any portion of its coding region is spanned by a loss/deletion segment; a gene is considered gained/amplified only if at least 50\% of the coding region is spanned by a gained/amplified segment.

\subsubsection{Identification of recurrent broad events}

Identification of recurrent CNAs above explicitly excluded broad events (based on a broad length cutoff in GISTIC and by a detrending procedure after CMDS). Therefore, broad events in the segmentation profiles were analyzed separately, using an approach similar to GISTIC’s broad event analysis \citeref{mermel11}. The log2 R ratio (LRR) of each chromosome was calculated using a size-weighted mean of all segments mapping to the chromosome. A chromosome was declared gained if its LRR was greater than 0.2, lost if the LRR was less than -0.2, and balanced otherwise. Unlike GISTIC, gained and lost broad events were analyzed together. The significance of the frequency of each broad event was tested using the exact binomial test. Each broad event frequency was compared to the background frequency, which was determined from a robust regression of the observed frequencies with respect to gene content (i.e. number of RefSeq genes) across all chromosomes.

\subsubsection{Identification of chromothripsis}

The occurrence of chromothripsis was identified using tumour copy number profiles as previously described \citeref{rausch12}. Using the post-processed segmentation profiles, chromothripsis was identified on a chromosome based on the presence of greater than 10 copy number state changes. The enrichment of chromothripsis on a chromosome for a subgroup was determined using the hypergeometric test, comparing the observed incidence to the background incidence (tallied across all samples). Select samples inferred to have chromothripsis were confirmed by WGS to identify rearrangements.

\subsection{Subgroup enrichment analysis of recurrent copy number aberrations}

Recurrent copy number aberrations (CNAs) identified by GISTIC2 in the unstratified and subgroup-stratified analysis were tested for subgroup enrichment. In the case of stratified analysis, regions identified in each subgroup were combined together. Since the reported region coordinates of common CNA events can differ among the strata, regions that reciprocally overlap (Dice coefficient > 0.2) were merged together using their union. In the enrichment analysis, the frequency of recurrent CNAs in each subgroup was compared against the remaining subgroups, and odds ratios were estimated using Fisher’s conditional maximum likelihood estimate. CNAs with odds ratios greater than 2.0 were considered subgroup-enriched. A similar enrichment analysis was repeated for comparing combinations of two subgroups against the remaining, in order to identify CNAs that are enriched in the ensemble of two subgroups (and are not otherwise enriched in a single subgroup).

\subsection{Integration of gene expression and copy number aberrations}

To directly assess the correlation between significant CNA and gene expression, expression profiles were generated on 285 medulloblastomas from our study, including samples from SHH ($n = 51$), Group3 ($n = 46$), and Group4 ($n = 188$), using the Affymetrix Gene 1.1 ST platform. Global integration of expression data was performed by comparing expression levels of amplified or deleted genes relative to genes in balanced regions (Mann-Whitney tests). Further, specific integration of expression data at each significant CNA locus were done by comparing expression of genes in samples with the aberration against those that do not, within in medulloblastoma subgroup. Multiple hypothesis correction by the false discovery rate method was applied to each locus independently, and the false discovery rate threshold was adaptively tuned for each locus so that the no false positives are expected. The resulting lists of genes with copy-number driven expression were used in candidate identification.

\subsection{Identification of candidate driver genes in each significant region}

Many of the significant regions identified using GISTIC and CMDS span multiple genes. Therefore, multiple lines of evidence were used to prioritize putative target genes within each region. Evidences collected from integrated expression data, the literature, and multiple datasets were classified into the tiers in \citetab{driver-evidence}.

\begin{table}[H]
	\caption[Tiered evidence framework for identifying candidate driver genes]
	{
		Tiered evidence framework for identifying candidate driver genes
	}
	\label{tab:driver-evidence}
	\scriptsize
	\setlength{\extrarowheight}{0.5em}
	\centering
	\begin{tabular}{ p{0.25\linewidth} | p{0.6\linewidth} | p{0.05\linewidth} }
		\hline
		\textbf{Type} & \textbf{Description} & \textbf{Tier} \\
		\hline
		Correlated expression & Gene expression is driven by SCNA in the integrated analysis & I \\
		Medulloblastoma literature & Implicated in medulloblastoma from the literature (PubMed) & I \\
		Cancer Gene Census & Documented in the Cancer Gene Census & I \\
		Parsons & Reported to be somatically mutated in the Parsons \emph{et al.}\ \citeref{parsons11} study on SNVs identified in medulloblastoma & III \\
		ICGC & Reported to be somatically mutated in the ICGC medulloblastoma study & III \\
		Northcott signature gene & Reported in the Northcott \emph{et al.}\ \citeref{northcott11a} medulloblastoma expression study & IV \\
		Cho signature gene & Reported in the Cho \emph{et al.}\ \citeref{cho11} medulloblastoma expression subgroup study & IV \\
		RNASeq SNV & Identified in the pilot RNASeq screen for SNV (Group 3 and 4 only) & IV \\
		Gli1 target & Identified as a Gli1 target in Lee \emph{et al.}\ \citeref{lee10} (SHH subgroup only) & IV \\
		Shh-inducible target & Identified as Shh-inducible gene by a screen in cerebellar granule neuron precursors (SHH subgroup only) & IV \\
		COSMIC & Documented in the Catalogue of Somatic Mutations in Cancer & V \\
		\hline
	\end{tabular}
\end{table}

The priority of each gene in a region was ranked based on the total score of the above lines of evidence, weighted by their respective tiers. A higher tier (lower number) is assigned a higher weight. The weights were assigned so that supporting evidence from the next tier is only considered for breaking ties at previous tiers. At most two genes from each region were selected for network analysis using this evidence-driven ranking.

\subsection{Mutual exclusivity analysis}

The significant gene lists were analyzed to detect mutual exclusive relationships by iterating through all possible combination of genes in the list (for combination sizes from 2 to 6). Within each subgroup, combinations of genes with the highest exclusivity scores were identified. The exclusivity score was defined as the number of samples that harbour exactly one aberration among the genes in the combination; in other words, it is the product of exclusivity and coverage as defined by Miller \emph{et al}\ \citeref{miller11}.

\subsection{Network analysis}

Pathway enrichment analysis of copy number aberrations was carried out using g:Profiler web server \citeref{reimand11} and visualized in Cyotscape software as an Enrichment Map network \citeref{merico10}. Candidate driver genes from selected GISTIC2 regions were compiled for SHH, Group3, and Group4, ranked by frequency, and subsequently queried for significantly enriched functional categories with g:Profiler using the ordered list algorithm (FDR-corrected cutoff p=0.05, hypergeometric test). Detected categories were filtered with a custom R script to only include Gene Ontology (GO) terms, and Reactome and KEGG pathways, using an upper limit of 500 genes per gene set and the requirement that at least two putative driver genes intersect with the gene set. A small number of non-informative KEGG pathways were removed from the final list. The overlap coefficient value 0.6 was used in Enrichment map visualization.  Enrichment maps were manually adjusted to highlight the most significant themes for visualization purposes.  J\''{u}ri Reimand performed the network analysis.

\subsection{Unsupervised clustering analysis of copy number events}

All significant broad events (spanning chromosome arms) and focal events (identified as described earlier) identified in the pan-cohort analysis were used in the unsupervised clustering of medulloblastoma samples, by the Ward linkage method and the Euclidean distance metric, as implemented in the R environment. The copy number states were converted to absolute values and samples with unknown subgroup affiliation were removed prior to clustering. The agreement between the observed clusters and the medulloblastoma subgroups were assessed by the Adjusted Rand Index and tested by the $\chi^2$ test.

\subsection{Expression array processing and data analysis}

For gene expression array profiling, 400ng total RNA was processed and hybridized to the Affymetrix Gene 1.1 ST array at TCAG according to the manufacturer’s instructions. The CEL files were quantile normalized using Expression Console (v1.1.2; Affymetrix, USA) and signal estimates determined using the RMA algorithm.  Prior to clustering analysis of Group 4 medulloblastomas, 500-1000 high standard deviation (SD) genes were selected and the expression signals unlogged.  Unsupervised clustering was carried out using the NMFConsensus module available on the GenePattern public server (Broad Institute, USA) with default parameters.  The cophenetic coefficient metric was used to assess stability of the tested sample clusters.

\subsection{nanoString CodeSets and data analysis}

To determine subgroup affiliation of the MAGIC cohort, a custom nanoString CodeSet was designed to assess the expression status of 22 medulloblastoma signature genes and samples processed as described in detail in a recent publication \citeself{northcott12}.  Samples were processed as recommended by nanoString at the University Health Network (UHN) Microarray Facility using an input of 100ng total RNA.  Raw nanoString counts for each gene within each experiment were subjected to a technical normalization using the counts obtained for positive control probe sets prior to a biological normalization using the three housekeeping genes included in the CodeSet. Normalized data was log2-transformed and then used as input for class prediction analysis using the PAM method.  A series of 101 medulloblastomas with known subgroup affiliation were used as a training dataset for class prediction \citeref{northcott11a}.

\subsection{Statistical and bioinformatic analyses}

Statistical and bioinformatic analyses were performed in the R statistical environment (v2.13) or using custom programs/scripts written in Python, C++, or Go. Enrichment analyses were done using the hypergeometric test. The significance of chromosome arm frequencies were done using the exact binomial test, comparing the observed frequency to the expected frequency derived from a robust regression of event frequency and gene content, in a similar manner to the broad analysis in GISTIC2. Comparisons of event frequencies across medulloblastoma subgroups were performed using Fisher’s exact test. Expressions of genes across samples were compared using the Mann-Whitney test (and confirmed with the Student’s independent t-test). Where applicable, multiple hypothesis corrections were performed using the false discovery rate method \citeref{storey03}. Univariate survival analyses were done using the log-rank test, as implemented in the survival R package (v2.36).


\section{Results}

\subsection{Molecular subgroups have disparate patterns of genomic events}

Copy-number profiles were generated on $> 1200$ medulloblastomas using the Affymetrix Genome-wide SNP6 platform. After quality control and clinical criteria filtering, copy-number profiles of $1087$ primary medulloblastomas were available for further analysis in identifying \gls{scna} events: regions of aberrant gains and losses in the tumour genome. The tumours were stratified based on molecular subgroups, as determined by the method described in \citech{mb-class}. The copy-number and cytogenetic profiles of medulloblastoma subgroups were highly divergent, demonstrating that medulloblastoma subgroups are genomically heterogeneous (appendix figure not shown). Indeed, when the cohort was analyzed by each subgroup independently, an increased number of \gls{scnas} were identified, many of which were subgroup-enriched (\citefig{subgroup-specificity}).

\begin{SCfigure}[5]
	\centering
	\includegraphics[width=0.22\textwidth]{fig/magic-cn/subgroup-specificity.pdf}
	\caption[Significant regions of focal SCNA identified by GISTIC2]
	{
	Significant regions of focal SCNA identified by GISTIC2 in pan-cohort or subgroup-stratified analyses.
	A total of 62 significant regions were identified when the cohort was analyzed as a single group, whereas 110 significant regions were captured when the cohort was analyzed according to subgroup. The number of significant subgroup-enriched regions identified more than doubled (73 vs. 30) when the subgroups were analyzed independently.
	}
	\label{fig:subgroup-specificity}
\end{SCfigure}

\subsection{Recurrent high-level amplifications and homozygous deletions target known cancer-associated genes}

Among the recurrent high-level amplifications (copy-number $\geq 5$) identified (\citefig{high-level-amps}), the most prevalent events targeted members of the MYC family (\gene{MYCN}, \gene{MYC}, \gene{MYCL}), with \gene{MYCN} predominantly amplified in SHH and Group4, \gene{MYC} in Group3, and \gene{MYCL} in SHH medulloblastomas. The most common homozygous deletions targeted known tumour suppressors \gene{PTEN}, \gene{PTCH1}, and \gene{CDKN2A/B}, all of which were enriched in SHH tumours (\citefig{homo-del}). A selected set of genes were assessed using custom nanoString assays, and 90.9\% of events were verified (\citefig{nanostr-verification}). Additional genes were validated on external cohorts by \gls{fish} \citeref{shih12}.

\subsection{Chromothripsis is rare in WNT medulloblastoma}

Chromothripsis (chromosome shattering) lead to co-occurrence of high-level amplifications and disruption of several genes localized to specific regions in one or multiple chromosomes. Several samples exhibit genomic aberrations reminiscent of chromothripsis, which has recently been implicated in cancer formation \citeref{stephens11, liu11, kloosterman11b, magrangeas11, crasta12, molenaar12}, as well as in medulloblastoma \citeself{rausch12}. In our cohort of chromothripsis, the incidence of chromothripsis is not uniform across subgroups ($p = 0.015$, Fisher's exact test). While the incidence of chromothripsis is depleted in WNT tumours compared to non-WNT tumours ($p = 0.0028$), chromothripsis is neither enriched nor depleted in the SHH, Group3, or Group4 ($p > 0.05$). The incidence of chromothripsis is about 12\% among SHH, Group3, and Group4 tumours. These findings are consistent with the observation that WNT medulloblastoma show no recurrent \gls{scna} other than chr6 loss and appear more genomically stable \citeref{shih12}.

\subsection{Subgroup-specific genomic events converge on oncogenic pathways}

The disparate genomic landscapes of medulloblastoma subgroups lead to the identification of a multitude of focal \gls{scnas} that characterize each molecular subgroup (appendix figures not shown). Novel genes identified in this study include: \gene{PPM1D}, \gene{PIK3C2B}, and \gene{MDM4} in SHH (\citefig{shh-amps-igv}); \gene{ACVR2A}, \gene{ACVR2B}, and \gene{TGFBR1} in Group3 (\citefig{group3-amps-igv}); and \gene{NFKBIA} and \gene{USP4} in Group4 (\citefig{group4-dels-igv}). Conversely, WNT medulloblastoma have few recurrent \gls{scnas} (\citefig{genome-coverage}). 

\begin{SCfigure}[5][b]
	\centering
	\includegraphics[width=0.15\textwidth]{fig/magic-cn/nanostr-verification.pdf}
	\caption[Verification of focal \gls{scnas} by nanoString]
	{
	Verification of focal \gls{scnas} by nanoString.
	Genes inferred to be focally amplified by SNP6 were interrogated using a custom nanoString CodeSet across a set of 192 medulloblastomas selected from our cohort. Bar-plot shows the number of samples for which each gene is verified (red) or not (black). An overall verification rate of 90.9\% was achieved.
	}
	\label{fig:nanostr-verification}
\end{SCfigure}

\clearpage

\begin{SCfigure}[5]
	\centering
	\includegraphics[width=0.4\textwidth]{fig/magic-cn/high-level-amps.pdf}
	\caption[Recurrent high-level amplifications in medulloblastoma]
	{
	Recurrent high-level amplifications in medulloblastoma.
	Frequency of genes amplified (segmented copy-number $\geq 5$) in at least two samples are shown with the distribution of the event across subgroups. The number of genes mapping to the peak region as defined by GISTIC2 (where applicable) are listed in parentheses after the candidate driver gene.
	}
	\label{fig:high-level-amps}
\end{SCfigure}

\begin{SCfigure}[5]
	\centering
	\includegraphics[width=0.4\textwidth]{fig/magic-cn/homo-del.pdf}
	\caption[Recurrent homozygous deletions in medulloblastoma]
	{
	Recurrent homozygous deletions in medulloblastoma.
	Frequency of genes targeted by homozygous deletion (segmented copy-number $\leq 0.7$) in at least two samples are shown.
	}
	\label{fig:homo-del}
\end{SCfigure}

\begin{SCfigure}[5]
	\centering
	\includegraphics[width=0.2\textwidth]{fig/magic-cn/genome-coverage.pdf}
	\caption[WNT medulloblastomas sustain a paucity of recurrent focal \gls{scnas}.]
	{
	WNT medulloblastomas sustain a paucity of recurrent focal \gls{scnas}.
	Bar-plots of the proportion of genome recurrently disrupted by focal \gls{scnas} are depicted for each medulloblastoma subgroup.
	}
	\label{fig:genome-coverage}
\end{SCfigure}

\clearpage

\begin{SCfigure}[1][t]
	\centering
	\includegraphics[width=0.5\textwidth]{fig/magic-cn/shh-amps-igv.pdf}
	\caption[Recurrent amplifications of \gene{PPMID}, \gene{MDM4}, and \gene{PIK3C2B} in SHH medulloblastoma]
	{
	Recurrent high-level amplifications of \gene{PPMID} and co-amplification of \gene{MDM4} and \gene{PIK3C2B} in SHH medulloblastoma.
	Segmented copy-number tracks are shown for the amplified loci (17q23 and 1q23).
	}
	\label{fig:shh-amps-igv}
\end{SCfigure}

\begin{SCfigure}[1][b]
	\centering
	\includegraphics[width=0.45\textwidth]{fig/magic-cn/group3-amps-igv.pdf}
	\caption[Recurrent amplifications target receptors of the TGF$\beta$ superfamily in Group3]
	{
	Recurrent amplifications target receptors of the TGF$\beta$ superfamily in Group3.
	Segmented copy-number tracks of Group3 medulloblastomas show recurrent high-level amplifications affecting \gene{ACVR2A} (2q22), \gene{ACVR2B} (3p22), and \gene{TGFBR1} (9q22).
	}
	\label{fig:group3-amps-igv}
\end{SCfigure}

\clearpage

SHH medulloblastoma, which is characterized by activation of Shh signaling \citeref{northcott11a,remke11,cho11,kool08,taylor12}, exhibit frequent \gls{scnas} in the Shh pathway. Genes involved in focal \gls{scnas} amplifications are significantly associated with SHH medulloblastoma signatures genes (appendix figure not shown), suggesting that copy-number changes contribute in part to the altered expression signatures previously observed in SHH tumours. Accordingly, positive regulators of Shh signaling (\gene{MYCN} and \gene{GLI2}) were recurrently amplified, while a negative regulator of Shh signaling (\gene{PTCH1}) was recurrently lost. Consistent with their functions in the same pathway, these events were mutually exclusive; however, they lead to different clinical outcomes (appendix figure not shown). In additional to Shh signaling, other core pathways recurrently disrupted in SHH medulloblastoma are TP53 signaling and RTK/PI3K signaling (\citefig{shh-pathways}).

\begin{SCfigure}[5]
	\centering
	\includegraphics[width=0.6\textwidth]{fig/magic-cn/shh-pathways.pdf}
	\caption[Core pathways genetically targeted in SHH medulloblastoma]
	{
	Core pathways genetically targeted in SHH medulloblastoma.
	Summary of \gls{scnas} affecting components of Shh signaling, TP53 signaling, and RTK/PI3K signaling are depicted. Colours reflect the frequency by which the respective genes are targeted by focal or broad events in SHH medulloblastomas (red for amplification, blue for deletion). Significance values indicate the prevalence with which each pathway is targeted in SHH vs. non-SHH cases (Fisher's exact test).
	}
	\label{fig:shh-pathways}
\end{SCfigure}

\begin{SCfigure}[5][b]
	\centering
	\includegraphics[width=0.3\textwidth]{fig/magic-cn/group3-pathways.pdf}
	\caption[TGF$\beta$ signaling is recurrently disrupted by \gls{scnas} in Group3]
	{
	TGF$\beta$ signaling is recurrently disrupted by \gls{scnas} in Group3.
	\gls{scnas} affecting the TGF$\beta$ pathway comprise 20.2\% of Group3 cases and are significantly enriched in Group3 compared to non-Group3 cases (Fisher's exact test).
	}
	\label{fig:group3-pathways}
\end{SCfigure}

\clearpage

The signaling pathways involved in Group3 and Group4 medulloblastomas are less well understood, as suggested by their names. Nonetheless, at the copy-number level, distinct pathways were dysregulated in Group3 and Group4 (appendix figure not shown). Group3 tumours are characterized by amplification of \gene{MYC} and \gene{OTX2}, which occur in a mutually exclusive pattern (appendix figure not shown). This observation is consistent with the tendency of the two oncogenic transcription factors to bind the same promoter regions \citeref{bunt11}. Further, the TGF$\beta$ signaling pathway is frequently disrupted by \gls{scnas} in Group3 (\citefig{group3-pathways}). Conversely, the NF-$\kappa$B pathway appear to genetically targeted in Group 4 medulloblastomas (\citefig{group4-dels-igv}).

\begin{SCfigure}[2]
	\centering
	\includegraphics[width=0.2\textwidth]{fig/magic-cn/group4-dels-igv.pdf}
	\caption[NF-$\kappa$B pathway is recurrently targeted in Group4]
	{
	NF-$\kappa$B pathway is recurrently targeted in Group4.
	Recurrent focal deletions disrupt \gene{NFKBIA} and \gene{USP4}, negative regulators of the NF-$\kappa$B pathway, in Group4 medulloblastoma.
	}
	\label{fig:group4-dels-igv}
\end{SCfigure}

While \gene{MYC} amplification is a known pivotal player Group3, our data indicated that others genes in close proximity to the \gene{MYC} locus may also play cooperative roles. This locus was frequently disrupted by a multitude of high-level amplicons (appendix figure not shown) as well as massively genomic rearrangements termed chromothripsis (\citefig{chromothr_myc_wgs}). As a consequence of these events, the adjacent \gene{PVT1} gene and miR-1204 are frequently co-amplified with \gene{MYC}. Moreover, amplifications of the \gene{MYC}/\gene{PVT1} locus frequently result in the formation of fusion transcripts. Concurrent with MYC-PTV1 fusion expression, miR-1204 (hosted in PTV1) is upregulated (appendix figure not shown). \gene{MYC}, \gene{PVT1}, and miR-1204 all have been previously shown to play independent functional roles in other tumours \citeref{guan07,carramusa07,huppi08,barsotti12}, and may synergistically promote tumourigenesis.

The most prevalent focal gain (and previously neglected) in Group4 was the somatic tandem duplication of the \gene{SNCAIP} gene (appendix figure not shown). SNCAIP expression is highly elevated in Group4 medulloblastomas (\citefig{sncaip-expr}). In fact, \gene{SNCAIP} duplication is further restricted to the Group4$\alpha$ and may play a functional role in this medulloblastoma subtype (\citefig{group4-alpha-beta}).


\clearpage


\begin{figure}[t]
	\begin{center}
		\includegraphics[width=0.6\textwidth]{fig/magic-cn/chromothr_myc_ai.pdf}
	\end{center}
	\caption[A multitude of amplicons disrupt the \gene{MYC}/\gene{PVT1} locus]
	{
	A multitude of amplicons disrupt the \gene{MYC}/\gene{PVT1} locus.
	Copy-number plot of 8q24.21 is shown for a representative sample. Dots represent raw copy-number estimates and lines denote copy-number segments and state (red: gain, blue: loss). 71.4\% of \gene{MYC}-amplified (20/28) cases exhibit (partial) co-amplification of adjacent non-coding \gene{PVT1} gene and miR-1204. PVT1-MYC fusion transcripts were detected by RNA-seq, qRT-PCR, and Sanger sequencing in 8/20 samples \citeself{shih12}.
	}
	\label{fig:chromothr_myc}
\end{figure}

\begin{figure}[b]
	\begin{center}
		\includegraphics[width=0.7\textwidth]{fig/magic-cn/chromothr_myc_wgs.pdf}
	\end{center}
	\caption[Chromothripsis disrupts the \gene{MYC}/\gene{PVT1} locus.]
	{
	Chromothripsis disrupts the \gene{MYC}/\gene{PVT1} locus.
	Whole-genome sequencing confirms a complex pattern of rearrangements on 8q24 in a representative sample, reminiscent of chromothripsis (chromosome shattering).
	Plot shows chromosome 8 copy-number estimates (derived from read depth ratio of tumour vs. matched germline).
	Complex rearrangements are observed in concordance with a multitude of amplicons in the 8q24 region.
	}
	\label{fig:chromothr_myc_wgs}
\end{figure}

\clearpage

\begin{figure}[t]
	\begin{center}
		\includegraphics[width=0.8\textwidth]{fig/magic-cn/sncaip-expr.pdf}
	\end{center}
	\caption[\gene{SNCAIP} is a Group4 signature gene]
	{
	\gene{SNCAIP} is a Group4 signature gene.
	\emph{Left}, Box-plot depicting SNCAIP significant upregulation in Group4 (Mann-Whitney test), as determined by expression analysis of a previously published cohort of 103 primary medulloblastoma \citeref{northcott11a}. SNCAIP ranks among the top 1\% of most highly expressed genes in Group4 medulloblastoma (rank 39 out of 16758).
	\emph{Right}, Validation of SNCAIP as a Group4 signature gene across five published medulloblastoma expression datasets: Thompson \citeref{thompson06}, Kool \citeref{kool08}, Fattet, Cho \citeref{cho11}, and Remke \citeref{remke11}. Expression datasets total 396 cases on four different array platforms. In all datasets, SNCAIP exhibited highest expression in Group4.
	}
	\label{fig:sncaip-expr}
\end{figure}

\begin{figure}[b]
	\begin{center}
		\includegraphics[width=0.7\textwidth]{fig/magic-cn/group4-alpha-beta.pdf}
	\end{center}
	\caption[\gene{SNCAIP} duplication is restricted to one subtype of Group4]
	{
	\gene{SNCAIP} duplication is restricted to one subtype of Group4.
	\textsf{a}, Non-negative matrix factorization (NMF) consensus clustering performed on expression profiles of Group4 cases ($n = 188$) reveal two transcriptionally distinct subtypes of Group4, designated $4\alpha$ and $4\beta$. SNCAIP duplicated is significantly enriched in the Group $4\alpha$ subtype (Fisher's exact test).
	\textsf{b}, SNCAIP expression is significantly elevated in Group $4\alpha$ compared to $4\beta$ (Mann-Whitney test).
	\textsf{c}, SNCAIP expression is copy number-driven in Group $4\alpha$. Group $4\alpha$ cases were stratified by \gene{SNCAIP} copy-number status. Samples harbouring SNCAIP duplication exhibit a significant \~1.5-fold increase in SNCAIP expression (Mann-Whitney test).
	}
	\label{fig:group4-alpha-beta}
\end{figure}

\clearpage

\section{Discussion}

This study raised several bioinformatic challenges particularly due to the lack of germline samples. Due to current and historical practices, the medulloblastoma samples amassed in this study were often not paired with germline samples. Owing to the rarity of large germline copy-number variants (spanning $> 1$ Mbp or $> 10$ probes) in control population \citeref{shih12}, somatic \gls{cna} and germline variants can be distinguished. We also dismissed copy-number variants observed in both cases and controls at high frequency ($> 90$\%). Idenitified likely germline copy-number variants were removed from copy-number profiles along with the breakpoints they introduced. Despite our best efforts, some likely germline events and hybridization artifacts (unrelated to submission batch but possibly related to reagent lot) remain in the copy-number profiles of the samples. Nonetheless, with additional post-processing, careful curation, focus on high frequency events, and integration with multiple sources of evidence, we successfully identified recurrently disrupted genomic loci, though we caution against the individual use of each tumour copy-number profile for future studies. The poor interpretability of individual copy-number events in a single tumour sample stems from non-systematic hybridization biases, poor sample DNA quality, the general genomic instability of some samples, and the notion that most genetic events during cancer evolution do not contribute to cancer cell survival and tumour formation. In light of these limitations, we prioritized genomic events observed frequently across multiple samples. Furthermore, we splitted samples into more homogeneous groups in order to identify such events above a background rate. Indeed, we discovered significantly more recurrent \gls{cna} when the samples were split by subgroup than when they were analyzed together, enforcing the utility of regarding medulloblastoma as four distinct diseases. To emphasize, we discovered patterns hidden in complex genomic profiles of tumours by contrasting biological similar groups of samples.

Another common issue that arises in cancer genomics studies is the role that genomic instability plays in tumour formation, progression, or recurrence. We considered different types of genomic instability separately and jointly, but we did not identify any significant and reproducible association between aneuploidy, frequent structural variations, or chromothripsis with patient survival (data not shown). We contend that genomic instability in itself does not impact outcome, though the underlying cause or the consequent disruption of specific genes may affect tumour aggressiveness. For example, \gene{TP53} mutation may conduce to chromothripsis, but only the former portends poor outcome in multivariate survival analyses. Chromothripsis does not exclusively occur in \gene{TP53} mutated cases; thus, it may be associated with mutations in other DNA damage response or repair genes, not all of which may contribute to tumour resistance against chemotherapy or radiotherapy. Further, chromothripsis in SHH medulloblastoma often results in amplification of \gene{GLI2}, but only the latter is independently prognostic for poor survival \citeref{shih14}. These findings in our medullobalstoma cohort contrast the association of chromothripsis with poorer survival (by univariate instead of multivariate analysis) in neuroblastoma \citeref{molenaar12}. Further, our pattern of chromothripsis incidence also contrasts that of a smaller cohort of medulloblastoma \citeref{rausch12}, in which the authors observed significantly higher frequency of chromothripsis in SHH medulloblastoma. In contrast, the incidence of chromothripsis is fairly uniform across SHH, Group3, and Group4 medulloblastomas and significantly depleted in WNT medulloblastoma, These results could be due to characteristic differences between the two cohorts of medulloblastoma examined. Both studies used the same method and algorithm to identify chromothripsis (the same researcher processed the raw SNP array data and made the chromothripsis calls in both studies). Given these discrepancies across studies, the interaction among competing covariates may obscure meaningful interpretation, and recall biases in retrospective studies may skew distributions of specific variables in studies with small sample sizes.

Notwithstanding the technical, biological, and statistical challenges, we have successfully identified recurrent amplifications and deletions of genes that converge in specific signaling pathways contributing to tumourigenesis within different molecular subgroups of medulloblastoma. WNT medulloblastoma is surprisingly devoid of recurrent genomic aberrations aside from chr6 loss and \gene{CTNNB1} activating mutation. While the functional consequence of chr6 is unclear, the high frequency of \gene{CTNNB1} mutation suggest that activated \gls{wnt} signaling may be the predominant tumourgenic mechanism in WNT medulloblastoma. Similarly, recurrent \gls{cna} events converge on and conspire to activate \gls{shh} signaling in SHH medulloblastoma. In a subset of SHH cases, TP53 signaling and \gls{pi3k} signaling are also recurrently disrupted by \gls{cna}, and the latter may be a candidate pathway for therapeutic intervention against tumours with aberrations upstream of the drug target. The dominant theme in Group3 medulloblastoma is recurrent dysruption of signaling of the super \gls{tgfb} family; in particular, \gls{cna} converge on activation of activin signaling, and this pathway may be the first candidate pathway for therapeutic intervention in this aggressive subgroup. While no signaling pathway is recurrently dysregulated in Group4 medulloblastoma, tandem duplications of \gene{SNCAIP} is frequently observed and may play an important role in the pathogenesis of Group4 medulloblastoma. Taken together, copy-number profiling of medulloblastoma samples have led to the discovery of several frequently disrupted genes and pathways that may serve as candidates for targeted therapeutic intervention.

The disparate patterns of recurrent \gls{cna} among the medulloblastoma subgroups support the distinct etiologies of the subgroups but also raises questions regarding the origin of the molecular subgroups of medulloblastoma. Most recurrent \gls{cna} were enriched (observed at higher frequency) in specific subgroups but were not exclusively found in one subgroup. Further, unsupervised clustering by copy-number profiles produced groups that showed only modest agreement with the molecular subgroups of medulloblastoma. These results taken together suggest that the \gls{cna} events do not determine the molecular subgroups, though they may modulate the expression patterns of the subgroups. In contrast, DNA methylation patterns can identify without supervision four medulloblastoma classes that show high concordance with the molecular subgroups identified by expression patterns \citeref{hovestadt13, schwalbe13}. These results indicate that the epigenetic landscape in the cell of origin controls the expression patterns of the developing tumour and may also provide the contexts that favours specific genomic aberrations, thus shaping different genomic landscapes for each molecular subgroup of medulloblastoma.

The Myc family of proto-oncogenes is amplified in a curious pattern across the medulloblastoma subgroups and pose the question regarding whether and how epigenetic patterns influence amplification patterns of the Myc family, comprising \gene{MYC}, \gene{MYCN}, and \gene{MYCL}. These homologous genes were found to be amplified in lymphoma, neuroblastoma, and lung carcinoma, respectively; these observations possibly points to functional differences in Myc family proteins in diverse cancer types. Similarly, Myc family genes are amplified in a non-random pattern in medulloblastoma: \gene{MYC} is amplified predominately (albeit not exclusively) in Group3, \gene{MYCN} is amplified preferentially in SHH and Group4 medulloblastomas, and rare \gene{MYCL} amplifications are observed only in SHH medulloblastoma. At the expression level, most medulloblastomas of either WNT or Group3 subgroup MYC independent of amplification. Similarly, SHH medulloblastomas express MYCN at a higher level than other subgroups and normal cerebellums even without \gene{MYCN} amplification. 
In neuroblastoma, \gene{MYCN} amplification is found in about 20\% of cases \citeref{brodeur84, olenaar12}, but amplification of \gene{MYC} or \gene{MYCL} is very rarely observed \citeref{molenaar12}. These observations raise the question of why \gene{MYC}, \gene{MYCN}, and \gene{MYCL} is preferentially dysregulated in one cancer type or subtype vs. another. Conceivably, Myc family genes could interact with different sets of co-transcription factors, drive different transcriptional programs, and cooperate with dysregulation of other signaling pathways in promoting tumour formation. Accordingly, the amplification of different Myc family genes could modulate the expression patterns of the medulloblastoma subgroups.

A alternative hypothesis is that differential accessibilities at genomic loci in the cell of origin govern which of the three human Myc family gene is amplified. Under this model, \gene{MYCN} amplification preferrentially occur in SHH medulloblastoma because \gene{MYCN} is expressed in the cell of origin for SHH medulloblastoma (e.g. external granule neural precursor) during normal development, rendering the genomic locus accessible to such enzymes as deaminases (e.g. \gene{AICDA}, \gene{APOBEC}), which initiate double-strand breaks that ultimately lead to DNA rearrangement. Similarly, \gene{MYC} amplification may preferrentially occur in Group3 medulloblastoma because the \gene{MYC} locus is permissible to structural rearrangement. In other words, the observed patterns of Myc family gene amplification arise mainly due to variations in accessibility of these genomic loci in different cells of origin instead of functional differences among Myc family members. Consistent with this model, MYCN is functionally redundant with MYC, and knock-in of \gene{MYCN} into the \gene{MYC} locus rescues the loss of \gene{MYC} in mouse development, cell growth, and cell differentiation \citeref{malynn00}. \gene{MYC} and \gene{MYCN} double knock-out mice exhibit much more severe phenotype in neurogensis than either one alone \citeref{wey10}. Further, \gene{MYC} and \gene{MYCN} are interchangeable in promoting tumour formation. \gene{MYCN} coding region can substitute for that of \gene{MYC} in transformation of pro-B cell \citeref{gostissa09}. Similarly, MYC can cooperative with SHH ligand to enhance formation of SHH medulloblastoma in a virus-induced mouse tumour model, despite that \gene{MYC} amplification is rarely observed in human SHH medulloblastoma \citeref{rao03, shih12}. Therefore, Myc family genes may indeed encode functionally similar proteins in promoting medulloblastoma formation.

Despite being an important player in transforming normals cells into cancer or in inducing pluripotency in differentiated cells, a Myc family
gene may exert different effects in different tumour contexts. In Group3 medulloblastoma, \gene{MYC} amplification is associated with poorer survival. However, \gene{MYCN} amplification is associated with poorer survival only in SHH medulloblastoma but not in Group4 medulloblastoma. The amplification of a Myc family gene is thus not universally associated with poorer outcome, and the function of Myc in medulloblastoma may be modulated by cooperating dysruption of other pathways in the tumour initiating cell and by extracellular signals in the context of neural development.

The story of the Myc family illustrates a broader theme of how genomic alterations lead to different subgroups medulloblastoma. \gene{MYC} amplifications can drive tumour formation, but only in the suitable cellular context and other co-occuring mutations \citeref{rao03, pei12}. Indeed, \gene{MYC} and \gene{SHH} overexpression are sufficient to induce tumour formation in the external granule layer of the cerebellum but not in the cerebrum. Accordingly, MYC overexpression likely do not create Group3 medulloblastoma. Rather, the (possibly multiple) cells of origin and cooperating alterations provide the necessary context for the amplification of \gene{MYC} and consequent induction of medulloblastoma, and the cell of origin govern the molecular subgroup of the resulting tumour. Similarly, \gene{MYCN} amplification is a common mechanism of tumourigenesis in subsets of SHH and Group4 medulloblastoma cases, and it does not define a molecular subgroup by itself. Indeed, gene amplification of the Myc family does not define a molecular entity, as the medulloblastoma subgroup are defined by epigenetic and expression profiles, and the Myc family members lead to disparate outcomes in different medulloblastoma subgroups. In experimental models, Myc family overexpression, along with cooperating mutations, can likely transform multiple cells of origin, and the subgroup of the resulting tumour would likely be determined by the cellular origin rather than the Myc family member. More generally, tumour of specific subtypes arise due to interaction between genetics and context. The cellular origin provides the permissive context for genomic alterations to exert their effects, and the nuclear organization, epigenetic landscape, chromatin architecture within the cell of origin likely influence the accessibility of genomic loci and shape the mutational landscape of the cancer cell \citeref{polak15}.

In summary, medulloblastoma subgroups are characterized by distinct genomic alterations that dysregulated disparate signaling pathways. The mutational profiles do not define each subgroup; instead, they are shaped by the molecular and cellular context within each subgroup, and the functional consequences of the mutations may also on the molecular subgroups. Hence, genes and pathways that are recurrently dysrupted by genomic alterations within each subgroup likely play pivotal roles in the formation and perhaps maintenance of the tumours, and such genes and pathways serve as prime candidates for design of rational, targeted therapuetic intervention.

\chapter{Molecular classification of medulloblastoma models}
\label{ch:mmu-mb-class}

\section{Materials and methods}

\section{Results}

\section{Discussion}

\chapter{Conclusions and future directions}
\label{ch:conclusion}

Molecular classification driving the next generation of clinical trials \citeref{leary12}.

Is medulloblastoma an entity?
Likely, given its distinguishing expression pattern.

What should be use for classifying cancer?
Anatomical location and histology will likely remain an integral core of classification.
What about genetics? Sure, some genes identifies a cancer type: RB (retinoblastoma), SMARCB1 (\gls{atrt}), PTCH1 (basal cell carcinoma and SHH medulloblastoma) but others are less definiing: TP53 (a spectrum of tumours; inhibiting TP53 will not restore the genomic damage already done). Most cancer mutations likely do not contribute to tumourigenesis.
What about epigenetics? reflect cell type and form contexts permissive for driver mutations. It has a similar goal as histology, but can be done high-throughput and objectively.

The discovery of the four molecular subgroups of medulloblastoma has paved the road for the rational development of specific targeted therapy and promises hope for achieving personalized medicine in the treatment of medulloblastoma, where each patient will receive an individualized regimen that maximizes efficacy and minimizes side-effects. Medulloblastomas, not unlike many other malignancies, arises due to a multitude of genetic aberrations, leading to disruptions of biological processes that differ from patient to patient. As we continue to tease out the intricate biological differences among patients with medulloblastomas, we must also provide medical practitioners with the means and the motivation to realize the potential of our discoveries to date, so that both science and medicine can move forward, side by side.

Two major obstacles in the adoption of molecular subgroup classification in the clinic are the lack of applicable biospecimen for genomic analysis and the discrepancy in quality control standards between research and clinical labs. To overcome these challenges, I have tested and validated an assay suitable for subgroup classifcation of \gls{ffpe} samples, and I have implemented rigorous quality control to ensure that the assay results are reproducible, credible, and suitable for guiding clinical decision-making.

If molecular subgroup information does not influence clinical treatment, however, the assay would have little clinical utility. I have therefore identified actionable signalling pathways within medulloblastoma subgroups through copy-number profiling that may serve as rational targets for future therapeutic development. To further fuel the motivation for classifying medulloblastoma subgroups in the clinic, I have identified molecular biomarkers that, together with clinical biomarkers, can stratify patients into risk groups using schemes specific to each molecular subgroup and attain unprecedented prognostic accuracy.

Accordingly, we have addressed some of the challenges that face the treatment of patients with medulloblastoma and provided evidence that supports the classification of medulloblastoma by molecular subgroups in the clinic. By providing medical practitioners with both the means and the motivation to realize the clinical potential of our findings, we strive to usher in a new era of medulloblastoma treatment based on individualized targeted therapy that enhances the quality of care and preserves the quality of life for the patients.



%% Appendix

\singlespacing
\small

\addcontentsline{toc}{chapter}{Publications Arising from Thesis}

\vspace*{-10em}
\bibliographystyleself{nature}
\bibliographyself{self}
\bigskip
$^*$ These authors contributed equally.

\clearpage

\clearpage

\addcontentsline{toc}{chapter}{References}

\bibliographystyleref{nature}
\bibliographyref{mb,tech,chromothr,cancer,array-qc}


\end{document}
